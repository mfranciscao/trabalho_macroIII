% Options for packages loaded elsewhere
\PassOptionsToPackage{unicode}{hyperref}
\PassOptionsToPackage{hyphens}{url}
%
\documentclass[
]{article}
\usepackage{amsmath,amssymb}
\usepackage{lmodern}
\usepackage{iftex}
\ifPDFTeX
  \usepackage[T1]{fontenc}
  \usepackage[utf8]{inputenc}
  \usepackage{textcomp} % provide euro and other symbols
\else % if luatex or xetex
  \usepackage{unicode-math}
  \defaultfontfeatures{Scale=MatchLowercase}
  \defaultfontfeatures[\rmfamily]{Ligatures=TeX,Scale=1}
\fi
% Use upquote if available, for straight quotes in verbatim environments
\IfFileExists{upquote.sty}{\usepackage{upquote}}{}
\IfFileExists{microtype.sty}{% use microtype if available
  \usepackage[]{microtype}
  \UseMicrotypeSet[protrusion]{basicmath} % disable protrusion for tt fonts
}{}
\makeatletter
\@ifundefined{KOMAClassName}{% if non-KOMA class
  \IfFileExists{parskip.sty}{%
    \usepackage{parskip}
  }{% else
    \setlength{\parindent}{0pt}
    \setlength{\parskip}{6pt plus 2pt minus 1pt}}
}{% if KOMA class
  \KOMAoptions{parskip=half}}
\makeatother
\usepackage{xcolor}
\usepackage[left=1cm, right=1cm, top=1cm, bottom=1.5cm]{geometry}
\usepackage{graphicx}
\makeatletter
\def\maxwidth{\ifdim\Gin@nat@width>\linewidth\linewidth\else\Gin@nat@width\fi}
\def\maxheight{\ifdim\Gin@nat@height>\textheight\textheight\else\Gin@nat@height\fi}
\makeatother
% Scale images if necessary, so that they will not overflow the page
% margins by default, and it is still possible to overwrite the defaults
% using explicit options in \includegraphics[width, height, ...]{}
\setkeys{Gin}{width=\maxwidth,height=\maxheight,keepaspectratio}
% Set default figure placement to htbp
\makeatletter
\def\fps@figure{htbp}
\makeatother
\setlength{\emergencystretch}{3em} % prevent overfull lines
\providecommand{\tightlist}{%
  \setlength{\itemsep}{0pt}\setlength{\parskip}{0pt}}
\setcounter{secnumdepth}{-\maxdimen} % remove section numbering
\usepackage{bbm}
\usepackage{amsmath}
\ifLuaTeX
  \usepackage{selnolig}  % disable illegal ligatures
\fi
\IfFileExists{bookmark.sty}{\usepackage{bookmark}}{\usepackage{hyperref}}
\IfFileExists{xurl.sty}{\usepackage{xurl}}{} % add URL line breaks if available
\urlstyle{same} % disable monospaced font for URLs
\hypersetup{
  pdftitle={Annex - Detailed derivation of all equations of paper ``Monetary Policy and Exchange Rate Volatility in a Small Open Economy''},
  pdfauthor={Anna Catarina Tavella e Matheus Franciscão},
  hidelinks,
  pdfcreator={LaTeX via pandoc}}

\title{Annex - Detailed derivation of all equations of paper ``Monetary
Policy and Exchange Rate Volatility in a Small Open Economy''}
\author{Anna Catarina Tavella e Matheus Franciscão}
\date{November/2022}

\begin{document}
\maketitle

\hypertarget{problem-of-the-consumer}{%
\section{1 - Problem of the consumer}\label{problem-of-the-consumer}}

\vspace{12pt}

\hypertarget{equation-1---the-representative-household}{%
\subsection{Equation (1) - The representative
household}\label{equation-1---the-representative-household}}

\(\displaystyle \max_{C_t,N_t} \ \ E_0 \sum_{t=0}^\infty \beta^tU(C_t,N_t)\)

\vspace{8pt}

This is the problem of the representative consumer in each economy,
where \(C_t\) represents the consumption basket and \(N_t\) denotes the
hours of labour.

\vspace{12pt}

\hypertarget{equation-2---consumption-index}{%
\subsection{Equation (2) - Consumption
index}\label{equation-2---consumption-index}}

\(C_t=\left[ (1-\alpha)^{\frac{1}{\eta}} (C_{H,t})^{\frac{\eta-1}{\eta}} + \alpha^{\frac{1}{\eta}} (C_{F,t})^{\frac{\eta-1}{\eta}} \right]^{\frac{\eta}{\eta-1}}\)

\vspace{8pt}

This is the composite consumption index, which the representative
consumer maximizes. Substituting in equation (1), the representative
consumer problem becomes

\(\displaystyle \max_{C_t,N_t} \ E_0 \sum_{t=0}^\infty \beta^tU(C_t,N_t) = \max_{C_{H,t},C_{F,t},N_t} \ E_0 \sum_{t=0}^\infty \beta^tU \left(\left[ (1-\alpha)^{\frac{1}{\eta}} (C_{H,t})^{\frac{\eta-1}{\eta}} + \alpha^{\frac{1}{\eta}} (C_{F,t})^{\frac{\eta-1}{\eta}} \right]^{\frac{\eta}{\eta-1}},N_t \right)\).

subject to the budget constraint specified in equation (3) below, where

\(C_{H,t} \equiv \displaystyle \left( \int_0^1 C_{H,t}(j)^{\frac{\varepsilon-1}{\varepsilon}}dj \right) ^{\frac{\varepsilon}{\varepsilon-1}}\),
\(C_{F,t} \equiv \displaystyle \left( \int_0^1 (C_{i,t})^{\frac{\gamma-1}{\gamma}}di \right) ^{\frac{\gamma}{\gamma-1}}\),
\(C_{i,t} \equiv \displaystyle \left( \int_0^1 C_{i,t}(j)^{\frac{\varepsilon-1}{\varepsilon}}dj \right) ^{\frac{\varepsilon}{\varepsilon-1}}\)

Substituting the expressions above and using equations (1) and (2) the
representative consumer problem can be written as:

\(\displaystyle \max_{C_{H,t}(j),C_{i,t},N_t} \ E_0 \sum_{t=0}^\infty \beta^tU \left(\left[ (1-\alpha)^{\frac{1}{\eta}} \left [\displaystyle \left( \int_0^1 C_{H,t}(j)^{\frac{\varepsilon-1}{\varepsilon}}dj \right) ^{\frac{\varepsilon}{\varepsilon-1}} \right]^{\frac{\eta-1}{\eta}} + \alpha^{\frac{1}{\eta}} \left[ \displaystyle \left( \int_0^1 (C_{i,t})^{\frac{\gamma-1}{\gamma}}di \right) ^{\frac{\gamma}{\gamma-1}} \right]^{\frac{\eta-1}{\eta}} \right]^{\frac{\eta}{\eta-1}},N_t \right)\)
=

\(\displaystyle \max_{C_{H,t}(j),C_{i,t}(j),N_t}\)

\(\ E_0 \sum_{t=0}^\infty \beta^tU \left(\left[ (1-\alpha)^{\frac{1}{\eta}} \left [\displaystyle \left( \int_0^1 C_{H,t}(j)^{\frac{\varepsilon-1}{\varepsilon}}dj \right) ^{\frac{\varepsilon}{\varepsilon-1}} \right]^{\frac{\eta-1}{\eta}} + \alpha^{\frac{1}{\eta}} \left[ \displaystyle \left( \int_0^1 \left( \displaystyle \left( \int_0^1 C_{i,t}(j)^{\frac{\varepsilon-1}{\varepsilon}}dj \right) ^{\frac{\varepsilon}{\varepsilon-1}} \right)^{\frac{\gamma-1}{\gamma}}di \right) ^{\frac{\gamma}{\gamma-1}} \right]^{\frac{\eta-1}{\eta}} \right]^{\frac{\eta}{\eta-1}},N_t \right)\)

\vspace{12pt}

\hypertarget{equation-3---consumer-budget-constraint}{%
\subsection{Equation (3) - Consumer budget
constraint}\label{equation-3---consumer-budget-constraint}}

\vspace{8pt}

\(\displaystyle \int_0^1 P_{H,t}(j)C_{H,t}(j)dj + \int_0^1\int_0^1 P_{i,t}(j)C_{i,t}(j)dj\ di + \mathbb{E}_t[ Q_{t,t+1}D_{t+1}] \leq D_t + W_tN_t + T_t\)

The representative households maximizes consumption subject to the
budget constrain above. The first integral represents the budget share
spent in domestic items. The double integral is the share spent in
imported goods: the sum of all items from all countries bought, in
domestic currency. The third term from the RHS is the budget share
allocated to assets due in t+1. The RHS incorporates the sum of all
payments received from assets summarized in \(D_t\) (including shares of
firms and other assets traded in an international market), the labour
income and transfers from/to the government.

\vspace{12pt}

\hypertarget{equations-4---demand-functions-for-each-specific-good}{%
\subsection{Equations (4) - Demand functions for each specific
good}\label{equations-4---demand-functions-for-each-specific-good}}

\(\displaystyle C_{H,t}(j)= \left( \frac{P_{H,t}(j)}{P_{H,t}}\right)^{-\varepsilon}C_{H,t}\)
and
\(\displaystyle C_{i,t}(j)= \left( \frac{P_{i,t}(j)}{P_{i,t}}\right)^{-\varepsilon}C_{i,t}\)

\vspace{8pt}

To get the demand functions first we write the Lagrangean:

\(\mathcal{L} = \displaystyle \mathbb{E}_t \sum_{t=0}^\infty \beta^t U \left(\left[ (1-\alpha)^{\frac{1}{\eta}} \left [\displaystyle \left( \int_0^1 C_{H,t}(j)^{\frac{\varepsilon-1}{\varepsilon}}dj \right) ^{\frac{\varepsilon}{\varepsilon-1}} \right]^{\frac{\eta-1}{\eta}} + \alpha^{\frac{1}{\eta}} \left[ \displaystyle \left( \int_0^1 \left( \displaystyle \left( \int_0^1 C_{i,t}(j)^{\frac{\varepsilon-1}{\varepsilon}}dj \right) ^{\frac{\varepsilon}{\varepsilon-1}} \right)^{\frac{\gamma-1}{\gamma}}di \right) ^{\frac{\gamma}{\gamma-1}} \right]^{\frac{\eta-1}{\eta}} \right]^{\frac{\eta}{\eta-1}},N_t \right)\)

\(+ \lambda_t \left(D_t + W_tN_t + Tt \displaystyle - \int_0^1 P_{H,t}(j)C_{H,t}(j)dj - \int_0^1\int_0^1 P_{i,t}(j)C_{i,t}(j)dj\ di - \mathbb{E}_t\{ Q_{t,t+1}D_{t+1}\}\right)\)

and calculate the MRS (marginal rate of substitution) between
\(C_{H,t}(j)\) and \(C_{H,t}\), as by the optimal allocation, it has to
be the rate of prices in every period of time (otherwise the consumer
could by a little less of the product with relative higher price and buy
another with relative lower price, increasing his utility).

\(\displaystyle \frac{\partial U(C_t,N_t)}{\partial C_{H,t}(j)} = U_c(C_t,N_t)\frac{\eta}{1-\eta}\left( C_t^{\frac{\eta-1}{\eta}} \right)^{\frac{\eta}{\eta-1}-1} (1-\alpha)^{\frac{1}{\eta}}\frac{\eta-1}{\eta}\left( C_{H,t}\right)^{-\frac{1}{\eta}}\frac{\varepsilon}{\varepsilon-1}\left(C_{H,t}^{\frac{\varepsilon-1}{\varepsilon}} \right)^{\frac{\varepsilon}{\varepsilon-1}-1}\int_0^1 \frac{\varepsilon-1}{\varepsilon} C_{H,t}(j)^{-\frac{1}{\varepsilon}}dj\)

After simplifying, we get

\(\displaystyle \frac{\partial U(C_t,N_t)}{\partial C_{H,t}(j)} = U_c(C_t,N_t) (1-\alpha)^{\frac{1}{\eta}} C_t^{\frac{1}{\eta}}\left( C_{H,t}\right)^{-\frac{1}{\eta}}C_{H,t}^{\frac{1}{\varepsilon}} \int_0^1 C_{H,t}(j)^{-\frac{1}{\varepsilon}}dj = U_c(C_t,N_t) \left[ (1-\alpha) \frac{C_t}{C_{H,t}} \right]^{\frac{1}{\eta}} \int_0^1 \left[\frac{C_{H,t}}{C_{H,t}(j)}\right]^{\frac{1}{\varepsilon}}dj\)

Similarly,

\(\displaystyle \frac{\partial U(C_t,N_t)}{\partial C_{H,t}} = U_c(C_t,N_t)\frac{\eta}{1-\eta}\left( C_t^{\frac{\eta-1}{\eta}} \right)^{\frac{\eta}{\eta-1}-1} (1-\alpha)^{\frac{1}{\eta}}\frac{\eta-1}{\eta}\left( C_{H,t}\right)^{-\frac{1}{\eta}}\)

\(\displaystyle \frac{\partial U(C_t,N_t)}{\partial C_{H,t}} = U_c(C_t,N_t) (1-\alpha)^{\frac{1}{\eta}} C_t^{\frac{1}{\eta}}\left( C_{H,t}\right)^{-\frac{1}{\eta}} = U_c(C_t,N_t) \left[ (1-\alpha) \frac{C_t}{C_{H,t}} \right]^{\frac{1}{\eta}}\)

\(\displaystyle \frac{\displaystyle \frac{\partial U(C_t,N_t)}{\displaystyle \partial C_{H,t}(j)}}{\frac{\displaystyle \partial U(C_t,N_t)}{\displaystyle \partial C_{H,t}}} = \frac{\displaystyle U_c(C_t,N_t) \left[ (1-\alpha) \frac{C_t}{C_{H,t}} \right]^{\frac{1}{\eta}} \int_0^1 \left[\frac{C_{H,t}}{C_{H,t}(j)}\right]^{\frac{1}{\varepsilon}}dj}{\displaystyle U_c(C_t,N_t) \left[ (1-\alpha) \frac{C_t}{C_{H,t}} \right]^{\frac{1}{\eta}}} = \frac{\displaystyle \int_0^1P_{H,t}(j)dj}{P_{H,t}}\)

After simplifying again, the expression is almost the demand function we
want.

\(\displaystyle \int_0^1 \left[\frac{C_{H,t}}{C_{H,t}(j)}\right]^{\frac{1}{\varepsilon}}dj = \displaystyle \int_0^1 \frac{P_{H,t}(j)}{P_{H,t}}dj\)

As the interval of both integrals are the same and the variable being
integrated is also the same, what is inside the integral in both sides
have also to be the same. So,

\(\displaystyle \left[\frac{C_{H,t}}{C_{H,t}(j)}\right]^{\frac{1}{\varepsilon}} = \frac{P_{H,t}(j)}{P_{H,t}} \Rightarrow \left[\frac{C_{H,t}(j)}{C_{H,t}}\right]^{-\frac{1}{\varepsilon}} = \frac{P_{H,t}(j)}{P_{H,t}} \Rightarrow C_{H,t}(j)= \left[ \frac{P_{H,t}(j)}{P_{H,t}} \right]^{-\varepsilon}C_{H,t}\)

Calculating now the MRS (marginal rate of substitution) between
\(C_{i,t}(j)\) and \(C_{i,t}\), which is also equal the rate of the
prices.

\(\displaystyle \frac{\partial U(C_t,N_t)}{\partial C_{i,t}(j)} =\)

\(\displaystyle U_c(C_t,N_t)\frac{\eta}{1-\eta}\left( C_t^{\frac{\eta-1}{\eta}} \right)^{\frac{\eta}{\eta-1}-1} \alpha^{\frac{1}{\eta}}\frac{\eta-1}{\eta}\left( C_{F,t}\right)^{-\frac{1}{\eta}}\frac{\gamma}{\gamma-1}\left(C_{H,t}^{\frac{\gamma-1}{\gamma}} \right)^{\frac{\gamma}{\gamma-1}-1}\int_0^1 \frac{\gamma-1}{\gamma} C_{i,t}^{-\frac{1}{\gamma}}\left[ \frac{\varepsilon}{\varepsilon-1}\left(C_{i,t}^{\frac{\varepsilon-1}{\varepsilon}} \right)^{\frac{\varepsilon}{\varepsilon-1}-1}\int_0^1 \frac{\varepsilon-1}{\varepsilon} C_{i,t}(j)^{-\frac{1}{\varepsilon}}dj \right]di\)

After simplifying, we get

\(\displaystyle \frac{\partial U(C_t,N_t)}{\partial C_{i,t}(j)} = U_c(C_t,N_t) \alpha^{\frac{1}{\eta}} C_t^{\frac{1}{\eta}}\left( C_{F,t}\right)^{-\frac{1}{\eta}}C_{F,t}^{\frac{1}{\gamma}} \int_0^1 C_{i,t}^{-\frac{1}{\gamma}} \left[ C_{i,t}^\frac{1}{\varepsilon} \int_0^1 C_{i,t}(j)^{-\frac{1}{\varepsilon}}dj \right] di\)

\(\displaystyle \frac{\partial U(C_t,N_t)}{\partial C_{i,t}(j)} = U_c(C_t,N_t) \left[ \alpha \frac{C_t}{C_{F,t}} \right]^{\frac{1}{\eta}} \int_0^1 \left[\frac{C_{F,t}}{C_{i,t}}\right]^{\frac{1}{\gamma}} \int_0^1 \left[\frac{C_{i,t}}{C_{i,t}(j)}\right]^{\frac{1}{\varepsilon}}dj \ di\)

\(\displaystyle \frac{\partial U(C_t,N_t)}{\partial C_{i,t}} = U_c(C_t,N_t)\frac{\eta}{1-\eta}\left( C_t^{\frac{\eta-1}{\eta}} \right)^{\frac{\eta}{\eta-1}-1} \alpha^{\frac{1}{\eta}}\frac{\eta-1}{\eta}\left( C_{F,t}\right)^{-\frac{1}{\eta}}\frac{\gamma}{\gamma-1}\left(C_{H,t}^{\frac{\gamma-1}{\gamma}} \right)^{\frac{\gamma}{\gamma-1}-1}\int_0^1 \frac{\gamma-1}{\gamma} C_{i,t}^{-\frac{1}{\gamma}}di\)

\(\displaystyle \frac{\partial U(C_t,N_t)}{\partial C_{i,t}} = U_c(C_t,N_t) \left[ \alpha \frac{C_t}{C_{F,t}} \right]^{\frac{1}{\eta}} \int_0^1 \left[\frac{C_{F,t}}{C_{i,t}}\right]^{\frac{1}{\gamma}} di\)

Calculating the MRS we have

\(\displaystyle \frac{\displaystyle \frac{\partial U(C_t,N_t)}{\displaystyle \partial C_{i,t}(j)}}{\frac{\displaystyle \partial U(C_t,N_t)}{\displaystyle \partial C_{i,t}}} = \frac{\displaystyle U_c(C_t,N_t) \left[ \alpha \frac{C_t}{C_{F,t}} \right]^{\frac{1}{\eta}} \int_0^1 \left[\frac{C_{F,t}}{C_{i,t}}\right]^{\frac{1}{\gamma}} \int_0^1 \left[\frac{C_{i,t}}{C_{i,t}(j)}\right]^{\frac{1}{\varepsilon}}dj \ di }{\displaystyle U_c(C_t,N_t) \left[ \alpha \frac{C_t}{C_{F,t}} \right]^{\frac{1}{\eta}} \int_0^1 \left[\frac{C_{F,t}}{C_{i,t}}\right]^{\frac{1}{\gamma}} di} = \frac{\displaystyle \int_0^1P_{i,t}(j)dj}{P_{i,t}}\)

As before, we can simplify again. Also, as there's a continuum of firms,
we can consider that the price of each product \(P_{i,t}(j)\) is only
correlated with its specific demand function and not with the demand
function of other in its country or another country, it follows that
each specific price is uncorrelated with \(C_{F,t}\) and \(C_{i,t}\).
Also, as each firm is very small, we can consider that it has negligible
influence on the aggregate index price of its country (\(P_{i,t}\)).
With these independence assumption, the joint distribution is equal to
the product of the marginal distributions.

\(\displaystyle \int_0^1 \int_0^1 \left[\frac{C_{F,t}}{C_{i,t}}\right]^{\frac{1}{\gamma}} \left[\frac{C_{i,t}}{C_{i,t}(j)}\right]^{\frac{1}{\varepsilon}}dj \ di = \int_0^1 \left[\frac{C_{F,t}}{C_{i,t}}\right]^{\frac{1}{\gamma}} di \int_0^1 \frac{\displaystyle P_{i,t}(j)}{P_{i,t}}dj = \int_0^1 \int_0^1 \frac{\displaystyle P_{i,t}(j)}{P_{i,t}} \left[\frac{C_{F,t}}{C_{i,t}}\right]^{\frac{1}{\gamma}} dj \ di\)

Now, as before, the integrand in both sides needs to be the same. The we
get the second demand equation.

\(\displaystyle \left[\frac{C_{F,t}}{C_{i,t}}\right]^{\frac{1}{\gamma}} \left[\frac{C_{i,t}}{C_{i,t}(j)}\right]^{\frac{1}{\varepsilon}} = \frac{\displaystyle P_{i,t}(j)}{P_{i,t}} \left[\frac{C_{F,t}}{C_{i,t}}\right]^{\frac{1}{\gamma}} \Rightarrow \left[\frac{C_{i,t}(j)}{C_{i,t}}\right]^{-\frac{1}{\varepsilon}} = \frac{\displaystyle P_{i,t}(j)}{P_{i,t}} \Rightarrow C_{i,t}(j) = \left[ \frac{\displaystyle P_{i,t}(j)}{P_{i,t}} \right]^{-\varepsilon}C_{i,t}\)

\vspace{12pt}

\hypertarget{equation-5---demand-for-foreign-goods-from-each-country}{%
\subsection{Equation (5) - Demand for foreign goods from each
country}\label{equation-5---demand-for-foreign-goods-from-each-country}}

\(\displaystyle C_{i,t}= \left( \frac{P_{i,t}}{P_{F,t}}\right)^{-\gamma}C_{F,t}\)

\vspace{8pt}

To find the aggregate demand for each country, in terms of total foreign
demand, we proceed by calculating the MRS between the aggregate
consumption for the country and the aggregate consumption of foreign
goods, which the optimal allocation resulting from the rate between the
prices, as before.

\(\displaystyle \frac{\partial U(C_t,N_t)}{\partial C_{F,t}} = U_c(C_t,N_t)\frac{\eta}{1-\eta}\left( C_t^{\frac{\eta-1}{\eta}} \right)^{\frac{\eta}{\eta-1}-1} \alpha^{\frac{1}{\eta}}\frac{\eta-1}{\eta}\left( C_{F,t}\right)^{-\frac{1}{\eta}} = U_c(C_t,N_t) \left[ \alpha \frac{C_t}{C_{F,t}} \right]^{\frac{1}{\eta}}\)

\(\displaystyle \frac{\displaystyle \frac{\partial U(C_t,N_t)}{\displaystyle \partial C_{i,t}}}{\frac{\displaystyle \partial U(C_t,N_t)}{\displaystyle \partial C_{F,t}}} = \frac{\displaystyle U_c(C_t,N_t) \left[ \alpha \frac{C_t}{C_{F,t}} \right]^{\frac{1}{\eta}} \int_0^1 \left[\frac{C_{F,t}}{C_{i,t}}\right]^{\frac{1}{\gamma}} di }{\displaystyle U_c(C_t,N_t) \left[ \alpha \frac{C_t}{C_{F,t}} \right]^{\frac{1}{\eta}} } = \frac{\displaystyle \int_0^1P_{i,t}di}{P_{F,t}}\)

As \(P_{F,t}\) doesn't depend on a specific i, we can put it inside the
integral. Then we get again two integrans which have to be the same for
the equality to hold.

\(\displaystyle \left[\frac{C_{F,t}}{C_{i,t}}\right]^{\frac{1}{\gamma}} = \frac{P_{i,t}}{P_{F,t}} \Rightarrow \left[\frac{C_{i,t}}{C_{F,t}}\right]^{-\frac{1}{\gamma}} = \frac{P_{i,t}}{P_{F,t}} \ \ \Rightarrow \ \ C_{i,t} = \left[ \frac{P_{i,t}}{P_{F,t}} \right]^{-\gamma}C_{F,t}\)

\vspace{12pt}

\hypertarget{equations-6---optimal-share-between-the-domestic-and-imported-goods}{%
\subsection{Equations (6) - Optimal share between the domestic and
imported
goods}\label{equations-6---optimal-share-between-the-domestic-and-imported-goods}}

\(\displaystyle C_{H,t} = (1-\alpha)\left( \frac{P_{H,t}}{P_t} \right)^{-\eta}C_t\)
and
\(\displaystyle C_{F,t} = \alpha \left( \frac{P_{F,t}}{P_t} \right)^{-\eta}C_t\)

\vspace{8pt}

Now we will calculate the MRS between the domestic products and the
total consumption, which has to be equal to the rate of prices. After,
we will do the same for the foreign products.

\(\displaystyle \frac{\displaystyle \frac{\partial U(C_t,N_t)}{\displaystyle \partial C_{H,t}}}{\frac{\displaystyle \partial U(C_t,N_t)}{\displaystyle \partial C_t}} = \frac{\displaystyle U_c(C_t,N_t) \left[ (1-\alpha) \frac{C_t}{C_{H,t}} \right]^{\frac{1}{\eta}}}{\displaystyle U_c(C_t,N_t) } = \frac{P_{H,t}}{P_t} \ \ \Rightarrow \ \ (1-\alpha) \frac{C_t}{C_{H,t}} = \left( \frac{P_{H,t}}{P_t} \right)^\eta \ \ \Rightarrow \ \ C_{H,t} = (1-\alpha)\left( \frac{P_{H,t}}{P_t} \right)^{-\eta}C_t\)

\(\displaystyle \frac{\displaystyle \frac{\partial U(C_t,N_t)}{\displaystyle \partial C_{F,t}}}{\frac{\displaystyle \partial U(C_t,N_t)}{\displaystyle \partial C_t}} = \frac{\displaystyle U_c(C_t,N_t) \left[ \alpha \frac{C_t}{C_{F,t}} \right]^{\frac{1}{\eta}}}{\displaystyle U_c(C_t,N_t) } = \frac{P_{F,t}}{P_t} \ \ \Rightarrow \ \ \alpha \frac{C_t}{C_{F,t}} = \left( \frac{P_{F,t}}{P_t} \right)^\eta \ \ \Rightarrow \ \ C_{F,t} = \alpha \left( \frac{P_{F,t}}{P_t} \right)^{-\eta}C_t\)

\vspace{12pt}

\hypertarget{equation-7---aggregated-expenditure}{%
\subsection{Equation (7) - Aggregated
expenditure}\label{equation-7---aggregated-expenditure}}

\(\displaystyle P_t C_t + \mathbb{E}_t[ Q_{t,t+1}D_{t+1}] \leq D_t + W_tN_t + T_t\)

\vspace{8pt}

Now that we have the demand functions

\(\displaystyle C_{H,t}(j)= \left( \frac{P_{H,t}(j)}{P_{H,t}}\right)^{-\varepsilon}C_{H,t}\);\\
\(\displaystyle C_{i,t}(j)= \left( \frac{P_{i,t}(j)}{P_{i,t}}\right)^{-\varepsilon}C_{i,t}\);
and
\(\displaystyle C_{i,t}= \left( \frac{P_{i,t}}{P_{F,t}}\right)^{-\gamma}C_{F,t}\)

let's prove that

\(\displaystyle \int_0^1 P_{H,t}(j)C_{H,t}(j)dj = P_{H,t}C_{H,t}\) ,
\(\displaystyle \int_0^1 P_{i,t}(j)C_{i,t}(j)dj = P_{i,t}C_{i,t}\) and
\(\displaystyle \int_0^1 P_{i,t}C_{i,t}dj = P_{F,t}C_{F,t}\)

using the definition of the price indexes:

\(\displaystyle P_{H,t} \equiv \left( \int_0^1 P_{H,t}(j)^{1-\varepsilon}dj \right)^{\frac{1}{1-\varepsilon}}\),
\(\displaystyle P_{i,t} \equiv \left( \int_0^1 P_{i,t}(j)^{1-\varepsilon}dj \right)^{\frac{1}{1-\varepsilon}}\)
and
\(\displaystyle P_{F,t} \equiv \left( \int_0^1 P_{i,t}^{1-\gamma}di \right)^{\frac{1}{1-\gamma}}\)

\(\displaystyle \int_0^1 P_{H,t}(j)C_{H,t}(j)dj = \int_0^1P_{H,t}(j)\left( \frac{P_{H,t}(j)}{P_{H,t}}\right)^{-\varepsilon}C_{H,t}dj = \frac{C_{H,t}}{P_{H,t}^{-\varepsilon}}\int_0^1P_{H,t}(j)^{1-\varepsilon}dj = \frac{C_{H,t}}{P_{H,t}^{-\varepsilon}}P_{H,t}^{1-\varepsilon} = P_{H,t}C_{H,t}\)

\(\displaystyle \int_0^1 P_{i,t}(j)C_{i,t}(j)dj = \int_0^1P_{i,t}(j)\left( \frac{P_{i,t}(j)}{P_{i,t}}\right)^{-\varepsilon}C_{i,t}dj = \frac{C_{i,t}}{P_{i,t}^{-\varepsilon}}\int_0^1 P_{i,t}(j)^{1-\varepsilon}dj = \frac{C_{i,t}}{P_{i,t}^{-\varepsilon}}P_{i,t}^{1-\varepsilon} = P_{i,t}C_{i,t}\)

\(\displaystyle \int_0^1 P_{i,t}C_{i,t}di = \int_0^1P_{i,t}\left( \frac{P_{i,t}}{P_{F,t}}\right)^{-\gamma}C_{F,t}di = \frac{C_{F,t}}{P_{F,t}^{-\gamma}}\int_0^1 P_{F,t}^{1-\gamma}dj = \frac{C_{F,t}}{P_{F,t}^{-\varepsilon}}P_{F,t}^{1-\varepsilon} = P_{F,t}C_{F,t}\)

With this aggregation, the budget constraint can be simplified

\(\displaystyle \int_0^1 P_{H,t}(j)C_{H,t}(j)dj + \int_0^1\int_0^1 P_{i,t}(j)C_{i,t}(j)dj\ di + \mathbb{E}_t\{ Q_{t,t+1}D_{t+1}\} \leq D_t + W_tN_t + T_t\)

\(\displaystyle P_{H,t}C_{H,t} + \int_0^1P_{i,t}C_{i,t}di = P_{H,t}C_{H,t} + P_{F,t}C_{F,t} \leq D_t + W_tN_t + Tt - \mathbb{E}_t\{ Q_{t,t+1}D_{t+1}\}\)

As the total consumption expenditure by the representative consumer is
with the domestic produts or foreig products, the budget constraint
becomes:

\(\displaystyle P_t C_t \leq D_t + W_tN_t + T_t - \mathbb{E}_t\{ Q_{t,t+1}D_{t+1}\}\)

\vspace{12pt}

\hypertarget{equation-8---intratemporal-substitution-between-consumption-and-labor}{%
\subsection{Equation (8) - Intratemporal substitution between
consumption and
labor}\label{equation-8---intratemporal-substitution-between-consumption-and-labor}}

\(\displaystyle C_t^{\sigma}N_t^{\varphi} = \frac{W_t}{P_t}\)

\vspace{8pt}

Now we arrived at a standard problem for the representative consumer.
Considering the functional form for the utility function as

\(\displaystyle U(C_t,N_t)=\frac{C_t^{1-\sigma}}{1-\sigma}-\frac{N_t^{1+\varphi}}{1+\varphi}\),

\(\displaystyle \max_{C_t,N_t}\ E_0 \sum_{t=0}^\infty \beta^t U(C_t,N_t) = \displaystyle \max_{C_t,N_t} \ E_0 \sum_{t=0}^\infty \beta^t \left( \frac{C_t^{1-\sigma}}{1-\sigma}-\frac{N_t^{1+\varphi}}{1+\varphi} \right)\)

subject to
\(\displaystyle D_t + W_tN_t + Tt - \mathbb{E}_t[ Q_{t,t+1}D_{t+1}] - P_t C_t = 0\),
as an optimal condition (the constraint is binding, otherwise the
household could consume a little more with the same budget and would not
be optimizing).

Now we can write the Lagrangean

\(\mathcal{L} = \displaystyle \mathbb{E}_t \sum_{t=0}^\infty \beta^t \left[ \left( \frac{C_t^{1-\sigma}}{1-\sigma}-\frac{N_t^{1+\varphi}}{1+\varphi} \right) + \lambda_t \left( D_t + W_tN_t + Tt - Q_{t,t+1}D_{t+1} - P_t C_t \right) \right]\)

with first order conditions (FOCs):

(\(C_t\)):
\(\displaystyle \beta^t U_C(C_t,N_t)= \beta^tC_t^{-\sigma} = \beta^t \lambda_t P_t \ \ \Rightarrow \ \ C_t^{-\sigma} = \lambda_t P_t\)

(\(N_t\)):
\(\displaystyle -\beta^t U_n(C_t,N_t)=\beta^tN_t^{\varphi} = \beta^t \lambda_t W_t \ \ \Rightarrow \ \ N_t^{\varphi} = \lambda_t W_t\)

(\(D_{t+1}\)):
\(\displaystyle \beta^t \lambda_t Q_{t,t+1} = \beta^{t+1} \lambda_{t+1} \ \ \Rightarrow \ \ \frac{\lambda_{t+1}}{\lambda_t} = \frac{Q_{t,t+1}}{\beta}\)

Dividing (\(N_t\)) FOC by (\(C_t\)) FOC, we have the standard equation
of intratemporal substitution between consumption and leisure
\(\displaystyle -\frac{U_C(C_t,N_t)}{U_N(C_t,N_t)} =C_t^{\sigma}N_t^{\varphi} = \frac{W_t}{P_t}\)

\vspace{12pt}

\hypertarget{equation-9---euler-equation}{%
\subsection{Equation (9) - Euler
equation}\label{equation-9---euler-equation}}

\(\displaystyle \beta \left( \frac{C_{t+1}}{C_t} \right)^{-\sigma} \frac{P_t}{P_{t+1}}= Q_{t,t+1}\)

\vspace{8pt}

Using the FOCs calculated before, we can advance one period for the
consumption FOC, we have
\(\mathbb{E}_t [C_{t+1}^{-\sigma}] = \mathbb{E}_t[\lambda_{t+1}P_{t+1}]\)

Dividing the consumption FOC in t+1 by the equation in t and
substituting by the \(\mathbb{E}_t[\lambda_{t+1}]/\lambda_t\) in the
\(D_{t+1}\) FOC, we get the Euler equation

\(\displaystyle \left( \frac{C_{t+1}}{C_t} \right)^{-\sigma} = \frac{\lambda_{t+1} P_{t+1}}{\lambda_t P_t} \ \ \Rightarrow \ \  \left( \frac{C_{t+1}}{C_t} \right)^{-\sigma} \frac{P_t}{P_{t+1}}= \frac{Q_{t,t+1}}{\beta} \ \ \Rightarrow \ \  \beta \left( \frac{C_{t+1}}{C_t} \right)^{-\sigma} \frac{P_t}{P_{t+1}}= Q_{t,t+1}\)

\vspace{12pt}

\hypertarget{equation-10---euler-equation-with-gross-returns}{%
\subsection{Equation (10) - Euler equation with gross
returns}\label{equation-10---euler-equation-with-gross-returns}}

\(\displaystyle \beta R_t \mathbb{E}_t \left[ \left( \frac{C_{t+1}}{C_t} \right)^{-\sigma} \frac{P_t}{P_{t+1}}\right]= 1\)

\vspace{8pt}

As \(Q_{t,t+1}\), is the price of a a riskless one-period discount bond
in domestic currency with gross return \(R_t\), we can substitute
\(\displaystyle R_t=\frac{1}{\mathbb{E}_t[Q_{t,t+1}]}\), we can take
expectations in both sides to get the equation (10)

\(\displaystyle \mathbb{E}_t \left[ \beta \left( \frac{C_{t+1}}{C_t} \right)^{-\sigma} \frac{P_t}{P_{t+1}}\right]= \mathbb{E}_t[Q_{t,t+1}] \ \ \Rightarrow \ \  \beta \mathbb{E}_t \left[ \left( \frac{C_{t+1}}{C_t} \right)^{-\sigma} \frac{P_t}{P_{t+1}}\right]= \frac{1}{R_t} \ \ \Rightarrow \ \  \beta R_t \mathbb{E}_t \left[ \left( \frac{C_{t+1}}{C_t} \right)^{-\sigma} \frac{P_t}{P_{t+1}}\right]= 1\)

\vspace{12pt}

\hypertarget{equation-11---log-linearized-euler-equation}{%
\subsection{Equation (11) - Log-linearized Euler
equation}\label{equation-11---log-linearized-euler-equation}}

\(\displaystyle c_t = \mathbb{E}_t \left[c_{t+1} \right] -\frac{1}{\sigma} \left( r_t - \mathbb{E}_t \left[ \pi_{t+1} \right] -\rho\right)\)

\vspace{8pt}

Log-linearizing
\(\displaystyle C_t^{\sigma}N_t^{\varphi} = \frac{W_t}{P_t}\) is
straight forward: \(w_t - p_t = \sigma c_t + \varphi n_t\)

To log-linerize the Euler equation, we'll use the Taylor expansion:
\(f(x) \approx f(x_0) + f'(x_o)(x-x_0)\). When expanding the exponential
function around 0, we get \(e^x=e^0+e^0(x-0)=1+x\)

\(\mathbb{E}_t \left[ \exp \left( \ln \left[ \beta R_t \left( \frac{C_{t+1}}{C_t} \right)^{-\sigma} \frac{P_t}{P_{t+1}} \right] \right) \right] = \mathbb{E}_t \left[ \exp \left( \ln(\beta) + r_t -\sigma(c_{t+1}-c_t) + p_t - p_{t+1} \right) \right] = 1 \ \ \Rightarrow\)

\(\displaystyle 1 + \ln(\beta) + r_t -\sigma(\mathbb{E}_t \left[c_{t+1} \right]-c_t) - \mathbb{E}_t \left[ \pi_{t+1} \right] = 1 \ \ \Rightarrow \ \ \sigma c_t = \sigma \mathbb{E}_t \left[c_{t+1} \right] + \mathbb{E}_t \left[ \pi_{t+1} \right] - r_t - \ln(\beta)\)

Rearranging, we get

\(\displaystyle \Rightarrow \ \ c_t = \mathbb{E}_t \left[c_{t+1} \right] -\frac{1}{\sigma} \left( r_t - \mathbb{E}_t \left[ \pi_{t+1} \right] -\rho\right)\)
as
\(\displaystyle \rho \equiv \frac{1-\beta}{\beta} \approx -\ln(\beta)\)

\vspace{12pt}

\hypertarget{external-sector}{%
\section{2 - External Sector}\label{external-sector}}

\hypertarget{equation-12---log-linear-effective-terms-of-trade}{%
\subsection{Equation (12) - Log-linear effective terms of
trade}\label{equation-12---log-linear-effective-terms-of-trade}}

\(\displaystyle s_t = \int_0^1 s_{i,t}di\)

\vspace{8pt}

Let's log-linearize the expression for the bilateral terms of trade
\(\displaystyle S_t \equiv \frac{P_{F,t}}{P_{H,t}}= \left( \int_0^1 S_{i,t}^{1-\gamma}di \right)^{\frac{1}{1-\gamma}} \ \ \Rightarrow \ \ S_t^{1-\gamma} = \int_0^1 S_{i,t}^{1-\gamma}di\)

\(\displaystyle \exp \left(\ln \left[ S_t^{1-\gamma}\right] \right) = \int_0^1 \exp \left(\ln \left[ S_{i,t}^{1-\gamma}\right] \right)di \ \ \Rightarrow \ \ \exp[(1-\gamma)s_t]=\int_0^1 \exp[(1-\gamma)s_{i,t}]di\)

Applying the exponential Taylor expansion (\(e^x = 1 + x\)) in both
sides, we get
\(\displaystyle 1+(1+\gamma)s_t = 1 + (1+\gamma) \int_0^1 s_{i,t}di \ \ \Rightarrow \ \ s_t = \int_0^1 s_{i,t}di\)

\vspace{12pt}

\hypertarget{equation-13---log-linear-cpi-around-a-symmetric-steady-state-satisfying-ppp}{%
\subsection{Equation (13) - Log-linear CPI around a symmetric steady
state satisfying
PPP}\label{equation-13---log-linear-cpi-around-a-symmetric-steady-state-satisfying-ppp}}

\(p_t = p_{H,t} + \alpha s_t\)

\vspace{8pt}

To log-linearize the CPI formula, considering that it is a symmetric
steady-state, we have \(P_{H,t}=P_{F,t}=P_t\). Now, taking logs in both
sides and using the Taylor expansion for a vector of two variables we
have
\(\displaystyle f(x,y) \approx f(x_0,y_0)+\frac{\partial f(x,y)}{\partial x} \biggr|_{x_0,y_0}(x-x_0)+\frac{\partial f(x,y)}{\partial y} \biggr|_{x_0,y_0}(y-y_0)\)
So,
\(\displaystyle (P_t)^{\eta-1} = (1-\alpha)(P_{H,t})^{1-\eta} + \alpha(P_{F,t})^{1-\eta}\)

By the CPI definition, we have
\(\displaystyle P_t \equiv \left[ (1-\alpha)(P_{H,t})^{1-\eta} + \alpha(P_{F,t})^{1-\eta} \right]^{\frac{1}{1-\eta}} \ \ \Rightarrow \ \ (P_t)^{1-\eta} = (1-\alpha)(P_{H,t})^{1-\eta} + \alpha(P_{F,t})^{1-\eta}\)

Taking logs, we have
\((1-\eta)\ln(P_t) = \ln\left[ (1-\alpha)(P_{H,t})^{1-\eta} + \alpha(P_{F,t})^{1-\eta} \right]=f(P_{H,t},P_{F,t})\)

Applying the Taylor expansion on the right side using \(x = P_{H,t}\),
\(y = P_{F,t}\) and \(x_0 = y_0 = P_t\),

\(\displaystyle (1-\eta)p_t \approx \ln \left[ (1-\alpha)(P_t)^{1-\eta} + \alpha(P_t)^{1-\eta} \right]+ \frac{(1-\alpha)(1-\eta)P_t^{-\eta}}{P_t^{1-\eta}}(P_{H,t}-P_t) + \frac{\alpha(1-\eta)P_t^{-\eta}}{P_t^{1-\eta}}(P_{F,t}-P_t)\)

\(\displaystyle (1-\eta)p_t = \ln(P_t^{1-\eta}) + (1-\alpha)(1-\eta)\frac{P_{H,t}-P_t}{P_t} + \alpha(1-\eta)\frac{P_{F,t}-P_t}{P_t}\).

\(p_t \approx p_t + (1-\alpha)[\ln(P_{H,t})-\ln(P_t)] + \alpha[\ln(P_{F,t})-\ln(P_t)] \ \ \Rightarrow \ \  p_t = (1-\alpha)p_{H,t}+\alpha \ p_{F,t}\),
as defined in the paper.

As \(s_t \equiv p_{F,t}-p_{H,t}\),
\(p_t=(1-\alpha)P_{H,t}+\alpha (s_t+p_{H,t}) = p_{H,t} + \alpha s_t\)

\vspace{12pt}

\hypertarget{equation-14---domestic-inflation}{%
\subsection{Equation (14) - Domestic
inflation}\label{equation-14---domestic-inflation}}

\(\pi_t = \pi_{H,t}+\alpha \Delta s_t\)

\vspace{8pt}

The domestic inflation rate is defined as
\(\displaystyle \pi_{H,t} \equiv p_{H,t}-p_{H,t-1}\), taking the
difference between the equation between t and t-1, we have
\(p_t-p_{t-1}=p_{H,t}-p_{H,t-1}+\alpha(s_t-s_{t-1}) \ \ \Rightarrow \ \ \pi_t = \pi_{H,t}+\alpha \Delta s_t\).

\vspace{12pt}

\hypertarget{equation-15---log-linear-terms-of-trade}{%
\subsection{Equation (15) - Log-linear terms of
trade}\label{equation-15---log-linear-terms-of-trade}}

\(s_t =\mathbb{E}_t + p_t^*-p_{H,t}\)

\vspace{8pt}

Assuming that the law of one price is valid in all times (the same goods
produced in different countries have the same price when converting to
the domestic currency, using the nominal interest rate), we have
\(P_{i,t}(j) = \mathcal{E}_{i,t}P_{i,t}^i(j)\) for all
\(i, j \in [0,1]\).

As
\(P_{i,t}^i \equiv \displaystyle \left( \int_0^1 P_{i,t}^i(j)^{{1-\varepsilon}}dj \right) ^{\frac{1}{1-\varepsilon}}\),
we have
\(\displaystyle \mathcal{E}_{i,t}P_{i,t}^i = \mathcal{E}_{i,t} \left( \int_0^1 P_{i,t}^i(j)^{{1-\varepsilon}}dj \right) ^{\frac{1}{1-\varepsilon}} = \left( (\mathcal{E}_{i,t})^{1-\varepsilon} \int_0^1 P_{i,t}^i(j)^{{1-\varepsilon}}dj \right) ^{\frac{1}{1-\varepsilon}}\)

\(\displaystyle = \left( \int_0^1 \left( \mathcal{E}_{i,t}P_{i,t}^i(j)\right) ^{{1-\varepsilon}}dj \right) ^{\frac{1}{1-\varepsilon}} = \left( \int_0^1 P_{i,t}(j)^{{1-\varepsilon}}dj \right) ^{\frac{1}{1-\varepsilon}} = P_{i,t}\)

Now, we have
\(\displaystyle P_{F,t}= \left( \int_0^1 P_{i,t}^{{1-\gamma}}di \right) ^{\frac{1}{1-\gamma}} = \left( \int_0^1 \left( \mathcal{E}_{i,t}P_{i,t}^i \right)^{{1-\gamma}}di \right)^{\frac{1}{1-\gamma}} \ \ \Rightarrow \ \ P_{F,t}^{1-\gamma} = \int_0^1 \left( \mathcal{E}_{i,t}P_{i,t}^i \right)^{{1-\gamma}}di\)

Log-linearizing the last expression, we get
\(\displaystyle \exp \left[(1-\gamma)\ln P_{F,t} \right] = \int_0^1 \exp \left[(1-\gamma) \ln \left( \mathcal{E}_{i,t}P_{i,t}^i \right) \right]di\)

\(\Rightarrow \ \  \displaystyle 1 + (1-\gamma)p_{F,t} = \int_0^1 \left[ 1+ (1-\gamma)(e_{i,t}+p_{i,t}^i)\right]di \ \ \Rightarrow \ \ p_{F,t} = \int_0^1 (e_{i,t}+p_{i,t}^i)di=\mathbb{E}_t+p_t^*\),

where \(\displaystyle \mathbb{E}_t \equiv \int_0^1 e_{i,t} \ di\) and
\(\displaystyle p_t^*\equiv \int_0^1 p_{i,t}^i \ di\). Also, we have
that \(s_t = p_{F,t}-p_{H,t}=\mathbb{E}_t + p_t^*-p_{H,t}\).

Defining the bilateral real exchange rate
\(\displaystyle \mathcal{Q}_{i,t} \equiv \frac{\mathcal{E}_{i,t}P_{i,t}}{P_t}\)
and the (log) effective real exchange rate
\(\displaystyle q_t \equiv \int_0^1 q_{i,t}di\) we have
\(\displaystyle q_t = \int_0^1 \ln \left( \frac{\mathcal{E}_{i,t}P_{i,t}}{P_t} \right)di= \int_0^1 \left(e_{i,t}+p_{i,t}-p_t \right)di = \mathbb{E}_t+p_t^*-p_t = s_t+p_{H,t}-(p_{H,t}+\alpha s_t) = (1-\alpha)s_t\)

\vspace{12pt}

\hypertarget{equation-16---international-risk-sharing}{%
\subsection{Equation (16) - International risk
sharing}\label{equation-16---international-risk-sharing}}

\(\displaystyle \beta \left( \frac{C_{t+1}^i}{C_t^i} \right)^{-\sigma} \left(\frac{P_t^i }{P_{t+1}^i}\right) \left(\frac{ \mathcal{E}_t^i}{ \mathcal{E}_{t+1}^i}\right) = Q_{t,t+1}\)

\vspace{8pt}

The problem of the representative household in any country is the same,
as the economies are all equal. There is, any country has an Euler
equation like
\(\displaystyle \beta \left( \frac{C_{t+1}}{C_t} \right)^{-\sigma} \frac{P_t}{P_{t+1}}= Q_{t,t+1}\).

The condition for the clearing of the international market is that
\(Q_{t,t+1}\) is unique. So the price converted to a common current has
to be the same. So, for every foreign country, the Euler equation
becomes

\(\displaystyle \beta \left( \frac{C_{t+1}^i}{C_t^i} \right)^{-\sigma} \frac{P_t^i \mathcal{E}_t^i}{P_{t+1}^i \mathcal{E}_{t+1}^i} = Q_{t,t+1}\).

\vspace{12pt}

\hypertarget{equation-17---consumption-and-the-bilateral-real-exchange-rate}{%
\subsection{Equation (17) - Consumption and the bilateral real exchange
rate}\label{equation-17---consumption-and-the-bilateral-real-exchange-rate}}

\(C_t = \vartheta_i C_t^i \mathcal{Q}_{i,t}^\frac{1}{\sigma}\)

\vspace{8pt}

Combining both equations, using the definition of the real exchange rate
\(\displaystyle \mathcal{Q}_{i,t} \equiv \frac{\mathcal{E}_{i,t}P_{i,t}}{P_t}\)
and solving for \(C_t\), we have

\(\displaystyle \beta \left( \frac{C_{t+1}}{C_t} \right)^{-\sigma} \frac{P_t}{P_{t+1}} = \beta \left( \frac{C_{t+1}^i}{C_t^i} \right)^{-\sigma} \frac{P_t^i \mathcal{E}_t^i}{P_{t+1}^i \mathcal{E}_{t+1}^i} \ \ \Rightarrow \ \ \left( \frac{C_{t+1}}{C_t} \right)^{-\sigma} \frac{1}{P_{t+1}} = \left( \frac{C_{t+1}^i}{C_t^i} \right)^{-\sigma} \frac{\mathcal{Q}_{i,t}}{P_{t+1}^i \mathcal{E}_{t+1}^i}\)

\(\displaystyle \Rightarrow \ \ (C_t)^\sigma = {C_{t+1}}^\sigma \left( C_{t+1}^i \right)^{-\sigma} \frac{P_{t+1}}{P_{t+1}^i \mathcal{E}_{t+1}^i}\mathcal{Q}_{i,t}(C_t^i)^\sigma \ \ \Rightarrow \ \ C_t = \frac{C_{t+1}}{C_{t+1}^i} (\mathcal{Q}_{i,t+1})^{-\frac{1}{\sigma}} C_t^i \mathcal{Q}_{i,t}^\frac{1}{\sigma} \ \ \Rightarrow \ \ C_t = \vartheta_i C_t^i \mathcal{Q}_{i,t}^\frac{1}{\sigma}\),

where
\(\displaystyle \vartheta_i = \frac{C_{t+1}}{C_{t+1}^i} (\mathcal{Q}_{i,t+1})^{-\frac{1}{\sigma}}\)
is a constant and generally will depend on initial relative net asset
positions.

\vspace{12pt}

\hypertarget{equations-a.1-and-a.2---domestic-production-and-external-production}{%
\subsection{Equations (A.1) and (A.2) - Domestic production and external
production}\label{equations-a.1-and-a.2---domestic-production-and-external-production}}

\(\displaystyle Y = \left[ (1-\alpha)h(\mathcal{S})^{\eta} q(\mathcal{S})^\frac{1}{\sigma} + \alpha q(\mathcal{S})^{\gamma} h(\mathcal{S})^{\gamma} \right]Y^* \equiv v (\mathcal{S})Y^*\)

\vspace{8pt}

Assuming identical conditions for all economies, the net asset position
for all of the is zero. In this case, \(\vartheta_i = \vartheta = 1\)
for all i. As the symmetric foresight steady-state in this condition is
shown in the appendix A.

The international market clearing implies that the total goods produced
in a country is consumed by domestically or it's exported. The integral
represents the sum of the demand for products of the economy analysed by
foreign countries. In a case with economies not with measure zero, we
need to exclude the economy analysed from the integral do avoid double
counting.

\(\displaystyle Y=C_H+C_i=(1-\alpha)\left( \frac{P_{H}}{P} \right)^{-\eta}C + \int_0^1 \left( \frac{P_i^i}{P_F} \right)^{-\gamma}C_F di = (1-\alpha)\left( \frac{P_{H}}{P} \right)^{-\eta}C + \alpha \int_0^1 \left( \frac{P_i^i}{P_F^i} \right)^{-\gamma} \left( \frac{P_{F}^i}{P^i} \right)^{-\eta}C^idi\)

\(\displaystyle Y= (1-\alpha)\left( \frac{P_{H}}{P} \right)^{-\eta}C + \alpha \int_0^1 \left( \frac{P_F^i}{P_i^i} \right)^{\gamma} \left( \frac{P_{F}^i}{P^i} \right)^{-\eta}C^idi= (1-\alpha)\left( \frac{P_{H}}{P} \right)^{-\eta}C + \alpha \int_0^1 \left( \frac{\mathcal{E}_i P_F^i }{P_H} \right)^{\gamma} \left( \frac{P_{F}^i}{P^i} \right)^{-\eta}C^idi\),

where \(P_i^i\) is the price in the domestic economy converted to the
currency of country i, or
\(\displaystyle P_i^i = \frac{P_i}{\mathcal{E}_i} = \frac{P_H}{\mathcal{E}_i}\),
as the goods have the same price in the international market, after
converting to the same currency. After simplifying, we have

\(\displaystyle Y = \left( \frac{P_{H}}{P} \right)^{-\eta} \left[ (1-\alpha)C + \alpha \int_0^1 \left( \frac{ \mathcal{E}_i P_F^i }{P_H} \right)^{\gamma-\eta} \left( \frac{\mathcal{E}_i P_i}{P} \right)^\eta C^idi \right] = \left( \frac{P_{H}}{P} \right)^{-\eta} \left[ (1-\alpha)C + \alpha \int_0^1 \left( \frac{ \mathcal{E}_i P_F^i }{P_H} \right)^{\gamma-\eta} Q_i^\eta C^idi \right]\)

Considering that
\(\displaystyle P = \left[ (1-\alpha) \left( P_H \right)^{1-\eta} + \alpha \left( P_F\right)^{1-\eta} \right]^{\frac{1}{1-\eta}}\)
in the steady-state,
\(\displaystyle P^{1-\eta} = (1-\alpha) \left( P_H \right)^{1-\eta} + \alpha \left( P_F \right)^{1-\eta}\),
as \(\displaystyle \mathcal{S}_i \equiv \frac{P_i}{P_H}\). So,

\(\displaystyle \left( \frac{P}{P_H} \right)^{1-\eta} = (1-\alpha) + \alpha \left( \frac{P_F}{P_H}\right)^{1-\eta} = (1-\alpha) + \alpha \mathcal{S}_i^{1-\eta} \ \ \Rightarrow \ \ \frac{P}{P_H}=\left[ (1-\alpha)+\alpha \mathcal{S}^{1-\eta}\right]^{\frac{1}{1-\eta}}= \left[ (1-\alpha) + \alpha \int_0^1(\mathcal{S}_i)^{1-\eta}di \right]^{\frac{1}{1-\eta}} \equiv h(\mathcal{S})\)

Defining
\(\displaystyle \mathcal{S}^i = \frac{\mathcal{E}_i P_F^i}{P_i}\) and
using the fact that
\(\displaystyle C^i=C \mathcal{Q}^{-\frac{1}{\sigma}}\) as
\(\vartheta_i = 1\) in a symmetric steady-state, we have

\(\displaystyle Y = h(\mathcal{S})^{\eta}C \left[ (1-\alpha) + \alpha \int_0^1 \left( \frac{ \mathcal{E}_i P_F^i }{P_i} \frac{P_i}{P_H} \right)^{\gamma-\eta} Q_i^{\eta-\frac{1}{\sigma}} di \right]= h(\mathcal{S})^{\eta}C \left[ (1-\alpha) + \alpha \int_0^1 \left( \mathcal{S}^i \frac{P_F}{P_H} \right)^{\gamma-\eta} Q_i^{\eta-\frac{1}{\sigma}} di \right]\)

As we will work with a first order approximation, the equality below is
valid.

\(\displaystyle Y = h(\mathcal{S})^{\eta}C \left[ (1-\alpha) + \alpha \int_0^1 \left( \mathcal{S}^i \mathcal{S}_i \right)^{\gamma-\eta} Q_i^{\eta-\frac{1}{\sigma}} di \right] = h(\mathcal{S})^{\eta}C \left[ (1-\alpha) + \alpha \int_0^1 ( \mathcal{S}^i)^{\gamma} di \int_0^1 ( \mathcal{S}_i)^{-\eta} di \int_0^1 Q_i^{\eta-\frac{1}{\sigma}} di \right]\)

\(\displaystyle Y = h(\mathcal{S})^{\eta}C \left[ (1-\alpha) + \alpha \mathcal{S}^{-\eta} \int_0^1 \left( \frac{P_F^i}{P_H} \right)^{\gamma} di \int_0^1 \left( \frac{\mathcal{E}_i P_F^i}{P} \right)^{\eta-\frac{1}{\sigma}} di \right]\),

as if
\(\displaystyle \mathcal{S}^{1-\gamma}=\int_0^1 \mathcal{S}^{1-\gamma}di\),
we can substitute variables \(-\eta=1-\gamma\) and we get the result.

\(\displaystyle Y = h(\mathcal{S})^{\eta}C \left[ (1-\alpha) + \alpha \mathcal{S}^{-\eta} \left( \frac{1}{P_H} \right)^{\gamma} \int_0^1 \left( P_F^i \right)^{\gamma} di \int_0^1 \left( \frac{\mathcal{E}_i P_F^i}{P} \right)^{\eta-\frac{1}{\sigma}} di \right]\)

\(\displaystyle Y= h(\mathcal{S})^{\eta}C \left[ (1-\alpha) + \alpha \mathcal{S}^{-\eta} \left( \frac{1}{P_H} \right)^{\gamma} (P^*)^{\gamma} \int_0^1 \left( \frac{\mathcal{E}_i P_F^i}{P_H} \frac{P_H}{P} \right)^{\eta-\frac{1}{\sigma}} di \right]\),

using the fact that
\(\left( P_F^i \right)^{1-\gamma} = \displaystyle \int_0^1 \left( P_i^i \right)^{1-\gamma}di\)
and using P* for the international price index of imported goods.

\(\displaystyle Y = h(\mathcal{S})^{\eta}C \left[ (1-\alpha) + \alpha \mathcal{S}^{-\eta} \left( \frac{P^*}{P_H} \right)^{\gamma} \int_0^1 \left( \frac{\mathcal{S}^i}{h(\mathcal{S})} \right)^{\eta-\frac{1}{\sigma}} di \right] =h(\mathcal{S})^{\eta}C \left[ (1-\alpha) + \alpha \mathcal{S}^{-\eta} \mathcal{S}^\gamma \left( \frac{1}{h(\mathcal{S})} \right)^{\eta-\frac{1}{\sigma}} \int_0^1 ( \mathcal{S}^i )^{\eta-\frac{1}{\sigma}} di \right]\)

\(\displaystyle Y = h(\mathcal{S})^{\eta}C \left[ (1-\alpha) + \alpha \mathcal{S}^{\gamma-\eta} \left( \frac{1}{h(\mathcal{S})} \right)^{\eta-\frac{1}{\sigma}} \mathcal{S} ^{\eta-\frac{1}{\sigma}} \right] = h(\mathcal{S})^{\eta}C \left[ (1-\alpha) + \alpha \mathcal{S}^{\gamma-\eta} \left( \frac{\mathcal{S}}{h(\mathcal{S})} \right)^{\eta-\frac{1}{\sigma}} \right]\),

which yields the result.
\(\displaystyle Y= h(\mathcal{S})^{\eta}C \left[ (1-\alpha) + \alpha \mathcal{S}^{\gamma-\eta} q(\mathcal{S})^{\eta-\frac{1}{\sigma}} \right]\),
where
\(\displaystyle \mathcal{Q}=\frac{\mathcal{S}}{h(\mathcal{S})} \equiv q(\mathcal{S})\)

Substituting \(\displaystyle C = C^*q(\mathcal{S})^\frac{1}{\sigma}\) in
the expression above, we have

\(\displaystyle Y= (1-\alpha)h(\mathcal{S})^{\eta}C + \alpha h(\mathcal{S})^{\eta} \mathcal{S}^{\gamma-\eta} q(\mathcal{S})^{\eta-\frac{1}{\sigma}} = (1-\alpha)h(\mathcal{S})^{\eta} C^*q(\mathcal{S})^\frac{1}{\sigma} + \alpha h(\mathcal{S})^{\eta} \mathcal{S}^{\gamma-\eta} q(\mathcal{S})^{\eta-\frac{1}{\sigma}} C^*q(\mathcal{S})^\frac{1}{\sigma}\)

Imposing market clearing \(C^*=Y^*\), we have

\(\displaystyle Y=\left[ (1-\alpha)h(\mathcal{S})^{\eta} q(\mathcal{S})^\frac{1}{\sigma} + \alpha \mathcal{S}^{\gamma-\eta} h(\mathcal{S})^{\eta} q(\mathcal{S})^{\eta} \right]Y^* = \left[ (1-\alpha)h(\mathcal{S})^{\eta} q(\mathcal{S})^\frac{1}{\sigma} + \alpha \mathcal{S}^{\gamma} h(\mathcal{S})^{-\eta} q(\mathcal{S})^{-\eta} h(\mathcal{S})^{\eta} q(\mathcal{S})^{\eta} \right]Y^*\)

\(\displaystyle Y= \left[ (1-\alpha)h(\mathcal{S})^{\eta} q(\mathcal{S})^\frac{1}{\sigma} + \alpha \mathcal{S}^{\gamma} \right]Y^* = \left[ (1-\alpha)h(\mathcal{S})^{\eta} q(\mathcal{S})^\frac{1}{\sigma} + \alpha q(\mathcal{S})^{\gamma} h(\mathcal{S})^{\gamma} \right]Y^* \equiv v (\mathcal{S})Y^*\),

where \(v (\mathcal{S})>0\), \(v' (\mathcal{S})>0\) and \(v (1)=1\)

\vspace{12pt}

\hypertarget{equation-a.3---production-in-the-steady-state}{%
\subsection{Equation (A.3) - Production in the
steady-state}\label{equation-a.3---production-in-the-steady-state}}

\(\displaystyle Y= A^{\frac{1+\varphi}{\varphi}} \left( \frac{1-\frac{1}{\varepsilon}}{(1-\tau)(Y^*)^\sigma \mathcal{S}} \right)^\frac{1}{\varphi}\)

\vspace{8pt}

The clearing of labour market in steady-state implies (the derivation of
the two equations below are in the firm's equations)

\(\displaystyle C^\sigma \left( \frac{Y}{A} \right)^\varphi= \frac{W}{P}\)

\(\displaystyle MC_t = \frac{W_t(1-\tau)}{P_{H,t} A_t} \ \ \Rightarrow \ \  MC = \frac{W(1-\tau)}{P_{H} A}\)

From the Taylor expansion in the price-setting problem of the firm, we
have

\(\displaystyle \sum_{k=0}^\infty (\beta \theta)^k\mathbb{E}_t \left\{ 1-\frac{\varepsilon}{\varepsilon-1} MC \right\}=0 \ \ \Rightarrow \ \ MC=1-\frac{1}{\varepsilon}\)

\(\displaystyle MC = \frac{W(1-\tau)}{P_{H} A} = 1-\frac{1}{\varepsilon} \ \ \Rightarrow \ \ \frac{W}{P}=A \frac{1-\frac{1}{\varepsilon}}{1-\tau} \frac{P_H}{P}=A \frac{1-\frac{1}{\varepsilon}}{1-\tau} \frac{1}{h (\mathcal{S})}\)

\(\displaystyle C^\sigma \left( \frac{Y}{A} \right)^\varphi = A \frac{1-\frac{1}{\varepsilon}}{1-\tau} \frac{1}{h (\mathcal{S})} \ \ \Rightarrow \ \ \left( C^*\mathcal{Q}^\frac{1}{\sigma} \right)^\frac{\sigma}{\varphi} \left( \frac{Y}{A} \right) = \left(A \frac{1-\frac{1}{\varepsilon}}{1-\tau} \frac{1}{h (\mathcal{S})} \right)^\frac{1}{\varphi}\)

\(\displaystyle Y=A^{1+\frac{1}{\varphi}} \left( \frac{1-\frac{1}{\varepsilon}}{(1-\tau)(C^*)^\sigma} \frac{1}{ h (\mathcal{S})\mathcal{Q}} \right)^\frac{1}{\varphi} = A^{\frac{1+\varphi}{\varphi}} \left( \frac{1-\frac{1}{\varepsilon}}{(1-\tau)(Y^*)^\sigma \mathcal{S}} \right)^\frac{1}{\varphi}\)

Substituting
\(\displaystyle Y=Y^*=A^{\frac{1+\varphi}{\sigma+\varphi}} \left( \frac{1-\frac{1}{\varepsilon}}{1-\tau} \right)^{\frac{1}{\sigma+\varphi}}\)
and solving for \(\mathcal{S}\), we have

\(\displaystyle A^{\frac{1+\varphi}{\sigma+\varphi}} \left( \frac{1-\frac{1}{\varepsilon}}{1-\tau} \right)^{\frac{1}{\sigma+\varphi}}=A^{\frac{1+\varphi}{\varphi}} \left( \frac{1-\frac{1}{\varepsilon}}{(1-\tau)(Y^*)^\sigma \mathcal{S}} \right)^\frac{1}{\varphi} \ \ \Rightarrow \ \ A^{\frac{\varphi({1+\varphi})}{\sigma+\varphi}} \left( \frac{1-\frac{1}{\varepsilon}}{1-\tau} \right)^{\frac{\varphi}{\sigma+\varphi}}=A^{1+\varphi} \frac{1-\frac{1}{\varepsilon}}{(1-\tau)(Y^*)^\sigma \mathcal{S}}\)

\(\displaystyle A^{\frac{\varphi+\varphi^2-\sigma-\varphi -\sigma\varphi-\varphi^2}{\sigma+\varphi}} \left( \frac{1-\frac{1}{\varepsilon}}{1-\tau} \right)^{\frac{\varphi-\sigma-\varphi}{\sigma+\varphi}} = \frac{1}{(Y^*)^\sigma \mathcal{S}} \ \ \Rightarrow \ \  A^{\frac{-\sigma(1+\varphi)}{\sigma+\varphi}} \left( \frac{1-\frac{1}{\varepsilon}}{1-\tau} \right)^{\frac{-\sigma}{\sigma+\varphi}} = \frac{1}{\left[A^{\frac{1+\varphi}{\sigma+\varphi}} \left( \frac{1-\frac{1}{\varepsilon}}{1-\tau} \right)^{\frac{1}{\sigma+\varphi}} \right]^\sigma \mathcal{S}}\)

\(\displaystyle A^{-\frac{1+\varphi}{\sigma+\varphi}} \left( \frac{1-\frac{1}{\varepsilon}}{1-\tau} \right)^{-\frac{1}{\sigma+\varphi}} = \frac{1}{\left[A^{\frac{1+\varphi}{\sigma+\varphi}} \left( \frac{1-\frac{1}{\varepsilon}}{1-\tau} \right)^{\frac{1}{\sigma+\varphi}} \right] \mathcal{S}^{\frac{1}{\sigma}}} \ \ \Rightarrow \ \ A^{-(1+\varphi)} \left( \frac{1-\frac{1}{\varepsilon}}{1-\tau} \right)^{-1} = \frac{1}{\left[A^{1+\varphi}\left( \frac{1-\frac{1}{\varepsilon}}{1-\tau} \right) \right] \mathcal{S}^{\frac{\sigma+\varphi}{\sigma}}} \ \ \Rightarrow \ \ \mathcal{S}^{\frac{\sigma+\varphi}{\sigma}}=1\),

which gives the result \(\mathcal{S}=1\), which in turn implies
\(\mathcal{S}_i=1\) for all \(i\) (purchasing parity holds).

\vspace{12pt}

\hypertarget{equation-18---log-linearized-relation-between-consumption-and-terms-of-trade}{%
\subsection{Equation (18) - Log-linearized relation between consumption
and terms of
trade}\label{equation-18---log-linearized-relation-between-consumption-and-terms-of-trade}}

\(\displaystyle c_t=c_t^*+\frac{1}{\sigma}q_t = c_t^*+\left( \frac{1-\alpha}{\sigma} \right)\)

\vspace{8pt}

Considering \(\vartheta_i=\vartheta=1\), we have
\(C_t=\vartheta_i C_t^i \mathcal{Q}_{i,t}^{\frac{1}{\sigma}}= C_t^i \mathcal{Q}_{i,t}^{\frac{1}{\sigma}}\).

Log-linearizing, we get \(\displaystyle c_t=c_t^i+\frac{1}{\sigma}q_t\).

Integrating both sides, we have
\(\displaystyle \int_0^1 c_t di = \int_0^1c_t^i di + \frac{1}{\sigma} \int_0^1q_t di \ \ \Rightarrow \ \ c_t=c_t^*+ \frac{1}{\sigma}q_t=c_t^*+ \frac{(1-\alpha)s_t}{\sigma}\),

where \(c_t^* \equiv \int_0^1 c_t di\)

\vspace{12pt}

\hypertarget{equation-19---uncovered-interest-parity}{%
\subsection{Equation (19) - Uncovered interest
parity}\label{equation-19---uncovered-interest-parity}}

\(r_t^i-r_t=\mathbb{E}_t[\Delta e_{i,t+1}]\)

\vspace{8pt}

As
\(\displaystyle \mathbb{E}_t\left[ \mathcal{Q}_{t,t+1} R_t^i \frac{\mathcal{E}_{i,t+1} }{\mathcal{E}_{i,t} }\right] = 1\)
and
\(\displaystyle \mathbb{E}_t\left[ \mathcal{Q}_{t,t+1} R_t \right] = 1\),
we have that

\(\displaystyle \mathbb{E}_t\left[ \mathcal{Q}_{t,t+1} R_t^i \frac{\mathcal{E}_{i,t+1} }{\mathcal{E}_{i,t} }\right] = \displaystyle \mathbb{E}_t\left[ \mathcal{Q}_{t,t+1} R_t \right] \ \ \Rightarrow \ \  \mathbb{E}_t\left[ \mathcal{Q}_{t,t+1} \left(R_t-R_t^i \left[ \frac{\mathcal{E}_{i,t+1} }{\mathcal{E}_{i,t} } \right] \right) \right] = 0\)

But to log-linearize it's better to do both sides separately.

\(\displaystyle \mathbb{E}_t \left[\exp \left( \ln \left[ \mathcal{Q}_{t,t+1} R_t^i \frac{\mathcal{E}_{i,t+1} }{\mathcal{E}_{i,t} }\right] \right) \right] \approx\)

\(\displaystyle \mathbb{E}_t \left[ 1 + \ln\frac{Q R^i \mathcal{E}_{i}}{\mathcal{E}_{i}}+\frac{1}{\mathcal{Q}R^i}R^i \frac{ \mathcal{E}_{i}}{\mathcal{E}_{i}}(\mathcal{Q}_{t,t+1}-\mathcal{Q}) + \frac{1}{QR^i}\mathcal{Q} \frac{ \mathcal{E}_{i}}{\mathcal{E}_{i}}(R_t^i-R^i) - \frac{1}{\mathcal{Q}R^i}\mathcal{Q} \frac{R^i}{\mathcal{E}_{i}}(\mathcal{E}_{i,t+1}-\mathcal{E}_i) + \frac{1}{\mathcal{Q}R^i}\mathcal{Q} \frac{R^i \mathcal{E}_{i}}{\mathcal{E}_{i}^2}(\mathcal{E}_{i,t}-\mathcal{E}_i) \right]\)

\(\displaystyle = 1+ \ln(\mathcal{Q}R^i) + \mathbb{E}_t \left[ \hat{q_t}+\hat{r_t}^i - \hat{e}_{i,t+1}+\hat{e}_{i,t}\right]\)

\(\displaystyle \mathbb{E}_t \left[\exp \left( \ln \left[ \mathcal{Q}_{t,t+1} R_t \right] \right) \right] \approx \mathbb{E}_t \left[ 1 + \ln(\mathcal{Q}R) +\frac{1}{\mathcal{Q}R}R (\mathcal{Q}_{t,t+1}-Q) +\frac{1}{\mathcal{Q}R} \mathcal{Q} (R_t-R)\right]\)

\(\displaystyle 1+ \ln(\mathcal{Q}R^i) + \mathbb{E}_t \left[ \hat{q}_t+\hat{r}_t^i - \hat{e}_{i,t+1} + \hat{e}_{i,t}\right] = 1 + \ln(\mathcal{Q}R) + \mathbb{E}_t \left[ \hat{q}_t + \hat{r}_t \right] \ \ \Rightarrow \ \ \hat{r}^i - \mathbb{E}_t \left[ e_{i,t+1}-e_{i,t}\right] = \hat{r} \ \ \Rightarrow \ \ r_t^i-r_t=\mathbb{E}_t[\Delta e_{i,t+1}]\)

\vspace{12pt}

\hypertarget{equation-20---terms-of-trade}{%
\subsection{Equation (20) - Terms of
trade}\label{equation-20---terms-of-trade}}

\(s_t =r_t^*-\mathbb{E}_t[\pi_{t+1}^*] - (r_t-\mathbb{E}_t[\pi_{H,t+1}])+\mathbb{E}_t[s_{t+1}]\)

\vspace{8pt}

The aggregation comes from the FOC. The uncovered interest rate parity
allow households to invest both in domestic and foreign assets:
\(B_t, B_t^*\). The budget constraint can be written as

\(P_t + Q_{t,t+1}D_{t+1}+Q_{t,t+1}^*\mathcal{E}_{t+1}D_{t+1}^* \leq D_t+\mathcal{E}_{t}D_t^*+W_tN_t+T_t\)

an the Lagrangean becomes

\(\mathcal{L} = \displaystyle \mathbb{E}_t \sum_{t=0}^\infty \beta^t \left[ \left( \frac{C_t^{1-\sigma}}{1-\sigma}-\frac{N_t^{1+\varphi}}{1+\varphi} \right) + \lambda_t \left(\mathcal{E}_t D_t^* + W_tN_t + T_t - Q_{t,t+1}^*\mathcal{E}_t D_{t+1}^* - Q_{t,t+1}D_{t+1} - P_t C_t \right) \right]\)

The FOCs are:

\((C_t)\)
\(\displaystyle C_t^{-\sigma} = \lambda_tP_t \ \ \Rightarrow \ \ \mathbb{E}_t[C_{t+1}^{-\sigma}] = \mathbb{E}_t[\lambda_{t+1}P_{t+1}] \ \ \Rightarrow \ \ \displaystyle \mathbb{E}_t\left[ \frac{\lambda_{t+1}}{\lambda_t}\right] = \mathbb{E}_t\left[ \left( \frac{C_{t+1}}{C_t} \right)^{-\sigma}\frac{P_t}{P_{t+1}}\right]\)

\((D_{t+1}^*)\)
\(\beta \ \mathbb{E}_t[\lambda_{t+1} \mathcal{E}_{t+1}] = \lambda_t Q_{t,t+1}^* \mathcal{E}_t \ \ \Rightarrow \ \ \displaystyle \mathbb{E}_t\left[ \frac{\lambda_{t+1}}{\lambda_t}\right] = \mathbb{E}_t\left[ \frac{Q_{t,t+1}^*\mathcal{E}_t}{\beta \mathcal{E}_{t+1}}\right]\)

\((D_{t}^*)\)
\(\beta \ \mathbb{E}_t[\lambda_{t+1} ] = \lambda_t Q_{t,t+1}\ \ \Rightarrow \ \ \displaystyle \mathbb{E}_t\left[ \frac{\lambda_{t+1}}{\lambda_t}\right] = \mathbb{E}_t\left[ \frac{Q_{t,t+1}}{\beta }\right]\)

\((N_t)\) \(N_t^\varphi=\lambda_t W_t\)

Combining the consumption FOC and the foreign bond's FOC we have

\(\displaystyle \mathbb{E}_t\left[ \left( \frac{C_{t+1}}{C_t} \right)^{-\sigma}\frac{P_t}{P_{t+1}}\right] = \mathbb{E}_t\left[ \frac{Q_{t,t+1}^*\mathcal{E}_t}{\beta \mathcal{E}_{t+1}}\right] \ \ \Rightarrow \ \ \beta \  \displaystyle \mathbb{E}_t\left[ \frac{1}{Q_{t,t+1}^*} \left( \frac{C_{t+1}}{C_t} \right)^{-\sigma}\frac{P_t}{P_{t+1}} \frac{\mathcal{E}_{t+1}}{\mathcal{E}_{t}} \right]=1\)

Doing the same steps for the domestic bonds, we have

\(\displaystyle \mathbb{E}_t\left[ \left( \frac{C_{t+1}}{C_t} \right)^{-\sigma}\frac{P_t}{P_{t+1}}\right] = \mathbb{E}_t\left[ \frac{Q_{t,t+1}}{\beta }\right] \ \ \Rightarrow \ \ \beta \  \displaystyle \mathbb{E}_t\left[ \frac{1}{Q_{t,t+1}} \left( \frac{C_{t+1}}{C_t} \right)^{-\sigma}\frac{P_t}{P_{t+1}} \right]=1\)

Dividing the equation for foreign bonds by the equation for domestic
bonds, we have

\(\displaystyle \mathbb{E}_t \left[ \frac{Q_{t,t+1}}{Q_{t,t+1}^*} \frac{\mathcal{E}_{t+1}}{\mathcal{E}_{t}} \right]=1 \ \ \Rightarrow \ \ \frac{Q_{t,t+1}}{Q_{t,t+1}^*} = \mathbb{E}_t \left[ \frac{\mathcal{E}_{t+1}}{\mathcal{E}_{t}} \right] \ \ \Rightarrow \ \ \ln(Q_{t,t+1})-\ln(Q_{t,t+1}^*) \approx \mathbb{E}_t[\ln{\mathcal{E}_{t+1} }-\ln \mathcal{E}_{t}] \ \ \Rightarrow \ \ r_t-r_t^*=\mathbb{E}_t[\Delta e_{t+1}]\)

From the definition of the log terms of trade, we have
\(s_t = \mathbb{E}_t + p_t^* - p_{H,t}\) and
\(\mathbb{E}_t[s_{t+1}] = \mathbb{E}_t[e_{t+1} + p_{t+1}^* - p_{H,t+1}]\).
Subtracting the first by the second, we get

\(s_t - \mathbb{E}_t[s_{t+1}] = \mathbb{E}_t + p_t^* - p_{H,t} - \mathbb{E}_t[e_{t+1}] - \mathbb{E}_t[p_{t+1}^*] - \mathbb{E}_t[p_{H,t+1}]\)

\(s_t = \mathbb{E}_t[s_{t+1}]-\mathbb{E}_t[\Delta e_{t+1}] -\mathbb{E}_t[\pi_{t+1}^*]-\mathbb{E}_t[\pi_{H,t+1}]=r_t^*-\mathbb{E}_t[\pi_{t+1}^*] - (r_t-\mathbb{E}_t[\pi_{H,t+1}])+\mathbb{E}_t[s_{t+1}]\)

\vspace{12pt}

\hypertarget{equation-21---solving-for-terms-of-trade}{%
\subsection{Equation (21) - Solving for terms of
trade}\label{equation-21---solving-for-terms-of-trade}}

\(\displaystyle s_t = \mathbb{E}_t \left[ \sum_{k=0}^\infty [\left( r_{t+t}^*-\pi_{t+k+1}^*\right)-\left( r_{t+t}-\pi_{t+k+1}\right)] \right]\)

\vspace{8pt}

Solving forward aquation (20), we have

\(s_t =r_t^*-\mathbb{E}_t[\pi_{t+1}^*] - (r_t-\mathbb{E}_t[\pi_{H,t+1}])+\mathbb{E}_t[r_{t+1}^*-\mathbb{E}_t[\pi_{t+2}^*] - (r_{t+1}-\mathbb{E}_t[\pi_{H,t+2}])+\mathbb{E}_t[s_{t+2}]]\)

\(s_t =r_t^*-\mathbb{E}_t[\pi_{t+1}^*] - (r_t-\mathbb{E}_t[\pi_{H,t+1}])+\mathbb{E}_t[r_{t+1}^*-\mathbb{E}_t[\pi_{t+2}^*] - (r_{t+1}-\mathbb{E}_t[\pi_{H,t+2}])+\mathbb{E}_t[{r_{t+2}^*-\mathbb{E}_t[\pi_{t+3}^*] - (r_{t+2}-\mathbb{E}_t[\pi_{H,t+3}])+\mathbb{E}_t[s_{t+3}]}]]\)

\(\displaystyle s_t = \mathbb{E}_t \left[ \sum_{k=0}^\infty [\left( r_{t+t}^*-\pi_{t+k+1}^*\right)-\left( r_{t+t}-\pi_{t+k+1}\right)] \right]\)

\vspace{12pt}

\hypertarget{firms}{%
\section{3 - Firms}\label{firms}}

\vspace{12pt}

\hypertarget{equation-22---log-linearized-production-function}{%
\subsection{Equation (22) - Log-linearized production
function}\label{equation-22---log-linearized-production-function}}

\(y_t=a_t+n_t\)

\vspace{8pt}

A representative firm has a technology with constant returns:

\(Y_t(j) = A_t N_t(j)\), where \(a_t = \ln(A_t)\) follows the AR(1)
process \(a_t = \rho_a a_{t-1} + \varepsilon_t\). Aggregating across
firms and log-linearizing we get that:

\(\displaystyle \int^1_0 Y_t(j) dj = \int^1_0 A_t N_t(j) dj \Rightarrow Y_t = A_t N_t \Rightarrow y_t = a_t + n_t\)

The technology defined above leads to a real marginal cost that does not
depend on the firm or the produced quantity:

\(\displaystyle MC_t = \frac{\partial}{\partial Y_t(j)} \frac{Y_t(j)}{A_t} \frac{W_t(1-\tau)}{P_{H,t}} =\frac{W_t(1-\tau)}{P_{H,t} A_t}\)
where \(\tau\) is an employment subsidy. In this case, the subsidy is
defined as the necessary for the production at the efficient level
(perfect competition).

Log-linearizing this expression, and setting \(\nu \equiv -\ln(1-\tau)\)
we get: \(mc_t(j) = -\nu + w_t - p_{H,t} - a_t\)

Also,
\(\displaystyle N_t \equiv \int_0^1 N_t(j)dj = \int_0^1\frac{Y_t(j)}{A_t}dj=\int_0^1\frac{Y_t(j)}{A_t}\frac{Y_t}{Y_t}dj=\int_0^1\frac{Y_t(j)}{Y_t}dj\frac{Y_t}{A_t}=\frac{Y_t Z_t}{A_t}\),
where \(\displaystyle Z_t \equiv \int_0^1\frac{Y_t(j)}{Y_t}dj\)

\vspace{12pt}

\hypertarget{equation-b.1-firm-price-setting}{%
\subsection{Equation (B.1) Firm
price-setting}\label{equation-b.1-firm-price-setting}}

\(\displaystyle \sum_{k=0}^\infty \theta^k\mathbb{E}_t \bigg\{ Q_{t,t+k} Y_{t+k } \left[ \overline{P}_{H,t}-\frac{\varepsilon}{\varepsilon-1} MC_{t+k}^n \right] \bigg\} = 0\)

\vspace{8pt}

If the firm can adjust price at time \(t\), it will set its price at
\(\overline P_{H,t} (j)\), which maximizes the present value of its
future profit:

\(\displaystyle \max_{\overline P_{H,t}(j)} \sum^\infty_{k=0} \theta^k \mathbb{E}_t \big[ Q_{t, t+k}[Y_{t+k}(j) (\overline P_{H,t}(j) - MC_{t+k} P_{H,t+k})] \big]\)

Also, we know that
\(\displaystyle Y_t^{\frac{\varepsilon-1}{\varepsilon}} = \int_0^1Y_t(j)^{\frac{\varepsilon-1}{\varepsilon}}dj\),
so
\(\displaystyle Y_{t+k}^{\frac{\varepsilon-1}{\varepsilon}} = \int_0^1Y_{t+k}(j)^{\frac{\varepsilon-1}{\varepsilon}}dj\)
and

\(\displaystyle Y_{t+k}(j) \leq \left( \frac{\overline{P}_{H,t}}{P_{H,t+k}} \right)^{-\varepsilon} \left( C_{H,t+k}+\int_0^1C_{H,t+k}^i di\right)\)

Then maximizing, the restriction above is binding (otherwise part of the
production would have been wasted, which contradicts the hypothesis of
optimization when the costs are not null). The total demand is

\(\displaystyle Y_{t+k}^{\frac{\varepsilon-1}{\varepsilon}} = \int_0^1 \left[ \left( \frac{\overline{P}_{H,t}}{P_{H,t+k}} \right)^{-\varepsilon} \left( C_{H,t+k}+\int_0^1C_{H,t+k}^i di\right) \right]^{\frac{\varepsilon-1}{\varepsilon}} dj\)

As nothing in the RHS of the above expression depends on (j), we can
take everything out of the integral.

\(\displaystyle Y_{t+k}^{\frac{\varepsilon-1}{\varepsilon}} = \left[ \left( \frac{\overline{P}_{H,t}}{P_{H,t+k}} \right)^{-\varepsilon} \left( C_{H,t+k}+\int_0^1C_{H,t+k}^i di\right) \right]^{\frac{\varepsilon-1}{\varepsilon}} \int_0^1 dj\)
and

\(\displaystyle Y_{t+k} = \left( \frac{\overline{P}_{H,t}}{P_{H,t+k}} \right)^{-\varepsilon} \left( C_{H,t+k}+\int_0^1C_{H,t+k}^i di\right)\)

The price setting problem of the firm becomes then:

\(\displaystyle \max_{\overline P_{H,t}(j)} \sum^\infty_{k=0} \theta^k \mathbb{E}_t \big[ Q_{t, t+k}[Y_{t+k}(j) (\overline P_{H,t}(j) - MC_{t+k} P_{H,t+k})] \big]\)

subject to
\(\displaystyle \left( \frac{\overline{P}_{H,t}}{P_{H,t+k}} \right)^{-\varepsilon} \left( C_{H,t+k}+\int_0^1C_{H,t+k}^i di\right)-Y_{t+k}(j) = 0\)

Substituting the restriction in the maximization problem, we get an
unrestricted maximization problem.

\(\displaystyle \underset{\overline{P}_{H,t}} \max \sum_{k=0}^\infty \theta^k\mathbb{E}_t \left\{Q_{t,t+k} \left[ \left( \frac{\overline{P}_{H,t}}{P_{H,t+k}} \right)^{-\varepsilon} \left( C_{H,t+k}+\int_0^1C_{H,t+k}^i di\right) (\overline{P}_{H,t}-MC_{t+k}^n) \right] \right\}\)

\(=\displaystyle \underset{\overline{P}_{H,t}} \max \sum_{k=0}^\infty \theta^k\mathbb{E}_t \left\{Q_{t,t+k} \left( C_{H,t+k}+\int_0^1C_{H,t+k}^i di\right) \left[\frac{\overline{P}_{H,t}^{1-\varepsilon}}{P_{H,t+k}^{-\varepsilon}} -\left( \frac{\overline{P}_{H,t}}{P_{H,t+k}} \right)^{-\varepsilon} MC_{t+k}^n \right] \right\}\),

which yields, after deriving with respect to \(\overline{P}_{H,t}\)

\(\displaystyle \sum_{k=0}^\infty \theta^k\mathbb{E}_t \left\{Q_{t,t+k} \left( C_{H,t+k}+\int_0^1C_{H,t+k}^i di\right) \left[(1-\varepsilon)\frac{\overline{P}_{H,t}^{-\varepsilon}}{P_{H,t+k}^{-\varepsilon}} +\varepsilon \frac{\overline{P}_{H,t}^{-\varepsilon-1}}{P_{H,t+k}^{-\varepsilon}} MC_{t+k}^n \right] \right\}=0\)

Under flexible prices, \(\overline P_{H,t}= P_{H,t}\) and
\(\overline P_{H,t+k}=P_{H, t+k}\)

\(\displaystyle (1-\varepsilon)+\varepsilon\frac{\overline{MC}_{t}^n}{\overline{P}_{H,t}}=0 \ \ \Rightarrow \ \ \overline{P}_{H,t} =\frac{\varepsilon}{\varepsilon-1} \overline{MC}_{t}^n\)

Substituting back to \(Q_{t, t+k}\) and rearranging, observing that
\(\displaystyle Y_{t+k} = \left( \frac{\overline{P}_{H,t}}{P_{H,t+k}} \right)^{-\varepsilon} \left( C_{H,t+k}+\int_0^1C_{H,t+k}^i di\right)\),
we get

\(\displaystyle \sum_{k=0}^\infty \theta^k\mathbb{E}_t \Bigg\{ Q_{t,t+k} \Bigg[ (-\varepsilon) \left( \frac{\overline{P}_{H,t}^{-\varepsilon-1} }{P_{H,t+k}^{-\varepsilon} } \right)\left( C_{H,t+k}+\int_0^1 C_{H,t+k}^idi \right) \left[ \overline{P}_{H,t}-MC_{t+k}^n\right] + Y_{t+k} \Bigg]\Bigg\} = 0\)

\(\displaystyle \sum_{k=0}^\infty \theta^k\mathbb{E}_t \Bigg\{ Q_{t,t+k} \Bigg[ \frac{(-\varepsilon)}{\overline{P}_{H,t}} \left( \frac{\overline{P}_{H,t}}{P_{H,t+k}} \right)^{-\varepsilon} \left( C_{H,t+k}+\int_0^1 C_{H,t+k}^idi \right) \left[ \overline{P}_{H,t}-MC_{t+k}^n\right] + Y_{t+k} \Bigg] \Bigg\} = 0\)

\(\displaystyle \sum_{k=0}^\infty \theta^k\mathbb{E}_t \Bigg\{ Q_{t,t+k} \Bigg[ \frac{(-\varepsilon)}{\overline{P}_{H,t}} Y_{t+k } \left[ \overline{P}_{H,t}-MC_{t+k}^n\right] + Y_{t+k} \Bigg]\Bigg\} = 0\)

As \(\varepsilon\) and \(\overline{P}_{H,t}\) don't depend on k and the
expression is equal zero, we can do the following operation

\(\displaystyle \sum_{k=0}^\infty \theta^k\mathbb{E}_t \Bigg\{ Q_{t,t+k} \Bigg[ Y_{t+k } \left[ -\varepsilon+\varepsilon \frac{MC_{t+k}^n}{\overline{P}_{H,t}} \right] \Bigg] + Y_{t+k} \Bigg\} = 0 \ \ \Rightarrow \ \ \sum_{k=0}^\infty \theta^k\mathbb{E}_t \Bigg\{ Q_{t,t+k} Y_{t+k } \left[ (1-\varepsilon)+\varepsilon \frac{MC_{t+k}^n}{\overline{P}_{H,t}} \right] \frac{\overline{P}_{H,t}}{1-\varepsilon} \Bigg\} = 0\)

\(\displaystyle \sum_{k=0}^\infty \theta^k\mathbb{E}_t \bigg\{ Q_{t,t+k} Y_{t+k } \left[ \overline{P}_{H,t}-\frac{\varepsilon}{\varepsilon-1} MC_{t+k}^n \right] \bigg\} = 0\)

\vspace{12pt}

\hypertarget{equation-b.2---log-linearization-of-price-setting}{%
\subsection{Equation (B.2) - log-linearization of price
setting}\label{equation-b.2---log-linearization-of-price-setting}}

\(\overline{p}_{H,t}-p_{H,t-1}=\beta \theta \mathbb{E}_t \{ \overline{p}_{H,t+1}-p_{H,t}\} + \pi_{H,t}+(1-\beta \theta)\widehat{mc}_t\)

\vspace{8pt}

Using the fact that
\(\displaystyle Q_{t,t+k}=\beta^k \left( \frac{C_{t+k}}{C_t} \right)^{-\sigma}\frac{P_t}{P_{t+k}}\),
we have

\(\displaystyle \sum_{k=0}^\infty \theta^k\mathbb{E}_t \left\{ \beta^k \left( \frac{C_{t+k}}{C_t} \right)^{-\sigma}\frac{P_t}{P_{t+k}} Y_{t+k } \left[ \overline{P}_{H,t}-\frac{\varepsilon}{\varepsilon-1} MC_{t+k}^n \right] \right\} = 0\)

As \(P_t\) and \(C_t\) doesn't depend on k, we can put it out of
summation and ignore (as the expression equals zero)

\(\displaystyle \sum_{k=0}^\infty (\beta \theta)^k\mathbb{E}_t \left\{ C_{t+k}^{-\sigma}P_{t+k}^{-1} Y_{t+k } \left[ \overline{P}_{H,t}-\frac{\varepsilon}{\varepsilon-1} MC_{t+k}^n \right] \right\} = 0\)

\(\displaystyle \sum_{k=0}^\infty (\beta \theta)^k\mathbb{E}_t \left\{ C_{t+k}^{-\sigma} Y_{t+k } \frac{P_{H,t-1}}{P_{t+k}} \left[ \frac{\overline{P}_{H,t}}{P_{H,t-1}}-\frac{\varepsilon}{\varepsilon-1}\frac{P_{H,t+k}}{P_{H,t-1}} \frac{MC_{t+k}^n}{P_{H,t+k}} \right] \right\} = 0\)

\(\displaystyle \sum_{k=0}^\infty (\beta \theta)^k\mathbb{E}_t \left\{ C_{t+k}^{-\sigma} Y_{t+k } \frac{P_{H,t-1}}{P_{t+k}} \left[ \frac{\overline{P}_{H,t}}{P_{H,t-1}}-\frac{\varepsilon}{\varepsilon-1}\Pi_{t-1,t+k}^H MC_{t+k}\right] \right\} = 0\)
where \(\displaystyle \Pi_{t-1,t+k}^H = \frac{P_{H,t+k}}{P_{H,t-1}}\)
and \(\displaystyle MC_{t+k} = \frac{MC_{t+k}^n}{P_{H,t+k}}\)

As we will use a first-order approximation, we can ignore the Jensen's
inequality. To make the multivariate Taylor expansion, we can use
\(f(x+\Delta x) \approx f(x)+ \Delta x^T \nabla f|_x(\Delta x)\), where
x is the vector of variables in the steady-state.

\(\displaystyle \sum_{k=0}^\infty (\beta \theta)^k\mathbb{E}_t \left\{ C_{t+k}^{-\sigma} Y_{t+k } \frac{P_{H,t-1}}{P_{t+k}} \left[ \frac{\overline{P}_{H,t}}{P_{H,t-1}}-\frac{\varepsilon}{\varepsilon-1}\Pi_{t-1,t+k}^H MC_{t+k}\right] \right\} = \sum_{k=0}^\infty (\beta \theta)^k\mathbb{E}_t \left\{ C^{-\sigma} Y \frac{P_{H}}{P} \left[ \frac{\overline{P}_{H}}{P_{H}}-\frac{\varepsilon}{\varepsilon-1} MC \right] \right\}\)

\(\displaystyle + \sum_{k=0}^\infty (\beta \theta)^k\mathbb{E}_t \left\{ -\sigma C^{-\sigma-1} Y \frac{P_{H}}{P} \left[ 1-\frac{\varepsilon}{\varepsilon-1} MC \right](C_{t+k}-C) \right\} + \sum_{k=0}^\infty (\beta \theta)^k\mathbb{E}_t \left\{ C^{-\sigma} \frac{P_{H}}{P} \left[ 1-\frac{\varepsilon}{\varepsilon-1} MC \right](Y_{t+k}-Y) \right\}\)

\(\displaystyle + \sum_{k=0}^\infty (\beta \theta)^k\mathbb{E}_t \left\{ \left( C^{-\sigma} Y \frac{1}{P} \left[ 1-\frac{\varepsilon}{\varepsilon-1} MC \right] + C^{-\sigma}Y \frac{P_H}{P} \left[ -\frac{1}{P_H} \right] \right)(P_{H,t-1}-P_H) \right\}\)

\(\displaystyle + \sum_{k=0}^\infty (\beta \theta)^k\mathbb{E}_t \left\{ -C^{-\sigma} Y \frac{P_{H}}{P^2} \left[ 1-\frac{\varepsilon}{\varepsilon-1} MC \right](P_{t+k}-P) \right\} + \sum_{k=0}^\infty (\beta \theta)^k\mathbb{E}_t \left\{ C^{-\sigma} Y \frac{P_{H}}{P} \left[ \frac{1}{P_H} \right](\overline{P}_{H,t}-P_H) \right\}\)

\(\displaystyle + \sum_{k=0}^\infty (\beta \theta)^k\mathbb{E}_t \left\{ C^{-\sigma} Y \frac{P_{H}}{P} \left[ -\frac{\varepsilon}{\varepsilon-1} MC \right](\Pi_{t-1,t+k}^H-\Pi^H) \right\} + \sum_{k=0}^\infty (\beta \theta)^k\mathbb{E}_t \left\{ C^{-\sigma} Y \frac{P_{H}}{P} \left[ -\frac{\varepsilon}{\varepsilon-1} \Pi^H \right](MC_{t+k}-MC) \right\}=0\)

As \(C^{-\sigma}, Y, P_H, P\) don't depend on k, we have

\(\displaystyle \sum_{k=0}^\infty (\beta \theta)^k\mathbb{E}_t \left\{ 1-\frac{\varepsilon}{\varepsilon-1} MC \right\} + \sum_{k=0}^\infty (\beta \theta)^k\mathbb{E}_t \left\{ -\sigma \left[ 1-\frac{\varepsilon}{\varepsilon-1} MC \right]\hat{c}_{t+k} \right\} + \sum_{k=0}^\infty (\beta \theta)^k\mathbb{E}_t \left\{ \left[ 1-\frac{\varepsilon}{\varepsilon-1} MC \right]\hat{y}_{t+k} \right\}\)

\(\displaystyle + \sum_{k=0}^\infty (\beta \theta)^k\mathbb{E}_t \left\{ \left(-\frac{\varepsilon}{\varepsilon-1} MC \right)\hat{p}_{H,t-1} \right\} + \sum_{k=0}^\infty (\beta \theta)^k\mathbb{E}_t \left\{ - \left[ 1-\frac{\varepsilon}{\varepsilon-1} MC \right](\hat{p}_{t+k}) \right\} + \sum_{k=0}^\infty (\beta \theta)^k\mathbb{E}_t \left\{ (\hat{\overline{p}}_{H,t}) \right\}\)

\(\displaystyle + \sum_{k=0}^\infty (\beta \theta)^k\mathbb{E}_t \left\{ \left[ -\frac{\varepsilon}{\varepsilon-1} MC \right]\pi_{t-1,t+k} \right\} + \sum_{k=0}^\infty (\beta \theta)^k\mathbb{E}_t \left\{ MC \left[ -\frac{\varepsilon}{\varepsilon-1} \right]\widehat{mc}_{t+k} \right\}=0\)

\(\displaystyle \sum_{k=0}^\infty (\beta \theta)^k\mathbb{E}_t \left\{ \left(1-\frac{\varepsilon}{\varepsilon-1} MC \right)(1-\sigma \hat{c}_{t+k} + \hat{y}_{t+k} + \hat{p}_{H,t-1} -\hat{p}_{t+k} + \pi_{t-1,t+k} + \widehat{mc}_{t+k}) -\hat{p}_{H,t-1} +\hat{\overline{p}}_{H,t}-\pi_{t-1,t+k} - \widehat{mc}_{t+k} \right\}\)

As the Taylor approximation is around zero (we are assuming regularity
conditions to all functions),

\(\displaystyle \sum_{k=0}^\infty (\beta \theta)^k\mathbb{E}_t \left\{ 1-\frac{\varepsilon}{\varepsilon-1} MC \right\}=0\)
, we have

\(\displaystyle \sum_{k=0}^\infty (\beta \theta)^k\mathbb{E}_t \left\{ -\hat{p}_{H,t-1} +\hat{\overline{p}}_{H,t}-\pi_{t-1,t+k} - \widehat{mc}_{t+k} \right\}=0\)

\(\displaystyle \frac{1}{1-\beta \theta}(\overline{p}_{H,t}-p_{H}) + \sum_{k=0}^\infty (\beta \theta)^k\mathbb{E}_t \left\{ -(p_{H,t-1}-p_{H}) -(p_{H,t+k}-p_{H,t-1}) - \widehat{mc}_{t+k} \right\}=0\)

\(\displaystyle \frac{1}{1-\beta \theta}(\overline{p}_{H,t}-p_{H,t-1}) + \sum_{k=0}^\infty (\beta \theta)^k\mathbb{E}_t \left\{ -p_{H,t+k}+p_{H,t}-p_{H,t}+p_{H,t-1} - \widehat{mc}_{t+k} \right\}=0\)

\(\displaystyle \overline{p}_{H,t} = p_{H,t-1} + \pi_{H,t}+ \sum_{k=0}^\infty (\beta \theta)^k \mathbb{E}_t \{\pi_{H,t+k} \} -\beta \theta\sum_{k=0}^\infty (\beta \theta)^k \mathbb{E}_t \{\pi_{H,t+k} \} + (1-\beta \theta)\sum_{k=0}^\infty (\beta \theta)^k \mathbb{E}_t \{\widehat{mc}_{t+k} \}\)

\(\displaystyle \overline{p}_{H,t} = p_{H,t-1} + \sum_{k=0}^\infty (\beta \theta)^k \mathbb{E}_t \{\pi_{H,t+k} \} + (1-\beta \theta)\sum_{k=0}^\infty (\beta \theta)^k \mathbb{E}_t \{\widehat{mc}_{t+k} \}\)

Rational expectations imply that the difference between the actual
inflation and the future expectations is the steady-state inflation,
which is zero. This expression can also be written as (it's easy to see
when trying to iterate forward).

\(\overline{p}_{H,t}-p_{H,t-1}=\beta \theta \mathbb{E}_t \{ \overline{p}_{H,t+1}-p_{H,t}\} + \pi_{H,t}+(1-\beta \theta)\widehat{mc}_t\)

\vspace{12pt}

\hypertarget{equation-b.3-price-index}{%
\subsection{Equation (B.3) Price index}\label{equation-b.3-price-index}}

\vspace{8pt}

The price-setting equation is
\(\displaystyle P_H \equiv \left[ \theta (P_{H,t-1})^{1-\varepsilon} + (1-\theta)(\overline{P}_{H,t})^{1-\varepsilon}\right]^{\frac{1}{1-\varepsilon}}\)

\vspace{12pt}

\hypertarget{equation-23---expected-nominal-marginal-costs}{%
\subsection{Equation (23) - Expected nominal marginal
costs}\label{equation-23---expected-nominal-marginal-costs}}

\(\displaystyle p_{H,t}=\mu + (1-\beta \theta)\sum_{k=0}^\infty(\beta \theta)^k\mathbb{E}_t \{ mc_{t+k}^n\}\)

\vspace{8pt}

Substituting \(\widehat{mc}_t = mc_t^n-p_{H,t}+\mu\), we have

\(\overline{p}_{H,t}-p_{H,t-1}=\beta \theta \mathbb{E}_t \{ \overline{p}_{H,t+1}-p_{H,t} + \pi_{H,t}+(1-\beta \theta)(mc_t^n-p_{H,t}+\mu)\}\)

\(\overline{p}_{H,t}-p_{H,t-1} = \mathbb{E}_t \{ \beta \theta\overline{p}_{H,t+1}-\beta \theta p_{H,t} + p_{H,t}- p_{H,t-1} +mc_t^n-p_{H,t}+\mu - \beta \theta mc_t^n + \beta \theta p_{H,t} - \beta \theta \mu \}\)

\(\overline{p}_{H,t} = \mathbb{E}_t \{ \beta \theta\overline{p}_{H,t+1} +mc_t^n +\mu - \beta \theta mc_t^n - \beta \theta \mu \}\)

which can also be written as
\(\displaystyle p_{H,t}=\mu + (1-\beta \theta)\sum_{k=0}^\infty(\beta \theta)^k\mathbb{E}_t \{ mc_{t+k}^n\}\)

\vspace{12pt}

\hypertarget{equilibrium}{%
\section{4 - Equilibrium}\label{equilibrium}}

\vspace{12pt}

\hypertarget{equation-24---demand-for-each-firm-in-the-small-open-economy}{%
\subsection{Equation (24) - Demand for each firm in the small open
economy}\label{equation-24---demand-for-each-firm-in-the-small-open-economy}}

\(\displaystyle Y_t(j) = \left( \frac{P_{H,t}(j)}{P_{H,t}}\right)^{-\varepsilon} \left[ (1-\alpha)\left( \frac{P_{H,t}}{P_t} \right)^{-\eta}C_t + \alpha \int_0^1 \left( \frac{P_{H,t}}{\mathcal{E}_{i,t}P_{F,t}^i}\right)^{-\gamma} \left( \frac{P_{F,t}^i}{P_t^i} \right)^{-\eta} C_t^i di \right]\)

\vspace{8pt}

Good market clearing in the representative small open economy (``home'')
requires that the total produced inside the small economy is consumed
either by the households of this economy or imported (from the small
economy, subscript H) from households of any other country (superscript
i).

\(\displaystyle Y_t(j) = C_{H,t}(j)+\int_0^1 C_{H,t}^i(j)di = \left( \frac{P_{H,t}(j)}{P_{H,t}}\right)^{-\varepsilon}C_{H,t} + \int_0^1 \displaystyle \left( \frac{P_{H,t}^i(j)}{P_{H,t}^i}\right)^{-\varepsilon}C_{H,t}^i di\)

\(\displaystyle Y_t(j) = \left( \frac{P_{H,t}(j)}{P_{H,t}}\right)^{-\varepsilon} (1-\alpha)\left( \frac{P_{H,t}}{P_t} \right)^{-\eta}C_t + \int_0^1 \left( \frac{P_{H,t}^i(j)}{P_{H,t}^i}\right)^{-\varepsilon} \left( \frac{P_{H,t}^i}{P_{F,t}^i}\right)^{-\gamma}C_{F,t}^i di\)

\(\displaystyle Y_t(j) = \left( \frac{P_{H,t}(j)}{P_{H,t}}\right)^{-\varepsilon} (1-\alpha)\left( \frac{P_{H,t}}{P_t} \right)^{-\eta}C_t + \int_0^1 \left( \frac{P_{H,t}^i(j)}{P_{H,t}^i}\right)^{-\varepsilon} \left( \frac{P_{H,t}}{\mathcal{E}_{i,t}P_{F,t}^i}\right)^{-\gamma} \alpha \left( \frac{P_{F,t}^i}{P_t^i} \right)^{-\eta} C_t^i di\)

Assuming symmetric preferences across countries, the home country will
sell the variety of good j for the same price, independently of which
country is buying, which implies
\(\displaystyle \frac{P_{H,t}^i (j)}{P_{H,t}^i}=\frac{P_{H,t} (j)}{P_{H,t}}\)

Thus,
\(\displaystyle Y_t(j) = \left( \frac{P_{H,t}(j)}{P_{H,t}}\right)^{-\varepsilon} \left[ (1-\alpha)\left( \frac{P_{H,t}}{P_t} \right)^{-\eta}C_t + \alpha \int_0^1 \left( \frac{P_{H,t}}{\mathcal{E}_{i,t}P_{F,t}^i}\right)^{-\gamma} \left( \frac{P_{F,t}^i}{P_t^i} \right)^{-\eta} C_t^i di \right]\)

\vspace{12pt}

\hypertarget{equation-25---aggregate-domestic-output}{%
\subsection{Equation (25) - Aggregate domestic
output}\label{equation-25---aggregate-domestic-output}}

\(\displaystyle Y_t = \left( \frac{P_{H,t}}{P_t} \right)^{-\eta}C_t \left[ (1-\alpha) + \alpha \int_0^1 \left( \mathcal{S}_{i,t} \mathcal{S}_t^i\right)^{\gamma-\eta} \mathcal{Q}_{i,t}^{\eta-\frac{1}{\sigma}} di \right]\)

\vspace{8pt}

Plugging on the definition of aggregate domestic output
\(\displaystyle Y_t^{\frac{\varepsilon-1}{\varepsilon}}= \int_0^1Y_t(j)^{\frac{\varepsilon-1}{\varepsilon}}dj\)
we have

\(\displaystyle Y_t^{\frac{\varepsilon-1}{\varepsilon}} = \int_0^1 \left( \left( \frac{P_{H,t}(j)}{P_{H,t}}\right)^{-\varepsilon} \left[ (1-\alpha)\left( \frac{P_{H,t}}{P_t} \right)^{-\eta}C_t + \alpha \int_0^1 \left( \frac{P_{H,t}}{\mathcal{E}_{i,t}P_{F,t}^i}\right)^{-\gamma} \left( \frac{P_{F,t}^i}{P_t^i} \right)^{-\eta} C_t^i di \right] \right)^{\frac{\varepsilon-1}{\varepsilon}}dj\)

\(\displaystyle Y_t^{\frac{\varepsilon-1}{\varepsilon}} = \left[ (1-\alpha)\left( \frac{P_{H,t}}{P_t} \right)^{-\eta}C_t + \alpha \int_0^1 \left( \frac{P_{H,t}}{\mathcal{E}_{i,t}P_{F,t}^i}\right)^{-\gamma} \left( \frac{P_{F,t}^i}{P_t^i} \right)^{-\eta} C_t^i di \right]^{\frac{\varepsilon-1}{\varepsilon}} \left( \frac{1}{P_{H,t}}\right)^{1-\varepsilon} \int_0^1 P_{H,t}(j) ^{1-\varepsilon} dj\)

By the definition of price index,
\(\displaystyle P_H^{1-\varepsilon} = \int_0^1 P_{H,t}(j)^{1-\varepsilon}dj\),
we have

\(\displaystyle Y_t^{\frac{\varepsilon-1}{\varepsilon}} = \left[ (1-\alpha)\left( \frac{P_{H,t}}{P_t} \right)^{-\eta}C_t + \alpha \int_0^1 \left( \frac{P_{H,t}}{\mathcal{E}_{i,t}P_{F,t}^i}\right)^{-\gamma} \left( \frac{P_{F,t}^i}{P_t^i} \right)^{-\eta} C_t^i di \right]^{\frac{\varepsilon-1}{\varepsilon}} \left( \frac{1}{P_{H,t}}\right)^{1-\varepsilon} P_H^{1-\varepsilon}\)

\(\displaystyle Y_t^{\frac{\varepsilon-1}{\varepsilon}} = \left[ (1-\alpha)\left( \frac{P_{H,t}}{P_t} \right)^{-\eta}C_t + \alpha \int_0^1 \left( \frac{P_{H,t}}{\mathcal{E}_{i,t}P_{F,t}^i}\right)^{-\gamma} \left( \frac{P_{F,t}^i}{P_t^i} \right)^{-\eta} C_t^i di \right]^{\frac{\varepsilon-1}{\varepsilon}}\)

\(\displaystyle Y_t = (1-\alpha)\left( \frac{P_{H,t}}{P_t} \right)^{-\eta}C_t + \alpha \int_0^1 \left( \frac{P_{H,t}}{\mathcal{E}_{i,t}P_{F,t}^i}\right)^{-\gamma} \left( \frac{P_{F,t}^i}{P_t^i} \right)^{-\eta} C_t^i di\)

Using the fact that
\(\displaystyle \mathcal{Q}_{i,t}= \frac{\mathcal{E}_{i,t} P_t^i}{P_t}\)

\(\displaystyle Y_t = (1-\alpha)\left( \frac{P_{H,t}}{P_t} \right)^{-\eta}C_t + \alpha \int_0^1 \left( \frac{\mathcal{E}_{i,t}P_{F,t}^i}{P_{H,t}}\right)^{\gamma} \left( P_{F,t}^i \right)^{-\eta} \left( \frac{\mathcal{Q}_{i,t}P_t}{\mathcal{E}_{i,t}} \right)^\eta C_t^i di = \left( \frac{P_{H,t}}{P_t} \right)^{-\eta} \left[ (1-\alpha)C_t + \alpha \int_0^1 \left( \frac{\mathcal{E}_{i,t}P_{F,t}^i}{P_{H,t}}\right)^{\gamma-\eta} \mathcal{Q}_{i,t}^\eta C_t^i di \right]\)

Considering that
\(\displaystyle \mathcal{S}_{i,t}=\frac{P_{i,t}}{P_{H,t}}\),
\(\displaystyle \mathcal{S}_t^i=\frac{\mathcal{E}_{i,t} P_{F,t}^i}{P_{i,t}}\),
and \(C_{t}^i=C_t \mathcal{Q}_{i,t}^{-\frac{1}{\sigma}}\) we have

\(\displaystyle Y_t = \left( \frac{P_{H,t}}{P_t} \right)^{-\eta} \left[ (1-\alpha)C_t + \alpha \int_0^1 \left( \frac{P_{i,t}}{P_{H,t}} \frac{\mathcal{E}_{i,t}P_{F,t}^i}{P_{i,t}}\right)^{\gamma-\eta} \mathcal{Q}_{i,t}^\eta \mathcal{Q}_{i,t}^{-\frac{1}{\sigma}} C_t di \right] = \left( \frac{P_{H,t}}{P_t} \right)^{-\eta}C_t \left[ (1-\alpha) + \alpha \int_0^1 \left( \mathcal{S}_{i,t} \mathcal{S}_t^i\right)^{\gamma-\eta} \mathcal{Q}_{i,t}^{\eta-\frac{1}{\sigma}} di \right]\)

\vspace{12pt}

\hypertarget{equation-26---small-economy-output-in-a-particular-case}{%
\subsection{Equation (26) - Small economy output in a particular
case}\label{equation-26---small-economy-output-in-a-particular-case}}

\(Y_t = \mathcal{S}_t^\alpha\)

\vspace{8pt}

If \(\sigma=\eta=\gamma=1\), then

\(\displaystyle Y_t = \left( \frac{P_{H,t}}{P_t} \right)^{-1} C_t \left[ (1-\alpha) + \alpha \int_0^1 \left( \mathcal{S}_{i,t} \mathcal{S}_t^i\right)^{0} \mathcal{Q}_{i,t}^{0} di \right] = \frac{P_{t}}{P_{H,t}} C_t=C_t \mathcal{S}_t^\alpha\)

as if \(\eta=1\), the CPI takes the form
\(P_t=(P_{H,t})^{1-\alpha}(P_{F,t})^\alpha\) implying
\(\displaystyle \frac{P_t}{P_{H,t}}=\left( \frac{P_{F,t}}{P_{H,t}} \right)^\alpha=\mathcal{S}^\alpha\)

\vspace{12pt}

\hypertarget{equation-27---log-linearized-output}{%
\subsection{Equation (27) - Log-linearized
output}\label{equation-27---log-linearized-output}}

\(\displaystyle y_t = c_t+\frac{\alpha \omega}{\sigma}s_t\)

\vspace{8pt}

Log-linearizing the equation (25) using the Taylor expansion around the
symmetric steady-state, we have

\(\displaystyle Y_t \approx Y + (-\eta)\left( P_H \right)^{-\eta-1}\left( \frac{1}{P} \right)^{-\eta}C \left[ (1-\alpha) + \alpha \int_0^1 \left( \mathcal{S}_{i} \mathcal{S}^i\right)^{\gamma-\eta} \mathcal{Q}_i^{\eta-\frac{1}{\sigma}} di \right](P_{H,t}-P_H)\)

\(\displaystyle + \eta\left( P_H \right)^{-\eta}\left( \frac{1}{P} \right)^{-\eta+1}C \left[ (1-\alpha) + \alpha \int_0^1 \left( \mathcal{S}_{i} \mathcal{S}^i\right)^{\gamma-\eta} \mathcal{Q}_i^{\eta-\frac{1}{\sigma}} di \right](P_t-P) +\left( \frac{P_H}{P} \right)^{-\eta} \left[ (1-\alpha) + \alpha \int_0^1 \left( \mathcal{S}_{i} \mathcal{S}^i\right)^{\gamma-\eta} \mathcal{Q}_i^{\eta-\frac{1}{\sigma}} di \right](C_{t}-C)\)

\(\displaystyle +\left( \frac{P_H}{P} \right)^{-\eta} C \left[\alpha \int_0^1 (\gamma-\eta) \left( \mathcal{S}_{i} \right)^{\gamma-\eta-1} \left( \mathcal{S}^i\right)^{\gamma-\eta} \mathcal{Q}_i^{\eta-\frac{1}{\sigma}} (\mathcal{S}_{i,t}-\mathcal{S}_{i}) di \right] +\left( \frac{P_H}{P} \right)^{-\eta} C \left[\alpha \int_0^1 (\gamma-\eta) \left( \mathcal{S}_{i} \right)^{\gamma-\eta} \left( \mathcal{S}^i\right)^{\gamma-\eta-1} \mathcal{Q}_i^{\eta-\frac{1}{\sigma}} (\mathcal{S}_{t}^i-\mathcal{S}^i) di \right]\)

\(\displaystyle +\left( \frac{P_H}{P} \right)^{-\eta} C \left[\alpha \int_0^1 \left(\eta-\frac{1}{\sigma} \right) \left( \mathcal{S}_{i} \mathcal{S}^i\right)^{\gamma-\eta} \mathcal{Q}_i^{\eta-\frac{1}{\sigma}-1} (\mathcal{Q}_{i,t}-\mathcal{Q}_i) di \right]\)

As shown in appendix A (and above, in the international risk sharing
section), in a symmetric steady-state
\(\mathcal{Q}_i=\mathcal{S}_i=\mathcal{S}^i=1\) for all i (purchasing
parity holds). Thus

\(\displaystyle Y_t-Y = \left( \frac{P_H}{P} \right)^{-\eta}C \left[ -\eta \frac{P_{H,t}-P_H}{P_H} +\eta \frac{P_{t}-P}{P} + \frac{C_t-C}{C} \right]\)

\(\displaystyle +\left( \frac{P_H}{P} \right)^{-\eta} C \left[\alpha \int_0^1 (\gamma-\eta) \left[ \frac{ \mathcal{S}_{t}^i-\mathcal{S}^i }{\mathcal{S}^i} + \frac{ \mathcal{S}_{i,t}-\mathcal{S}_i }{\mathcal{S}_i}\right] + \left(\eta-\frac{1}{\sigma} \right) \frac{\mathcal{Q}_{i,t}-\mathcal{Q}_i}{\mathcal{Q}_i} di \right]\)

As
\(\displaystyle \left( \frac{P}{P_H} \right)^{1-\eta}=(1-\alpha)+\alpha \int_o^1(\mathcal{S}_i)^{1-\eta}di =1\)
(from international risk sharing section) in a symmetric steady-state
and \(C=Y\) in the international market clearing, we have

\(\displaystyle \frac{Y_t-Y}{Y} = -\eta \hat{p}_{H,t} +\eta \hat{p}_t + \hat{c_t} + \left[\alpha (\gamma-\eta) \int_0^1 s_t^i di + \alpha (\gamma-\eta) \int_0^1 s_{i,t}di + \alpha\left(\eta-\frac{1}{\sigma} \right) \int_0^1 q_{i,t} di \right]\)

Considering that \(p_t-p_{H,t}=\alpha s_t\) and recalling that
\(\displaystyle \int_0^1s_t^idi=0\),
\(\displaystyle s_t= \int_0^1s_{i,t}di\) and
\(\displaystyle q_t \equiv \int_0^1 q_{i,t}di\), we have

\(\displaystyle y_t-y = \eta \alpha s_t +c_t-c + \alpha (\gamma-\eta) s_t + \alpha\left(\eta-\frac{1}{\sigma} \right) q_{i,t} \ \ \Rightarrow \ \ y_t=c_t+ \alpha \gamma s_t + \alpha\left(\eta-\frac{1}{\sigma} \right) q_{i,t}\)

As \(q_t = (1-\alpha)s_t\), derived in the section domestic inflation
and CPI inflation, we have

\(\displaystyle y_t=c_t+ \alpha \gamma s_t + \alpha\left(\eta-\frac{1}{\sigma} \right) (1-\alpha)s_t=c_t+\frac{\alpha}{\sigma}\left[ \sigma \gamma + (\sigma \eta - 1)(1-\alpha) \right] s_t = c_t+\frac{\alpha \omega}{\sigma}s_t\)

where \(\omega \equiv \sigma \gamma + (\sigma \eta - 1)(1-\alpha)\)

\vspace{12pt}

\hypertarget{equation-28---international-market-clearing}{%
\subsection{Equation (28) - International market
clearing}\label{equation-28---international-market-clearing}}

\(\displaystyle y_t^* \equiv \int_0^1y_t^i di=\int_0^1 c_t^i di \equiv c_t^*\)

\vspace{8pt}

The result comes directly from the definition above

\vspace{12pt}

\hypertarget{equation-29---relation-between-output-and-terms-of-trade}{%
\subsection{Equation (29) - Relation between output and terms of
trade}\label{equation-29---relation-between-output-and-terms-of-trade}}

\(\displaystyle y_t = y_t^* + \frac{1}{\sigma_\alpha}s_t\)

\vspace{8pt}

As the equation (28) equation holds for all countries, from a generic
country we have

\(\displaystyle y_t^i=c_t^i+ \frac{\alpha \omega}{\sigma} s_t^i\)

Aggregating over all countries we have the world market clearing
condition. The equatily follows from the fact that
\(\int_0^1 s_t^i di =0\). International market clearing implies
\(y_t^* \equiv \int_0^1y_t^idi = \int_0^1c_t^idi \equiv c_t^*\).

As
\(\displaystyle c_t = c_t^* + \left(\frac{1-\alpha}{\sigma} \right)s_t\),
we have that

\(\displaystyle y_t - \frac{\alpha \omega}{\sigma}s_t = c_t^* + \left(\frac{1-\alpha}{\sigma} \right)s_t \ \ \Rightarrow \ \ y_t = y_t^* + \left(\frac{1-\alpha}{\sigma} + \frac{\alpha \omega}{\sigma} \right)s_t \ \ \Rightarrow \ \ y_t = y_t^* + \frac{1}{\sigma_\alpha}s_t\)

where
\(\displaystyle \sigma_\alpha \equiv \frac{\sigma}{(1-\alpha)+ \alpha \omega}\)

\vspace{12pt}

\hypertarget{equation-30---log-linearized-euler-equation}{%
\subsection{Equation (30) - Log-linearized Euler
equation}\label{equation-30---log-linearized-euler-equation}}

\(\displaystyle y_t= \mathbb{E}_t\{y_{t+1}\} -\frac{1}{\sigma_\alpha}(r_t-\mathbb{E}_t\{\pi_{H,t+1}\} -\rho)+ \alpha \Theta \mathbb{E}_t\{\Delta y_{t+1}^*\}\)

\vspace{8pt}

Recalling that
\(\displaystyle c_t=\mathbb{E}_t\{c_{t+1}\}-\frac{1}{\sigma}(r_t-\mathbb{E}_t\{\pi_{t+1}\}-\rho)\),
we have

\(\displaystyle y_t - \frac{\alpha \omega}{\sigma}s_t = \mathbb{E}_t\{c_{t+1}\}-\frac{1}{\sigma}(r_t-\mathbb{E}_t\{\pi_{t+1}\}-\rho) \ \ \Rightarrow \ \ y_t = \frac{\alpha \omega}{\sigma}s_t + \mathbb{E}_t\{y_{t+1}\} - \frac{\alpha \omega}{\sigma}\mathbb{E}_t\{s_{t+1}\} -\frac{1}{\sigma}(r_t-\mathbb{E}_t\{\pi_{t+1}\}-\rho)\)

\(\displaystyle y_t = \mathbb{E}_t\{y_{t+1}\} -\frac{1}{\sigma}(r_t-\mathbb{E}_t\{\pi_{t+1}\}-\rho)- \frac{\alpha \omega}{\sigma}\mathbb{E}_t\{\Delta s_{t+1}\}\)

As \(\pi_t = \pi_{H,t}+\alpha\Delta s_t\), we have

\(\displaystyle y_t = \mathbb{E}_t\{y_{t+1}\} -\frac{1}{\sigma}(r_t-\mathbb{E}_t\{\pi_{H,t+1}\}-\alpha \mathbb{E}_t\{\Delta s_{t+1} \} -\rho)- \frac{\alpha \omega}{\sigma}\mathbb{E}_t\{\Delta s_{t+1}\} = \mathbb{E}_t\{y_{t+1}\} -\frac{1}{\sigma}(r_t-\mathbb{E}_t\{\pi_{H,t+1}\} -\rho)- \frac{\alpha (\omega-1)}{\sigma}\mathbb{E}_t\{\Delta s_{t+1}\}\)

\(\displaystyle y_t = \mathbb{E}_t\{y_{t+1}\} -\frac{1}{\sigma}(r_t-\mathbb{E}_t\{\pi_{H,t+1}\} -\rho)- \frac{\alpha \Theta}{\sigma}\mathbb{E}_t\{\Delta s_{t+1}\}\),

where
\(\Theta = \omega-1 = \sigma \gamma + (\sigma \eta - 1)(1-\alpha)-1=(\sigma \gamma-1) + (\sigma \eta - 1)(1-\alpha)\)

As \(\Delta s_t= \sigma_\alpha (\Delta y_t - \Delta y_t^*)\)

\(\displaystyle y_t = \mathbb{E}_t\{y_{t+1}\} -\frac{1}{\sigma}(r_t-\mathbb{E}_t\{\pi_{H,t+1}\} -\rho)- \frac{\alpha \Theta}{\sigma}\mathbb{E}_t\{\sigma_\alpha (\Delta y_{t+1} - \Delta y_{t+1}^*)\}\)

\(\displaystyle \sigma y_t = \sigma \mathbb{E}_t\{y_{t+1}\} -(r_t-\mathbb{E}_t\{\pi_{H,t+1}\} -\rho)- \alpha \Theta \sigma_\alpha (\mathbb{E}_t\{ \Delta y_{t+1} \} - \mathbb{E}_t\{\Delta y_{t+1}^*\})\)

\(\displaystyle \sigma y_t=\sigma \mathbb{E}_t\{y_{t+1}\} -(r_t-\mathbb{E}_t\{\pi_{H,t+1}\} -\rho)- \alpha (\omega-1) \frac{\sigma}{(1-\alpha)+\alpha \omega} (\mathbb{E}_t\{ y_{t+1} \} -y_t - \mathbb{E}_t\{\Delta y_{t+1}^*\})\)

\(\displaystyle \frac{\sigma y_t -\alpha \sigma y_t + \sigma \alpha \omega y_t -\sigma \alpha \omega y_t + \alpha \sigma y_t}{(1-\alpha)+\alpha \omega}= \frac{\sigma \mathbb{E}_t\{y_{t+1}\} -\alpha \sigma \mathbb{E}_t\{y_{t+1}\} + \sigma \alpha \omega \mathbb{E}_t\{y_{t+1}\} -\sigma \alpha \omega \mathbb{E}_t\{y_{t+1}\} + \alpha \sigma \mathbb{E}_t\{y_{t+1}\}}{(1-\alpha)+\alpha \omega}\)

\(\displaystyle -(r_t-\mathbb{E}_t\{\pi_{H,t+1}\} -\rho)+ \alpha (\omega-1) \frac{\sigma}{(1-\alpha)+\alpha \omega} \mathbb{E}_t\{\Delta y_{t+1}^*\}\)

\(\displaystyle \frac{\sigma y_t}{(1-\alpha)+\alpha \omega}= \frac{\sigma \mathbb{E}_t\{y_{t+1}\} }{(1-\alpha)+\alpha \omega}-(r_t-\mathbb{E}_t\{\pi_{H,t+1}\} -\rho)+ \alpha \Theta \sigma_\alpha \mathbb{E}_t\{\Delta y_{t+1}^*\}\)

\(\displaystyle \sigma_\alpha y_t= \sigma_\alpha \mathbb{E}_t\{y_{t+1}\} -(r_t-\mathbb{E}_t\{\pi_{H,t+1}\} -\rho)+ \alpha \Theta \sigma_\alpha \mathbb{E}_t\{\Delta y_{t+1}^*\} \ \ \Rightarrow \ \  y_t= \mathbb{E}_t\{y_{t+1}\} -\frac{1}{\sigma_\alpha}(r_t-\mathbb{E}_t\{\pi_{H,t+1}\} -\rho)+ \alpha \Theta \mathbb{E}_t\{\Delta y_{t+1}^*\}\)

\vspace{12pt}

\hypertarget{equation-31---the-trade-balance}{%
\subsection{Equation (31) - The trade
balance}\label{equation-31---the-trade-balance}}

\(\displaystyle nx_t = \alpha \left(\frac{\omega}{\sigma} - 1 \right) s_t\)

\vspace{8pt}

Defining net exports in terms of domestic output
\(\displaystyle nx_t \equiv \frac{1}{Y}\left( Y_t - \frac{P_t}{P_{H,t}}C_t \right)\),
defined as a fraction of steady-state output. If
\(\sigma=\eta=\gamma=1\),

\(\displaystyle Y_t = \left( \frac{P_{H,t}}{P_t} \right)^{-1} C_t \left[ (1-\alpha) + \alpha \int_0^1 \left( \mathcal{S}_{i,t} \mathcal{S}_t^i\right)^{0} \mathcal{Q}^{0} di \right] = \frac{P_{t}}{P_{H,t}} C_t=C_t \mathcal{S}_t^\alpha\)

\(\displaystyle Y_t = \frac{P_{t}}{P_{H,t}} C_t \ \ \Rightarrow \ \ Y_t P_{H,t}=P_tC_t\),
implying a balanced trade.

Log-linearizing \(Y_t=C_t \mathcal{S}_t^\alpha\), we have
\(y_t = c_t + \alpha s_t\)

A first-order approximation for \(nx_t\) is (noting that nx is zero in
the steady-state), recalling that in the steady-state, \(P_H=P\) and
\(Y=C\)

\(\displaystyle nx_t = \frac{1}{Y} \left( (Y_t-Y)-\frac{C}{P}(P_t-P)+\frac{C}{P}(P_{H,t}-P) - (C_t-C) \right)=y_t-p_t+p_{H,t}-c_t\)

As \(p_t-p_{H,t}=\alpha s_t\) and
\(\displaystyle y_t = c_t+\frac{\alpha \omega}{\sigma}\), we have
\(\displaystyle nx_t=y_t-c_t-\alpha s_t= \frac{\alpha \omega}{\sigma}s_t - \alpha s_t \ \ \Rightarrow \ \ nx_t = \alpha \left(\frac{\omega}{\sigma} - 1 \right) s_t\)

In this model, \(nx_t\) iz zero if
\(\displaystyle \frac{\omega}{\sigma} - 1=0\), or
\(\displaystyle \frac{\sigma \gamma + (\sigma \eta - 1)(1-\alpha)}{\sigma}=1\),
which means that \(\sigma (\gamma-1) + (1-\alpha)(\sigma \eta - 1)=0\)

\vspace{12pt}

\hypertarget{equation-32---coefficient-of-marginal-cost}{%
\subsection{Equation (32) - Coefficient of marginal
cost}\label{equation-32---coefficient-of-marginal-cost}}

\(\pi_{H,t} = \beta \mathbb{E}_t \left\{ \pi_{H,t+1}\right\}+ \lambda\widehat{mc}_t\)
where
\(\displaystyle \lambda = \frac{(1-\theta) (1-\beta \theta)}{\theta}\)

\vspace{8pt}

Log-linearizing the equation above around the steady-state with the
Taylor expansion gives

\(\displaystyle P_{H,t} \approx [\theta(P_H)^{1-\varepsilon} + (1-\theta)(P_H)^{1-\varepsilon}]^{\frac{1}{1-\varepsilon}} + \frac{1}{1-\varepsilon} \left[ (P_H)^{1-\varepsilon} \right]^{\frac{1}{1-\varepsilon}-1} \theta(1-\varepsilon) P_H^{-\varepsilon} (P_{H,t-1}-P_H)\)

\(\displaystyle + \frac{1}{1-\varepsilon} \left[ (P_H)^{1-\varepsilon} \right]^{\frac{\varepsilon}{1-\varepsilon}} (1-\theta)(1-\varepsilon) P_H^{-\varepsilon} (\overline{P}_{H,t}-P_H)\)

\(\displaystyle P_{H,t} \approx P_H + \theta(P_H)^{\varepsilon}(P_H)^{-\varepsilon} (P_{H,t-1}-P_H) + (1-\theta)(P_H)^{\varepsilon}(P_H)^{-\varepsilon} (\overline{P}_{H,t}-P_H)=P_H+\theta P_{H,t-1} -\theta P_H +\overline{P}_{H,t}-P_H - \theta \overline{P}_{H,t} + \theta P_H\)

\(\displaystyle P_{H,t} = \theta P_{H,t-1} + (1-\theta) \overline{P}_{H,t} \ \ \Rightarrow \ \ \frac{P_{H,t}-P_{H,t-1}}{P_H} = \frac{\theta P_{H,t-1} - P_{H,t-1} + (1-\theta) \overline{P}_{H,t}}{P_H} \ \ \Rightarrow \ \ \pi_{H,t}=(1-\theta)(\overline{p}_{H,t}-p_{H,t-1})\)

Substituting \(\overline{p}_{H,t}-p_{H,t-1}\) we have

\(\displaystyle \pi_{H,t}=(1-\theta)(\beta \theta \mathbb{E}_t \{ \overline{p}_{H,t+1}-p_{H,t} + \pi_{H,t}+(1-\beta \theta)\widehat{mc}_t\}) = (1-\theta) \left(\beta \theta \mathbb{E}_t \left\{ \frac{\pi_{H,t+1}}{1-\theta} + \pi_{H,t}+(1-\beta \theta)\widehat{mc}_t \right\} \right)\)

\(\displaystyle \theta \pi_{H,t} = \beta \theta \mathbb{E}_t \left\{ \pi_{H,t+1}\right\} +(1-\theta) (1-\beta \theta)\widehat{mc}_t \ \ \Rightarrow \ \  \pi_{H,t} = \beta \mathbb{E}_t \left\{ \pi_{H,t+1}\right\} +\frac{(1-\theta) (1-\beta \theta)}{\theta}\widehat{mc}_t\)

which gives
\(\pi_{H,t} = \beta \mathbb{E}_t \left\{ \pi_{H,t+1}\right\}+ \lambda\widehat{mc}_t\)
where
\(\displaystyle \lambda = \frac{(1-\theta) (1-\beta \theta)}{\theta}\)

\vspace{12pt}

\hypertarget{inflation-dynamics}{%
\section{5 - Inflation dynamics}\label{inflation-dynamics}}

\vspace{12pt}

\hypertarget{equation-33---log-linearized-marginal-cost}{%
\subsection{Equation (33) - Log-linearized marginal
cost}\label{equation-33---log-linearized-marginal-cost}}

\(\displaystyle mc_t =-\nu + \sigma y_t^*+ \varphi y_t+ s_t -(1+\varphi)a_t\)

\vspace{8pt}

The two relations below were derived in the previous sections

\(\pi_{H,t} = \beta \mathbb{E}_t \{ \pi_{H,t+1}\} + \lambda \widehat{mc}_t\)
where
\(\displaystyle \lambda \equiv \frac{(1-\beta \theta)(1-\theta)}{\theta}\)

The log-linearized equation of marginal cost is
\(mc_t = -\nu+w_t-p_{H,t}-a_t\) where \(\nu\equiv -\ln(1-\tau)\) and
\(\tau\) is the subsidy. Thus,
\(mc_t= -\nu + (w_t-p_t) - (p_{H,t}-p_t)-a_t\)

From the log-linearized FOC, we have
\(w_t-p_t=\sigma c_t + \varphi n_t\)

Thus, \(mc_t= -\nu + \sigma c_t + \varphi n_t + \alpha s_t -a_t\)

Using
\(\displaystyle c_t=c_t^*+ \left( \frac{1-\alpha}{\sigma} \right)s_t\)
and \(y_t = a_t + n_t\), we have

\(\displaystyle mc_t= -\nu + \sigma \left[ c_t^*+ \left( \frac{1-\alpha}{\sigma} \right)s_t \right] + \varphi (y_t-a_t) + \alpha s_t -a_t=-\nu + \sigma c_t^*+ ( 1-\alpha )s_t + \varphi (y_t-a_t) + \alpha s_t -a_t\)

As the world consumption is equal to its production (international
market clearing),

\(\displaystyle mc_t =-\nu + \sigma y_t^*+ \varphi y_t+ s_t -(1+\varphi)a_t\)

\vspace{12pt}

\hypertarget{equation-34---alternative-log-linearized-marginal-cost}{%
\subsection{Equation (34) - Alternative log-linearized marginal
cost}\label{equation-34---alternative-log-linearized-marginal-cost}}

\(mc_t = -\nu + (\sigma_\alpha+\varphi)y_t + (\sigma-\sigma_\alpha)y_t^*-(1+\varphi)a_t\)

\vspace{18pt}

Using \(\displaystyle y_t = y_t^* + \frac{1}{\sigma_\alpha}s_t\) we can
substitute for \(s_t\) in the expression above

\(\displaystyle mc_t =-\nu + \sigma y_t^*+ \varphi y_t+ \sigma_\alpha(y_t-y_t^*) -(1+\varphi)a_t \ \ \Rightarrow \ \ mc_t = -\nu + (\sigma_\alpha+\varphi)y_t + (\sigma-\sigma_\alpha)y_t^*-(1+\varphi)a_t\)

\vspace{12pt}

\hypertarget{equation-35---domestic-natural-output}{%
\subsection{Equation (35) - domestic natural
output}\label{equation-35---domestic-natural-output}}

\(\overline{y_t} =\Omega+\Gamma a_t+ \alpha \Psi y_t^*\)

\vspace{8pt}

The output gap is defined as \(x_t \equiv y_t - \overline{y}_t\)

To find The domestic natural level of output we impose \(mc_t=-\mu\) and
solving for domestic output as \(y_t= \overline{y}_t\)

\(-\mu = -\nu + (\sigma_\alpha+\varphi)\overline{y_t} + (\sigma-\sigma_\alpha)y_t^*-(1+\varphi)a_t \ \ \Rightarrow \ \ (\sigma_\alpha+\varphi)\overline{y_t} = \nu -\mu +(1+\varphi)a_t - (\sigma-\sigma_\alpha)y_t^*\)

\(\displaystyle \overline{y_t} = \frac{\nu -\mu}{\sigma_\alpha+\varphi} + \frac{1+\varphi}{\sigma_\alpha+\varphi}a_t - \frac{\sigma-\sigma_\alpha}{\sigma_\alpha+\varphi}y_t^*\)

As \(\Theta = (\sigma \gamma-1)+(\sigma \eta-1)(1-\alpha)\),
\(\omega=\sigma \gamma+(\sigma \eta-1)(1-\alpha)=\Theta+1\)
\(\displaystyle \sigma_\alpha = \frac{\sigma}{(1-\alpha)+\alpha \omega}\),
we have

\(\displaystyle \sigma-\sigma_\alpha=\sigma- \frac{\sigma}{(1-\alpha)+\alpha \omega}=\frac{-\alpha \sigma +\alpha \omega\sigma}{(1-\alpha)+\alpha \omega}\)

Let's verify that
\(\sigma-\sigma_\alpha = - \alpha \Theta \sigma_\alpha\)

\(\displaystyle \alpha \Theta \sigma_\alpha= \alpha (\omega-1) \frac{\sigma}{(1-\alpha)+\alpha \omega} = \frac{\alpha \sigma \omega -\alpha \sigma}{(1-\alpha)+\alpha \omega}=\sigma-\sigma_\alpha\)

Thus,
\(\displaystyle \overline{y_t} = \frac{\nu -\mu}{\sigma_\alpha+\varphi} + \frac{1+\varphi}{\sigma_\alpha+\varphi}a_t - \frac{ \alpha \Theta \sigma_\alpha}{\sigma_\alpha+\varphi}y_t^* \ \ \Rightarrow \ \ \overline{y_t} =\Omega+\Gamma a_t+ \alpha \Psi y_t^*\)
where

\(\displaystyle \Omega = \frac{\nu -\mu}{\sigma_\alpha+\varphi}\),
\(\displaystyle \Gamma = \frac{1+\varphi}{\sigma_\alpha+\varphi}>0\) as
\(\sigma_\alpha>0\) and
\(\displaystyle \Psi=-\frac{\Theta \sigma_\alpha}{\sigma_\alpha + \varphi}\)

\vspace{12pt}

\hypertarget{equation-36---new-keynesian-phillips-curve}{%
\subsection{Equation (36) - New Keynesian Phillips
Curve}\label{equation-36---new-keynesian-phillips-curve}}

\(\pi_{H,t} = \beta \mathbb{E}_t \{ \pi_{H,t+1}\} + \kappa_\alpha x_t\)

Substituting \(y_t\) in
\(mc_t = -\nu + (\sigma_\alpha+\varphi)y_t + (\sigma-\sigma_\alpha)y_t^*-(1+\varphi)a_t\),

we have
\(mc_t = -\nu + (\sigma_\alpha+\varphi)(x_t+\overline{y_t}) + (\sigma-\sigma_\alpha)y_t^*-(1+\varphi)a_t\)

\(\displaystyle mc_t = -\nu + (\sigma_\alpha+\varphi)\left(x_t+\frac{\nu -\mu}{\sigma_\alpha+\varphi} + \frac{1+\varphi}{\sigma_\alpha+\varphi}a_t - \frac{ \alpha \Theta \sigma_\alpha}{\sigma_\alpha+\varphi}y_t^* \right) + (\sigma-\sigma_\alpha)y_t^*-(1+\varphi)a_t\)

\(\displaystyle mc_t = -\nu + (\sigma_\alpha+\varphi)x_t + \nu - \mu - \alpha \Theta \sigma_\alpha y_t^* + (\sigma-\sigma_\alpha)y_t^*\)

\(\displaystyle mc_t-(-\mu)=(\sigma_\alpha+\varphi)x_t - \mu - \alpha \left[ \omega-1 \right] \frac{\sigma}{(1-\alpha)+\alpha \omega} y_t^* + \frac{-\alpha \sigma +\alpha \omega\sigma}{(1-\alpha)+\alpha \omega} y_t^* + \mu\)

\(\displaystyle \widehat{mc_t}=(\sigma_\alpha+\varphi)x_t + \frac{-\alpha \omega \sigma + \alpha \sigma}{(1-\alpha)+\alpha \omega} y_t^* + \frac{-\alpha \sigma +\alpha \omega\sigma}{(1-\alpha)+\alpha \omega} y_t^* \ \ \Rightarrow \ \ \widehat{mc_t}=(\sigma_\alpha+\varphi)x_t\)

Substituting the equation above into
\(\pi_{H,t} = \beta \mathbb{E}_t \{ \pi_{H,t+1} + \lambda \widehat{mc}_t\}\),
we have the a version of the New Keynesian Phillips Curve (NKPC)

\(\pi_{H,t} = \beta \mathbb{E}_t \{ \pi_{H,t+1}\} + \lambda (\sigma_\alpha+\varphi)x_t = \beta \mathbb{E}_t \{ \pi_{H,t+1}\} + \kappa_\alpha x_t\),
where \(\kappa_\alpha \equiv \lambda (\sigma_\alpha+\varphi)\)

\vspace{12pt}

\hypertarget{equation-37---output-gap}{%
\subsection{Equation (37) - output gap}\label{equation-37---output-gap}}

\(\displaystyle x_t = \mathbb{E}_t\{x_{t+1}\} -\frac{1}{\sigma_\alpha}(r_t-\mathbb{E}_t\{\pi_{H,t+1}\} -\overline{rr}_t)\)

\vspace{8pt}

Substituting \(y_t\) and \(\mathbb{E}_t\{y_{t+1}\}\) in
\(\displaystyle y_t= \mathbb{E}_t\{y_{t+1}\} -\frac{1}{\sigma_\alpha}(r_t-\mathbb{E}_t\{\pi_{H,t+1}\} -\rho)+ \alpha \Theta \mathbb{E}_t\{\Delta y_{t+1}^*\}\),
we have

\(x_t + \Omega+\Gamma a_t+ \alpha \Psi y_t^* = \mathbb{E}_t\{x_{t+1} + \Omega+\Gamma a_{t+1}+ \alpha \Psi y_{t+1}^*\} -\frac{1}{\sigma_\alpha}(r_t-\mathbb{E}_t\{\pi_{H,t+1}\} -\rho)+ \alpha \Theta \mathbb{E}_t\{\Delta y_{t+1}^*\}\)

\(\displaystyle x_t = \mathbb{E}_t\{x_{t+1}\} +\Gamma (\rho_a a_{t}-a_t)+ \alpha \Psi \mathbb{E}_t\{\Delta y_{t+1}^*\} -\frac{1}{\sigma_\alpha}(r_t-\mathbb{E}_t\{\pi_{H,t+1}\} -\rho)+ \alpha \Theta \mathbb{E}_t\{\Delta y_{t+1}^*\}\)

\(\displaystyle x_t = \mathbb{E}_t\{x_{t+1}\} -\frac{1}{\sigma_\alpha}(r_t-\mathbb{E}_t\{\pi_{H,t+1}\} -[\rho-\sigma_\alpha \Gamma (1-\rho_a)a_t +\alpha \sigma_\alpha(\Theta+\Psi) \mathbb{E}_t\{\Delta y_{t+1}^*\}])\)

\(\displaystyle x_t = \mathbb{E}_t\{x_{t+1}\} -\frac{1}{\sigma_\alpha}(r_t-\mathbb{E}_t\{\pi_{H,t+1}\} -\overline{rr}_t)\)

where
\(\overline{rr}_t \equiv \rho-\sigma_\alpha \Gamma (1-\rho_a)a_t +\alpha \sigma_\alpha(\Theta+\Psi) \mathbb{E}_t\{\Delta y_{t+1}^*\}\)
is the small open economy's natural rate of interest.

\vspace{12pt}

\hypertarget{optimal-monetary-policy-a-special-case}{%
\subsection{Optimal Monetary Policy: a special
case}\label{optimal-monetary-policy-a-special-case}}

\vspace{8pt}

The problem of the central planner is:

\(\displaystyle \max E_0\sum_{i=0}^\infty \beta^t U(C_t,N_t)\) subject
to \(Y_t=A_tN_t\), \(\displaystyle C_t=C_t^i\mathcal{Q}_{i,t}\) and
\(\displaystyle Y_t=\frac{P_t}{P_{H,t}} C_t\), as
\(\eta=\sigma=\gamma=1\).

In this case,
\(\displaystyle Y_t=\frac{(P_{H,t})^{1-\alpha}(P_{F,t})^{\alpha}}{P_{H,t}} C_t = \frac{(P_{F,t})^{\alpha}}{(P_{H,t})^{\alpha}}C_t=\mathcal{S}_t^{\alpha}C_t \ \ \Rightarrow \ \ Y_t=\mathcal{S}_t^{\alpha}C_t \ \ \Rightarrow \ \ \mathcal{S}_t= \left( \frac{Y_t}{C_t} \right)^{\frac{1}{\alpha}}\).

By equation 29, we have \(y_t=y_t^*+s_t\) or
\(\displaystyle Y_t=Y_t^{*} \mathcal{S}_t= Y_t^{*} \left( \frac{Y_t}{C_t} \right)^{\frac{1}{\alpha}} \ \ \Rightarrow \ \ C_t Y_t^\alpha = (Y_t^*)^\alpha Y_t \ \ \Rightarrow \ \ C_t=Y_t^{1-\alpha}(Y_t^*)^\alpha\).

The Central planner problem then becomes

\(\displaystyle E_0\max \sum_{i=0}^\infty \beta^t U(C_t,N_t)\) subject
to \(Y_t=A_tN_t\) and \(C_t=Y_t^{1-\alpha}(Y_t^*)^\alpha\).

The Lagrangean can be written as

\(\displaystyle \mathcal{L}= E_0 \sum_{i=0}^\infty \beta^t \left\{ U(C_t,N_t) +\Lambda_t(A_t N_t - Y_t) + \Phi_t \left( Y_t^{1-\alpha}(Y_t^*)^\alpha -C_t \right) \right\}\)

which yields the FOCS:

\((C_t)\) \(U_c(C_t,N_t)=\Phi_t\)

\((N_t)\) \(U_N(C_t,N_t)=-\Lambda_t A_t\)

\((Y_t)\) \(\Phi_t (1-\alpha)Y_t^{-\alpha}(Y_t^*)^\alpha =\Lambda_t\)

\(U_N(C_t,N_t) = -A_t \Phi_t (1-\alpha)Y_t^{-\alpha}(Y_t^*)^\alpha = -A_t U_c(C_t,N_t) (1-\alpha)Y_t^{-\alpha}(Y_t^*)^\alpha\)

\(\displaystyle -\frac{U_N(C_t,N_t)}{U_c(C_t,N_t)} = \frac{Y_t}{N_t} (1-\alpha)Y_t^{-\alpha}(Y_t^*)^\alpha=(1-\alpha) \frac{C_t}{N_t}\)

Using the utility function
\(\displaystyle U(C_t,N_t)=\log(C_t)-\frac{N_t^{1+\varphi}}{1+\varphi}\)
as \(\sigma=1\).

We have \(\displaystyle U_C(C_t,N_t)=\frac{1}{C_t}\) and
\(\displaystyle U_N(C_t,N_t)=-N_t^\varphi\).

Substituting in the FOCs relation, we have

\(\displaystyle N_t^\varphi C_t=(1-\alpha)\frac{C_t}{N_t} \ \ \Rightarrow \ \ N_t^{1+\varphi}=(1-\alpha) \ \ \Rightarrow \ \ N_t = N = (1-\alpha)^{\frac{1}{1+\varphi}}\),

which is a constant employment.

We know that, under flexible prices,
\(\displaystyle \frac{\overline{MC}_t^n}{\overline{P}_{H,t}}=\overline{MC}_t=\frac{\varepsilon-1}{\varepsilon}\)
(see section 9 for the derivation).

From the representative consumer FOCs, we have the standard relation:
\(\displaystyle -\frac{U_C(\overline{C}_t,\overline{N}_t)}{U_N(\overline{C}_t,\overline{N}_t)}=\frac{\overline{W}_t}{\overline{P}_t}\).

The subsidy \(\tau\) is chosen to achieve the optimal level of
production if the prices were fully flexible.

Thus, we have
\(\displaystyle \overline{MC}_t=\frac{\varepsilon-1}{\varepsilon}=\frac{\overline{W}_t (1-\tau)}{\overline{P}_{H,t} A_t}=-\frac{U_C(\overline{C}_t,\overline{N}_t)}{U_N(\overline{C}_t,\overline{N}_t)}\frac{\overline{P}_t (1-\tau)}{\overline{P}_{H,t} A_t}\).

As \(\eta=1\),
\(\overline{P}_t=(\overline{P}_{H,t})^{1-\alpha}(\overline{P}_{F,t})^\alpha\).
Substituting,

\(\displaystyle \overline{MC}_t= -\frac{U_C(\overline{C}_t,\overline{N}_t)}{U_N(\overline{C}_t,\overline{N}_t)}\frac{(\overline{P}_{H,t})^{1-\alpha}(\overline{P}_{F,t})^\alpha (1-\tau)}{\overline{P}_{H,t} \overline{A}_t}= -\frac{U_C(\overline{C}_t,\overline{N}_t)}{U_N(\overline{C}_t,\overline{N}_t)} \left( \frac{\overline{P}_{F,t}}{\overline{P}_{H,t}} \right)^\alpha \frac{1-\tau}{\overline{A}_t}= -\frac{(1-\tau)}{\overline{A}_t} (\mathcal{\overline{S}}_i)^\alpha \frac{U_C(\overline{C}_t,\overline{N}_t)}{U_N(\overline{C}_t,\overline{N}_t)}\)

As \(\sigma=1\),
\(\displaystyle -\frac{U_C(\overline{C}_t,\overline{N}_t)}{U_N(\overline{C}_t,\overline{N}_t)}=\overline{C}_t \overline{N}_t^\varphi \ \ \Rightarrow \ \ \overline{MC}_t= \frac{(1-\tau)}{\overline{A}_t} (\mathcal{\overline{S}}_i)^\alpha \overline{C}_t \overline{N}_t^\varphi\)

From equation 26 (equilibrium part), we know that
\(\displaystyle \overline{Y}_t=\overline{C}_t \mathcal{\overline{S}}_t^\alpha \ \ \Rightarrow \ \  \mathcal{\overline{S}}_t^\alpha = \frac{\overline{Y}_t}{\overline{C}_t}\).
Thus,
\(\displaystyle \overline{MC}_t= \frac{(1-\tau)}{\overline{A}_t} \frac{\overline{Y}_t}{\overline{C}_t} \overline{C}_t \overline{N}_t^\varphi\)

Substituting the technological constraint

\(\displaystyle 1-\frac{1}{\varepsilon}=\overline{MC}_t= \frac{(1-\tau)}{\overline{A}_t} \overline{A}_t \overline{N}_t \overline{N}_t^\varphi=(1-\tau) \overline{N}_t^{1+\varphi}=(1-\tau)\left( (1-\alpha)^{1+\varphi} \right)^{\frac{1}{1+\varphi}}=(1-\tau)(1-\alpha)\).

Hence, if \(\tau\) is set to satisfy
\(\displaystyle (1-\tau)(1-\alpha)= 1-\frac{1}{\varepsilon}\), we have
also the log-linear form \(\nu=\mu+\log(1-\alpha)\), where \(\mu\)
(defined earlier) is \(\log(1-\tau)\) and the flexible price allocation
is guaranteed.

\hypertarget{equation-38---central-bank-commitment-rule}{%
\subsection{Equation (38) - Central bank commitment
rule}\label{equation-38---central-bank-commitment-rule}}

Solving forward equation (36), we have

\(\displaystyle \pi_{H,t}=\beta \mathbb{E}_t\{ \pi_{H,t+1} \}+ \kappa_\alpha x_t = \beta \mathbb{E}_t\{ \beta \mathbb{E}_t \{ \pi_{H,t+2} \}+ \kappa_\alpha x_{t+1}\}+ \kappa_\alpha x_t=\beta^T \mathbb{E}_t\{\pi_{H,T}\} + \sum_{j=t}^T \beta^{j-t}\kappa_\alpha x_j\)

To stabilize inflation,
\(\displaystyle \pi_{H,t} =\underset{T \rightarrow \infty} \lim\beta^T \mathbb{E}_t\{\pi_{H,T}\} + \sum_{j=t}^T \beta^{j-t}\kappa_\alpha x_j\)
which will stabilized (\(\pi_{H,t}=0\)) only if the output gap is zero
for every period.

As
\(x_t \equiv y_t-\overline{y}_t \ \ \Rightarrow \ \ y_t=\overline{y}_t\).

In equation (37) we have
\(\displaystyle x_t = \mathbb{E}_t\{x_{t+1}\} -\frac{1}{\sigma_\alpha}(r_t-\mathbb{E}_t\{\pi_{H,t+1}\} -\overline{rr}_t)\).

In this case, we have \(\mathbb{E}_t\{x_{t+1}\}=0\),
\(\mathbb{E}_t\{\pi_{H,t+1}\}=0\), and \(\sigma_\alpha=1\), as

\(\displaystyle \sigma_\alpha \equiv \frac{\sigma}{(1-\alpha)+\alpha \omega}=\frac{1}{(1-\alpha)+\alpha (\sigma \gamma+(\sigma \eta -1 )(1-\alpha))}=\frac{1}{(1-\alpha)+\alpha (1+(1 -1 )(1-\alpha))}=1\).

Thus, \(r_t=\overline{rr}_t\).

Equation (38) comes from a hypothesis that the Central Bank commits
itself with the rule
\(r_t=\overline{rr}_t+\phi_\pi \pi_{H,t}+\phi_x x_t\).

\vspace{12pt}

\hypertarget{equation-c.1---difference-equation-system}{%
\subsection{Equation (C.1) - Difference equation
system}\label{equation-c.1---difference-equation-system}}

\(\displaystyle \left[ \begin{matrix} x_t\\ \pi_t \end{matrix} \right] =\mathbf{A_O} \left[ \begin{matrix} \mathbb{E}_t\{x_{t+1} \}\\ \mathbb{E}_t \{\pi_{t+1} \} \end{matrix} \right]\)

\vspace{8pt}

After setting \(r_t=\overline{rr}_t\) in a closed economy we have

\(\displaystyle x_t = \mathbb{E}_t\{x_{t+1}\} -\frac{1}{\sigma_\alpha}(r_t-\mathbb{E}_t\{\pi_{H,t+1}\} -\overline{rr}_t) = \mathbb{E}_t\{x_{t+1}\} -\frac{1}{\sigma}(\overline{rr}_t-\mathbb{E}_t\{\pi_{H,t+1}\} -\overline{rr}_t) = \mathbb{E}_t\{x_{t+1}\} -\frac{1}{\sigma}\mathbb{E}_t\{\pi_{H,t+1}\}\)

We have now a system with 2 equations

\(\displaystyle x_t = \mathbb{E}_t\{x_{t+1}\} +\frac{1}{\sigma}\mathbb{E}_t\{\pi_{H,t+1}\}\)

\(\pi_{H,t}=\beta \mathbb{E}_t\{ \pi_{H,t+1}\} + \kappa_\alpha x_t\),

which can be summarized as
\(\displaystyle \left[ \begin{matrix} 1 & 0\\ -\kappa & 1 \end{matrix} \right] \left[ \begin{matrix} x_t\\ \pi_t \end{matrix} \right] = \left[ \begin{matrix} 1 & \sigma^{-1} \\ 0 & \beta \end{matrix} \right] \left[ \begin{matrix} \mathbb{E}_t\{x_{t+1} \}\\ \mathbb{E}_t \{\pi_{t+1} \} \end{matrix} \right]\)

\(\displaystyle \left[ \begin{matrix} x_t\\ \pi_t \end{matrix} \right] = \left[ \begin{matrix} 1 & 0\\ \kappa & 1 \end{matrix} \right] \left[ \begin{matrix} 1 & \sigma^{-1} \\ 0 & \beta \end{matrix} \right] \left[ \begin{matrix} \mathbb{E}_t\{x_{t+1} \}\\ \mathbb{E}_t \{\pi_{t+1} \} \end{matrix} \right] = \left[ \begin{matrix} 1 & \sigma^{-1} \\ \kappa & \kappa \sigma^{-1}+ \beta \end{matrix} \right] \left[ \begin{matrix} \mathbb{E}_t\{x_{t+1} \}\\ \mathbb{E}_t \{\pi_{t+1} \} \end{matrix} \right] \ \ \Rightarrow \ \ \left[ \begin{matrix} x_t\\ \pi_t \end{matrix} \right] = \left[ \begin{matrix} 1 & \sigma^{-1} \\ \kappa & \beta +\kappa \sigma^{-1} \end{matrix} \right] \left[ \begin{matrix} \mathbb{E}_t\{x_{t+1} \}\\ \mathbb{E}_t \{\pi_{t+1} \} \end{matrix} \right]\).

Defining
\(\mathbf{A_O}=\left[ \begin{matrix} 1 & \sigma^{-1} \\ \kappa & \beta +\kappa \sigma^{-1} \end{matrix} \right]\),
we get the expression

\(\displaystyle \left[ \begin{matrix} x_t\\ \pi_t \end{matrix} \right] =\mathbf{A_O} \left[ \begin{matrix} \mathbb{E}_t\{x_{t+1} \}\\ \mathbb{E}_t \{\pi_{t+1} \} \end{matrix} \right]\)

\vspace{12pt}

\hypertarget{equations-c.2-and-c.3---non-explosive-solution}{%
\subsection{Equations (C.2) and (C.3) - Non-explosive
solution}\label{equations-c.2-and-c.3---non-explosive-solution}}

\(\kappa (\phi_\pi-1)+\phi_y(1-\beta)>0\)

\vspace{8pt}

From the system above, we can calculate its eigenvalues:

\(\left| \begin{matrix} 1-\xi & \sigma^{-1} \\ \kappa & \beta +\kappa \sigma^{-1}-\xi \end{matrix} \right|=(1-\xi)(\beta +\kappa \sigma^{-1}-\xi)-\kappa \sigma^{-1}= \beta +\kappa \sigma^{-1}-\xi - \beta \xi -\kappa \sigma^{-1} \xi+\xi^2 -\kappa \sigma^{-1}\)

Now we have a quadratic function whose roots are the eigenvalues of the
system \(f(\xi)=\xi^2-(1+\beta+\kappa \sigma^{-1})\xi+\beta\),

with the product of roots being \(\beta<1\) and the sum of the roots
greater than 1 \(1+\beta+\kappa \sigma^{-1}\).

We can see that if \(f(\xi)>0\) if \(\xi \rightarrow \infty\),
\(f(0)=\beta>0\) and
\(f(1)=1^2-(1+\beta+\kappa \sigma^{-1})1+\beta=-\kappa \sigma^{-1}<0\),

as
\(\displaystyle \kappa=\lambda(\sigma+\varphi)=\frac{(1-\beta \theta)(1-\theta)}{\theta}(\sigma+\varphi)>0\).
Thus, we have one root between 0 and 1 and another one greater than 1
(from the intermediate value theorem). As there's one eigenvalue outside
the unit root circle, there are infinite solution for this system, as
both variables are forward looking.

Now if the Central bank has commits to the rule
\(r_t=\overline{rr}_t+\phi_\pi \pi_t+\phi_x x_t\), which is the equation
equation (C.2), we have

\(\displaystyle x_t = \mathbb{E}_t\{x_{t+1}\} -\frac{1}{\sigma}(\overline{rr}_t+\phi_\pi \pi_{t}+\phi_x x_t-\mathbb{E}_t\{\pi_{t+1}\} -\overline{rr}_t)= -\phi_\pi \sigma^{-1} -\pi_{t}+\phi_x \sigma^{-1} x_t + \mathbb{E}_t\{x_{t+1}\}+ \sigma^{-1}\mathbb{E}_t\{\pi_{t+1}\}\)

and the system becomes

\(\displaystyle (1+\phi_x \sigma^{-1})x_t +\phi_\pi \sigma^{-1}\pi_{t} = \mathbb{E}_t\{x_{t+1}\} +\sigma^{-1}\mathbb{E}_t\{\pi_{t+1}\}\)

\(-\kappa x_t +\pi_{H,t}=\beta \mathbb{E}_t\{ \pi_{H,t+1}\}\),

\(\displaystyle \left[ \begin{matrix} 1+\phi_x \sigma^{-1} & \phi_\pi \sigma^{-1}\\ -\kappa & 1 \end{matrix} \right] \left[ \begin{matrix} x_t\\ \pi_t \end{matrix} \right] = \left[ \begin{matrix} 1 & \sigma^{-1} \\ 0 & \beta \end{matrix} \right] \left[ \begin{matrix} \mathbb{E}_t\{x_{t+1} \}\\ \mathbb{E}_t \{\pi_{t+1} \} \end{matrix} \right]\)

\(\displaystyle \left[ \begin{matrix} x_t\\ \pi_t \end{matrix} \right] = \frac{1}{1+\phi_x \sigma^{-1}+ \phi_\pi \sigma^{-1} \kappa} \left[ \begin{matrix} 1 & -\phi_\pi \sigma^{-1}\\ \kappa & 1+\phi_x \sigma^{-1} \end{matrix} \right] \left[ \begin{matrix} 1 & \sigma^{-1} \\ 0 & \beta \end{matrix} \right] \left[ \begin{matrix} \mathbb{E}_t\{x_{t+1} \}\\ \mathbb{E}_t \{\pi_{t+1} \} \end{matrix} \right]=\frac{\sigma}{\sigma+\phi_x + \phi_\pi \kappa}\left[ \begin{matrix} 1& \sigma^{-1}(1-\beta \phi_\pi)\\ \kappa & \sigma^{-1}(\kappa+\beta \sigma+\beta\phi_x) \end{matrix} \right] \left[ \begin{matrix} \mathbb{E}_t\{x_{t+1} \}\\ \mathbb{E}_t \{\pi_{t+1} \} \end{matrix} \right]\)

\(\displaystyle \left[ \begin{matrix} x_t\\ \pi_t \end{matrix} \right] =\frac{1}{\sigma+\phi_x + \phi_\pi \kappa}\left[ \begin{matrix} \sigma& 1-\beta \phi_\pi\\ \kappa \sigma & \kappa+\beta (\sigma+\phi_x) \end{matrix} \right] \left[ \begin{matrix} \mathbb{E}_t\{x_{t+1} \}\\ \mathbb{E}_t \{\pi_{t+1} \} \end{matrix} \right]=\mathbf{A}_T\left[ \begin{matrix} \mathbb{E}_t\{x_{t+1} \}\\ \mathbb{E}_t \{\pi_{t+1} \} \end{matrix} \right]\),
where

\(\displaystyle \mathbf{A}_T \equiv\Omega\left[ \begin{matrix} \sigma& 1-\beta \phi_\pi\\ \kappa \sigma & \kappa+\beta (\sigma+\phi_x) \end{matrix} \right]\)
and
\(\displaystyle \Omega \equiv\frac{1}{\sigma+\phi_x + \phi_\pi\kappa}\)

If we restrict \(\phi_\pi>0\) and \(\phi_x>0\), \(\Omega>0\).

To satisfy the Blanchard and Khan conditions, we need that both
eigenvalues are inside the unit circle.

\(|\mathbf{A}_T-\lambda \mathbf{I}| = 0\)

\(\displaystyle \left| \frac{1}{\sigma+\phi_x + \phi_\pi \kappa}\left[ \begin{matrix} \sigma & 1-\beta \phi_\pi\\ \kappa \sigma & \kappa+\beta (\sigma+\phi_x) \end{matrix} \right] -\lambda \left[ \begin{matrix} 1 & 0\\ 0 & 1 \end{matrix} \right] \right|=0\)

\(\displaystyle \left| \left[ \begin{matrix} \displaystyle \frac{\sigma}{\sigma+\phi_x + \phi_\pi \kappa}-\lambda & \displaystyle \frac{1-\beta \phi_\pi}{\sigma+\phi_x + \phi_\pi \kappa}\\ \displaystyle \frac{\kappa \sigma}{\sigma+\phi_x + \phi_\pi \kappa} & \displaystyle \frac{\kappa+\beta (\sigma+\phi_x)}{\sigma+\phi_x + \phi_\pi \kappa}-\lambda \end{matrix} \right] \right|=0\)

\(\displaystyle \frac{\sigma \kappa + \beta\sigma (\sigma+\phi_y)}{(\sigma+\phi_x + \phi_\pi \kappa)^2}- \frac{\sigma +\kappa + \beta (\sigma+ \phi_y)}{(\sigma+\phi_x + \phi_\pi \kappa)} \lambda + \lambda^2-\frac{\sigma \kappa-\beta \phi_\pi \sigma \kappa}{(\sigma+\phi_x + \phi_\pi \kappa)^2}=\lambda^2- \frac{\sigma +\kappa + \beta (\sigma+ \phi_y)}{(\sigma+\phi_x + \phi_\pi \kappa)} \lambda + \frac{\sigma \beta(\sigma+\phi_y+\phi_\pi \kappa)}{(\sigma+\phi_x + \phi_\pi \kappa)^2}=0\)

\(\displaystyle \lambda^2- \frac{\sigma +\kappa + \beta (\sigma+ \phi_y)}{\sigma+\phi_x + \phi_\pi \kappa} \lambda + \frac{\sigma \beta}{\sigma+\phi_x + \phi_\pi \kappa}=0\)

LaSalle (1986) showed that both roots of the equation \(x^2+bx+c=0\) are
less than 1 if and only if \(|c|<1\) and \(|b|<1+c\). This comment was
taken from Drago Bergholt notes.

We have that
\(\displaystyle \left| \frac{\sigma \beta}{\sigma+ \phi_y+\phi_\pi \kappa} \right|<1\)

As \(\sigma>0\), \(\beta>0\), \(\kappa>0\), \(\phi_\pi>0\) and
\(\phi_y>0\)

\(\displaystyle \frac{\sigma \beta}{\sigma+ \phi_y+\phi_\pi \kappa} <1 \ \ \Rightarrow \ \ \sigma \beta<\sigma+ \phi_y+\phi_\pi \kappa \ \ \Rightarrow \ \ \sigma (\beta-1)< \phi_y+\phi_\pi \kappa\)
This condition is always satisfied, as \(\beta<1\).

The second condition is
\(\displaystyle \left| \frac{\sigma+\kappa+ \beta(\sigma+\phi_y)}{\sigma+ \phi_y+\phi_\pi \kappa} \right|<1+\frac{\sigma \beta}{\sigma+ \phi_y+\phi_\pi \kappa}\)

\(\displaystyle \frac{\sigma+\kappa+ \beta(\sigma+\phi_y)}{\sigma+ \phi_y+\phi_\pi \kappa} < 1+\frac{\sigma \beta}{\sigma+ \phi_y+\phi_\pi \kappa}\)

\(\sigma+\kappa+ \beta(\sigma+\phi_y) < \sigma+ \phi_y+\phi_\pi \kappa + \sigma \beta \ \ \Rightarrow \ \ \kappa + \beta \phi_y<\phi_y+\phi_\pi \kappa \ \ \Rightarrow \ \ \kappa (\phi_\pi-1)+\phi_y(1-\beta)>0\)

\vspace{12pt}

\hypertarget{equation-39---equation-c.3-for-the-open-economy}{%
\subsection{Equation (39) - Equation (C.3) for the open
economy}\label{equation-39---equation-c.3-for-the-open-economy}}

\(\kappa_{\alpha} (\phi_\pi-1)+\phi_y(1-\beta)>0\)

\vspace{8pt}

The system is analogous to the matrix equation (C.1). The only
difference is that the parameter \(\kappa\) is now \(\kappa_{\alpha}\),
as defined in equation (36). In our particular case
\(\sigma=\eta=\gamma=1\), \(\sigma_\alpha=\sigma\) and
\(\kappa_\alpha=\kappa\), the standard parameter in the New Keynesian
Phillips Curve.

\vspace{12pt}

\hypertarget{macroeconomic-implications}{%
\subsection{Macroeconomic
implications}\label{macroeconomic-implications}}

By equation (35)
\(\displaystyle \overline{y}_t=\Omega+\Gamma a_t -\alpha \Psi y_t^*=\frac{\nu-\mu}{\sigma_\alpha+\varphi}+\frac{1+\varphi}{\sigma_\alpha+\varphi} a_t -\alpha \frac{\Theta \sigma_\alpha}{\sigma_\alpha+\varphi}y_t^*\),
we can see that a technological shock always increase the output level,
as \(\sigma_\alpha>0\) and \(\varphi>0\).

Computing the natual level of the terms of trade, we have

\(\displaystyle \overline{s}_t=\sigma_\alpha (\overline{y}_t-y_t^*)=\sigma_\alpha \left(\frac{\nu-\mu}{\sigma_\alpha+\varphi}+\frac{1+\varphi}{\sigma_\alpha+\varphi} a_t -\alpha \frac{\Theta \sigma_\alpha}{\sigma_\alpha+\varphi}y_t^* -y_t^* \right)=\sigma_\alpha \left(\frac{\nu-\mu}{\sigma_\alpha+\varphi}+\frac{1+\varphi}{\sigma_\alpha+\varphi} a_t - \frac{\alpha \Theta \sigma_\alpha+\sigma_\alpha+\varphi}{\sigma_\alpha+\varphi}y_t^* \right)\)

\(\displaystyle \overline{s}_t=\sigma_\alpha \left(\frac{\nu-\mu}{\sigma_\alpha+\varphi}+\frac{1+\varphi}{\sigma_\alpha+\varphi} a_t - \frac{\alpha (\omega-1) \sigma_\alpha+\sigma_\alpha+\varphi}{\sigma_\alpha+\varphi}y_t^* \right)=\sigma_\alpha \left(\frac{\nu-\mu}{\sigma_\alpha+\varphi}+\frac{1+\varphi}{\sigma_\alpha+\varphi} a_t - \frac{(\alpha \omega- \alpha+1)\displaystyle \frac{\sigma}{1-\alpha+\alpha \omega} +\varphi}{\sigma_\alpha+\varphi}y_t^* \right)\)

\(\displaystyle \overline{s}_t=\sigma_\alpha \left(\frac{\nu-\mu}{\sigma_\alpha+\varphi}+\frac{1+\varphi}{\sigma_\alpha+\varphi} a_t - \frac{\sigma +\varphi}{\sigma_\alpha+\varphi}y_t^* \right)=\sigma_\alpha \Omega+ \sigma_\alpha \Gamma a_t - \sigma_\alpha \Phi y_t^*\),
where
\(\displaystyle \Phi=\frac{\sigma+\varphi}{\sigma_\alpha+\varphi}>0\)

As
\(\overline{p}_t=(1-\alpha) \overline{p}_{H,t}+\alpha\overline{p}_{F,t} = (1-\alpha)\overline{p}_{H,t}+\alpha(\overline{e}_{t}+p_t^*)\).
As domestic prices are fully stabilized,
\((1-\alpha)\overline{p}_{H,t}\) is a constant, so \(\overline{p}_t\) is
proportional to
\(\alpha(\overline{e}_{t}+p_t^*) = \alpha \overline{s}_t\)

\vspace{12pt}

\hypertarget{the-welfare-costs-of-deviations-from-the-optimal}{%
\section{6 - The welfare costs of deviations from the
optimal}\label{the-welfare-costs-of-deviations-from-the-optimal}}

\vspace{12pt}

\hypertarget{equation-40---function-with-welfare-losses}{%
\subsection{Equation (40) - Function with welfare
losses}\label{equation-40---function-with-welfare-losses}}

\(\displaystyle \mathbb{W}= -\frac{(1-\alpha)}{2}\sum_{i=0}^{\infty} \beta^t \left[ \frac{\varepsilon}{\lambda}\pi_{H,t}^2+ (1+\varphi)x_t^2 \right] + t.i.p+ o(||a||^3)\)

\vspace{8pt}

The calculations for this expressions are summarized in appendix D. By
the Taylor's rule, the second order approximation is:

\(\displaystyle f(x_t) = f(a) + f'(a)(x_t-a) + \frac{1}{2}f''(a)(x_t-a)^2 + \frac{1}{6}f'''(\widetilde{a})(x_t-a)^3\).
We assume that the third term is small, as the deviations from the
steady-state are assumed to be small (by the intermediate value theorem,
\(\widetilde{a}\) is between \(x_t\) and \(a\).

We can take

\(\displaystyle \frac{Y_t}{Y}=e^{\ln \frac{Y_t}{Y}}=1+\ln \frac{Y_t}{Y}+\frac{1}{2} \left( \ln \frac{Y_t}{Y} \right)^2 + \frac{1}{3!}\left( \ln \frac{Y_t}{Y} \right)^3 + ... = 1+y_t+\frac{1}{2} \left( y_t \right)^2+o(||a||^n)\),
there \(a\) is the bound for the high order terms.

\(\displaystyle \frac{Y_t-Y}{Y}= y_t+\frac{y_t^2}{2} +o(||a||^n)\)

Combining equations (18):
\(\displaystyle c_t=c_t^*+ \left(\frac{1-\alpha}{\sigma} \right) s_t\)
and (29):
\(\displaystyle y_t = y_t^*+\frac{1}{\sigma_\alpha}s_t = y_t^*+\frac{1}{\sigma}s_t\)
as \(\omega=1\), we have:

\(\displaystyle c_t=c_t^*+\left( \frac{1-\alpha}{\sigma} \right)\sigma(y_t-y_t^*)\).
As \(c_t^*=y_t^*\) because of global market clearing,
\(c_t=y_t^* + y_t -y_t^* -\alpha y_t + \alpha y_t^* \ \ \Rightarrow \ \ c_t = (1-\alpha)y_t+\alpha y_t^*\).

As \(x_t \equiv y_t-\overline{y}_t\), in the stabilized economy,
\(x_t=0\) and \(y_t=\overline{y}_t\). Thus,
\(\overline{c}_t = (1-\alpha)(0-\overline{y}_t)+\alpha y_t^*=\alpha y_t^*-(1-\alpha)\overline {y}_t\)

Substituting, we get:
\(c_t = (1-\alpha)(\overline{y}_t+x_t)+\alpha y_t^*=(1-\alpha)\overline{y}_t+\alpha y_t^* +(1-\alpha)x_t \ \ \Rightarrow \ \ c_t=\overline{c}_t+(1-\alpha)x_t\).

Expanding the log-deviation of the disutility of work, we have:

\(\displaystyle \left( \frac{N_t}{\overline{N} } \right)^{1+\varphi}=\exp[(1+\varphi)\widetilde{n}]=1+(1+\varphi)\widetilde{n}_t + \frac{1}{2}\widetilde{n}_t^2 + o(||a||^3) \ \ \Rightarrow \ \ N_t^{1+\varphi} = \overline{N}^{1+\varphi} \left( 1+(1+\varphi)\widetilde{n}_t + \frac{1}{2}\widetilde{n}_t^2 + o(||b||^3) \right)\)

\(\displaystyle \frac{N_t^{1+\varphi}}{1+\varphi} = \frac{\overline{N}^{1+\varphi}}{1+\varphi} +\overline{N}^{1+\varphi}\left[ \widetilde{n}_t + \frac{1}{2}(1+\varphi)\widetilde{n}_t^2 \right] + o(||a||^3)\)

Using the fact that
\(\displaystyle N_t=\left(\frac{Y_t}{A_t} \right) \int_0^1 \left(\frac{P_{H,t}(i)}{P_{H,t}} \right)^{-\varepsilon}di\),
\(\displaystyle \int_0^1 \left(\frac{P_{H,t}(i)}{P_{H,t}} \right)^{-\varepsilon}di=\frac{N_t A_t}{Y_t}\)

\(\displaystyle \Rightarrow \ \ \log \int_0^1 \left(\frac{P_{H,t}(i)}{P_{H,t}} \right)^{-\varepsilon}di= \log \left( \frac{N_t A_t}{Y_t} \right)=n_t+a_t-y_t\)

If we define
\(\displaystyle z_t \equiv \log \int_0^1 \left(\frac{P_{H,t}(i)}{P_{H,t}} \right)^{-\varepsilon}di\),
then \(z_t=n_t+a_t-y_t=n_t+a_t-(\overline{y}_t+x_t)\).

When prices are stabilized, \(P_{H,t}(i)=P_{H,t}\) and
\(\overline{z}_t=0\). Also, there are no productivity shocks, so
\(a_t= \overline{y}_t-\overline{n}_t\). Thus,

\(z_t=\overline{n}_t+\widetilde{n}_t+\overline{y}_t-\overline{n}_t-(\overline{y}_t+x_t) \ \ \Rightarrow \ \ \widetilde{n}_t=z_t+x_t\)

Lemma 1 (appendix D): The proof is in the paper. There's just one
passage that it took time to figure out what happened.

From
\(\displaystyle \mathbb{E}_i \{ \widehat{p}_{H,t}(i)\} = \frac{(\varepsilon-1)}{2}\mathbb{E}_i \{ \widehat{p}_{H,t}(i)^2\}\)
and
\(\displaystyle \left(\frac{P_{H,t}(i)}{P_{H,t}} \right)^{-\varepsilon}=1-\varepsilon \widehat{p}_{H,t}(i)+ \frac{\varepsilon^2}{2} \widehat{p}_{H,t}(i)^2+o(||a||^3)\)

\(\displaystyle \mathbb{E}_i\left[ \left(\frac{P_{H,t}(i)}{P_{H,t}} \right)^{-\varepsilon} \right]= \mathbb{E}_i \left[ 1-\varepsilon \widehat{p}_{H,t}(i)+ \frac{\varepsilon^2}{2} \widehat{p}_{H,t}(i)^2+o(||a||^3) \right]\)

\(\displaystyle \int_0^1 \left(\frac{P_{H,t}(i)}{P_{H,t}} \right)^{-\varepsilon} di= 1-\varepsilon \mathbb{E}_i[\widehat{p}_{H,t}(i)]+ \frac{\varepsilon^2}{2} \mathbb{E}_i[\widehat{p}_{H,t}(i)^2]=1-\varepsilon \frac{(\varepsilon-1)}{2}\mathbb{E}_i \{ \widehat{p}_{H,t}(i)^2\} + \frac{\varepsilon^2}{2} \mathbb{E}_i[\widehat{p}_{H,t}(i)^2]=1+\frac{\varepsilon}{2}\mathbb{E}_i \{ \widehat{p}_{H,t}(i)^2\}\)

\(\Rightarrow \ \ \displaystyle \int_0^1 \left(\frac{P_{H,t}(i)}{P_{H,t}} \right)^{-\varepsilon} di=1+\frac{\varepsilon}{2}\mathbb{V}_i\{p_{H,t}(i)\}\)

\(\displaystyle z_t= \log \int_0^1 \left(\frac{P_{H,t}(i)}{P_{H,t}} \right)^{-\varepsilon} di = \log \left( 1+\frac{\varepsilon}{2}\mathbb{V}_i\{p_{H,t}(i)\} \right)= \frac{\varepsilon}{2}\mathbb{V}_i\{p_{H,t}(i)\} +o(||a||^3)\)

Rewriting the second-order approximation disutility of labor:

\(\displaystyle \frac{N_t^{1+\varphi}}{1+\varphi} = \frac{\overline{N}^{1+\varphi}}{1+\varphi} +\overline{N}^{1+\varphi}\left[ x_t+z_t + \frac{1}{2}(1+\varphi)(x_t+z_t)^2 \right] + o(||a||^3)\)

\(\displaystyle \frac{N_t^{1+\varphi}}{1+\varphi} = \frac{\overline{N}^{1+\varphi}}{1+\varphi} +\overline{N}^{1+\varphi}\left[ x_t+z_t + \frac{1}{2}(1+\varphi)(x_t^2+2 x_tz_t + z_t^2) \right] + o(||a||^3)\)

As \(z_t\) is a variance (second-order term), \(z_t^2\) and \(x_t z_t\)
are terms of greater order, so they can be included in the remanescent
terms \(o(||a||^3)\). Thus, we get

\(\displaystyle \frac{N_t^{1+\varphi}}{1+\varphi} = \frac{\overline{N}^{1+\varphi}}{1+\varphi} +\overline{N}^{1+\varphi}\left[ x_t+z_t + \frac{1}{2}(1+\varphi)x_t^2 \right] + o(||a||^3)\).

Under the optimal subsidy assumption, from the consumer's FOC, we have
that \(\overline{N}_t^{1+\varphi}=(1-\alpha)\) (constant employment).
Thus,

\(\displaystyle U(C_t,N_t)\equiv \frac{C_t^{1-\sigma}}{1-\sigma}-\frac{N_t^{1+\varphi}}{1+\varphi}= \frac{C_t^{1-\sigma}}{1-\sigma}- \frac{\overline{N}^{1+\varphi}}{1+\varphi} -\overline{N}^{1+\varphi}\left[ x_t+z_t + \frac{1}{2}(1+\varphi)x_t^2 \right] + o(||a||^3)\)

\(\displaystyle U(C_t,N_t) = - \frac{(1-\alpha)}{1+\varphi}-(1-\alpha)\left[ x_t+z_t + \frac{1}{2}(1+\varphi)x_t^2 \right] + \frac{C_t^{1-\sigma}}{1-\sigma}+ o(||a||^3)\)

\(\displaystyle U(C_t,N_t) = -(1-\alpha)\left[ x_t+z_t + \frac{1}{2}(1+\varphi)x_t^2 \right] + \frac{C_t^{1-\sigma}}{1-\sigma}- \frac{(1-\alpha)}{1+\varphi} o(||a||^3)= -(1-\alpha)\left[ z_t + \frac{1}{2}(1+\varphi)x_t^2 \right] + t.i.p+ o(||a||^3)\),
there t.i.p. denotes terms independent of policy and under the optimal
policy, \(x_t=0\).

\(\displaystyle \mathbb{W}= \sum_{i=0}^{\infty} \beta^tU(C_t,N_t) = \sum_{i=0}^{\infty} \beta^t \left[-(1-\alpha)\left( z_t + \frac{1}{2}(1+\varphi)x_t^2 \right) + t.i.p+ o(||a||^3) \right]\)

\(\displaystyle \mathbb{W}= \sum_{i=0}^{\infty} \beta^t \left[-(1-\alpha)\left( \frac{\varepsilon}{2}\mathbb{V}_i[p_{H,t}(i)] +o(||a||^3) + \frac{1}{2}(1+\varphi)x_t^2 \right) \right]+ t.i.p+ o(||a||^3)\)

Taking out the constant (\(t.i.p\)), which are the terms independent of
monetary policy and \(o(||a||^3)\), which is a third order term, we
achieve the desired welfare loss equation:

\(\displaystyle \mathbb{W}= \sum_{i=0}^{\infty} \beta^t \left[-(1-\alpha)\left( \frac{\varepsilon}{2}\mathbb{V}_i[p_{H,t}(i)] + \frac{1}{2}(1+\varphi)x_t^2 \right) \right]\).

Lemma 2 from Woodford(2003):
\(\displaystyle \sum_{t=0}^\infty \beta^t \mathbb{V}_i[p_{H,t}(i)]= \frac{1}{\lambda} \sum_{t=0}^\infty \beta^t \pi_{H,t}^2\),
where
\(\displaystyle \lambda \equiv \frac{(1-\theta)(1-\beta \theta)}{\theta}\)

Using Lemma 2, we have:

\(\displaystyle \mathbb{W}= -\frac{(1-\alpha)}{2}\sum_{i=0}^{\infty} \beta^t \left[ \frac{\varepsilon}{\lambda}\pi_{H,t}^2+ (1+\varphi)x_t^2 \right] + t.i.p + o(||a||^3)\)

To arrive at Lemma 2, we can calculate the price variance:

\(\mathbb{V}_i=\mathbb{E}_i[(p_{H,t}(i)-\mathbb{E}_i[p_{H,t-1}(i)])^2]-(\mathbb{E}_i[p_{H,t}(i)]-\mathbb{E}_i[p_{H,t-1}(i)])^2\)

Because only an exogenous draw of \(1-\theta\) firms are able to reset
their price:

\(\mathbb{E}_i[(p_{H,t}(i)-\mathbb{E}_i[p_{H,t-1}(i)])^2]=\theta\mathbb{E}_i[(p_{H,t-1}(i)-\mathbb{E}_i[p_{H,t-1}(i)])^2] + (1-\theta)\mathbb{E}_i[(p_{H,t}^*-\mathbb{E}_i[p_{H,t-1}(i)])^2]\)

The expected price of good \(i\) produced at home country is

\(\mathbb{E}_i[(p_{H,t}(i)]=(1-\theta)p_{H,t}^*+ \theta \mathbb{E}_i[(p_{H,t-1}(i)]\).

Solving for \(p_t^*\), we have

\(\displaystyle (1-\theta)p_{H,t}^* = \mathbb{E}_i[(p_{H,t}(i)]- \theta \mathbb{E}_i[(p_{H,t-1}(i)] \ \ \Rightarrow \ \ p_{H,t}^* = \frac{1}{1-\theta}\mathbb{E}_i[(p_{H,t}(i)]- \frac{\theta}{1-\theta} \mathbb{E}_i[(p_{H,t-1}(i)]\)

Substituting \(p_{H,t}^*\) in the above equation, we have

\(\displaystyle \mathbb{E}_i[(p_{H,t}(i)-\mathbb{E}_i[p_{H,t-1}(i)])^2]=\)

\(\displaystyle =\theta \mathbb{E}_i[(p_{H,t-1}(i)-\mathbb{E}_i[p_{H,t-1}(i)])^2] + (1-\theta)\mathbb{E}_i\left[ \left( \left[ \frac{1}{1-\theta}\mathbb{E}_i[(p_{H,t}(i)]- \frac{\theta}{1-\theta} \mathbb{E}_i[(p_{H,t-1}(i)] \right]-\mathbb{E}_i[p_{H,t-1}(i)]\right)^2 \right]\)

\(\displaystyle = \theta \mathbb{E}_i[(p_{H,t-1}(i)-\mathbb{E}_i[p_{H,t-1}(i)])^2] + (1-\theta)\mathbb{E}_i\left[ \left( \frac{1}{1-\theta}\mathbb{E}_i[(p_{H,t}(i)]- \frac{1}{1-\theta} \mathbb{E}_i[(p_{H,t-1}(i)]\right)^2 \right]\)

\(\displaystyle \mathbb{E}_i[(p_{H,t}(i)-\mathbb{E}_i[p_{H,t-1}(i)])^2] = \theta \mathbb{E}_i[(p_{H,t-1}(i)-\mathbb{E}_i[p_{H,t-1}(i)])^2] + \frac{1}{1-\theta}\left( \mathbb{E}_i[(p_{H,t}(i)]- \mathbb{E}_i[(p_{H,t-1}(i)]\right)^2\)

\(\displaystyle \mathbb{V}_i[p_{H,t}(i)] = \mathbb{E}_i[(p_{H,t}(i)-\mathbb{E}_i[p_{H,t-1}(i)])^2]-(\mathbb{E}_i[p_{H,t}(i)]-\mathbb{E}_i[p_{H,t-1}(i)])^2\)

Substituting the expression above into the variance, we get

\(\displaystyle \mathbb{V}_i[p_{H,t}(i)] = \theta \mathbb{E}_i[(p_{H,t-1}(i)-\mathbb{E}_i[p_{H,t-1}(i)])^2] + \frac{1}{1-\theta}\left( \mathbb{E}_i[(p_{H,t}(i)]- \mathbb{E}_i[(p_{H,t-1}(i)]\right)^2 - (\mathbb{E}_i[p_{H,t}(i)]-\mathbb{E}_i[p_{H,t-1}(i)])^2\)

\(\displaystyle \mathbb{V}_i = \theta \mathbb{E}_i[(p_{H,t-1}(i)-\mathbb{E}_i[p_{H,t-1}(i)])^2] + \frac{\theta}{1-\theta}\left( \mathbb{E}_i[(p_{H,t}(i)]- \mathbb{E}_i[(p_{H,t-1}(i)]\right)^2 \approx \theta \mathbb{V}_i[p_{H,t-1}] + \frac{\theta}{1-\theta} \pi_{H,t}^2\)

\(\displaystyle \mathbb{V}_i[p_{H,t}(i)] \approx \theta \mathbb{V}_i[p_{H,t-1}] + \frac{\theta}{1-\theta} \pi_{H,t}^2\)

Using the approximation above and Iterating backwards, we have:

\(\displaystyle \mathbb{V}_i[p_{H,t}(i)]= \frac{\theta}{1-\theta} \pi_t^2+\theta \left\{\frac{\theta}{1-\theta} \pi_{t-1}^2+\theta \left[ \frac{\theta}{1-\theta} \pi_{t-2}^2+\theta \left( \frac{\theta}{1-\theta} \pi_{t-3}^2+\theta \mathbb{V}_i[p_{H,t-4}(i)] \right) \right] \right\}\)

\(\displaystyle \mathbb{V}_i[p_{H,t}(i)]= \theta^{t+1}\mathbb{V}_i[p_{H,-1}(i)]+ \sum_{s=0}^{t} \theta^s \frac{\theta}{1-\theta} \pi_{H,t-s}^2\)

Taking the discounted value of \(\mathbb{V}_i[p_{H,t}(i)]\) over all
periods, as \(t \to \infty\), the first term of equation above converges
to zero almost surely as \(\theta < 1\). Then we have,

\(\displaystyle \sum_{t=0}^{\infty} \beta^t \mathbb{V}_i[p_{H,t}(i)] = \sum_{t=0}^{\infty} \beta^t \sum_{t=0}^\infty\theta^t \frac{\theta}{1-\theta} \pi_{H,t}^2 = \frac{\theta}{1-\theta} \sum_{t=0}^{\infty} \sum_{t=0}^{\infty} \beta^{t} (\beta \theta)^t \pi_{H,t}^2 = \frac{\theta}{(1-\theta)(1-\beta \theta)}\sum_{t=0}^{\infty}\beta^t\pi_{H,t}^2 = \frac{1}{\lambda}\sum_{t=0}^{\infty}\beta^t\pi_{H,t}^2\)

\vspace{12pt}

\hypertarget{equation-41---variances-of-inflation-and-output-gap}{%
\subsection{Equation (41) - Variances of inflation and output
gap}\label{equation-41---variances-of-inflation-and-output-gap}}

\(\displaystyle \mathbb{V} = -\frac{(1-\alpha)}{2}\sum_{i=0}^{\infty} \beta^t \left[ \frac{\varepsilon}{\lambda}\mathbb{V}(\pi_{H,t})+ (1+\varphi)\mathbb{V}(x_t) \right]\)

\vspace{8pt}

Now
\(\displaystyle \mathbb{E}_t \left[ -\frac{(1-\alpha)}{2}\sum_{i=0}^{\infty} \beta^t \left[ \frac{\varepsilon}{\lambda}\pi_{H,t}^2+ (1+\varphi)x_t^2 \right] + t.i.p+ o(||a||^3) \right] = -\frac{(1-\alpha)}{2}\sum_{i=0}^{\infty} \beta^t \left[ \frac{\varepsilon}{\lambda}\mathbb{V}(\pi_{H,t})+ (1+\varphi)\mathbb{V}(x_t) \right]\)

as \(E[x_t]=E[\pi_t]=0\)

If \(\beta \to 1\), any policy deviation in a period can be calculated
as

\(\displaystyle \mathbb{V} = -\frac{(1-\alpha)}{2}\sum_{i=0}^{\infty} \beta^t \left[ \frac{\varepsilon}{\lambda}\mathbb{V}(\pi_{H,t})+ (1+\varphi)\mathbb{V}(x_t) \right]\)

\hypertarget{appendix-e---optimal-policy-for-the-modification}{%
\subsection{Appendix E - Optimal policy for the
modification}\label{appendix-e---optimal-policy-for-the-modification}}

\(\displaystyle \mathbb{W}= -\frac{(1-\alpha)}{2}\sum_{i=0}^{\infty} \beta^t \left[ \frac{\varepsilon}{\lambda}\pi_{H,t}^2+\frac{\zeta}{\Lambda}\pi_{w,t}^2+ (1+\varphi)x_t^2 \right] + t.i.p+ o(||a||^3)\)

\vspace{8pt}

Here we'll derive the loss function with wage rigidity, adapted from
Rhee. Then, considering the limit case, we get the original paper
formula. By the Taylor's rule, the second order approximation is:

\(\displaystyle Y_t = Y+y_t+\frac{1}{2} \left( y_t \right)^2+o(||a||^n) \ \ \Rightarrow \ \ \displaystyle \frac{Y_t-Y}{Y}= y_t+\frac{y_t^2}{2} +o(||a||^n) \approx y_t+\frac{y_t^2}{2}\),
there \(a\) is the bound for the high order terms.

Let's consider and additive welfare function, defined as the sum of the
utility of all households:

\(\displaystyle \mathbb{W}=\int_0^1 U_t(\ell)d\ell = \int_0^1 E_0\sum_{t=0}^\infty \beta^t U(C_t(\ell),N_t(\ell))d\ell = E_0 \sum_{t=0}^{\infty} \beta^t \int_0^1 U(C_t(\ell),N_t(\ell))d\ell=E_0 \sum_{t=0}^{\infty} \beta^t \int_0^1 \left(\log C_t(\ell)-\frac{N_t(\ell)}{1+\varphi}\right)d\ell\)

In the extended model, as the demanded work for each type of labor is
different (as they are inside a continuum of measure 1), the consumption
will also be different and will depend on the total income. To make the
problem tractable, we will suppose that the government makes a lump-sum
transfer for each household to compensate the income differences so
everyone consume the same basket and the only source of heterogeneity is
the disutility of labor. We also consider the special parametrization
\(\sigma=\gamma=\eta=1\)

With flexible salary prices, there's no monopolistic competition in the
labor market. Thus, we have \(\displaystyle U_C(C_t,N_t)=\frac{1}{C_t}\)
and \(\displaystyle U_N(C_t,N_t)=-N_t^\varphi\). Substituting in the
FOCs relation, with flexible prices we have

\(\displaystyle \overline{N}_t^\varphi \overline{C}_t= (1-\alpha)\frac{\overline{C}_t}{\overline{N}_t} \ \ \Rightarrow \ \ \overline{N}_t^{\varphi+1} = \overline{N}^{1+\varphi}(1-\alpha) \ \ \Rightarrow \ \ \overline{N}=(1-\alpha)^{\frac{1}{1+\varphi}}\)
which is a constant employment.

From log-linearized equations in the special case, we have

\(\displaystyle c_t=c_t^*+(1-\alpha)s_t \ \ \Rightarrow \ \ s_t=\frac{c_t-c_t^*}{1-\alpha}\)

\(\displaystyle y_t=c_t+\alpha s_t=c_t+ \alpha \frac{c_t-c_t^*}{1-\alpha} \ \ \Rightarrow \ \ (1-\alpha) y_t=(1-\alpha)c_t+\alpha c_t-\alpha c_t^* \ \ \Rightarrow \ \ c_t=(1-\alpha)y_t+\alpha y_t^*\)

Back to our welfare function

\(\displaystyle \mathbb{W} \approx E_0 \sum_{t=0}^{\infty} \beta^t \left[ (1-\alpha)y_t+\alpha y_t^* -\int_0^1 \left(\frac{(N_t(\ell)-\overline{N}_t)^{\varphi+1}}{\varphi+1} \right)d\ell \right] \approx E_0 \sum_{t=0}^{\infty} \beta^t \left[ (1-\alpha)y_t+\alpha y_t^* -\int_0^1 \left(\frac{\overline{N}_t^{\varphi+1}+(1+\varphi)N_t(\ell)\overline{N}_t^\varphi}{1+\varphi} \right)d\ell \right]\)

\(\displaystyle \mathbb{W} = E_0 \sum_{t=0}^{\infty} \beta^t \left[ (1-\alpha)y_t+\alpha y_t^* - \frac{\overline{N}_t^{\varphi+1}}{1+\varphi}\int_0^1 \left(1+(1+\varphi)\frac{N_t(\ell)}{\overline{N}_t} \right)d\ell \right]\)

\(\displaystyle \mathbb{W} \approx E_0 \sum_{t=0}^{\infty} \beta^t \left[ (1-\alpha)y_t+\alpha y_t^* - \frac{1-\alpha}{1+\varphi}- (1-\alpha)\int_0^1 \left(\widehat{n}_t(\ell) + \frac{1+\varphi}{2} \widehat{n}_t(\ell)^2 \right)d\ell \right]\)

\(\displaystyle \mathbb{W} \approx E_0 \sum_{t=0}^{\infty} \beta^t \left[ (1-\alpha)y_t- (1-\alpha)\widehat{n}_t - \frac{(1-\alpha)(1+\varphi)}{2}\int_0^1 \widehat{n}_t(\ell)^2 d\ell + t.i.p\right]\),

where \(t.i.p\) contains the terms independent of policy, which we
remove from the function, as they are constant terms from the point of
view of the policy maker.

Using equation proposed in the modification
\(\displaystyle N_t(j,\ell)=\left(\frac{W_t(\ell)}{W_t} \right)^{-\zeta} N_t(j)\),
where
\(\displaystyle W_t \equiv \left(\int_0^1 N_t(j,\ell)^{\frac{\zeta-1}{\zeta}}d\ell\right)^{\frac{\zeta}{1-\eta}}\)

and summing the labor for all j firms, we have
\(\displaystyle N_t(j,\ell)=\left(\frac{ W_t(\ell)}{W_t}\right)^\zeta N_t(j) \ \ \Rightarrow \ \ \int_0^1N_t(j,\ell)dj= \int_0^1\left(\frac{ W_t(\ell)}{W_t}\right)^\zeta N_t(j) dj \ \ \Rightarrow \ \ N_t(\ell)=\left(\frac{ W_t(\ell)}{W_t}\right)^\zeta N_t\)

Taking the log and subtracting the same equation in the steady-state, we
have

\(\displaystyle n_t(\ell)=\zeta(w_t(\ell)-w_t) + n_t \ \ \Rightarrow \ \  \widehat{n}_t(\ell)=\zeta(\widehat{w}_t(\ell)-\widehat{w}_t) + n_t \ \ \Rightarrow \ \ \widehat{n}_t(\ell) - n_t=\zeta(\widehat{w}_t(\ell)-\widehat{w}_t)\)

\(\displaystyle \int_0^1 \widehat{n}_t(j)^2 dj = \int_0^1 (\zeta(\widehat{w}_t(\ell)-\widehat{w}_t)+\widehat{n}_t)^2 dj = \zeta^2\int_0^1(\widehat{w}_t(\ell)-\widehat{w}_t)^2d\ell + 2 \zeta \int_0^1 (\widehat{w}_t(\ell)-\widehat{w}_t) d\ell + \int_0^1 \widehat{n}_t^2 d\ell = \widehat{n}_t^2+ \zeta^2 \mathbb{V}[w_t(\ell)]\)

using the fact that
\(\int_0^1 (\widehat{w}_t(\ell)-\widehat{w}_t) d\ell=0\) and
\(\int_0^1 (\widehat{w}_t(\ell)-\widehat{w}_t)^2 d\ell=\mathbb{V}[w_t(\ell)]\)

Now we need to write the employment in terms of output and prices:

\(\displaystyle N_t=\int_0^1 \int_0^1 N_t(j,\ell)d\ell dj = \int_0^1N_t(j) \int_0^1 \frac{N_t(j,\ell)}{N_t(\ell)} d\ell dj = \Delta_{w,t} \int_0^1N_t(j)dj\)

where
\(\Delta_{w,t} \equiv \int_0^1 \frac{N_t(j,\ell)}{N_t(\ell)}d\ell = \int_0^1 \left(\frac{W_t(\ell)}{W_t} \right)^{-\zeta}d\ell\).

Now we'll derive the demand curve for \(Y_t(j)\). Good market clearing
in the representative small open economy (``home'') requires that the
total produced inside the small economy is consumed either by the
households of this economy or imported (from the small economy,
subscript H) from households of any other country (superscript i).

\(\displaystyle Y_t(j) = C_{H,t}(j)+\int_0^1 C_{H,t}^i(j)di = \left( \frac{P_{H,t}(j)}{P_{H,t}}\right)^{-\varepsilon}C_{H,t} + \int_0^1 \displaystyle \left( \frac{P_{H,t}^i(j)}{P_{H,t}^i}\right)^{-\varepsilon}C_{H,t}^i di\)

\(\displaystyle Y_t(j) = \left( \frac{P_{H,t}(j)}{P_{H,t}}\right)^{-\varepsilon} (1-\alpha)\left( \frac{P_{H,t}}{P_t} \right)^{-\eta}C_t + \int_0^1 \left( \frac{P_{H,t}^i(j)}{P_{H,t}^i}\right)^{-\varepsilon} \left( \frac{P_{H,t}^i}{P_{F,t}^i}\right)^{-\gamma}C_{F,t}^i di\)

\(\displaystyle Y_t(j) = \left( \frac{P_{H,t}(j)}{P_{H,t}}\right)^{-\varepsilon} (1-\alpha)\left( \frac{P_{H,t}}{P_t} \right)^{-\eta}C_t + \int_0^1 \left( \frac{P_{H,t}^i(j)}{P_{H,t}^i}\right)^{-\varepsilon} \left( \frac{P_{H,t}}{\mathcal{E}_{i,t}P_{F,t}^i}\right)^{-\gamma} \alpha \left( \frac{P_{F,t}^i}{P_t^i} \right)^{-\eta} C_t^i di\)

Assuming symmetric preferences across countries, the home country will
sell the variety of good j for the same price, independently of which
country is buying, which implies
\(\displaystyle \frac{P_{H,t}^i (j)}{P_{H,t}^i}=\frac{P_{H,t} (j)}{P_{H,t}}\)

Thus,
\(\displaystyle Y_t(j) = \left( \frac{P_{H,t}(j)}{P_{H,t}}\right)^{-\varepsilon} \left[ (1-\alpha)\left( \frac{P_{H,t}}{P_t} \right)^{-\eta}C_t + \alpha \int_0^1 \left( \frac{P_{H,t}}{\mathcal{E}_{i,t}P_{F,t}^i}\right)^{-\gamma} \left( \frac{P_{F,t}^i}{P_t^i} \right)^{-\eta} C_t^i di \right] = \left( \frac{P_{H,t}(j)}{P_{H,t}}\right)^{-\varepsilon}Y_t\)

Back to the employment expression, using the above expression and using
\(Y_t(j)=A_tN_t(j)\) and

\(\displaystyle N_t \equiv \left(\int_0^1 N_t(j,\ell)^{\frac{\zeta-1}{\zeta}} d\ell \right)^{\frac{\zeta}{\zeta-1}}\)

in the modification proposed, we have

\(\displaystyle N_t= \Delta_{w,t} \int_0^1 \frac{Y_t(j)}{A_t}dj= \Delta_{w,t} \frac{Y_t}{A_t}\int_0^1 \frac{Y_t(j)}{Y_t}dj = \Delta_{w,t} \frac{Y_t}{A_t}\int_0^1 \frac{Y_t(j)}{Y_t}dj=\Delta_{w,t} \Delta_{P_H,t}\frac{Y_t}{A_t}\)

where
\(\Delta_{P_H,t}=\int_0^1 \left( \frac{P_{H,t}(j)}{P_{H,t}}\right)^{-\varepsilon}dj\)

Log-linearizing around the steady-state (the potential, or natural
product), where there are no differences between firms and households we
have:

\(\widehat{n}_t=y_t-\overline{y}_t+d_{w,t}+d_{P_H,t}\), where
\(d_{w,t}=\log \Delta_{w,t}\) and \(d_{P_H,t}=\log \Delta_{P_H,t}\)

Lemma 1:
\(\displaystyle d_{P_H,t}=\frac{\varepsilon}{2}\mathbb{V}_j[P_{H,t}(j)]\).
The proof is in the appendix D of the original paper.

Lemma 2:
\(\displaystyle d_{p_H,t}=\frac{\zeta}{2}\mathbb{V}_\ell[w_t(\ell)]\).
Proof is Erced et al.

Lemma 3: (proof in Woodford, chapter 6)

\(\displaystyle \sum_{i=0}^{\infty} \beta^t \mathbb{V}_j[p_{H,t}(j)]=\frac{\theta}{(1-\beta\theta)(1-\theta)}\sum_{i=0}^\infty \beta^t \pi_{H,t}^2\)

\(\displaystyle \sum_{i=0}^{\infty} \beta^t \mathbb{V}_\ell[w_t(\ell)]=\frac{\varsigma}{(1-\beta \varsigma)(1-\varsigma)}\sum_{i=0}^\infty \beta^t \pi_{w,t}^2\)

Back to the welfare function, we have

\(\displaystyle \mathbb{W} = E_0 \sum_{t=0}^{\infty} \beta^t \left[ (1-\alpha)y_t- (1-\alpha)(y_t-\overline{y}_t+d_{w,t}+d_{P_H,t}) - \frac{(1-\alpha)(1+\varphi)}{2} (\widehat{n}_t^2+ \zeta^2 \mathbb{V}_\ell[w_t(\ell)]) + t.i.p\right]\)

\(\displaystyle \mathbb{W} = E_0 \sum_{t=0}^{\infty} \beta^t \left[ - (1-\alpha)\left(\frac{\varepsilon}{2}\mathbb{V}_j[p_{H,t}(j)]+ \frac{\zeta}{2}\mathbb{V}_\ell[w_t(\ell)]\right) - \frac{(1-\alpha)(1+\varphi)}{2} (\widehat{n}_t^2+ \zeta^2 \mathbb{V}_\ell[w_t(\ell)]) + t.i.p\right]\)

\(\displaystyle \mathbb{W} = - \frac{1-\alpha}{2} E_0 \sum_{t=0}^{\infty} \beta^t \left[\varepsilon\mathbb{V}_j[p_{H,t}(j)]+ \zeta\mathbb{V}_\ell[w_t(\ell)] + (1+\varphi) (x_t^2+ \zeta^2 \mathbb{V}_\ell[w_t(\ell)])\right] + t.i.p\)

\(\displaystyle \mathbb{W} = - \frac{1-\alpha}{2} E_0 \sum_{t=0}^{\infty} \beta^t \left\{(1+\varphi) x_t^2 + \varepsilon\mathbb{V}_j[p_{H,t}(j)] + [\zeta(1+\varphi \zeta)] \mathbb{V}_\ell[w_t(\ell)]\right\} + t.i.p\),

where \(\zeta^2\mathbb{V}_\ell[w_t(\ell)]\) is a term independent of
policy.

Using both equations from lemma 3, we have

\(\displaystyle \mathbb{W} = - \frac{1-\alpha}{2} E_0 \sum_{t=0}^{\infty} \beta^t \left\{(1+\varphi) x_t^2 + \varepsilon\frac{\theta}{(1-\beta\theta)(1-\theta)}\pi_{H,t}^2 + \zeta(1+\varphi \zeta) \frac{\varsigma}{(1-\beta\varsigma)(1-\varsigma)} \pi_{w,t}^2\right\} + t.i.p\),

\(\displaystyle \mathbb{W} = - \frac{1-\alpha}{2} E_0 \sum_{t=0}^{\infty} \beta^t \left\{(1+\varphi) x_t^2 + \frac{\varepsilon}{\lambda}\pi_{H,t}^2 + \frac{\zeta}{\Lambda} \pi_{w,t}^2\right\} + t.i.p\),

where \(\lambda=\frac{\theta}{(1-\beta\theta)(1-\theta)}\) and
\(\Lambda=\frac{\varsigma (1-\beta \varsigma)}{(1-\beta\varsigma)(1-\varsigma)}\)

In the limit case, when \(\Lambda \to \infty\), the third term of the
equation above vanishes and we have the result for the original paper,
with flexible prices in the labor market.

\vspace{12pt}

\end{document}
