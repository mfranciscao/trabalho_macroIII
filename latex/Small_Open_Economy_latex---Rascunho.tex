% Options for packages loaded elsewhere
\PassOptionsToPackage{unicode}{hyperref}
\PassOptionsToPackage{hyphens}{url}
%
\documentclass[
]{article}
\usepackage{amsmath,amssymb}
\usepackage{lmodern}
\usepackage{iftex}
\ifPDFTeX
  \usepackage[T1]{fontenc}
  \usepackage[utf8]{inputenc}
  \usepackage{textcomp} % provide euro and other symbols
\else % if luatex or xetex
  \usepackage{unicode-math}
  \defaultfontfeatures{Scale=MatchLowercase}
  \defaultfontfeatures[\rmfamily]{Ligatures=TeX,Scale=1}
\fi
% Use upquote if available, for straight quotes in verbatim environments
\IfFileExists{upquote.sty}{\usepackage{upquote}}{}
\IfFileExists{microtype.sty}{% use microtype if available
  \usepackage[]{microtype}
  \UseMicrotypeSet[protrusion]{basicmath} % disable protrusion for tt fonts
}{}
\makeatletter
\@ifundefined{KOMAClassName}{% if non-KOMA class
  \IfFileExists{parskip.sty}{%
    \usepackage{parskip}
  }{% else
    \setlength{\parindent}{0pt}
    \setlength{\parskip}{6pt plus 2pt minus 1pt}}
}{% if KOMA class
  \KOMAoptions{parskip=half}}
\makeatother
\usepackage{xcolor}
\usepackage[left=1cm, right=1cm, top=1cm, bottom=1.5cm]{geometry}
\usepackage{graphicx}
\makeatletter
\def\maxwidth{\ifdim\Gin@nat@width>\linewidth\linewidth\else\Gin@nat@width\fi}
\def\maxheight{\ifdim\Gin@nat@height>\textheight\textheight\else\Gin@nat@height\fi}
\makeatother
% Scale images if necessary, so that they will not overflow the page
% margins by default, and it is still possible to overwrite the defaults
% using explicit options in \includegraphics[width, height, ...]{}
\setkeys{Gin}{width=\maxwidth,height=\maxheight,keepaspectratio}
% Set default figure placement to htbp
\makeatletter
\def\fps@figure{htbp}
\makeatother
\setlength{\emergencystretch}{3em} % prevent overfull lines
\providecommand{\tightlist}{%
  \setlength{\itemsep}{0pt}\setlength{\parskip}{0pt}}
\setcounter{secnumdepth}{-\maxdimen} % remove section numbering
\usepackage{bbm}
\usepackage{amsmath}
\ifLuaTeX
  \usepackage{selnolig}  % disable illegal ligatures
\fi
\IfFileExists{bookmark.sty}{\usepackage{bookmark}}{\usepackage{hyperref}}
\IfFileExists{xurl.sty}{\usepackage{xurl}}{} % add URL line breaks if available
\urlstyle{same} % disable monospaced font for URLs
\hypersetup{
  pdftitle={Small Open Economy Model},
  pdfauthor={Anna Catarina Tavella e Matheus Franciscão},
  hidelinks,
  pdfcreator={LaTeX via pandoc}}

\title{Small Open Economy Model}
\author{Anna Catarina Tavella e Matheus Franciscão}
\date{2022-xx-xx}

\begin{document}
\maketitle

\hypertarget{problem-of-the-consumer-for-reference}{%
\subsection{1 - Problem of the consumer (for
reference)}\label{problem-of-the-consumer-for-reference}}

\(\displaystyle {max} \ \mathbb{E}_0 \sum_{t=0}^\infty \beta^tU(C_t,N_t) = \displaystyle {max} \ E_0 \sum_{t=0}^\infty \beta^tU \left(\left[ (1-\alpha)^{\frac{1}{\eta}} (C_{H,t})^{\frac{\eta-1}{\eta}} + \alpha^{\frac{1}{\eta}} (C_{F,t})^{\frac{\eta-1}{\eta}} \right]^{\frac{\eta}{\eta-1}},N_t \right)\)

subject to the budget constraint (specified below), where

\(C_{H,t} \equiv \displaystyle \left( \int_0^1 C_{H,t}(j)^{\frac{\varepsilon-1}{\varepsilon}}dj \right) ^{\frac{\varepsilon}{\varepsilon-1}}\),
\(C_{F,t} \equiv \displaystyle \left( \int_0^1 (C_{i,t})^{\frac{\gamma-1}{\gamma}}di \right) ^{\frac{\gamma}{\gamma-1}}\),
\(C_{i,t} \equiv \displaystyle \left( \int_0^1 C_{i,t}(j)^{\frac{\varepsilon-1}{\varepsilon}}dj \right) ^{\frac{\varepsilon}{\varepsilon-1}}\)

and
\(C_t \equiv \left[ (1-\alpha)^{\frac{1}{\eta}} (C_{H,t})^{\frac{\eta-1}{\eta}} + \alpha^{\frac{1}{\eta}} (C_{F,t})^{\frac{\eta-1}{\eta}} \right]^{\frac{\eta}{\eta-1}}\)

Substituting, we get

\(\displaystyle {max} \ \mathbb{E}_0 \sum_{t=0}^\infty \beta^tU \left(\left[ (1-\alpha)^{\frac{1}{\eta}} \left [\displaystyle \left( \int_0^1 C_{H,t}(j)^{\frac{\varepsilon-1}{\varepsilon}}dj \right) ^{\frac{\varepsilon}{\varepsilon-1}} \right]^{\frac{\eta-1}{\eta}} + \alpha^{\frac{1}{\eta}} \left[ \displaystyle \left( \int_0^1 (C_{F,t})^{\frac{\gamma-1}{\gamma}}di \right) ^{\frac{\gamma}{\gamma-1}} \right]^{\frac{\eta-1}{\eta}} \right]^{\frac{\eta}{\eta-1}},N_t \right)\)
=

\(\displaystyle {max} \ \mathbb{E}_0 \sum_{t=0}^\infty \beta^tU \left(\left[ (1-\alpha)^{\frac{1}{\eta}} \left [\displaystyle \left( \int_0^1 C_{H,t}(j)^{\frac{\varepsilon-1}{\varepsilon}}dj \right) ^{\frac{\varepsilon}{\varepsilon-1}} \right]^{\frac{\eta-1}{\eta}} + \alpha^{\frac{1}{\eta}} \left[ \displaystyle \left( \int_0^1 \left( \displaystyle \left( \int_0^1 C_{i,t}(j)^{\frac{\varepsilon-1}{\varepsilon}}dj \right) ^{\frac{\varepsilon}{\varepsilon-1}} \right)^{\frac{\gamma-1}{\gamma}}di \right) ^{\frac{\gamma}{\gamma-1}} \right]^{\frac{\eta-1}{\eta}} \right]^{\frac{\eta}{\eta-1}},N_t \right)\)

subject to the budget constraint:

\(\displaystyle \int_0^1 P_{H,t}(j)C_{H,t}(j)dj + \int_0^1\int_0^1 P_{i,t}(j)C_{i,t}(j)dj\ di + \mathbb{E}_t\{ Q_{t,t+1}D_{t+1}\} \leq D_t + W_tN_t + Tt\)

\(\mathcal{L} = \displaystyle \mathbb{E}_t \sum_{t=0}^\infty \beta^t U \left(\left[ (1-\alpha)^{\frac{1}{\eta}} \left [\displaystyle \left( \int_0^1 C_{H,t}(j)^{\frac{\varepsilon-1}{\varepsilon}}dj \right) ^{\frac{\varepsilon}{\varepsilon-1}} \right]^{\frac{\eta-1}{\eta}} + \alpha^{\frac{1}{\eta}} \left[ \displaystyle \left( \int_0^1 \left( \displaystyle \left( \int_0^1 C_{i,t}(j)^{\frac{\varepsilon-1}{\varepsilon}}dj \right) ^{\frac{\varepsilon}{\varepsilon-1}} \right)^{\frac{\gamma-1}{\gamma}}di \right) ^{\frac{\gamma}{\gamma-1}} \right]^{\frac{\eta-1}{\eta}} \right]^{\frac{\eta}{\eta-1}},N_t \right)\)

\(+ \lambda_t \left(D_t + W_tN_t + Tt \displaystyle - \int_0^1 P_{H,t}(j)C_{H,t}(j)dj - \int_0^1\int_0^1 P_{i,t}(j)C_{i,t}(j)dj\ di - \mathbb{E}_t\{ Q_{t,t+1}D_{t+1}\}\right)\)

\hypertarget{finding-the-demand-function-for-each-specific-good}{%
\subsection{2 - Finding the demand function for each specific
good}\label{finding-the-demand-function-for-each-specific-good}}

\(\displaystyle C_{H,t}(j)= \left( \frac{P_{H,t}(j)}{P_{H,t}}\right)^{-\varepsilon}C_{H,t}\),
\(\displaystyle C_{i,t}(j)= \left( \frac{P_{i,t}(j)}{P_{i,t}}\right)^{-\varepsilon}C_{i,t}\)
and
\(\displaystyle C_{i,t}= \left( \frac{P_{i,t}}{P_{F,t}}\right)^{-\gamma}C_{F,t}\)

It's easier by calculating the MRS (marginal rate of substitution)
between \(C_{H,t}(j)\) and \(C_{H,t}\), as by the optimal allocation, it
has to be the rate of prices in every period of time (otherwise the
consumer could by a little less of the product with relative higher
price and buy another with relative lower price, increasing his
utility).

\(\displaystyle \frac{\partial U(C_t,N_t)}{\partial C_{H,t}(j)} = U_c(C_t,N_t)\frac{\eta}{1-\eta}\left( C_t^{\frac{\eta-1}{\eta}} \right)^{\frac{\eta}{\eta-1}-1} (1-\alpha)^{\frac{1}{\eta}}\frac{\eta-1}{\eta}\left( C_{H,t}\right)^{-\frac{1}{\eta}}\frac{\varepsilon}{\varepsilon-1}\left(C_{H,t}^{\frac{\varepsilon-1}{\varepsilon}} \right)^{\frac{\varepsilon}{\varepsilon-1}-1}\int_0^1 \frac{\varepsilon-1}{\varepsilon} C_{H,t}(j)^{-\frac{1}{\varepsilon}}dj\)

After simplifying, we get

\(\displaystyle \frac{\partial U(C_t,N_t)}{\partial C_{H,t}(j)} = U_c(C_t,N_t) (1-\alpha)^{\frac{1}{\eta}} C_t^{\frac{1}{\eta}}\left( C_{H,t}\right)^{-\frac{1}{\eta}}C_{H,t}^{\frac{1}{\varepsilon}} \int_0^1 C_{H,t}(j)^{-\frac{1}{\varepsilon}}dj = U_c(C_t,N_t) \left[ (1-\alpha) \frac{C_t}{C_{H,t}} \right]^{\frac{1}{\eta}} \int_0^1 \left[\frac{C_{H,t}}{C_{H,t}(j)}\right]^{\frac{1}{\varepsilon}}dj\)

Similarly,

\(\displaystyle \frac{\partial U(C_t,N_t)}{\partial C_{H,t}} = U_c(C_t,N_t)\frac{\eta}{1-\eta}\left( C_t^{\frac{\eta-1}{\eta}} \right)^{\frac{\eta}{\eta-1}-1} (1-\alpha)^{\frac{1}{\eta}}\frac{\eta-1}{\eta}\left( C_{H,t}\right)^{-\frac{1}{\eta}} = U_c(C_t,N_t) (1-\alpha)^{\frac{1}{\eta}} C_t^{\frac{1}{\eta}}\left( C_{H,t}\right)^{-\frac{1}{\eta}} = U_c(C_t,N_t) \left[ (1-\alpha) \frac{C_t}{C_{H,t}} \right]^{\frac{1}{\eta}}\)

\(\displaystyle \frac{\displaystyle \frac{\partial U(C_t,N_t)}{\displaystyle \partial C_{H,t}(j)}}{\frac{\displaystyle \partial U(C_t,N_t)}{\displaystyle \partial C_{H,t}}} = \frac{\displaystyle U_c(C_t,N_t) \left[ (1-\alpha) \frac{C_t}{C_{H,t}} \right]^{\frac{1}{\eta}} \int_0^1 \left[\frac{C_{H,t}}{C_{H,t}(j)}\right]^{\frac{1}{\varepsilon}}dj}{\displaystyle U_c(C_t,N_t) \left[ (1-\alpha) \frac{C_t}{C_{H,t}} \right]^{\frac{1}{\eta}}} = \frac{\displaystyle \int_0^1P_{H,t}(j)dj}{P_{H,t}}\)

After simplifying again, the expression is almost the demand function we
want.

\(\displaystyle \int_0^1 \left[\frac{C_{H,t}}{C_{H,t}(j)}\right]^{\frac{1}{\varepsilon}}dj = \displaystyle \int_0^1 \frac{P_{H,t}(j)}{P_{H,t}}dj\)

As the interval of both integrals are the same and the variable being
integrated is also the same, what is inside the integral in both sides
have also to be the same. So,

\(\displaystyle \left[\frac{C_{H,t}}{C_{H,t}(j)}\right]^{\frac{1}{\varepsilon}} = \frac{P_{H,t}(j)}{P_{H,t}} \Rightarrow \left[\frac{C_{H,t}(j)}{C_{H,t}}\right]^{-\frac{1}{\varepsilon}} = \frac{P_{H,t}(j)}{P_{H,t}} \Rightarrow C_{H,t}(j)= \left[ \frac{P_{H,t}(j)}{P_{H,t}} \right]^{-\varepsilon}C_{H,t}\)

Calculating now the MRS (marginal rate of substitution) between
\(C_{i,t}(j)\) and \(C_{i,t}\), which is also equal the rate of the
prices.

\(\displaystyle \frac{\partial U(C_t,N_t)}{\partial C_{i,t}(j)} =\)

\(\displaystyle U_c(C_t,N_t)\frac{\eta}{1-\eta}\left( C_t^{\frac{\eta-1}{\eta}} \right)^{\frac{\eta}{\eta-1}-1} \alpha^{\frac{1}{\eta}}\frac{\eta-1}{\eta}\left( C_{F,t}\right)^{-\frac{1}{\eta}}\frac{\gamma}{\gamma-1}\left(C_{H,t}^{\frac{\gamma-1}{\gamma}} \right)^{\frac{\gamma}{\gamma-1}-1}\int_0^1 \frac{\gamma-1}{\gamma} C_{i,t}^{-\frac{1}{\gamma}}\left[ \frac{\varepsilon}{\varepsilon-1}\left(C_{i,t}^{\frac{\varepsilon-1}{\varepsilon}} \right)^{\frac{\varepsilon}{\varepsilon-1}-1}\int_0^1 \frac{\varepsilon-1}{\varepsilon} C_{i,t}(j)^{-\frac{1}{\varepsilon}}dj \right]di\)

After simplifying, we get

\(\displaystyle \frac{\partial U(C_t,N_t)}{\partial C_{i,t}(j)} = U_c(C_t,N_t) \alpha^{\frac{1}{\eta}} C_t^{\frac{1}{\eta}}\left( C_{F,t}\right)^{-\frac{1}{\eta}}C_{F,t}^{\frac{1}{\gamma}} \int_0^1 C_{i,t}^{-\frac{1}{\gamma}} \left[ C_{i,t}^\frac{1}{\varepsilon} \int_0^1 C_{i,t}(j)^{-\frac{1}{\varepsilon}}dj \right] di\)

\(\displaystyle \frac{\partial U(C_t,N_t)}{\partial C_{i,t}(j)} = U_c(C_t,N_t) \left[ \alpha \frac{C_t}{C_{F,t}} \right]^{\frac{1}{\eta}} \int_0^1 \left[\frac{C_{F,t}}{C_{i,t}}\right]^{\frac{1}{\gamma}} \int_0^1 \left[\frac{C_{i,t}}{C_{i,t}(j)}\right]^{\frac{1}{\varepsilon}}dj \ di\)

\(\displaystyle \frac{\partial U(C_t,N_t)}{\partial C_{i,t}} = U_c(C_t,N_t)\frac{\eta}{1-\eta}\left( C_t^{\frac{\eta-1}{\eta}} \right)^{\frac{\eta}{\eta-1}-1} \alpha^{\frac{1}{\eta}}\frac{\eta-1}{\eta}\left( C_{F,t}\right)^{-\frac{1}{\eta}}\frac{\gamma}{\gamma-1}\left(C_{H,t}^{\frac{\gamma-1}{\gamma}} \right)^{\frac{\gamma}{\gamma-1}-1}\int_0^1 \frac{\gamma-1}{\gamma} C_{i,t}^{-\frac{1}{\gamma}}di\)

\(\displaystyle \frac{\partial U(C_t,N_t)}{\partial C_{i,t}} = U_c(C_t,N_t) \left[ \alpha \frac{C_t}{C_{F,t}} \right]^{\frac{1}{\eta}} \int_0^1 \left[\frac{C_{F,t}}{C_{i,t}}\right]^{\frac{1}{\gamma}} di\)

Calculating the MRS we have

\(\displaystyle \frac{\displaystyle \frac{\partial U(C_t,N_t)}{\displaystyle \partial C_{i,t}(j)}}{\frac{\displaystyle \partial U(C_t,N_t)}{\displaystyle \partial C_{i,t}}} = \frac{\displaystyle U_c(C_t,N_t) \left[ \alpha \frac{C_t}{C_{F,t}} \right]^{\frac{1}{\eta}} \int_0^1 \left[\frac{C_{F,t}}{C_{i,t}}\right]^{\frac{1}{\gamma}} \int_0^1 \left[\frac{C_{i,t}}{C_{i,t}(j)}\right]^{\frac{1}{\varepsilon}}dj \ di }{\displaystyle U_c(C_t,N_t) \left[ \alpha \frac{C_t}{C_{F,t}} \right]^{\frac{1}{\eta}} \int_0^1 \left[\frac{C_{F,t}}{C_{i,t}}\right]^{\frac{1}{\gamma}} di} = \frac{\displaystyle \int_0^1P_{i,t}(j)dj}{P_{i,t}}\)

As before, we can simplify again. Also, as there's a continuum of firms,
we can consider that the price of each product \(P_{i,t}(j)\) is only
correlated with its specific demand function and not with the demand
function of other in its country or another country, it follows that
each specific price is uncorrelated with \(C_{F,t}\) and \(C_{i,t}\).
Also, as each firm is very small, we can consider that it has negligible
influence on the aggregate index price of its country (\(P_{i,t}\)).
Whith these independence assumption, the joint distribution is equal to
the product of the marginal distributions.

\(\displaystyle \int_0^1 \int_0^1 \left[\frac{C_{F,t}}{C_{i,t}}\right]^{\frac{1}{\gamma}} \left[\frac{C_{i,t}}{C_{i,t}(j)}\right]^{\frac{1}{\varepsilon}}dj \ di = \int_0^1 \left[\frac{C_{F,t}}{C_{i,t}}\right]^{\frac{1}{\gamma}} di \int_0^1 \frac{\displaystyle P_{i,t}(j)}{P_{i,t}}dj = \int_0^1 \int_0^1 \frac{\displaystyle P_{i,t}(j)}{P_{i,t}} \left[\frac{C_{F,t}}{C_{i,t}}\right]^{\frac{1}{\gamma}} dj \ di\)

Now, as before, the integrand in both sides needs to be the same. The we
get the second demand equation.

\(\displaystyle \left[\frac{C_{F,t}}{C_{i,t}}\right]^{\frac{1}{\gamma}} \left[\frac{C_{i,t}}{C_{i,t}(j)}\right]^{\frac{1}{\varepsilon}} = \frac{\displaystyle P_{i,t}(j)}{P_{i,t}} \left[\frac{C_{F,t}}{C_{i,t}}\right]^{\frac{1}{\gamma}} \Rightarrow \left[\frac{C_{i,t}(j)}{C_{i,t}}\right]^{-\frac{1}{\varepsilon}} = \frac{\displaystyle P_{i,t}(j)}{P_{i,t}} \Rightarrow C_{i,t}(j) = \left[ \frac{\displaystyle P_{i,t}(j)}{P_{i,t}} \right]^{-\varepsilon}C_{i,t}\)

To find the aggregate demand for each country, in terms of total foreign
demand, we proceed by calculating the MRS between the aggregate
consumption for the country and the aggregate consumption of foreign
goods, which the optimal allocation resulting from the rate between the
prices, as before.

\(\displaystyle \frac{\partial U(C_t,N_t)}{\partial C_{F,t}} = U_c(C_t,N_t)\frac{\eta}{1-\eta}\left( C_t^{\frac{\eta-1}{\eta}} \right)^{\frac{\eta}{\eta-1}-1} \alpha^{\frac{1}{\eta}}\frac{\eta-1}{\eta}\left( C_{F,t}\right)^{-\frac{1}{\eta}} = U_c(C_t,N_t) \left[ \alpha \frac{C_t}{C_{F,t}} \right]^{\frac{1}{\eta}}\)

\(\displaystyle \frac{\displaystyle \frac{\partial U(C_t,N_t)}{\displaystyle \partial C_{i,t}}}{\frac{\displaystyle \partial U(C_t,N_t)}{\displaystyle \partial C_{F,t}}} = \frac{\displaystyle U_c(C_t,N_t) \left[ \alpha \frac{C_t}{C_{F,t}} \right]^{\frac{1}{\eta}} \int_0^1 \left[\frac{C_{F,t}}{C_{i,t}}\right]^{\frac{1}{\gamma}} di }{\displaystyle U_c(C_t,N_t) \left[ \alpha \frac{C_t}{C_{F,t}} \right]^{\frac{1}{\eta}} } = \frac{\displaystyle \int_0^1P_{i,t}di}{P_{F,t}}\)

As \(P_{F,t}\) doesn't depend on a specific i, we can put it inside the
integral. Then we get again two integrans which have to be the same for
the equality to hold.

\(\displaystyle \left[\frac{C_{F,t}}{C_{i,t}}\right]^{\frac{1}{\gamma}} = \frac{P_{i,t}}{P_{F,t}} \Rightarrow \left[\frac{C_{i,t}}{C_{F,t}}\right]^{-\frac{1}{\gamma}} = \frac{P_{i,t}}{P_{F,t}} \ \ \Rightarrow \ \ C_{i,t} = \left[ \frac{P_{i,t}}{P_{F,t}} \right]^{-\gamma}C_{F,t}\)

\hypertarget{aggregating-the-expenditure}{%
\subsection{3 - Aggregating the
expenditure}\label{aggregating-the-expenditure}}

Now that we have the demand functions

\(\displaystyle C_{H,t}(j)= \left( \frac{P_{H,t}(j)}{P_{H,t}}\right)^{-\varepsilon}C_{H,t}\),
\(\displaystyle C_{i,t}(j)= \left( \frac{P_{i,t}(j)}{P_{i,t}}\right)^{-\varepsilon}C_{i,t}\)
and
\(\displaystyle C_{i,t}= \left( \frac{P_{i,t}}{P_{F,t}}\right)^{-\gamma}C_{F,t}\)

let's prove that

\(\displaystyle \int_0^1 P_{H,t}(j)C_{H,t}(j)dj = P_{H,t}C_{H,t}\) ,
\(\displaystyle \int_0^1 P_{i,t}(j)C_{i,t}(j)dj = P_{i,t}C_{i,t}\) and
\(\displaystyle \int_0^1 P_{i,t}C_{i,t}dj = P_{F,t}C_{F,t}\)

using the definition of the price indexes:

\(\displaystyle P_{H,t} \equiv \left( \int_0^1 P_{H,t}(j)^{1-\varepsilon}dj \right)^{\frac{1}{1-\varepsilon}}\),
\(\displaystyle P_{i,t} \equiv \left( \int_0^1 P_{i,t}(j)^{1-\varepsilon}dj \right)^{\frac{1}{1-\varepsilon}}\)
and
\(\displaystyle P_{F,t} \equiv \left( \int_0^1 P_{i,t}^{1-\gamma}di \right)^{\frac{1}{1-\gamma}}\)

\(\displaystyle \int_0^1 P_{H,t}(j)C_{H,t}(j)dj = \int_0^1P_{H,t}(j)\left( \frac{P_{H,t}(j)}{P_{H,t}}\right)^{-\varepsilon}C_{H,t}dj = \frac{C_{H,t}}{P_{H,t}^{-\varepsilon}}\int_0^1P_{H,t}(j)^{1-\varepsilon}dj = \frac{C_{H,t}}{P_{H,t}^{-\varepsilon}}P_{H,t}^{1-\varepsilon} = P_{H,t}C_{H,t}\)

\(\displaystyle \int_0^1 P_{i,t}(j)C_{i,t}(j)dj = \int_0^1P_{i,t}(j)\left( \frac{P_{i,t}(j)}{P_{i,t}}\right)^{-\varepsilon}C_{i,t}dj = \frac{C_{i,t}}{P_{i,t}^{-\varepsilon}}\int_0^1 P_{i,t}(j)^{1-\varepsilon}dj = \frac{C_{i,t}}{P_{i,t}^{-\varepsilon}}P_{i,t}^{1-\varepsilon} = P_{i,t}C_{i,t}\)

\(\displaystyle \int_0^1 P_{i,t}C_{i,t}di = \int_0^1P_{i,t}\left( \frac{P_{i,t}}{P_{F,t}}\right)^{-\gamma}C_{F,t}di = \frac{C_{F,t}}{P_{F,t}^{-\gamma}}\int_0^1 P_{F,t}^{1-\gamma}dj = \frac{C_{F,t}}{P_{F,t}^{-\varepsilon}}P_{F,t}^{1-\varepsilon} = P_{F,t}C_{F,t}\)

With this aggregation, the budget constraint can be simplified

\(\displaystyle \int_0^1 P_{H,t}(j)C_{H,t}(j)dj + \int_0^1\int_0^1 P_{i,t}(j)C_{i,t}(j)dj\ di + \mathbb{E}_t\{ Q_{t,t+1}D_{t+1}\} \leq D_t + W_tN_t + Tt\)

\(\displaystyle P_{H,t}C_{H,t} + \int_0^1P_{i,t}C_{i,t}di = P_{H,t}C_{H,t} + P_{F,t}C_{F,t} \leq D_t + W_tN_t + Tt - \mathbb{E}_t\{ Q_{t,t+1}D_{t+1}\}\)

As the total consumption expenditure by the representative consumer is
with the domestic produts or foreig products, the budget constraint
becomes:

\(\displaystyle P_t C_t \leq D_t + W_tN_t + T_t - \mathbb{E}_t\{ Q_{t,t+1}D_{t+1}\}\)

\hypertarget{finding-the-optimal-share-between-the-domestic-and-imported-goods}{%
\subsection{4 - Finding the optimal share between the domestic and
imported
goods}\label{finding-the-optimal-share-between-the-domestic-and-imported-goods}}

Now we will calculate the MRS between the domestic products and the
total consumption, which has to be equal to the rate of prices. After,
we will do the same for the foreign products.

\(\displaystyle \frac{\displaystyle \frac{\partial U(C_t,N_t)}{\displaystyle \partial C_{H,t}}}{\frac{\displaystyle \partial U(C_t,N_t)}{\displaystyle \partial C_t}} = \frac{\displaystyle U_c(C_t,N_t) \left[ (1-\alpha) \frac{C_t}{C_{H,t}} \right]^{\frac{1}{\eta}}}{\displaystyle U_c(C_t,N_t) } = \frac{P_{H,t}}{P_t} \ \ \Rightarrow \ \ (1-\alpha) \frac{C_t}{C_{H,t}} = \left( \frac{P_{H,t}}{P_t} \right)^\eta \ \ \Rightarrow \ \ C_{H,t} = (1-\alpha)\left( \frac{P_{H,t}}{P_t} \right)^{-\eta}C_t\)

\(\displaystyle \frac{\displaystyle \frac{\partial U(C_t,N_t)}{\displaystyle \partial C_{F,t}}}{\frac{\displaystyle \partial U(C_t,N_t)}{\displaystyle \partial C_t}} = \frac{\displaystyle U_c(C_t,N_t) \left[ \alpha \frac{C_t}{C_{F,t}} \right]^{\frac{1}{\eta}}}{\displaystyle U_c(C_t,N_t) } = \frac{P_{F,t}}{P_t} \ \ \Rightarrow \ \ \alpha \frac{C_t}{C_{F,t}} = \left( \frac{P_{F,t}}{P_t} \right)^\eta \ \ \Rightarrow \ \ C_{F,t} = \alpha \left( \frac{P_{F,t}}{P_t} \right)^{-\eta}C_t\)

\hypertarget{standard-problem-of-the-representative-consumer}{%
\subsection{5 - Standard problem of the representative
consumer}\label{standard-problem-of-the-representative-consumer}}

Now we arrive at a standard problem for the representative consumer

\(\displaystyle \underset{C_t,N_t,D_{t+1}} {max} \ \mathbb{E}_0 \sum_{t=0}^\infty \beta^t U(C_t,N_t) = \displaystyle \underset{C_t,N_t} {max} \ E_0 \sum_{t=0}^\infty \beta^t \left( \frac{C_t^{1-\sigma}}{1-\sigma}-\frac{N_t^{1+\varphi}}{1+\varphi} \right)\)

subject to
\(\displaystyle D_t + W_tN_t + Tt - \mathbb{E}_t\{ Q_{t,t+1}D_{t+1}\} - P_t C_t = 0\),
as an optimal condition.

using the separable utility function specified as
\(\displaystyle U(C_t,N_t)=\frac{C_t^{1-\sigma}}{1-\sigma}-\frac{N_t^{1+\varphi}}{1+\varphi}\)

\(\mathcal{L} = \displaystyle \mathbb{E}_t \sum_{t=0}^\infty \beta^t \left[ \left( \frac{C_t^{1-\sigma}}{1-\sigma}-\frac{N_t^{1+\varphi}}{1+\varphi} \right) + \lambda_t \left( D_t + W_tN_t + Tt - Q_{t,t+1}D_{t+1} - P_t C_t \right) \right]\)

with first order conditions (FOCs):

(\(C_t\)):
\(\displaystyle \beta^t U_C(C_t,N_t)= \beta^tC_t^{-\sigma} = \beta^t \lambda_t P_t \ \ \Rightarrow \ \ C_t^{-\sigma} = \lambda_t P_t\)

(\(N_t\)):
\(\displaystyle -\beta^t U_n(C_t,N_t)=\beta^tN_t^{\varphi} = \beta^t \lambda_t W_t \ \ \Rightarrow \ \ N_t^{\varphi} = \lambda_t W_t\)

(\(D_{t+1}\)):
\(\displaystyle \beta^t \lambda_t Q_{t,t+1} = \beta^{t+1} \mathbb {E}[\lambda_{t+1}] \ \ \Rightarrow \ \ \frac{\mathbb{E}[\lambda_{t+1}]}{\lambda_t} = \frac{Q_{t,t+1}}{\beta}\)

Dividing (\(N_t\)) FOC by (\(C_t\)) FOC, we have the standard equation
of intratemporal substitution between consumption and leisure
\(\displaystyle -\frac{U_C(C_t,N_t)}{U_N(C_t,N_t)} =C_t^{\sigma}N_t^{\varphi} = \frac{W_t}{P_t}\)

Advancing one period for the consumption FOC, we have
\(\mathbb{E} [C_{t+1}^{-\sigma}] = \mathbb{E}[\lambda_{t+1}P_{t+1}]\)

As the model will be log-linearized and an approximation of the first
order will be used to solve it, we can ignore the Jensen's inequality
where there is an expectation operator. Up to first order approximation,
\(\mathbb{E}[xy] \approx \mathbb{E}[x] \mathbb{E}[y]\).

Dividing the consumption FOC in t+1 by the equation in t and
substituting by the \(\mathbb{E}[\lambda_{t+1}]/\lambda_t\) in the
\(D_{t+1}\) FOC, we get the Euler equation

\(\displaystyle \mathbb{E} \left[ \left( \frac{C_{t+1}}{C_t} \right)^{-\sigma} \right] = \mathbb{E} \left[ \frac{\lambda_{t+1} P_{t+1}}{\lambda_t P_t} \right] \ \ \Rightarrow \ \  \mathbb{E} \left[ \left( \frac{C_{t+1}}{C_t} \right)^{-\sigma} \frac{P_t}{P_{t+1}}\right]= \frac{\mathbb{E}[Q_{t,t+1}]}{\beta} \ \ \Rightarrow \ \ \beta R_t \mathbb{E} \left[ \left( \frac{C_{t+1}}{C_t} \right)^{-\sigma} \frac{P_t}{P_{t+1}}\right]= 1\)

as \(Q_{t,t+1}\), is the price of a a riskless one-period discount bond
in domestic currency with gross return \(R_t\).

Log-linearizing
\(\displaystyle C_t^{\sigma}N_t^{\varphi} = \frac{W_t}{P_t}\) is
straight forward: \(w_t - p_t = \sigma c_t + \varphi n_t\)

To log-linerize the Euler equation, we'll use the Taylor expansion:
\(f(x) \approx f(x_0) + f'(x_o)(x-x_0)\). When expanding the exponential
function around 0, we get \(e^x=e^0+e^0(x-0)=1+x\)

\(\mathbb{E} \left[ \exp \left( \ln \left[ \beta R_t \left( \frac{C_{t+1}}{C_t} \right)^{-\sigma} \frac{P_t}{P_{t+1}} \right] \right) \right] = \mathbb{E} \left[ \exp \left( \ln(\beta) + r_t -\sigma(c_{t+1}-c_t) + p_t - p_{t+1} \right) \right] = 1 \ \ \Rightarrow\)

\(\displaystyle 1 + \ln(\beta) + r_t -\sigma(\mathbb{E} \left[c_{t+1} \right]-c_t) - \mathbb{E} \left[ \pi_{t+1} \right] = 1 \ \ \Rightarrow \ \ \sigma c_t = \sigma \mathbb{E} \left[c_{t+1} \right] + \mathbb{E} \left[ \pi_{t+1} \right] - r_t - \ln(\beta)\)
\(\displaystyle \Rightarrow \ \ c_t = \mathbb{E} \left[c_{t+1} \right] -\frac{1}{\sigma} \left( r_t - \mathbb{E} \left[ \pi_{t+1} \right] -\rho\right)\)
as \(\rho \equiv \frac{1-\beta}{\beta} \approx -\ln(\beta)\)

\hypertarget{terms-of-trade}{%
\subsection{6 - Terms of trade}\label{terms-of-trade}}

Let's log-linearize the expression for the bilateral terms of trade
\(\displaystyle S_t \equiv \frac{P_{F,t}}{P_{H,t}}= \left( \int_0^1 S_{i,t}^{1-\gamma}di \right)^{\frac{1}{1-\gamma}} \ \ \Rightarrow \ \ S_t^{1-\gamma} = \int_0^1 S_{i,t}^{1-\gamma}di\)

\(\displaystyle \exp \left(\ln \left[ S_t^{1-\gamma}\right] \right) = \int_0^1 \exp \left(\ln \left[ S_{i,t}^{1-\gamma}\right] \right)di \ \ \Rightarrow \ \ \exp[(1-\gamma)s_t]=\int_0^1 \exp[(1-\gamma)s_{i,t}]di\)

Applying the exponential Taylor expansion (\(e^x = 1 + x\)) in both
sides, we get
\(\displaystyle 1+(1+\gamma)s_t = 1 + (1+\gamma) \int_0^1 s_{i,t}di \ \ \Rightarrow \ \ s_t = \int_0^1 s_{i,t}di\)

To log-linearize the CPI formula, considering that it is a symmetric
steady-state, we have \(P_{H,t}=P_{F,t}=P_t\). Now, taking logs in both
sides and using the Taylor expansion for a vector of two variables we
have
\(\displaystyle f(x,y) \approx f(x_0,y_0)+\frac{\partial f(x,y)}{\partial x} \biggr|_{x_0,y_0}(x-x_0)+\frac{\partial f(x,y)}{\partial y} \biggr|_{x_0,y_0}(y-y_0)\)
So,
\(\displaystyle (P_t)^{\eta-1} = (1-\alpha)(P_{H,t})^{1-\eta} + \alpha(P_{F,t})^{1-\eta}\)

By the CPI definition, we have
\(\displaystyle P_t \equiv \left[ (1-\alpha)(P_{H,t})^{1-\eta} + \alpha(P_{F,t})^{1-\eta} \right]^{\frac{1}{1-\eta}} \ \ \Rightarrow \ \ (P_t)^{1-\eta} = (1-\alpha)(P_{H,t})^{1-\eta} + \alpha(P_{F,t})^{1-\eta}\)

Taking logs, we have
\((1-\eta)\ln(P_t) = \ln\left[ (1-\alpha)(P_{H,t})^{1-\eta} + \alpha(P_{F,t})^{1-\eta} \right]=f(P_{H,t},P_{F,t})\)

Applying the Taylor expansion on the right side using \(x = P_{H,t}\),
\(y = P_{F,t}\) and \(x_0 = y_0 = P_t\),

\(\displaystyle (1-\eta)p_t \approx \ln \left[ (1-\alpha)(P_t)^{1-\eta} + \alpha(P_t)^{1-\eta} \right]+ \frac{(1-\alpha)(1-\eta)P_t^{-\eta}}{P_t^{1-\eta}}(P_{H,t}-P_t) + \frac{\alpha(1-\eta)P_t^{-\eta}}{P_t^{1-\eta}}(P_{F,t}-P_t)\)

\(\displaystyle (1-\eta)p_t = \ln(P_t^{1-\eta}) + (1-\alpha)(1-\eta)\frac{P_{H,t}-P_t}{P_t} + \alpha(1-\eta)\frac{P_{F,t}-P_t}{P_t}\).

\(p_t \approx p_t + (1-\alpha)[\ln(P_{H,t})-\ln(P_t)] + \alpha[\ln(P_{F,t})-\ln(P_t)] \ \ \Rightarrow \ \  p_t = (1-\alpha)p_{H,t}+\alpha \ p_{F,t}\),
as defined in the paper.

As \(s_t \equiv p_{F,t}-p_{H,t}\),
\(p_t=(1-\alpha)P_{H,t}+\alpha (s_t+p_{H,t}) = p_{H,t} + \alpha s_t\)

The domestic inflation rate is defined as
\(\displaystyle \pi_{H,t} \equiv p_{H,t}-p_{H,t-1}\), taking the
difference between the equation between t and t-1, we have
\(p_t-p_{t-1}=p_{H,t}-p_{H,t-1}+\alpha(s_t-s_{t-1}) \ \ \Rightarrow \ \ \pi_t = \pi_{H,t}+\alpha \Delta s_t\).

Assuming that the law of one price is valid in all times (the same goods
produced in different countries have the same price when converting to
the domestic currency, using the nominal interest rate), we have
\(P_{i,t}(j) = \mathcal{E}_{i,t}P_{i,t}^i(j)\) for all
\(i, j \in [0,1]\).

As
\(P_{i,t}^i \equiv \displaystyle \left( \int_0^1 P_{i,t}^i(j)^{{1-\varepsilon}}dj \right) ^{\frac{1}{1-\varepsilon}}\),
we have
\(\displaystyle \mathcal{E}_{i,t}P_{i,t}^i = \mathcal{E}_{i,t} \left( \int_0^1 P_{i,t}^i(j)^{{1-\varepsilon}}dj \right) ^{\frac{1}{1-\varepsilon}} = \left( (\mathcal{E}_{i,t})^{1-\varepsilon} \int_0^1 P_{i,t}^i(j)^{{1-\varepsilon}}dj \right) ^{\frac{1}{1-\varepsilon}}\)

\(\displaystyle = \left( \int_0^1 \left( \mathcal{E}_{i,t}P_{i,t}^i(j)\right) ^{{1-\varepsilon}}dj \right) ^{\frac{1}{1-\varepsilon}} = \left( \int_0^1 P_{i,t}(j)^{{1-\varepsilon}}dj \right) ^{\frac{1}{1-\varepsilon}} = P_{i,t}\)

Now, we have
\(\displaystyle P_{F,t}= \left( \int_0^1 P_{i,t}^{{1-\gamma}}di \right) ^{\frac{1}{1-\gamma}} = \left( \int_0^1 \left( \mathcal{E}_{i,t}P_{i,t}^i \right)^{{1-\gamma}}di \right)^{\frac{1}{1-\gamma}} \ \ \Rightarrow \ \ P_{F,t}^{1-\gamma} = \int_0^1 \left( \mathcal{E}_{i,t}P_{i,t}^i \right)^{{1-\gamma}}di\)

Log-linearizing the last expression, we get
\(\displaystyle \exp \left[(1-\gamma)\ln P_{F,t} \right] = \int_0^1 \exp \left[(1-\gamma) \ln \left( \mathcal{E}_{i,t}P_{i,t}^i \right) \right]di\)

\(\Rightarrow \ \  \displaystyle 1 + (1-\gamma)p_{F,t} = \int_0^1 \left[ 1+ (1-\gamma)(e_{i,t}+p_{i,t}^i)\right]di \ \ \Rightarrow \ \ p_{F,t} = \int_0^1 (e_{i,t}+p_{i,t}^i)di=e_t+p_t^*\),

where \(\displaystyle e_t \equiv \int_0^1 e_{i,t} \ di\) and
\(\displaystyle p_t^*\equiv \int_0^1 p_{i,t}^i \ di\). Also, we have
that \(s_t = p_{F,t}-p_{H,t}=e_t + p_t^*-p_{H,t}\).

Defining the bilateral real exchange rate
\(\displaystyle \mathcal{Q}_{i,t} \equiv \frac{\mathcal{E}_{i,t}P_{i,t}}{P_t}\)
and the (log) effective real exchange rate
\(\displaystyle q_t \equiv \int_0^1 q_{i,t}di\) we have
\(\displaystyle q_t = \int_0^1 \ln \left( \frac{\mathcal{E}_{i,t}P_{i,t}}{P_t} \right)di= \int_0^1 \left(e_{i,t}+p_{i,t}-p_t \right)di = e_t+p_t^*-p_t = s_t+p_{H,t}-(p_{H,t}+\alpha s_t) = (1-\alpha)s_t\)

\hypertarget{international-risk-sharing}{%
\subsection{7 - International risk
sharing}\label{international-risk-sharing}}

The problem of the representative household in any country is the same,
as the economies are all equal. There is, any country has an Euler
equation like
\(\displaystyle \beta \mathbb{E} \left[ \left( \frac{C_{t+1}}{C_t} \right)^{-\sigma} \frac{P_t}{P_{t+1}}\right]= Q_{t,t+1}\).

The condition for the clearing of the international market is that
\(Q_{t,t+1}\) is unique. So the price converted to a common current has
to be the same. So, for every foreign country, the Euler equation
becomes

\(\displaystyle \beta \mathbb{E} \left[ \left( \frac{C_{t+1}^i}{C_t^i} \right)^{-\sigma} \frac{P_t^i \mathcal{E}_t^i}{P_{t+1}^i \mathcal{E}_{t+1}^i}\right] = Q_{t,t+1}\).

Combining both equations, using the definition of the real exchange rate
\(\displaystyle \mathcal{Q}_{i,t} \equiv \frac{\mathcal{E}_{i,t}P_{i,t}}{P_t}\)
and solving for \(C_t\), we have

\(\displaystyle \beta \mathbb{E} \left[ \left( \frac{C_{t+1}}{C_t} \right)^{-\sigma} \frac{P_t}{P_{t+1}}\right] = \beta \mathbb{E} \left[ \left( \frac{C_{t+1}^i}{C_t^i} \right)^{-\sigma} \frac{P_t^i \mathcal{E}_t^i}{P_{t+1}^i \mathcal{E}_{t+1}^i}\right] \ \ \Rightarrow \ \ \mathbb{E} \left[ \left( \frac{C_{t+1}}{C_t} \right)^{-\sigma} \frac{1}{P_{t+1}}\right] = \mathbb{E} \left[ \left( \frac{C_{t+1}^i}{C_t^i} \right)^{-\sigma} \frac{\mathcal{Q}_{i,t}}{P_{t+1}^i \mathcal{E}_{t+1}^i}\right]\)

\(\displaystyle \Rightarrow \ \ (C_t)^\sigma = \mathbb{E} \left[ {C_{t+1}}^\sigma \left( C_{t+1}^i \right)^{-\sigma} \frac{P_{t+1}}{P_{t+1}^i \mathcal{E}_{t+1}^i}\right]\mathcal{Q}_{i,t}(C_t^i)^\sigma \ \ \Rightarrow \ \ C_t = \mathbb{E} \left[ \frac{C_{t+1}}{C_{t+1}^i} (\mathcal{Q}_{i,t+1})^{-\frac{1}{\sigma}}\right] C_t^i \mathcal{Q}_{i,t}^\frac{1}{\sigma} \ \ \Rightarrow \ \ C_t = \vartheta_i C_t^i \mathcal{Q}_{i,t}^\frac{1}{\sigma}\),

where
\(\displaystyle \vartheta_i = \mathbb{E} \left[ \frac{C_{t+1}}{C_{t+1}^i} (\mathcal{Q}_{i,t+1})^{-\frac{1}{\sigma}}\right]\)
is a constant and generally will depend on initial relative net asset
positions. Assuming identical conditions for all economies, the net
asset position for all of the is zero. In this case,
\(\vartheta_i = \vartheta = 1\) for all i. As the symmetric foresight
steady-state in this condition is shown in the appendix A.

The international market clearing implies that the total goods produced
in a country is consumed by domestically or it's exported. The integral
represents the sum of the demand for products of the economy analysed by
foreign countries. In a case with economies not with measure zero, we
need to exclude the economy analysed from the integral do avoid double
counting.

\(\displaystyle Y=C_H+C_i=(1-\alpha)\left( \frac{P_{H}}{P} \right)^{-\eta}C + \int_0^1 \left( \frac{P_i^i}{P_F} \right)^{-\gamma}C_F di = (1-\alpha)\left( \frac{P_{H}}{P} \right)^{-\eta}C + \alpha \int_0^1 \left( \frac{P_i^i}{P_F^i} \right)^{-\gamma} \left( \frac{P_{F}^i}{P^i} \right)^{-\eta}C^idi\)

\(\displaystyle Y= (1-\alpha)\left( \frac{P_{H}}{P} \right)^{-\eta}C + \alpha \int_0^1 \left( \frac{P_F^i}{P_i^i} \right)^{\gamma} \left( \frac{P_{F}^i}{P^i} \right)^{-\eta}C^idi= (1-\alpha)\left( \frac{P_{H}}{P} \right)^{-\eta}C + \alpha \int_0^1 \left( \frac{\mathcal{E}_i P_F^i }{P_H} \right)^{\gamma} \left( \frac{P_{F}^i}{P^i} \right)^{-\eta}C^idi\),

where \(P_i^i\) is the price in the domestic economy converted to the
currency of country i, or
\(\displaystyle P_i^i = \frac{P_i}{\mathcal{E}_i} = \frac{P_H}{\mathcal{E}_i}\),
as the goods have the same price in the international market, after
converting to the same currency. After simplifying, we have

\(\displaystyle Y = \left( \frac{P_{H}}{P} \right)^{-\eta} \left[ (1-\alpha)C + \alpha \int_0^1 \left( \frac{ \mathcal{E}_i P_F^i }{P_H} \right)^{\gamma-\eta} \left( \frac{\mathcal{E}_i P_i}{P} \right)^\eta C^idi \right] = \left( \frac{P_{H}}{P} \right)^{-\eta} \left[ (1-\alpha)C + \alpha \int_0^1 \left( \frac{ \mathcal{E}_i P_F^i }{P_H} \right)^{\gamma-\eta} Q_i^\eta C^idi \right]\)

Considering that
\(\displaystyle P = \left[ (1-\alpha) \left( P_H \right)^{1-\eta} + \alpha \left( P_F\right)^{1-\eta} \right]^{\frac{1}{1-\eta}}\)
in the steady-state,
\(\displaystyle P^{1-\eta} = (1-\alpha) \left( P_H \right)^{1-\eta} + \alpha \left( P_F \right)^{1-\eta}\),
as \(\displaystyle \mathcal{S}_i \equiv \frac{P_i}{P_H}\). So,

\(\displaystyle \left( \frac{P}{P_H} \right)^{1-\eta} = (1-\alpha) + \alpha \left( \frac{P_F}{P_H}\right)^{1-\eta} = (1-\alpha) + \alpha \mathcal{S}_i^{1-\eta} \ \ \Rightarrow \ \ \frac{P}{P_H}=\left[ (1-\alpha)+\alpha \mathcal{S}^{1-\eta}\right]^{\frac{1}{1-\eta}}= \left[ (1-\alpha) + \alpha \int_0^1(\mathcal{S}_i)^{1-\eta}di \right]^{\frac{1}{1-\eta}} \equiv h(\mathcal{S})\)

Defining
\(\displaystyle \mathcal{S}^i = \frac{\mathcal{E}_i P_F^i}{P_i}\) and
using the fact that
\(\displaystyle C^i=C \mathcal{Q}^{-\frac{1}{\sigma}}\) as
\(\vartheta_i = 1\) in a symmetric steady-state, we have

\(\displaystyle Y = h(\mathcal{S})^{\eta}C \left[ (1-\alpha) + \alpha \int_0^1 \left( \frac{ \mathcal{E}_i P_F^i }{P_i} \frac{P_i}{P_H} \right)^{\gamma-\eta} Q_i^{\eta-\frac{1}{\sigma}} di \right]= h(\mathcal{S})^{\eta}C \left[ (1-\alpha) + \alpha \int_0^1 \left( \mathcal{S}^i \frac{P_F}{P_H} \right)^{\gamma-\eta} Q_i^{\eta-\frac{1}{\sigma}} di \right]\)

As we will work with a first order approximation, the equality below is
valid.

\(\displaystyle Y = h(\mathcal{S})^{\eta}C \left[ (1-\alpha) + \alpha \int_0^1 \left( \mathcal{S}^i \mathcal{S}_i \right)^{\gamma-\eta} Q_i^{\eta-\frac{1}{\sigma}} di \right] = h(\mathcal{S})^{\eta}C \left[ (1-\alpha) + \alpha \int_0^1 ( \mathcal{S}^i)^{\gamma} di \int_0^1 ( \mathcal{S}_i)^{-\eta} di \int_0^1 Q_i^{\eta-\frac{1}{\sigma}} di \right]\)

\(\displaystyle Y = h(\mathcal{S})^{\eta}C \left[ (1-\alpha) + \alpha \mathcal{S}^{-\eta} \int_0^1 \left( \frac{P_F^i}{P_H} \right)^{\gamma} di \int_0^1 \left( \frac{\mathcal{E}_i P_F^i}{P} \right)^{\eta-\frac{1}{\sigma}} di \right]\),

as if
\(\displaystyle \mathcal{S}^{1-\gamma}=\int_0^1 \mathcal{S}^{1-\gamma}di\),
we can substitute variables \(-\eta=1-\gamma\) and we get the result.

\(\displaystyle Y = h(\mathcal{S})^{\eta}C \left[ (1-\alpha) + \alpha \mathcal{S}^{-\eta} \left( \frac{1}{P_H} \right)^{\gamma} \int_0^1 \left( P_F^i \right)^{\gamma} di \int_0^1 \left( \frac{\mathcal{E}_i P_F^i}{P} \right)^{\eta-\frac{1}{\sigma}} di \right]\)

\(\displaystyle Y= h(\mathcal{S})^{\eta}C \left[ (1-\alpha) + \alpha \mathcal{S}^{-\eta} \left( \frac{1}{P_H} \right)^{\gamma} (P^*)^{\gamma} \int_0^1 \left( \frac{\mathcal{E}_i P_F^i}{P_H} \frac{P_H}{P} \right)^{\eta-\frac{1}{\sigma}} di \right]\),

using the fact that
\(\left( P_F^i \right)^{1-\gamma} = \displaystyle \int_0^1 \left( P_i^i \right)^{1-\gamma}di\)
and using P* for the international price index of imported goods.

\(\displaystyle Y = h(\mathcal{S})^{\eta}C \left[ (1-\alpha) + \alpha \mathcal{S}^{-\eta} \left( \frac{P^*}{P_H} \right)^{\gamma} \int_0^1 \left( \frac{\mathcal{S}^i}{h(\mathcal{S})} \right)^{\eta-\frac{1}{\sigma}} di \right] =h(\mathcal{S})^{\eta}C \left[ (1-\alpha) + \alpha \mathcal{S}^{-\eta} \mathcal{S}^\gamma \left( \frac{1}{h(\mathcal{S})} \right)^{\eta-\frac{1}{\sigma}} \int_0^1 ( \mathcal{S}^i )^{\eta-\frac{1}{\sigma}} di \right]\)

\(\displaystyle Y = h(\mathcal{S})^{\eta}C \left[ (1-\alpha) + \alpha \mathcal{S}^{\gamma-\eta} \left( \frac{1}{h(\mathcal{S})} \right)^{\eta-\frac{1}{\sigma}} \mathcal{S} ^{\eta-\frac{1}{\sigma}} \right] = h(\mathcal{S})^{\eta}C \left[ (1-\alpha) + \alpha \mathcal{S}^{\gamma-\eta} \left( \frac{\mathcal{S}}{h(\mathcal{S})} \right)^{\eta-\frac{1}{\sigma}} \right]\),

which yields the result.
\(\displaystyle Y= h(\mathcal{S})^{\eta}C \left[ (1-\alpha) + \alpha \mathcal{S}^{\gamma-\eta} q(\mathcal{S})^{\eta-\frac{1}{\sigma}} \right]\),
where
\(\displaystyle \mathcal{Q}=\frac{\mathcal{S}}{h(\mathcal{S})} \equiv q(\mathcal{S})\)

Substituting \(\displaystyle C = C^*q(\mathcal{S})^\frac{1}{\sigma}\) in
the expression above, we have

\(\displaystyle Y= (1-\alpha)h(\mathcal{S})^{\eta}C + \alpha h(\mathcal{S})^{\eta} \mathcal{S}^{\gamma-\eta} q(\mathcal{S})^{\eta-\frac{1}{\sigma}} = (1-\alpha)h(\mathcal{S})^{\eta} C^*q(\mathcal{S})^\frac{1}{\sigma} + \alpha h(\mathcal{S})^{\eta} \mathcal{S}^{\gamma-\eta} q(\mathcal{S})^{\eta-\frac{1}{\sigma}} C^*q(\mathcal{S})^\frac{1}{\sigma}\)

Imposing market clearing \(C^*=Y^*\), we have

\(\displaystyle Y=\left[ (1-\alpha)h(\mathcal{S})^{\eta} q(\mathcal{S})^\frac{1}{\sigma} + \alpha \mathcal{S}^{\gamma-\eta} h(\mathcal{S})^{\eta} q(\mathcal{S})^{\eta} \right]Y^* = \left[ (1-\alpha)h(\mathcal{S})^{\eta} q(\mathcal{S})^\frac{1}{\sigma} + \alpha \mathcal{S}^{\gamma} h(\mathcal{S})^{-\eta} q(\mathcal{S})^{-\eta} h(\mathcal{S})^{\eta} q(\mathcal{S})^{\eta} \right]Y^*\)

\(\displaystyle Y= \left[ (1-\alpha)h(\mathcal{S})^{\eta} q(\mathcal{S})^\frac{1}{\sigma} + \alpha \mathcal{S}^{\gamma} \right]Y^* = \left[ (1-\alpha)h(\mathcal{S})^{\eta} q(\mathcal{S})^\frac{1}{\sigma} + \alpha q(\mathcal{S})^{\gamma} h(\mathcal{S})^{\gamma} \right]Y^* \equiv v (\mathcal{S})Y^*\),

where \(v (\mathcal{S})>0\), \(v' (\mathcal{S})>0\) and \(v (1)=1\)

The clearing of labour market in steady-state implies (the derivation of
the two equations below are in the firm's equations)

\(\displaystyle C^\sigma \left( \frac{Y}{A} \right)^\varphi= \frac{W}{P}\);

\(\displaystyle\displaystyle C^\sigma \left( \frac{Y}{A} \right)^\varphi= \frac{W}{P}\)
\(\displaystyle MC_t = \frac{W_t(1-\tau)}{P_{H,t} A_t} \ \ \Rightarrow \ \  MC = \frac{W(1-\tau)}{P_{H} A}\)

From the Taylor expansion in the price-setting problem of the firm, we
have

\(\displaystyle \sum_{k=0}^\infty (\beta \theta)^kE_t \left\{ 1-\frac{\varepsilon}{\varepsilon-1} MC \right\}=0 \ \ \Rightarrow \ \ MC=1-\frac{1}{\varepsilon}\)

\(\displaystyle MC = \frac{W(1-\tau)}{P_{H} A} = 1-\frac{1}{\varepsilon} \ \ \Rightarrow \ \ \frac{W}{P}=A \frac{1-\frac{1}{\varepsilon}}{1-\tau} \frac{P_H}{P}=A \frac{1-\frac{1}{\varepsilon}}{1-\tau} \frac{1}{h (\mathcal{S})}\)

\(\displaystyle C^\sigma \left( \frac{Y}{A} \right)^\varphi = A \frac{1-\frac{1}{\varepsilon}}{1-\tau} \frac{1}{h (\mathcal{S})} \ \ \Rightarrow \ \ \left( C^*\mathcal{Q}^\frac{1}{\sigma} \right)^\frac{\sigma}{\varphi} \left( \frac{Y}{A} \right) = \left(A \frac{1-\frac{1}{\varepsilon}}{1-\tau} \frac{1}{h (\mathcal{S})} \right)^\frac{1}{\varphi}\)

\(\displaystyle Y=A^{1+\frac{1}{\varphi}} \left( \frac{1-\frac{1}{\varepsilon}}{(1-\tau)(C^*)^\sigma} \frac{1}{ h (\mathcal{S})\mathcal{Q}} \right)^\frac{1}{\varphi} = A^{\frac{1+\varphi}{\varphi}} \left( \frac{1-\frac{1}{\varepsilon}}{(1-\tau)(Y^*)^\sigma \mathcal{S}} \right)^\frac{1}{\varphi}\)

Substituting
\(\displaystyle Y=Y^*=A^{\frac{1+\varphi}{\sigma+\varphi}} \left( \frac{1-\frac{1}{\varepsilon}}{1-\tau} \right)^{\frac{1}{\sigma+\varphi}}\)
and solving for \(\mathcal{S}\), we have

\(\displaystyle A^{\frac{1+\varphi}{\sigma+\varphi}} \left( \frac{1-\frac{1}{\varepsilon}}{1-\tau} \right)^{\frac{1}{\sigma+\varphi}}=A^{\frac{1+\varphi}{\varphi}} \left( \frac{1-\frac{1}{\varepsilon}}{(1-\tau)(Y^*)^\sigma \mathcal{S}} \right)^\frac{1}{\varphi} \ \ \Rightarrow \ \ A^{\frac{\varphi({1+\varphi})}{\sigma+\varphi}} \left( \frac{1-\frac{1}{\varepsilon}}{1-\tau} \right)^{\frac{\varphi}{\sigma+\varphi}}=A^{1+\varphi} \frac{1-\frac{1}{\varepsilon}}{(1-\tau)(Y^*)^\sigma \mathcal{S}}\)

\(\displaystyle A^{\frac{\varphi+\varphi^2-\sigma-\varphi -\sigma\varphi-\varphi^2}{\sigma+\varphi}} \left( \frac{1-\frac{1}{\varepsilon}}{1-\tau} \right)^{\frac{\varphi-\sigma-\varphi}{\sigma+\varphi}} = \frac{1}{(Y^*)^\sigma \mathcal{S}} \ \ \Rightarrow \ \  A^{\frac{-\sigma(1+\varphi)}{\sigma+\varphi}} \left( \frac{1-\frac{1}{\varepsilon}}{1-\tau} \right)^{\frac{-\sigma}{\sigma+\varphi}} = \frac{1}{\left[A^{\frac{1+\varphi}{\sigma+\varphi}} \left( \frac{1-\frac{1}{\varepsilon}}{1-\tau} \right)^{\frac{1}{\sigma+\varphi}} \right]^\sigma \mathcal{S}}\)

\(\displaystyle A^{-\frac{1+\varphi}{\sigma+\varphi}} \left( \frac{1-\frac{1}{\varepsilon}}{1-\tau} \right)^{-\frac{1}{\sigma+\varphi}} = \frac{1}{\left[A^{\frac{1+\varphi}{\sigma+\varphi}} \left( \frac{1-\frac{1}{\varepsilon}}{1-\tau} \right)^{\frac{1}{\sigma+\varphi}} \right] \mathcal{S}^{\frac{1}{\sigma}}} \ \ \Rightarrow \ \ A^{-(1+\varphi)} \left( \frac{1-\frac{1}{\varepsilon}}{1-\tau} \right)^{-1} = \frac{1}{\left[A^{1+\varphi}\left( \frac{1-\frac{1}{\varepsilon}}{1-\tau} \right) \right] \mathcal{S}^{\frac{\sigma+\varphi}{\sigma}}} \ \ \Rightarrow \ \ \mathcal{S}^{\frac{\sigma+\varphi}{\sigma}}=1\),

which gives the result \(\mathcal{S}=1\), which in turn implies
\(\mathcal{S}_i=1\) for all \(i\) (purchasing parity holds).

\hypertarget{uncovered-interest-parity-and-the-terms-of-trade}{%
\subsection{8 - Uncovered interest parity and the terms of
trade}\label{uncovered-interest-parity-and-the-terms-of-trade}}

As
\(\displaystyle \mathbb{E}\left[ \mathcal{Q}_{t,t+1} R_t^i \frac{\mathcal{E}_{i,t+1} }{\mathcal{E}_{i,t} }\right] = 1\)
and
\(\displaystyle \mathbb{E}\left[ \mathcal{Q}_{t,t+1} R_t \right] = 1\),
we have that

\(\displaystyle \mathbb{E}\left[ \mathcal{Q}_{t,t+1} R_t^i \frac{\mathcal{E}_{i,t+1} }{\mathcal{E}_{i,t} }\right] = \displaystyle \mathbb{E}\left[ \mathcal{Q}_{t,t+1} R_t \right] \ \ \Rightarrow \ \  \mathbb{E}\left[ \mathcal{Q}_{t,t+1} \left(R_t-R_t^i \left[ \frac{\mathcal{E}_{i,t+1} }{\mathcal{E}_{i,t} } \right] \right) \right] = 0\)

But to log-linearize it's better to do both sides separately.

\(\displaystyle \mathbb{E} \left[\exp \left( \ln \left[ \mathcal{Q}_{t,t+1} R_t^i \frac{\mathcal{E}_{i,t+1} }{\mathcal{E}_{i,t} }\right] \right) \right] \approx\)

\(\displaystyle \mathbb{E} \left[ 1 + \ln\frac{Q R^i \mathcal{E}_{i}}{\mathcal{E}_{i}}+\frac{1}{\mathcal{Q}R^i}R^i \frac{ \mathcal{E}_{i}}{\mathcal{E}_{i}}(\mathcal{Q}_{t,t+1}-\mathcal{Q}) + \frac{1}{QR^i}\mathcal{Q} \frac{ \mathcal{E}_{i}}{\mathcal{E}_{i}}(R_t^i-R^i) - \frac{1}{\mathcal{Q}R^i}\mathcal{Q} \frac{R^i}{\mathcal{E}_{i}}(\mathcal{E}_{i,t+1}-\mathcal{E}_i) + \frac{1}{\mathcal{Q}R^i}\mathcal{Q} \frac{R^i \mathcal{E}_{i}}{\mathcal{E}_{i}^2}(\mathcal{E}_{i,t}-\mathcal{E}_i) \right]\)

\(\displaystyle = 1+ \ln(\mathcal{Q}R^i) + \mathbb{E} \left[ \hat{q_t}+\hat{r_t}^i - \hat{e}_{i,t+1}+\hat{e}_{i,t}\right]\)

\(\displaystyle \mathbb{E} \left[\exp \left( \ln \left[ \mathcal{Q}_{t,t+1} R_t \right] \right) \right] \approx \mathbb{E} \left[ 1 + \ln(\mathcal{Q}R) +\frac{1}{\mathcal{Q}R}R (\mathcal{Q}_{t,t+1}-Q) +\frac{1}{\mathcal{Q}R} \mathcal{Q} (R_t-R)\right]\)

\(\displaystyle 1+ \ln(\mathcal{Q}R^i) + \mathbb{E} \left[ \hat{q}_t+\hat{r}_t^i - \hat{e}_{i,t+1} + \hat{e}_{i,t}\right] = 1 + \ln(\mathcal{Q}R) + \mathbb{E} \left[ \hat{q}_t + \hat{r}_t \right] \ \ \Rightarrow \ \ \hat{r}^i - \mathbb{E} \left[ e_{i,t+1}-e_{i,t}\right] = \hat{r} \ \ \Rightarrow \ \ r_t^i-r_t=\mathbb{E}_t[\Delta e_{i,t+1}]\)

The aggregation comes from the FOC. The uncovered interest rate parity
allow households to invest both in domestic and foreign assets:
\(B_t, B_t^*\). The budget constraint can be written as

\(P_t + Q_{t,t+1}D_{t+1}+Q_{t,t+1}^*\mathcal{E}_{t+1}D_{t+1}^* \leq D_t+\mathcal{E}_{t}D_t^*+W_tN_t+T_t\)

an the lagrangean becomes

\(\mathcal{L} = \displaystyle \mathbb{E}_t \sum_{t=0}^\infty \beta^t \left[ \left( \frac{C_t^{1-\sigma}}{1-\sigma}-\frac{N_t^{1+\varphi}}{1+\varphi} \right) + \lambda_t \left(\mathcal{E}_t D_t^* + W_tN_t + T_t - Q_{t,t+1}^*\mathcal{E}_t D_{t+1}^* - Q_{t,t+1}D_{t+1} - P_t C_t \right) \right]\)

The FOCs are:

\((C_t)\)
\(\displaystyle C_t^{-\sigma} = \lambda_tP_t \ \ \Rightarrow \ \ \mathbb{E}[C_{t+1}^{-\sigma}] = \mathbb{E}[\lambda_{t+1}P_{t+1}] \ \ \Rightarrow \ \ \displaystyle \mathbb{E}\left[ \frac{\lambda_{t+1}}{\lambda_t}\right] = \mathbb{E}\left[ \left( \frac{C_{t+1}}{C_t} \right)^{-\sigma}\frac{P_t}{P_{t+1}}\right]\)

\((D_{t+1}^*)\)
\(\beta \ \mathbb{E}[\lambda_{t+1} \mathcal{E}_{t+1}] = \lambda_t Q_{t,t+1}^* \mathcal{E}_t \ \ \Rightarrow \ \ \displaystyle \mathbb{E}\left[ \frac{\lambda_{t+1}}{\lambda_t}\right] = \mathbb{E}\left[ \frac{Q_{t,t+1}^*\mathcal{E}_t}{\beta \mathcal{E}_{t+1}}\right]\)

\((D_{t+1}^*)\)
\(\beta \ \mathbb{E}[\lambda_{t+1} ] = \lambda_t Q_{t,t+1}\ \ \Rightarrow \ \ \displaystyle \mathbb{E}\left[ \frac{\lambda_{t+1}}{\lambda_t}\right] = \mathbb{E}\left[ \frac{Q_{t,t+1}}{\beta }\right]\)

\((N_t)\) \(N_t^\varphi=\lambda_t W_t\)

Combining the consumption FOC and the foreign bond's FOC we have

\(\displaystyle \mathbb{E}\left[ \left( \frac{C_{t+1}}{C_t} \right)^{-\sigma}\frac{P_t}{P_{t+1}}\right] = \mathbb{E}\left[ \frac{Q_{t,t+1}^*\mathcal{E}_t}{\beta \mathcal{E}_{t+1}}\right] \ \ \Rightarrow \ \ \beta \  \displaystyle \mathbb{E}\left[ \frac{1}{Q_{t,t+1}^*} \left( \frac{C_{t+1}}{C_t} \right)^{-\sigma}\frac{P_t}{P_{t+1}} \frac{\mathcal{E}_{t+1}}{\mathcal{E}_{t}} \right]=1\)

Doing the same steps for the domestic bonds, we have

\(\displaystyle \mathbb{E}\left[ \left( \frac{C_{t+1}}{C_t} \right)^{-\sigma}\frac{P_t}{P_{t+1}}\right] = \mathbb{E}\left[ \frac{Q_{t,t+1}}{\beta }\right] \ \ \Rightarrow \ \ \beta \  \displaystyle \mathbb{E}\left[ \frac{1}{Q_{t,t+1}} \left( \frac{C_{t+1}}{C_t} \right)^{-\sigma}\frac{P_t}{P_{t+1}} \right]=1\)

Dividing the equation for foreign bonds by the equation for domestic
bonds, we have

\(\displaystyle \mathbb{E} \left[ \frac{Q_{t,t+1}}{Q_{t,t+1}^*} \frac{\mathcal{E}_{t+1}}{\mathcal{E}_{t}} \right]=1 \ \ \Rightarrow \ \ \frac{Q_{t,t+1}}{Q_{t,t+1}^*} = \mathbb{E} \left[ \frac{\mathcal{E}_{t+1}}{\mathcal{E}_{t}} \right] \ \ \Rightarrow \ \ \ln(Q_{t,t+1})-\ln(Q_{t,t+1}^*) \approx \mathbb{E}[\ln{\mathcal{E}_{t+1} }-\ln \mathcal{E}_{t}] \ \ \Rightarrow \ \ r_t-r_t^*=\mathbb{E}[\Delta e_{t+1}]\)

From the definition of the log terms of trade, we have
\(s_t = e_t + p_t^* - p_{H,t}\) and
\(\mathbb{E}[s_{t+1}] = \mathbb{E}[e_{t+1} + p_{t+1}^* - p_{H,t+1}]\).
Subtracting the first by the second, we get

\(s_t - \mathbb{E}[s_{t+1}] = e_t + p_t^* - p_{H,t} - \mathbb{E}[e_{t+1}] - \mathbb{E}[p_{t+1}^*] - \mathbb{E}[p_{H,t+1}] \ \ \Rightarrow \ \ s_t = \mathbb{E}[s_{t+1}]-\mathbb{E}[\Delta e_{t+1}] -\mathbb{E}[\pi_{t+1}^*]-\mathbb{E}[\pi_{H,t+1}]=r_t^*-\mathbb{E}[\pi_{t+1}^*] - (r_t-\mathbb{E}[\pi_{H,t+1}])+\mathbb{E}[s_{t+1}]\)

\hypertarget{firms}{%
\subsection{9 Firms}\label{firms}}

\(\displaystyle N_t \equiv \int_0^1 N_t(j)dj = \int_0^1\frac{Y_t(j)}{A_t}dj=\int_0^1\frac{Y_t(j)}{A_t}\frac{Y_t}{Y_t}dj=\int_0^1\frac{Y_t(j)}{Y_t}dj\frac{Y_t}{A_t}=\frac{Y_t Z_t}{A_t}\),
where \(\displaystyle Z_t \equiv \int_0^1\frac{Y_t(j)}{Y_t}dj\)

The marginal cost is the same for all firms because of constant returns
to scale. In this case, the quantity produced is determined by the cost
function. For each additional worker, the firm has to pay \(W_t\) and
produces \(\displaystyle \frac{Y_t(j)}{A_t}\) and receives \(P_t\) by
product.

So, abstracting from the subsidy, the marginal cost is
\(\displaystyle MC_t = \frac{W_t}{P_{H,t} A_t}\) If we include the
subsidy, it's as the firm pays a smaller salary. So, the marginal cost
becomes \(\displaystyle MC_t = \frac{W_t(1-\tau)}{P_{H,t} A_t}\)

Combining the consumption FOC and the labor FOC int he steady-state, we
have

\(\displaystyle \frac{C^{-\sigma}}{P} = \lambda = N^{\varphi}=\left( \frac{Y}{A} \right)^\varphi\ \ \Rightarrow \ \ C^\sigma \left( \frac{Y}{A} \right)^\varphi=\frac{W}{P}\)

Also, we know that
\(\displaystyle Y_t^{\frac{\varepsilon-1}{\varepsilon}} = \int_0^1Y_t(j)^{\frac{\varepsilon-1}{\varepsilon}}dj\),
so
\(\displaystyle Y_{t+k}^{\frac{\varepsilon-1}{\varepsilon}} = \int_0^1Y_{t+k}(j)^{\frac{\varepsilon-1}{\varepsilon}}dj\)
and

\(\displaystyle Y_{t+k}(j) \leq \left( \frac{\bar{P}_{H,t}}{P_{H,t+k}} \right)^{-\varepsilon} \left( C_{H,t+k}+\int_0^1C_{H,t+k}^i di\right)\)

Then maximizing, the restriction above is binding (otherwise part of the
production would have been wasted, which contradicts the hypothesis of
optimization when the costs are not null). The total demand is

\(\displaystyle Y_{t+k}^{\frac{\varepsilon-1}{\varepsilon}} = \int_0^1 \left[ \left( \frac{\bar{P}_{H,t}}{P_{H,t+k}} \right)^{-\varepsilon} \left( C_{H,t+k}+\int_0^1C_{H,t+k}^i di\right) \right]^{\frac{\varepsilon-1}{\varepsilon}} dj\)

As nothing in the RHS of the above expression depends on (j), we can
take everything out of the integral.

\(\displaystyle Y_{t+k}^{\frac{\varepsilon-1}{\varepsilon}} = \left[ \left( \frac{\bar{P}_{H,t}}{P_{H,t+k}} \right)^{-\varepsilon} \left( C_{H,t+k}+\int_0^1C_{H,t+k}^i di\right) \right]^{\frac{\varepsilon-1}{\varepsilon}} \int_0^1 dj\)
and

\(\displaystyle Y_{t+k} = \left( \frac{\bar{P}_{H,t}}{P_{H,t+k}} \right)^{-\varepsilon} \left( C_{H,t+k}+\int_0^1C_{H,t+k}^i di\right)\)

The problem of the firm (price setting) is:

\[\displaystyle \underset{\bar{P}_{H,t}} \max \sum_{k=0}^\infty \theta^kE_t \left\{\beta^k \left( \frac{C_{H,t+k}}{C_{H,t}} \right)^{-\sigma} \left( \frac{P_{H,t}}{P_{H,t+k}} \right) \left[ \left( \frac{\bar{P}_{H,t}}{P_{H,t+k}} \right)^{-\varepsilon} \left( C_{H,t+k}+\int_0^1C_{H,t+k}^i di\right) (\bar{P}_{H,t}-MC_{t+k}^n) \right] \right\}\]

\(\displaystyle \underset{\bar{P}_{H,t}} \max \sum_{k=0}^\infty \theta^kE_t \left\{\beta^k \left( \frac{C_{H,t+k}}{C_{H,t}} \right)^{-\sigma} \left( C_{H,t+k}+\int_0^1C_{H,t+k}^i di\right) \left( \frac{P_{H,t}}{P_{H,t+k}} \right) \left[\frac{\bar{P}_{H,t}^{1-\varepsilon}}{P_{H,t+k}^{-\varepsilon}} -\left( \frac{\bar{P}_{H,t}}{P_{H,t+k}} \right)^{-\varepsilon} MC_{t+k}^n) \right] \right\}\)

, which yields, after deriving

\(\displaystyle \underset{\bar{P}_{H,t}} \max \sum_{k=0}^\infty \theta^kE_t \left\{\beta^k \left( \frac{C_{H,t+k}}{C_{H,t}} \right)^{-\sigma} \left( C_{H,t+k}+\int_0^1C_{H,t+k}^i di\right) \left( \frac{P_{H,t}}{P_{H,t+k}} \right) \left[(1-\varepsilon)\frac{\bar{P}_{H,t}^{-\varepsilon}}{P_{H,t+k}^{-\varepsilon}} +\varepsilon \frac{\bar{P}_{H,t}^{-\varepsilon-1}}{P_{H,t+k}^{-\varepsilon}} MC_{t+k}^n) \right] \right\}=0\)

Under flexible prices,

\(\displaystyle (1-\varepsilon)+\varepsilon\frac{MC_{t}^n}{\bar{P}_{H,t}}=0 \ \ \Rightarrow \ \ \bar{P}_{H,t} =\frac{\varepsilon}{\varepsilon-1}MC_{t}^n\)

From appendix B,
\(\displaystyle \underset{\bar{P}_{H,t}} \max \sum_{k=0}^\infty \theta^kE_t \left\{ Q_{t,t+k} Y_{t+k} \left[ \bar{P}_{H,t}-MC_{t+k}^n\right] \right\}\)
subject to the demand constraints

\(\displaystyle Y_{t+k} \leq \left( \frac{\bar{P}_{H,t}}{P_{H,t+k}} \right)^{-\varepsilon} \left( C_{H,t+k}+\int_0^1 C_{H,t+k}^idi\right)\)

Because there's monopolistic competition, the demand depends on the
price set. Substituting the restriction in the maximization problem we
have

\[\displaystyle \underset{\bar{P}_{H,t}} \max \sum_{k=0}^\infty \theta^kE_t \left\{ Q_{t,t+k} \left[   \left( \frac{\bar{P}_{H,t}(j)}{P_{H,t+k}(j)} \right)^{-\varepsilon} \left( C_{H,t+k}+\int_0^1 C_{H,t+k}^idi\right) \right] \left[ \bar{P}_{H,t}-MC_{t+k}^n\right]\right\}\]

Calculating the CPO, we have:

\(\displaystyle \sum_{k=0}^\infty \theta^kE_t \Bigg\{ Q_{t,t+k} \Bigg[ (-\varepsilon) \left( \frac{\bar{P}_{H,t}^{-\varepsilon-1} }{P_{H,t+k}^{-\varepsilon} } \right)\left( C_{H,t+k}+\int_0^1 C_{H,t+k}^idi \right) \left[ \bar{P}_{H,t}-MC_{t+k}^n\right] \Bigg] + Y_{t+k} \Bigg\} = 0\)

\(\displaystyle \sum_{k=0}^\infty \theta^kE_t \Bigg\{ Q_{t,t+k} \Bigg[ \frac{(-\varepsilon)}{\bar{P}_{H,t}} \left( \frac{\bar{P}_{H,t}}{P_{H,t+k}} \right)^{-\varepsilon} \left( C_{H,t+k}+\int_0^1 C_{H,t+k}^idi \right) \left[ \bar{P}_{H,t}-MC_{t+k}^n\right] \Bigg] + Y_{t+k} \Bigg\} = 0\)

\(\displaystyle \sum_{k=0}^\infty \theta^kE_t \Bigg\{ Q_{t,t+k} \Bigg[ \frac{(-\varepsilon)}{\bar{P}_{H,t}} Y_{t+k } \left[ \bar{P}_{H,t}-MC_{t+k}^n\right] \Bigg] + Y_{t+k} \Bigg\} = 0\)

As \(\varepsilon\) and \(\bar{P}_{H,t}\) don't depend on k and the
expression is equal zero, we can do the following operation

\(\displaystyle \sum_{k=0}^\infty \theta^kE_t \Bigg\{ Q_{t,t+k} \Bigg[ Y_{t+k } \left[ -\varepsilon+\varepsilon \frac{MC_{t+k}^n}{\bar{P}_{H,t}} \right] \Bigg] + Y_{t+k} \Bigg\} = 0 \ \ \Rightarrow \ \ \sum_{k=0}^\infty \theta^kE_t \Bigg\{ Q_{t,t+k} Y_{t+k } \left[ (1-\varepsilon)+\varepsilon \frac{MC_{t+k}^n}{\bar{P}_{H,t}} \right] \frac{\bar{P}_{H,t}}{1-\varepsilon} \Bigg\} = 0\)

\(\displaystyle \sum_{k=0}^\infty \theta^kE_t \bigg\{ Q_{t,t+k} Y_{t+k } \left[ \bar{P}_{H,t}-\frac{\varepsilon}{\varepsilon-1} MC_{t+k}^n \right] \bigg\} = 0\)

Using the fact that
\(\displaystyle Q_{t,t+k}=\beta^k \left( \frac{C_{t+k}}{C_t} \right)^{-\sigma}\frac{P_t}{P_{t+k}}\),
we have

\(\displaystyle \sum_{k=0}^\infty \theta^kE_t \left\{ \beta^k \left( \frac{C_{t+k}}{C_t} \right)^{-\sigma}\frac{P_t}{P_{t+k}} Y_{t+k } \left[ \bar{P}_{H,t}-\frac{\varepsilon}{\varepsilon-1} MC_{t+k}^n \right] \right\} = 0\)

As \(P_t\) and \(C_t\) doesn't depend on k, we can put it out of
summation and ignore (as the expression equals zero)

\(\displaystyle \sum_{k=0}^\infty (\beta \theta)^kE_t \left\{ C_{t+k}^{-\sigma}P_{t+k}^{-1} Y_{t+k } \left[ \bar{P}_{H,t}-\frac{\varepsilon}{\varepsilon-1} MC_{t+k}^n \right] \right\} = 0\)

\(\displaystyle \sum_{k=0}^\infty (\beta \theta)^kE_t \left\{ C_{t+k}^{-\sigma} Y_{t+k } \frac{P_{H,t-1}}{P_{t+k}} \left[ \frac{\bar{P}_{H,t}}{P_{H,t-1}}-\frac{\varepsilon}{\varepsilon-1}\frac{P_{H,t+k}}{P_{H,t-1}} \frac{MC_{t+k}^n}{P_{H,t+k}} \right] \right\} = 0\)

\(\displaystyle \sum_{k=0}^\infty (\beta \theta)^kE_t \left\{ C_{t+k}^{-\sigma} Y_{t+k } \frac{P_{H,t-1}}{P_{t+k}} \left[ \frac{\bar{P}_{H,t}}{P_{H,t-1}}-\frac{\varepsilon}{\varepsilon-1}\Pi_{t-1,t+k}^H MC_{t+k}\right] \right\} = 0\)
where \(\displaystyle \Pi_{t-1,t+k}^H = \frac{P_{H,t+k}}{P_{H,t-1}}\)
and \(\displaystyle MC_{t+k} = \frac{MC_{t+k}^n}{P_{H,t+k}}\)

As we will use a first-order approximation, we can ignore the Jensen's
inequality. To make the multivariate Taylor expansion, we can use
\(f(x+\Delta x) \approx f(x)+ \Delta x^T \nabla f|_x(\Delta x)\), where
x is the vector of variables in the steady-state.

\(\displaystyle \sum_{k=0}^\infty (\beta \theta)^kE_t \left\{ C_{t+k}^{-\sigma} Y_{t+k } \frac{P_{H,t-1}}{P_{t+k}} \left[ \frac{\bar{P}_{H,t}}{P_{H,t-1}}-\frac{\varepsilon}{\varepsilon-1}\Pi_{t-1,t+k}^H MC_{t+k}\right] \right\} = \sum_{k=0}^\infty (\beta \theta)^kE_t \left\{ C^{-\sigma} Y \frac{P_{H}}{P} \left[ \frac{\bar{P}_{H}}{P_{H}}-\frac{\varepsilon}{\varepsilon-1} MC \right] \right\}\)

\(\displaystyle + \sum_{k=0}^\infty (\beta \theta)^kE_t \left\{ -\sigma C^{-\sigma-1} Y \frac{P_{H}}{P} \left[ 1-\frac{\varepsilon}{\varepsilon-1} MC \right](C_{t+k}-C) \right\} + \sum_{k=0}^\infty (\beta \theta)^kE_t \left\{ C^{-\sigma} \frac{P_{H}}{P} \left[ 1-\frac{\varepsilon}{\varepsilon-1} MC \right](Y_{t+k}-Y) \right\}\)

\(\displaystyle + \sum_{k=0}^\infty (\beta \theta)^kE_t \left\{ \left( C^{-\sigma} Y \frac{1}{P} \left[ 1-\frac{\varepsilon}{\varepsilon-1} MC \right] + C^{-\sigma}Y \frac{P_H}{P} \left[ -\frac{1}{P_H} \right] \right)(P_{H,t-1}-P_H) \right\}\)

\(\displaystyle + \sum_{k=0}^\infty (\beta \theta)^kE_t \left\{ -C^{-\sigma} Y \frac{P_{H}}{P^2} \left[ 1-\frac{\varepsilon}{\varepsilon-1} MC \right](P_{t+k}-P) \right\} + \sum_{k=0}^\infty (\beta \theta)^kE_t \left\{ C^{-\sigma} Y \frac{P_{H}}{P} \left[ \frac{1}{P_H} \right](\bar{P}_{H,t}-P_H) \right\}\)

\(\displaystyle + \sum_{k=0}^\infty (\beta \theta)^kE_t \left\{ C^{-\sigma} Y \frac{P_{H}}{P} \left[ -\frac{\varepsilon}{\varepsilon-1} MC \right](\Pi_{t-1,t+k}^H-\Pi^H) \right\} + \sum_{k=0}^\infty (\beta \theta)^kE_t \left\{ C^{-\sigma} Y \frac{P_{H}}{P} \left[ -\frac{\varepsilon}{\varepsilon-1} \Pi^H \right](MC_{t+k}-MC) \right\}=0\)

As \(C^{-\sigma}, Y, P_H, P\) don't depend on k, we have

\(\displaystyle \sum_{k=0}^\infty (\beta \theta)^kE_t \left\{ 1-\frac{\varepsilon}{\varepsilon-1} MC \right\} + \sum_{k=0}^\infty (\beta \theta)^kE_t \left\{ -\sigma \left[ 1-\frac{\varepsilon}{\varepsilon-1} MC \right]\hat{c}_{t+k} \right\} + \sum_{k=0}^\infty (\beta \theta)^kE_t \left\{ \left[ 1-\frac{\varepsilon}{\varepsilon-1} MC \right]\hat{y}_{t+k} \right\}\)

\(\displaystyle + \sum_{k=0}^\infty (\beta \theta)^kE_t \left\{ \left(-\frac{\varepsilon}{\varepsilon-1} MC \right)\hat{p}_{H,t-1} \right\} + \sum_{k=0}^\infty (\beta \theta)^kE_t \left\{ - \left[ 1-\frac{\varepsilon}{\varepsilon-1} MC \right](\hat{p}_{t+k}) \right\} + \sum_{k=0}^\infty (\beta \theta)^kE_t \left\{ (\hat{\bar{p}}_{H,t}) \right\}\)

\(\displaystyle + \sum_{k=0}^\infty (\beta \theta)^kE_t \left\{ \left[ -\frac{\varepsilon}{\varepsilon-1} MC \right]\pi_{t-1,t+k} \right\} + \sum_{k=0}^\infty (\beta \theta)^kE_t \left\{ MC \left[ -\frac{\varepsilon}{\varepsilon-1} \right]\widehat{mc}_{t+k} \right\}=0\)

\(\displaystyle \sum_{k=0}^\infty (\beta \theta)^kE_t \left\{ \left(1-\frac{\varepsilon}{\varepsilon-1} MC \right)(1-\sigma \hat{c}_{t+k} + \hat{y}_{t+k} + \hat{p}_{H,t-1} -\hat{p}_{t+k} + \pi_{t-1,t+k} + \widehat{mc}_{t+k}) -\hat{p}_{H,t-1} +\hat{\bar{p}}_{H,t}-\pi_{t-1,t+k} - \widehat{mc}_{t+k} \right\}\)

As the Taylor approximation is around zero (we are assuming regularity
conditions to all functions),

\(\displaystyle \sum_{k=0}^\infty (\beta \theta)^kE_t \left\{ 1-\frac{\varepsilon}{\varepsilon-1} MC \right\}=0\)
, we have

\(\displaystyle \sum_{k=0}^\infty (\beta \theta)^kE_t \left\{ -\hat{p}_{H,t-1} +\hat{\bar{p}}_{H,t}-\pi_{t-1,t+k} - \widehat{mc}_{t+k} \right\}=0\)

\(\displaystyle \frac{1}{1-\beta \theta}(\bar{p}_{H,t}-p_{H}) + \sum_{k=0}^\infty (\beta \theta)^kE_t \left\{ -(p_{H,t-1}-p_{H}) -(p_{H,t+k}-p_{H,t-1}) - \widehat{mc}_{t+k} \right\}=0\)

\(\displaystyle \frac{1}{1-\beta \theta}(\bar{p}_{H,t}-p_{H,t-1}) + \sum_{k=0}^\infty (\beta \theta)^kE_t \left\{ -p_{H,t+k}+p_{H,t}-p_{H,t}+p_{H,t-1} - \widehat{mc}_{t+k} \right\}=0\)

\(\displaystyle \bar{p}_{H,t} = p_{H,t-1} + \pi_{H,t}+ \sum_{k=0}^\infty (\beta \theta)^k E_t \{\pi_{H,t+k} \} -\beta \theta\sum_{k=0}^\infty (\beta \theta)^k E_t \{\pi_{H,t+k} \} + (1-\beta \theta)\sum_{k=0}^\infty (\beta \theta)^k E_t \{\widehat{mc}_{t+k} \}\)

\(\displaystyle \bar{p}_{H,t} = p_{H,t-1} + \sum_{k=0}^\infty (\beta \theta)^k E_t \{\pi_{H,t+k} \} + (1-\beta \theta)\sum_{k=0}^\infty (\beta \theta)^k E_t \{\widehat{mc}_{t+k} \}\)

Rational expectations imply that the difference between the actual
inflation and the future expectations is the steady-state inflation,
which is zero. This expression can also be written as (it's easy to see
when trying to iterate forward).

\(\bar{p}_{H,t}-p_{H,t-1}=\beta \theta E_t \{ \bar{p}_{H,t+1}-p_{H,t} + \pi_{H,t}+(1-\beta \theta)\widehat{mc}_t\}\)

Substituting \(\widehat{mc}_t = mc_t^n-p_{H,t}+\mu\), we have

\(\bar{p}_{H,t}-p_{H,t-1}=\beta \theta E_t \{ \bar{p}_{H,t+1}-p_{H,t} + \pi_{H,t}+(1-\beta \theta)(mc_t^n-p_{H,t}+\mu)\}\)

\(\bar{p}_{H,t}-p_{H,t-1} = E_t \{ \beta \theta\bar{p}_{H,t+1}-\beta \theta p_{H,t} + p_{H,t}- p_{H,t-1} +mc_t^n-p_{H,t}+\mu - \beta \theta mc_t^n + \beta \theta p_{H,t} - \beta \theta \mu \}\)

\(\bar{p}_{H,t} = E_t \{ \beta \theta\bar{p}_{H,t+1} +mc_t^n +\mu - \beta \theta mc_t^n - \beta \theta \mu \}\)

which can also be written as
\(\displaystyle p_{H,t}=\mu + (1-\beta \theta)\sum_{k=0}^\infty(\beta \theta)^kE_t \{ mc_{t+k}^n\}\)

The price-setting equation is
\(\displaystyle P_H \equiv \left[ \theta (P_{H,t-1})^{1-\varepsilon} + (1-\theta)(\bar{P}_{H,t})^{1-\varepsilon}\right]^{\frac{1}{1-\varepsilon}}\)

Log-linearizing the equation above around the steady-state with the
Taylor expansion gives

\(\displaystyle P_{H,t} \approx [\theta(P_H)^{1-\varepsilon} + (1-\theta)(P_H)^{1-\varepsilon}]^{\frac{1}{1-\varepsilon}} + \frac{1}{1-\varepsilon} \left[ (P_H)^{1-\varepsilon} \right]^{\frac{1}{1-\varepsilon}-1} \theta(1-\varepsilon) P_H^{-\varepsilon} (P_{H,t-1}-P_H)\)

\(\displaystyle + \frac{1}{1-\varepsilon} \left[ (P_H)^{1-\varepsilon} \right]^{\frac{\varepsilon}{1-\varepsilon}} (1-\theta)(1-\varepsilon) P_H^{-\varepsilon} (\bar{P}_{H,t}-P_H)\)

\(\displaystyle P_{H,t} \approx P_H + \theta(P_H)^{\varepsilon}(P_H)^{-\varepsilon} (P_{H,t-1}-P_H) + (1-\theta)(P_H)^{\varepsilon}(P_H)^{-\varepsilon} (\bar{P}_{H,t}-P_H)=P_H+\theta P_{H,t-1} -\theta P_H +\bar{P}_{H,t}-P_H - \theta \bar{P}_{H,t} + \theta P_H\)

\(\displaystyle P_{H,t} = \theta P_{H,t-1} + (1-\theta) \bar{P}_{H,t} \ \ \Rightarrow \ \ \frac{P_{H,t}-P_{H,t-1}}{P_H} = \frac{\theta P_{H,t-1} - P_{H,t-1} + (1-\theta) \bar{P}_{H,t}}{P_H} \ \ \Rightarrow \ \ \pi_{H,t}=(1-\theta)(\bar{p}_{H,t}-p_{H,t-1})\)

Substituting \(\bar{p}_{H,t}-p_{H,t-1}\) we have

\(\displaystyle \pi_{H,t}=(1-\theta)(\beta \theta E_t \{ \bar{p}_{H,t+1}-p_{H,t} + \pi_{H,t}+(1-\beta \theta)\widehat{mc}_t\}) = (1-\theta) \left(\beta \theta E_t \left\{ \frac{\pi_{H,t+1}}{1-\theta} + \pi_{H,t}+(1-\beta \theta)\widehat{mc}_t \right\} \right)\)

\(\displaystyle \theta \pi_{H,t} = \beta \theta E_t \left\{ \pi_{H,t+1}\right\} +(1-\theta) (1-\beta \theta)\widehat{mc}_t \ \ \Rightarrow \ \  \pi_{H,t} = \beta E_t \left\{ \pi_{H,t+1}\right\} +\frac{(1-\theta) (1-\beta \theta)}{\theta}\widehat{mc}_t\)

which gives
\(\pi_{H,t} = \beta E_t \left\{ \pi_{H,t+1}\right\}+ \lambda\widehat{mc}_t\)
where
\(\displaystyle \lambda = \frac{(1-\theta) (1-\beta \theta)}{\theta}\)

\hypertarget{equilibrium}{%
\subsection{10 - Equilibrium}\label{equilibrium}}

Good market clearing in the representative small open economy (``home'')
requires that the total produced inside the small economy is consumed
either by the households of this economy or imported (from the small
economy, subscript H) from households of any other country (superscript
i).

\(\displaystyle Y_t(j) = C_{H,t}(j)+\int_0^1 C_{H,t}^i(j)di = \left( \frac{P_{H,t}(j)}{P_{H,t}}\right)^{-\varepsilon}C_{H,t} + \int_0^1 \displaystyle \left( \frac{P_{H,t}^i(j)}{P_{H,t}^i}\right)^{-\varepsilon}C_{H,t}^i di\)

\(\displaystyle Y_t(j) = \left( \frac{P_{H,t}(j)}{P_{H,t}}\right)^{-\varepsilon} (1-\alpha)\left( \frac{P_{H,t}}{P_t} \right)^{-\eta}C_t + \int_0^1 \left( \frac{P_{H,t}^i(j)}{P_{H,t}^i}\right)^{-\varepsilon} \left( \frac{P_{H,t}^i}{P_{F,t}^i}\right)^{-\gamma}C_{F,t}^i di\)

\(\displaystyle Y_t(j) = \left( \frac{P_{H,t}(j)}{P_{H,t}}\right)^{-\varepsilon} (1-\alpha)\left( \frac{P_{H,t}}{P_t} \right)^{-\eta}C_t + \int_0^1 \left( \frac{P_{H,t}^i(j)}{P_{H,t}^i}\right)^{-\varepsilon} \left( \frac{P_{H,t}}{\mathcal{E}_{i,t}P_{F,t}^i}\right)^{-\gamma} \alpha \left( \frac{P_{F,t}^i}{P_t^i} \right)^{-\eta} C_t^i di\)

Assuming symmetric preferences across countries, the home country will
sell the variety of good j for the same price, independently of which
country is buying, which implies
\(\displaystyle \frac{P_{H,t}^i (j)}{P_{H,t}^i}=\frac{P_{H,t} (j)}{P_{H,t}}\)

Thus,
\(\displaystyle Y_t(j) = \left( \frac{P_{H,t}(j)}{P_{H,t}}\right)^{-\varepsilon} \left[ (1-\alpha)\left( \frac{P_{H,t}}{P_t} \right)^{-\eta}C_t + \alpha \int_0^1 \left( \frac{P_{H,t}}{\mathcal{E}_{i,t}P_{F,t}^i}\right)^{-\gamma} \left( \frac{P_{F,t}^i}{P_t^i} \right)^{-\eta} C_t^i di \right]\)

Plugging on the definition of aggregate domestic output
\(\displaystyle Y_t^{\frac{\varepsilon-1}{\varepsilon}}= \int_0^1Y_t(j)^{\frac{\varepsilon-1}{\varepsilon}}dj\)
we have

\(\displaystyle Y_t^{\frac{\varepsilon-1}{\varepsilon}} = \int_0^1 \left( \left( \frac{P_{H,t}(j)}{P_{H,t}}\right)^{-\varepsilon} \left[ (1-\alpha)\left( \frac{P_{H,t}}{P_t} \right)^{-\eta}C_t + \alpha \int_0^1 \left( \frac{P_{H,t}}{\mathcal{E}_{i,t}P_{F,t}^i}\right)^{-\gamma} \left( \frac{P_{F,t}^i}{P_t^i} \right)^{-\eta} C_t^i di \right] \right)^{\frac{\varepsilon-1}{\varepsilon}}dj\)

\(\displaystyle Y_t^{\frac{\varepsilon-1}{\varepsilon}} = \left[ (1-\alpha)\left( \frac{P_{H,t}}{P_t} \right)^{-\eta}C_t + \alpha \int_0^1 \left( \frac{P_{H,t}}{\mathcal{E}_{i,t}P_{F,t}^i}\right)^{-\gamma} \left( \frac{P_{F,t}^i}{P_t^i} \right)^{-\eta} C_t^i di \right]^{\frac{\varepsilon-1}{\varepsilon}} \left( \frac{1}{P_{H,t}}\right)^{1-\varepsilon} \int_0^1 P_{H,t}(j) ^{1-\varepsilon} dj\)

By the definition of price index,
\(\displaystyle P_H^{1-\varepsilon} = \int_0^1 P_{H,t}(j)^{1-\varepsilon}dj\),
we have

\(\displaystyle Y_t^{\frac{\varepsilon-1}{\varepsilon}} = \left[ (1-\alpha)\left( \frac{P_{H,t}}{P_t} \right)^{-\eta}C_t + \alpha \int_0^1 \left( \frac{P_{H,t}}{\mathcal{E}_{i,t}P_{F,t}^i}\right)^{-\gamma} \left( \frac{P_{F,t}^i}{P_t^i} \right)^{-\eta} C_t^i di \right]^{\frac{\varepsilon-1}{\varepsilon}} \left( \frac{1}{P_{H,t}}\right)^{1-\varepsilon} P_H^{1-\varepsilon}\)

\(\displaystyle Y_t^{\frac{\varepsilon-1}{\varepsilon}} = \left[ (1-\alpha)\left( \frac{P_{H,t}}{P_t} \right)^{-\eta}C_t + \alpha \int_0^1 \left( \frac{P_{H,t}}{\mathcal{E}_{i,t}P_{F,t}^i}\right)^{-\gamma} \left( \frac{P_{F,t}^i}{P_t^i} \right)^{-\eta} C_t^i di \right]^{\frac{\varepsilon-1}{\varepsilon}}\)

\(\displaystyle Y_t = (1-\alpha)\left( \frac{P_{H,t}}{P_t} \right)^{-\eta}C_t + \alpha \int_0^1 \left( \frac{P_{H,t}}{\mathcal{E}_{i,t}P_{F,t}^i}\right)^{-\gamma} \left( \frac{P_{F,t}^i}{P_t^i} \right)^{-\eta} C_t^i di\)

Using the fact that
\(\displaystyle \mathcal{Q}_{i,t}= \frac{\mathcal{E}_{i,t} P_t^i}{P_t}\)

\(\displaystyle Y_t = (1-\alpha)\left( \frac{P_{H,t}}{P_t} \right)^{-\eta}C_t + \alpha \int_0^1 \left( \frac{\mathcal{E}_{i,t}P_{F,t}^i}{P_{H,t}}\right)^{\gamma} \left( P_{F,t}^i \right)^{-\eta} \left( \frac{\mathcal{Q}_{i,t}P_t}{\mathcal{E}_{i,t}} \right)^\eta C_t^i di = \left( \frac{P_{H,t}}{P_t} \right)^{-\eta} \left[ (1-\alpha)C_t + \alpha \int_0^1 \left( \frac{\mathcal{E}_{i,t}P_{F,t}^i}{P_{H,t}}\right)^{\gamma-\eta} \mathcal{Q}^\eta C_t^i di \right]\)

Considering that
\(\displaystyle \mathcal{S}_{i,t}=\frac{P_{i,t}}{P_{H,t}}\),
\(\displaystyle \mathcal{S}_t^i=\frac{\mathcal{E}_{i,t} P_{F,t}^i}{P_{i,t}}\),
and \(C_{t}^i=C_t \mathcal{Q}^{-\frac{1}{\sigma}}\) we have

\(\displaystyle Y_t = \left( \frac{P_{H,t}}{P_t} \right)^{-\eta} \left[ (1-\alpha)C_t + \alpha \int_0^1 \left( \frac{P_{i,t}}{P_{H,t}} \frac{\mathcal{E}_{i,t}P_{F,t}^i}{P_{i,t}}\right)^{\gamma-\eta} \mathcal{Q}^\eta \mathcal{Q}^{-\frac{1}{\sigma}} C_t di \right] = \left( \frac{P_{H,t}}{P_t} \right)^{-\eta}C_t \left[ (1-\alpha) + \alpha \int_0^1 \left( \mathcal{S}_{i,t} \mathcal{S}_t^i\right)^{\gamma-\eta} \mathcal{Q}^{\eta-\frac{1}{\sigma}} di \right]\)

If \(\sigma=\eta=\gamma=1\), then

\(\displaystyle Y_t = \left( \frac{P_{H,t}}{P_t} \right)^{-1} C_t \left[ (1-\alpha) + \alpha \int_0^1 \left( \mathcal{S}_{i,t} \mathcal{S}_t^i\right)^{0} \mathcal{Q}^{0} di \right] = \frac{P_{t}}{P_{H,t}} C_t=C_t \mathcal{S}_t^\alpha\)

as if \(\eta=1\), the CPI takes the form
\(P_t=(P_{H,t})^{1-\alpha}(P_{F,t})^\alpha\) implying
\(\displaystyle \frac{P_t}{P_{H,t}}=\left( \frac{P_{F,t}}{P_{H,t}} \right)^\alpha=\mathcal{S}^\alpha\)

Log-linearizing the equation (25) using the Taylor expansion around the
symmetric steady-state, we have

\(\displaystyle Y_t \approx Y + (-\eta)\left( P_H \right)^{-\eta-1}\left( \frac{1}{P} \right)^{-\eta}C \left[ (1-\alpha) + \alpha \int_0^1 \left( \mathcal{S}_{i} \mathcal{S}^i\right)^{\gamma-\eta} \mathcal{Q}_i^{\eta-\frac{1}{\sigma}} di \right](P_{H,t}-P_H)\)

\(\displaystyle + \eta\left( P_H \right)^{-\eta}\left( \frac{1}{P} \right)^{-\eta+1}C \left[ (1-\alpha) + \alpha \int_0^1 \left( \mathcal{S}_{i} \mathcal{S}^i\right)^{\gamma-\eta} \mathcal{Q}_i^{\eta-\frac{1}{\sigma}} di \right](P_t-P) +\left( \frac{P_H}{P} \right)^{-\eta} \left[ (1-\alpha) + \alpha \int_0^1 \left( \mathcal{S}_{i} \mathcal{S}^i\right)^{\gamma-\eta} \mathcal{Q}_i^{\eta-\frac{1}{\sigma}} di \right](C_{t}-C)\)

\(\displaystyle +\left( \frac{P_H}{P} \right)^{-\eta} C \left[\alpha \int_0^1 (\gamma-\eta) \left( \mathcal{S}_{i} \right)^{\gamma-\eta-1} \left( \mathcal{S}^i\right)^{\gamma-\eta} \mathcal{Q}_i^{\eta-\frac{1}{\sigma}} (\mathcal{S}_{i,t}-\mathcal{S}_{i}) di \right] +\left( \frac{P_H}{P} \right)^{-\eta} C \left[\alpha \int_0^1 (\gamma-\eta) \left( \mathcal{S}_{i} \right)^{\gamma-\eta} \left( \mathcal{S}^i\right)^{\gamma-\eta-1} \mathcal{Q}_i^{\eta-\frac{1}{\sigma}} (\mathcal{S}_{t}^i-\mathcal{S}^i) di \right]\)

\(\displaystyle +\left( \frac{P_H}{P} \right)^{-\eta} C \left[\alpha \int_0^1 \left(\eta-\frac{1}{\sigma} \right) \left( \mathcal{S}_{i} \mathcal{S}^i\right)^{\gamma-\eta} \mathcal{Q}_i^{\eta-\frac{1}{\sigma}-1} (\mathcal{Q}_{i,t}-\mathcal{Q}_i) di \right]\)

As shown in appendix A (and above, in the international risk sharing
section), in a symmetric steady-state
\(\mathcal{Q}_i=\mathcal{S}_i=\mathcal{S}^i=1\) for all i (purchasing
parity holds). Thus

\(\displaystyle Y_t-Y = \left( \frac{P_H}{P} \right)^{-\eta}C \left[ -\eta \frac{P_{H,t}-P_H}{P_H} +\eta \frac{P_{t}-P}{P} + \frac{C_t-C}{C} \right]\)

\(\displaystyle +\left( \frac{P_H}{P} \right)^{-\eta} C \left[\alpha \int_0^1 (\gamma-\eta) \left[ \frac{ \mathcal{S}_{t}^i-\mathcal{S}^i }{\mathcal{S}^i} + \frac{ \mathcal{S}_{i,t}-\mathcal{S}_i }{\mathcal{S}_i}\right] + \left(\eta-\frac{1}{\sigma} \right) \frac{\mathcal{Q}_{i,t}-\mathcal{Q}_i}{\mathcal{Q}_i} di \right]\)

As
\(\displaystyle \left( \frac{P}{P_H} \right)^{1-\eta}=(1-\alpha)+\alpha \int_o^1(\mathcal{S}_i)^{1-\eta}di =1\)
(from international risk sharing section) in a symmetric steady-state
and \(C=Y\) in the international market clearing, we have

\(\displaystyle \frac{Y_t-Y}{Y} = -\eta \hat{p}_{H,t} +\eta \hat{p}_t + \hat{c_t} + \left[\alpha (\gamma-\eta) \int_0^1 s_t^i di + \alpha (\gamma-\eta) \int_0^1 s_{i,t}di + \alpha\left(\eta-\frac{1}{\sigma} \right) \int_0^1 q_{i,t} di \right]\)

Considering that \(p_t-p_{H,t}=\alpha s_t\) and recalling that
\(\displaystyle \int_0^1s_t^idi=0\),
\(\displaystyle s_t= \int_0^1s_{i,t}di\) and
\(\displaystyle q_t \equiv \int_0^1 q_{i,t}di\), we have

\(\displaystyle y_t-y = \eta \alpha s_t +c_t-c + \alpha (\gamma-\eta) s_t + \alpha\left(\eta-\frac{1}{\sigma} \right) q_{i,t} \ \ \Rightarrow \ \ y_t=c_t+ \alpha \gamma s_t + \alpha\left(\eta-\frac{1}{\sigma} \right) q_{i,t}\)

As \(q_t = (1-\alpha)s_t\), derived in the section domestic inflation
and CPI inflation, we have

\(\displaystyle y_t=c_t+ \alpha \gamma s_t + \alpha\left(\eta-\frac{1}{\sigma} \right) (1-\alpha)s_t=c_t+\frac{\alpha}{\sigma}\left[ \sigma \gamma + (\sigma \eta - 1)(1-\alpha) \right] s_t = c_t+\frac{\alpha \omega}{\sigma}s_t\)

where \(\omega \equiv \sigma \gamma + (\sigma \eta - 1)(1-\alpha)\)

As this equation holds for all countries, from a generic country we have

\(\displaystyle y_t^i=c_t^i+ \frac{\alpha \omega}{\sigma} s_t^i\)

Aggregating over all countries we have the world market clearing
condition. The equatily follows from the fact that
\(\int_0^1 s_t^i di =0\)

\(y_t^* \equiv \int_0^1y_t^idi = \int_0^1c_t^idi \equiv c_t^*\)

As
\(\displaystyle c_t = c_t^* + \left(\frac{1-\alpha}{\sigma} \right)s_t\),
we have that

\(\displaystyle y_t - \frac{\alpha \omega}{\sigma}s_t = c_t^* + \left(\frac{1-\alpha}{\sigma} \right)s_t \ \ \Rightarrow \ \ y_t = y_t^* + \left(\frac{1-\alpha}{\sigma} + \frac{\alpha \omega}{\sigma} \right)s_t \ \ \Rightarrow \ \ y_t = y_t^* + \frac{1}{\sigma_\alpha}s_t\)

where
\(\displaystyle \sigma_\alpha \equiv \frac{\sigma}{(1-\alpha)+ \alpha \omega}\)

Recalling that
\(\displaystyle c_t=E_t\{c_{t+1}\}-\frac{1}{\sigma}(r_t-E_t\{\pi_{t+1}\}-\rho)\),
we have

\(\displaystyle y_t - \frac{\alpha \omega}{\sigma}s_t = E_t\{c_{t+1}\}-\frac{1}{\sigma}(r_t-E_t\{\pi_{t+1}\}-\rho) \ \ \Rightarrow \ \ y_t = \frac{\alpha \omega}{\sigma}s_t + E_t\{y_{t+1}\} - \frac{\alpha \omega}{\sigma}E_t\{s_{t+1}\} -\frac{1}{\sigma}(r_t-E_t\{\pi_{t+1}\}-\rho)\)

\(\displaystyle y_t = E_t\{y_{t+1}\} -\frac{1}{\sigma}(r_t-E_t\{\pi_{t+1}\}-\rho)- \frac{\alpha \omega}{\sigma}E_t\{\Delta s_{t+1}\}\)

As \(\pi_t = \pi_{H,t}+\alpha\Delta s_t\), we have

\(\displaystyle y_t = E_t\{y_{t+1}\} -\frac{1}{\sigma}(r_t-E_t\{\pi_{H,t+1}\}-\alpha E_t\{\Delta s_{t+1} \} -\rho)- \frac{\alpha \omega}{\sigma}E_t\{\Delta s_{t+1}\} = E_t\{y_{t+1}\} -\frac{1}{\sigma}(r_t-E_t\{\pi_{H,t+1}\} -\rho)- \frac{\alpha (\omega-1)}{\sigma}E_t\{\Delta s_{t+1}\}\)

\(\displaystyle y_t = E_t\{y_{t+1}\} -\frac{1}{\sigma}(r_t-E_t\{\pi_{H,t+1}\} -\rho)- \frac{\alpha \Theta}{\sigma}E_t\{\Delta s_{t+1}\}\),

where
\(\Theta = \omega-1 = \sigma \gamma + (\sigma \eta - 1)(1-\alpha)-1=(\sigma \gamma-1) + (\sigma \eta - 1)(1-\alpha)\)

As \(\Delta s_t= \sigma_\alpha (\Delta y_t - \Delta y_t^*)\)

\(\displaystyle y_t = E_t\{y_{t+1}\} -\frac{1}{\sigma}(r_t-E_t\{\pi_{H,t+1}\} -\rho)- \frac{\alpha \Theta}{\sigma}E_t\{\sigma_\alpha (\Delta y_{t+1} - \Delta y_{t+1}^*)\}\)

\(\displaystyle \sigma y_t = \sigma E_t\{y_{t+1}\} -(r_t-E_t\{\pi_{H,t+1}\} -\rho)- \alpha \Theta \sigma_\alpha (E_t\{ \Delta y_{t+1} \} - E_t\{\Delta y_{t+1}^*\})\)

\(\displaystyle \sigma y_t=\sigma E_t\{y_{t+1}\} -(r_t-E_t\{\pi_{H,t+1}\} -\rho)- \alpha (\omega-1) \frac{\sigma}{(1-\alpha)+\alpha \omega} (E_t\{ y_{t+1} \} -y_t - E_t\{\Delta y_{t+1}^*\})\)

\(\displaystyle \frac{\sigma y_t -\alpha \sigma y_t + \sigma \alpha \omega y_t -\sigma \alpha \omega y_t + \alpha \sigma y_t}{(1-\alpha)+\alpha \omega}= \frac{\sigma E_t\{y_{t+1}\} -\alpha \sigma E_t\{y_{t+1}\} + \sigma \alpha \omega E_t\{y_{t+1}\} -\sigma \alpha \omega E_t\{y_{t+1}\} + \alpha \sigma E_t\{y_{t+1}\}}{(1-\alpha)+\alpha \omega}\)

\(\displaystyle -(r_t-E_t\{\pi_{H,t+1}\} -\rho)+ \alpha (\omega-1) \frac{\sigma}{(1-\alpha)+\alpha \omega} E_t\{\Delta y_{t+1}^*\}\)

\(\displaystyle \frac{\sigma y_t}{(1-\alpha)+\alpha \omega}= \frac{\sigma E_t\{y_{t+1}\} }{(1-\alpha)+\alpha \omega}-(r_t-E_t\{\pi_{H,t+1}\} -\rho)+ \alpha \Theta \sigma_\alpha E_t\{\Delta y_{t+1}^*\}\)

\(\displaystyle \sigma_\alpha y_t= \sigma_\alpha E_t\{y_{t+1}\} -(r_t-E_t\{\pi_{H,t+1}\} -\rho)+ \alpha \Theta \sigma_\alpha E_t\{\Delta y_{t+1}^*\} \ \ \Rightarrow \ \  y_t= E_t\{y_{t+1}\} -\frac{1}{\sigma_\alpha}(r_t-E_t\{\pi_{H,t+1}\} -\rho)+ \alpha \Theta E_t\{\Delta y_{t+1}^*\}\)

\hypertarget{trade-balance}{%
\subsection{Trade balance}\label{trade-balance}}

Defining net exports in terms of domestic output
\(\displaystyle nx_t \equiv \frac{1}{Y}\left( Y_t - \frac{P_t}{P_{H,t}}C_t \right)\),
defined as a fraction of steady-state output. If
\(\sigma=\eta=\gamma=1\),

\(\displaystyle Y_t = \left( \frac{P_{H,t}}{P_t} \right)^{-1} C_t \left[ (1-\alpha) + \alpha \int_0^1 \left( \mathcal{S}_{i,t} \mathcal{S}_t^i\right)^{0} \mathcal{Q}^{0} di \right] = \frac{P_{t}}{P_{H,t}} C_t=C_t \mathcal{S}_t^\alpha\)

\(\displaystyle Y_t = \frac{P_{t}}{P_{H,t}} C_t \ \ \Rightarrow \ \ Y_t P_{H,t}=P_tC_t\),
implying a balanced trade.

Log-linearizing \(Y_t=C_t \mathcal{S}_t^\alpha\), we have
\(y_t = c_t + \alpha s_t\)

A first-order approximation for \(nx_t\) is (noting that nx is zero in
the steady-state), recalling that in the steady-state, \(P_H=P\) and
\(Y=C\)

\(\displaystyle nx_t = \frac{1}{Y} \left( (Y_t-Y)-\frac{C}{P}(P_t-P)+\frac{C}{P}(P_{H,t}-P) - (C_t-C) \right)=y_t-p_t+p_{H,t}-c_t\)

As \(p_t-p_{H,t}=\alpha s_t\) and
\(\displaystyle y_t = c_t+\frac{\alpha \omega}{\sigma}\), we have
\(\displaystyle nx_t=y_t-c_t-\alpha s_t= \frac{\alpha \omega}{\sigma}s_t - \alpha s_t \ \ \Rightarrow \ \ nx_t = \alpha \left(\frac{\omega}{\sigma} - 1 \right) s_t\)

In this model, \(nx_t\) iz zero if
\(\displaystyle \frac{\omega}{\sigma} - 1=0\), or
\(\displaystyle \frac{\sigma \gamma + (\sigma \eta - 1)(1-\alpha)}{\sigma}=1\),
which means that \(\sigma (\gamma-1) + (1-\alpha)(\sigma \eta - 1)=0\)

\hypertarget{marginal-cost-and-inflation-dynamics-in-the-small-open-economy}{%
\subsection{Marginal cost and inflation dynamics in the small open
economy}\label{marginal-cost-and-inflation-dynamics-in-the-small-open-economy}}

The two relations below were derived in the section ``Firms''

\(\pi_{H,t} = \beta E_t \{ \pi_{H,t+1}\} + \lambda \widehat{mc}_t\)
where
\(\displaystyle \lambda \equiv \frac{(1-\beta \theta)(1-\theta)}{\theta}\)

The log-linearized equation of marginal cost is
\(mc_t = -\nu+w_t-p_{H,t}-a_t\) where \(\nu\equiv -\ln(1-\tau)\) and
\(\tau\) is the subsidy. Thus,
\(mc_t= -\nu + (w_t-p_t) - (p_{H,t}-p_t)-a_t\)

From the log-linearized FOC, we have
\(w_t-p_t=\sigma c_t + \varphi n_t\)

Thus, \(mc_t= -\nu + \sigma c_t + \varphi n_t + \alpha s_t -a_t\)

Using
\(\displaystyle c_t=c_t^*+ \left( \frac{1-\alpha}{\sigma} \right)s_t\)
and \(y_t = a_t + n_t\), we have

\(\displaystyle mc_t= -\nu + \sigma \left[ c_t^*+ \left( \frac{1-\alpha}{\sigma} \right)s_t \right] + \varphi (y_t-a_t) + \alpha s_t -a_t=-\nu + \sigma c_t^*+ ( 1-\alpha )s_t + \varphi (y_t-a_t) + \alpha s_t -a_t\)

As the world consumption is equal to its production,

\(\displaystyle mc_t =-\nu + \sigma y_t^*+ \varphi y_t+ s_t -(1+\varphi)a_t\)

Using \(\displaystyle y_t = y_t^* + \frac{1}{\sigma_\alpha}s_t\) we can
substitute for \(s_t\) in the expression above

\(\displaystyle mc_t =-\nu + \sigma y_t^*+ \varphi y_t+ \sigma_\alpha(y_t-y_t^*) -(1+\varphi)a_t \ \ \Rightarrow \ \ mc_t = -\nu + (\sigma_\alpha+\varphi)y_t + (\sigma-\sigma_\alpha)y_t^*-(1+\varphi)a_t\)

\hypertarget{equilibrium-dynamics}{%
\subsection{Equilibrium dynamics}\label{equilibrium-dynamics}}

The output gap is defined as \(x_t \equiv y_t - \bar{y}_t\)

To find The domestic natural level of output we impose \(mc_t=-\mu\) and
solving for domestic output as \(y_t= \bar{y}_t\)

\(-\mu = -\nu + (\sigma_\alpha+\varphi)\bar{y_t} + (\sigma-\sigma_\alpha)y_t^*-(1+\varphi)a_t \ \ \Rightarrow \ \ (\sigma_\alpha+\varphi)\bar{y_t} = \nu -\mu +(1+\varphi)a_t - (\sigma-\sigma_\alpha)y_t^*\)

\(\displaystyle \bar{y_t} = \frac{\nu -\mu}{\sigma_\alpha+\varphi} + \frac{1+\varphi}{\sigma_\alpha+\varphi}a_t - \frac{\sigma-\sigma_\alpha}{\sigma_\alpha+\varphi}y_t^*\)

As \(\Theta = (\sigma \gamma-1)+(\sigma \eta-1)(1-\alpha)\),
\(\omega=\sigma \gamma+(\sigma \eta-1)(1-\alpha)=\Theta+1\)
\(\displaystyle \sigma_\alpha = \frac{\sigma}{(1-\alpha)+\alpha \omega}\),
we have

\(\displaystyle \sigma-\sigma_\alpha=\sigma- \frac{\sigma}{(1-\alpha)+\alpha \omega}=\frac{-\alpha \sigma +\alpha \omega\sigma}{(1-\alpha)+\alpha \omega}\)

Let's verify that
\(\sigma-\sigma_\alpha = - \alpha \Theta \sigma_\alpha\)

\(\displaystyle \alpha \Theta \sigma_\alpha= \alpha (\omega-1) \frac{\sigma}{(1-\alpha)+\alpha \omega} = \frac{\alpha \sigma \omega -\alpha \sigma}{(1-\alpha)+\alpha \omega}=\sigma-\sigma_\alpha\)

Thus,
\(\displaystyle \bar{y_t} = \frac{\nu -\mu}{\sigma_\alpha+\varphi} + \frac{1+\varphi}{\sigma_\alpha+\varphi}a_t - \frac{ \alpha \Theta \sigma_\alpha}{\sigma_\alpha+\varphi}y_t^* \ \ \Rightarrow \ \ \bar{y_t} =\Omega+\Gamma a_t+ \alpha \Psi y_t^*\)
where

\(\displaystyle \Omega = \frac{\nu -\mu}{\sigma_\alpha+\varphi}\),
\(\displaystyle \Gamma = \frac{1+\varphi}{\sigma_\alpha+\varphi}>0\) as
\(\sigma_\alpha>0\) and
\(\displaystyle \Psi=-\frac{\Theta \sigma_\alpha}{\sigma_\alpha + \varphi}\)

Substituting \(y_t\) in
\(mc_t = -\nu + (\sigma_\alpha+\varphi)y_t + (\sigma-\sigma_\alpha)y_t^*-(1+\varphi)a_t\),
we have
\(mc_t = -\nu + (\sigma_\alpha+\varphi)(x_t+\bar{y_t}) + (\sigma-\sigma_\alpha)y_t^*-(1+\varphi)a_t\)

\(\displaystyle mc_t = -\nu + (\sigma_\alpha+\varphi)\left(x_t+\frac{\nu -\mu}{\sigma_\alpha+\varphi} + \frac{1+\varphi}{\sigma_\alpha+\varphi}a_t - \frac{ \alpha \Theta \sigma_\alpha}{\sigma_\alpha+\varphi}y_t^* \right) + (\sigma-\sigma_\alpha)y_t^*-(1+\varphi)a_t = -\nu + (\sigma_\alpha+\varphi)x_t + \nu - \mu - \alpha \Theta \sigma_\alpha y_t^* + (\sigma-\sigma_\alpha)y_t^*\)

\(\displaystyle mc_t-(-\mu)=(\sigma_\alpha+\varphi)x_t - \mu - \alpha \left[ \omega-1 \right] \frac{\sigma}{(1-\alpha)+\alpha \omega} y_t^* + \frac{-\alpha \sigma +\alpha \omega\sigma}{(1-\alpha)+\alpha \omega} y_t^* + \mu\)

\(\displaystyle \widehat{mc_t}=(\sigma_\alpha+\varphi)x_t + \frac{-\alpha \omega \sigma + \alpha \sigma}{(1-\alpha)+\alpha \omega} y_t^* + \frac{-\alpha \sigma +\alpha \omega\sigma}{(1-\alpha)+\alpha \omega} y_t^* \ \ \Rightarrow \ \ \widehat{mc_t}=(\sigma_\alpha+\varphi)x_t\)

Substituting the equation above into
\(\pi_{H,t} = \beta E_t \{ \pi_{H,t+1} + \lambda \widehat{mc}_t\}\), we
have the a version of the New Keynesian Phillips Curve (NKPC)

\(\pi_{H,t} = \beta E_t \{ \pi_{H,t+1}\} + \lambda (\sigma_\alpha+\varphi)x_t = \beta E_t \{ \pi_{H,t+1}\} + \kappa_\alpha x_t\),
where \(\kappa_\alpha \equiv \lambda (\sigma_\alpha+\varphi)\)

Substituting \(y_t\) and \(E_t\{y_{t+1}\}\) in
\(\displaystyle y_t= E_t\{y_{t+1}\} -\frac{1}{\sigma_\alpha}(r_t-E_t\{\pi_{H,t+1}\} -\rho)+ \alpha \Theta E_t\{\Delta y_{t+1}^*\}\),
we have

\[x_t + \Omega+\Gamma a_t+ \alpha \Psi y_t^* = E_t\{x_{t+1} + \Omega+\Gamma a_{t+1}+ \alpha \Psi y_{t+1}^*\} -\frac{1}{\sigma_\alpha}(r_t-E_t\{\pi_{H,t+1}\} -\rho)+ \alpha \Theta  E_t\{\Delta y_{t+1}^*\}\]
\(\displaystyle x_t = E_t\{x_{t+1}\} +\Gamma (\rho_a a_{t}-a_t)+ \alpha \Psi E_t\{\Delta y_{t+1}^*\} -\frac{1}{\sigma_\alpha}(r_t-E_t\{\pi_{H,t+1}\} -\rho)+ \alpha \Theta E_t\{\Delta y_{t+1}^*\}\)

\(\displaystyle x_t = E_t\{x_{t+1}\} -\frac{1}{\sigma_\alpha}(r_t-E_t\{\pi_{H,t+1}\} -[\rho-\sigma_\alpha \Gamma (1-\rho_a)a_t +\alpha \sigma_\alpha(\Theta+\Psi) E_t\{\Delta y_{t+1}^*\}])\)

\(\displaystyle x_t = E_t\{x_{t+1}\} -\frac{1}{\sigma_\alpha}(r_t-E_t\{\pi_{H,t+1}\} -\overline{rr}_t)\)
where
\(\overline{rr}_t \equiv \rho-\sigma_\alpha \Gamma (1-\rho_a)a_t +\alpha \sigma_\alpha(\Theta+\Psi) E_t\{\Delta y_{t+1}^*\}\)
is the small open economy's natural rate of interest.

\hypertarget{optimal-monetary-policy-a-special-case}{%
\subsection{Optimal Monetary Policy: a special
case}\label{optimal-monetary-policy-a-special-case}}

The problem of the central planner is:

\(\displaystyle \max E_0\sum_{i=0}^\infty \beta^t U(C_t,N_t)\) subject
to \(Y_t=A_tN_t\), \(\displaystyle C_t=C_t^i\mathcal{Q}_{i,t}\) and
\(\displaystyle Y_t=\frac{P_t}{P_{H,t}} C_t\), as
\(\eta=\sigma=\gamma=1\).

In this case,
\(\displaystyle Y_t=\frac{(P_{H,t})^{1-\alpha}(P_{F,t})^{\alpha}}{P_{H,t}} C_t = \frac{(P_{F,t})^{\alpha}}{(P_{H,t})^{\alpha}}C_t=\mathcal{S}_t^{\alpha}C_t \ \ \Rightarrow \ \ Y_t=\mathcal{S}_t^{\alpha}C_t \ \ \Rightarrow \ \ \mathcal{S}_t= \left( \frac{Y_t}{C_t} \right)^{\frac{1}{\alpha}}\).
By equation 29, we have \(y_t=y_t^*+s_t\) or
\(\displaystyle Y_t=Y_t^{*} \mathcal{S}_t= Y_t^{*} \left( \frac{Y_t}{C_t} \right)^{\frac{1}{\alpha}} \ \ \Rightarrow \ \ C_t Y_t^\alpha = (Y_t^*)^\alpha Y_t \ \ \Rightarrow \ \ C_t=Y_t^{1-\alpha}(Y_t^*)^\alpha\).

The Central planner problem then becomes

\(\displaystyle E_0\max \sum_{i=0}^\infty \beta^t U(C_t,N_t)\) subject
to \(Y_t=A_tN_t\) and \(C_t=Y_t^{1-\alpha}(Y_t^*)^\alpha\).

The Lagrangean can be written as

\(\displaystyle \mathcal{L}= E_0 \sum_{i=0}^\infty \beta^t \left\{ U(C_t,N_t) +\Lambda_t(A_t N_t - Y_t) + \Phi_t \left( Y_t^{1-\alpha}(Y_t^*)^\alpha -C_t \right) \right\}\)

which yields the FOCS:

\((C_t)\) \(U_c(C_t,N_t)=\Phi_t\)

\((N_t)\) \(U_N(C_t,N_t)=-\Lambda_t A_t\)

\((Y_t)\) \(\Phi_t (1-\alpha)Y_t^{-\alpha}(Y_t^*)^\alpha =\Lambda_t\)

\(U_N(C_t,N_t) = -A_t \Phi_t (1-\alpha)Y_t^{-\alpha}(Y_t^*)^\alpha = -A_t U_c(C_t,N_t) (1-\alpha)Y_t^{-\alpha}(Y_t^*)^\alpha\)

\(\displaystyle -\frac{U_N(C_t,N_t)}{U_c(C_t,N_t)} = \frac{Y_t}{N_t} (1-\alpha)Y_t^{-\alpha}(Y_t^*)^\alpha=(1-\alpha) \frac{C_t}{N_t}\)

Using the utility function
\(\displaystyle U(C_t,N_t)=\log(C_t)-\frac{N_t^{1+\varphi}}{1+\varphi}\)
as \(\sigma=1\).

We have \(\displaystyle U_C(C_t,N_t)=\frac{1}{C_t}\) and
\(\displaystyle U_N(C_t,N_t)=-N_t^\varphi\). Substituting in the FOCs
relation, we have
\(\displaystyle N_t^\varphi C_t=(1-\alpha)\frac{C_t}{N_t} \ \ \Rightarrow \ \ N_t^{1+\varphi}=(1-\alpha) \ \ \Rightarrow \ \ N_t = N = (1-\alpha)^{\frac{1}{1+\varphi}}\),
which is a constant employment.

We know that, under flexible prices,
\(\displaystyle \frac{\overline{MC}_t^n}{\overline{P}_{H,t}}=\overline{MC}_t=\frac{\varepsilon-1}{\varepsilon}\)
(see section 9 for the derivation).

From the representative consumer FOCs (problem of item 5), we have the
standard relation:
\(\displaystyle -\frac{U_C(\overline{C}_t,\overline{N}_t)}{U_N(\overline{C}_t,\overline{N}_t)}=\frac{\overline{W}_t}{\overline{P}_t}\).
Thus,

The subsidy \(\tau\) is chosen to achieve the optimal level of
production if the prices were fully flexible. Thus, we have
\(\displaystyle \overline{MC}_t=\frac{\varepsilon-1}{\varepsilon}=\frac{\overline{W}_t (1-\tau)}{\overline{P}_{H,t} A_t}=-\frac{U_C(\overline{C}_t,\overline{N}_t)}{U_N(\overline{C}_t,\overline{N}_t)}\frac{\overline{P}_t (1-\tau)}{\overline{P}_{H,t} A_t}\).

As \(\eta=1\),
\(\overline{P}_t=(\overline{P}_{H,t})^{1-\alpha}(\overline{P}_{F,t})^\alpha\).
Substituting,

\(\displaystyle \overline{MC}_t= -\frac{U_C(\overline{C}_t,\overline{N}_t)}{U_N(\overline{C}_t,\overline{N}_t)}\frac{(\overline{P}_{H,t})^{1-\alpha}(\overline{P}_{F,t})^\alpha (1-\tau)}{\overline{P}_{H,t} \overline{A}_t}= -\frac{U_C(\overline{C}_t,\overline{N}_t)}{U_N(\overline{C}_t,\overline{N}_t)} \left( \frac{\overline{P}_{F,t}}{\overline{P}_{H,t}} \right)^\alpha \frac{1-\tau}{\overline{A}_t}= -\frac{(1-\tau)}{\overline{A}_t} (\mathcal{\overline{S}}_i)^\alpha \frac{U_C(\overline{C}_t,\overline{N}_t)}{U_N(\overline{C}_t,\overline{N}_t)}\)

As \(\sigma=1\),
\(\displaystyle -\frac{U_C(\overline{C}_t,\overline{N}_t)}{U_N(\overline{C}_t,\overline{N}_t)}=\overline{C}_t \overline{N}_t^\varphi \ \ \Rightarrow \ \ \overline{MC}_t= \frac{(1-\tau)}{\overline{A}_t} (\mathcal{\overline{S}}_i)^\alpha \overline{C}_t \overline{N}_t^\varphi\)

From equation 26 (equilibrium part), we know that
\(\displaystyle \overline{Y}_t=\overline{C}_t \mathcal{\overline{S}}_t^\alpha \ \ \Rightarrow \ \  \mathcal{\overline{S}}_t^\alpha = \frac{\overline{Y}_t}{\overline{C}_t}\).
Thus,
\(\displaystyle \overline{MC}_t= \frac{(1-\tau)}{\overline{A}_t} \frac{\overline{Y}_t}{\overline{C}_t} \overline{C}_t \overline{N}_t^\varphi\)

Substituting the technological constraint,
\(\displaystyle 1-\frac{1}{\varepsilon}=\overline{MC}_t= \frac{(1-\tau)}{\overline{A}_t} \overline{A}_t \overline{N}_t \overline{N}_t^\varphi=(1-\tau) \overline{N}_t^{1+\varphi}=(1-\tau)\left( (1-\alpha)^{1+\varphi} \right)^{\frac{1}{1+\varphi}}=(1-\tau)(1-\alpha)\).

Hence, if \(\tau\) is set to satisfy
\(\displaystyle (1-\tau)(1-\alpha)= 1-\frac{1}{\varepsilon}\), we have
also the log-linear form \(\nu=\mu+\log(1-\alpha)\), where \(\mu\)
(defined earlier) is \(\log(1-\tau)\) and the flexible price allocation
is guaranteed.

\hypertarget{optimal-policy-implementation}{%
\subsection{Optimal policy
Implementation}\label{optimal-policy-implementation}}

Solving forward equation (36), we have
\(\displaystyle \pi_{H,t}=\beta E_t\{ \pi_{H,t+1} \}+ \kappa_\alpha x_t = \beta E_t\{ \beta E_t \{ \pi_{H,t+2} \}+ \kappa_\alpha x_{t+1}\}+ \kappa_\alpha x_t=\beta^T E_t\{\pi_{H,T}\} + \sum_{j=t}^T \beta^{j-t}\kappa_\alpha x_j\)

To stabilize inflation,

\(\displaystyle \pi_{H,t} =\underset{T \rightarrow \infty} \lim\beta^T E_t\{\pi_{H,T}\} + \sum_{j=t}^T \beta^{j-t}\kappa_\alpha x_j\)
which will stabilized (\(\pi_{H,t}=0\)) only if the output gap is zero
for every period. As
\(x_t \equiv y_t-\overline{y}_t \ \ \Rightarrow \ \ y_t=\overline{y}_t\).

In equation (37) we have
\(\displaystyle x_t = E_t\{x_{t+1}\} -\frac{1}{\sigma_\alpha}(r_t-E_t\{\pi_{H,t+1}\} -\overline{rr}_t)\).
In this case, we have \(E_t\{x_{t+1}\}=0\), \(E_t\{\pi_{H,t+1}\}=0\),
and \(\sigma_\alpha=1\), as

\(\displaystyle \sigma_\alpha \equiv \frac{\sigma}{(1-\alpha)+\alpha \omega}=\frac{1}{(1-\alpha)+\alpha (\sigma \gamma+(\sigma \eta -1 )(1-\alpha))}=\frac{1}{(1-\alpha)+\alpha (1+(1 -1 )(1-\alpha))}=1\).

Thus, \(r_t=\overline{rr}_t\).

Now we suppose that the Central Bank follows this rule when the economy
is not in its steady-state:
\(r_t=\overline{rr}_t+\phi_\pi \pi_{H,t}+\phi_x x_t\).

After setting \(r_t=\overline{rr}_t\) in a closed economy we have
\(\displaystyle x_t = E_t\{x_{t+1}\} -\frac{1}{\sigma_\alpha}(r_t-E_t\{\pi_{H,t+1}\} -\overline{rr}_t) = E_t\{x_{t+1}\} -\frac{1}{\sigma}(\overline{rr}_t-E_t\{\pi_{H,t+1}\} -\overline{rr}_t) = E_t\{x_{t+1}\} -\frac{1}{\sigma}E_t\{\pi_{H,t+1}\}\)

We have now a system with 2 equations

\(\displaystyle x_t = E_t\{x_{t+1}\} +\frac{1}{\sigma}E_t\{\pi_{H,t+1}\}\)

\(\pi_{H,t}=\beta E_t\{ \pi_{H,t+1}\} + \kappa_\alpha x_t\),

which can be summarized as
\(\displaystyle \left[ \begin{matrix} 1 & 0\\ -\kappa & 1 \end{matrix} \right] \left[ \begin{matrix} x_t\\ \pi_t \end{matrix} \right] = \left[ \begin{matrix} 1 & \sigma^{-1} \\ 0 & \beta \end{matrix} \right] \left[ \begin{matrix} E_t\{x_{t+1} \}\\ E_t \{\pi_{t+1} \} \end{matrix} \right]\)

\(\displaystyle \left[ \begin{matrix} x_t\\ \pi_t \end{matrix} \right] = \left[ \begin{matrix} 1 & 0\\ \kappa & 1 \end{matrix} \right] \left[ \begin{matrix} 1 & \sigma^{-1} \\ 0 & \beta \end{matrix} \right] \left[ \begin{matrix} E_t\{x_{t+1} \}\\ E_t \{\pi_{t+1} \} \end{matrix} \right] = \left[ \begin{matrix} 1 & \sigma^{-1} \\ \kappa & \kappa \sigma^{-1}+ \beta \end{matrix} \right] \left[ \begin{matrix} E_t\{x_{t+1} \}\\ E_t \{\pi_{t+1} \} \end{matrix} \right] \ \ \Rightarrow \ \ \left[ \begin{matrix} x_t\\ \pi_t \end{matrix} \right] = \left[ \begin{matrix} 1 & \sigma^{-1} \\ \kappa & \beta +\kappa \sigma^{-1} \end{matrix} \right] \left[ \begin{matrix} E_t\{x_{t+1} \}\\ E_t \{\pi_{t+1} \} \end{matrix} \right]\).

Defining
\(\mathbf{A_O}=\left[ \begin{matrix} 1 & \sigma^{-1} \\ \kappa & \beta +\kappa \sigma^{-1} \end{matrix} \right]\),
we can calculate its eigenvalues:

\(\left| \begin{matrix} 1-\xi & \sigma^{-1} \\ \kappa & \beta +\kappa \sigma^{-1}-\xi \end{matrix} \right|=(1-\xi)(\beta +\kappa \sigma^{-1}-\xi)-\kappa \sigma^{-1}= \beta +\kappa \sigma^{-1}-\xi - \beta \xi -\kappa \sigma^{-1} \xi+\xi^2 -\kappa \sigma^{-1}\)

Now we have a quadratic function whose roots are the eigenvalues of the
system \(f(\xi)=\xi^2-(1+\beta+\kappa \sigma^{-1})\xi+\beta\),

with the product of roots being \(\beta<1\) and the sum of the roots
greater than 1 \(1+\beta+\kappa \sigma^{-1}\).

We can see that if \(f(\xi)>0\) if \(\xi \rightarrow \infty\),
\(f(0)=\beta>0\) and
\(f(1)=1^2-(1+\beta+\kappa \sigma^{-1})1+\beta=-\kappa \sigma^{-1}<0\),
as
\(\displaystyle \kappa=\lambda(\sigma+\varphi)=\frac{(1-\beta \theta)(1-\theta)}{\theta}(\sigma+\varphi)>0\).
Thus, we have one root between 0 and 1 and another one greater than 1
(from the intermediate value theorem). As there's one eigenvalue outside
the unit root circle, there are infinite solution for this system, as
both variables are forward looking.

Now if the Central bank has commits to the rule
\(r_t=\overline{rr}_t+\phi_\pi \pi_t+\phi_x x_t\), we have

\(\displaystyle x_t = E_t\{x_{t+1}\} -\frac{1}{\sigma}(\overline{rr}_t+\phi_\pi \pi_{t}+\phi_x x_t-E_t\{\pi_{t+1}\} -\overline{rr}_t)= -\phi_\pi \sigma^{-1} -\pi_{t}+\phi_x \sigma^{-1} x_t + E_t\{x_{t+1}\}+ \sigma^{-1}E_t\{\pi_{t+1}\}\)

and the system becomes

\(\displaystyle (1+\phi_x \sigma^{-1})x_t +\phi_\pi \sigma^{-1}\pi_{t} = E_t\{x_{t+1}\} +\sigma^{-1}E_t\{\pi_{t+1}\}\)

\(-\kappa x_t +\pi_{H,t}=\beta E_t\{ \pi_{H,t+1}\}\),

\(\displaystyle \left[ \begin{matrix} 1+\phi_x \sigma^{-1} & \phi_\pi \sigma^{-1}\\ -\kappa & 1 \end{matrix} \right] \left[ \begin{matrix} x_t\\ \pi_t \end{matrix} \right] = \left[ \begin{matrix} 1 & \sigma^{-1} \\ 0 & \beta \end{matrix} \right] \left[ \begin{matrix} E_t\{x_{t+1} \}\\ E_t \{\pi_{t+1} \} \end{matrix} \right]\)

\(\displaystyle \left[ \begin{matrix} x_t\\ \pi_t \end{matrix} \right] = \frac{1}{1+\phi_x \sigma^{-1}+ \phi_\pi \sigma^{-1} \kappa} \left[ \begin{matrix} 1 & -\phi_\pi \sigma^{-1}\\ \kappa & 1+\phi_x \sigma^{-1} \end{matrix} \right] \left[ \begin{matrix} 1 & \sigma^{-1} \\ 0 & \beta \end{matrix} \right] \left[ \begin{matrix} E_t\{x_{t+1} \}\\ E_t \{\pi_{t+1} \} \end{matrix} \right]=\frac{\sigma}{\sigma+\phi_x + \phi_\pi \kappa}\left[ \begin{matrix} 1& \sigma^{-1}(1-\beta \phi_\pi)\\ \kappa & \sigma^{-1}(\kappa+\beta \sigma+\beta\phi_x) \end{matrix} \right] \left[ \begin{matrix} E_t\{x_{t+1} \}\\ E_t \{\pi_{t+1} \} \end{matrix} \right]\)

\(\displaystyle \left[ \begin{matrix} x_t\\ \pi_t \end{matrix} \right] =\frac{1}{\sigma+\phi_x + \phi_\pi \kappa}\left[ \begin{matrix} \sigma& 1-\beta \phi_\pi\\ \kappa \sigma & \kappa+\beta (\sigma+\phi_x) \end{matrix} \right] \left[ \begin{matrix} E_t\{x_{t+1} \}\\ E_t \{\pi_{t+1} \} \end{matrix} \right]=\mathbf{A}_T\left[ \begin{matrix} E_t\{x_{t+1} \}\\ E_t \{\pi_{t+1} \} \end{matrix} \right]\),
where

\(\displaystyle \mathbf{A}_T \equiv\Omega\left[ \begin{matrix} \sigma& 1-\beta \phi_\pi\\ \kappa \sigma & \kappa+\beta (\sigma+\phi_x) \end{matrix} \right]\)
and
\(\displaystyle \Omega \equiv\frac{1}{\sigma+\phi_x + \phi_\pi\kappa}\)

If we restrict \(\phi_\pi>0\) and \(\phi_x>0\), \(\Omega>0\).

To satisfy the Blanchard and Khan conditions, we need that both
eigenvalues are inside the unit circle.

\(|\mathbf{A}_T-\lambda \mathbf{I}| = 0\)

\(\displaystyle \left| \frac{1}{\sigma+\phi_x + \phi_\pi \kappa}\left[ \begin{matrix} \sigma & 1-\beta \phi_\pi\\ \kappa \sigma & \kappa+\beta (\sigma+\phi_x) \end{matrix} \right] -\lambda \left[ \begin{matrix} 1 & 0\\ 0 & 1 \end{matrix} \right] \right|=0\)

\(\displaystyle \left| \left[ \begin{matrix} \displaystyle \frac{\sigma}{\sigma+\phi_x + \phi_\pi \kappa}-\lambda & \displaystyle \frac{1-\beta \phi_\pi}{\sigma+\phi_x + \phi_\pi \kappa}\\ \displaystyle \frac{\kappa \sigma}{\sigma+\phi_x + \phi_\pi \kappa} & \displaystyle \frac{\kappa+\beta (\sigma+\phi_x)}{\sigma+\phi_x + \phi_\pi \kappa}-\lambda \end{matrix} \right] \right|=0\)

\(\displaystyle \frac{\sigma \kappa + \beta\sigma (\sigma+\phi_y)}{(\sigma+\phi_x + \phi_\pi \kappa)^2}- \frac{\sigma +\kappa + \beta (\sigma+ \phi_y)}{(\sigma+\phi_x + \phi_\pi \kappa)} \lambda + \lambda^2-\frac{\sigma \kappa-\beta \phi_\pi \sigma \kappa}{(\sigma+\phi_x + \phi_\pi \kappa)^2}=\lambda^2- \frac{\sigma +\kappa + \beta (\sigma+ \phi_y)}{(\sigma+\phi_x + \phi_\pi \kappa)} \lambda + \frac{\sigma \beta(\sigma+\phi_y+\phi_\pi \kappa)}{(\sigma+\phi_x + \phi_\pi \kappa)^2}=0\)

\(\displaystyle \lambda^2- \frac{\sigma +\kappa + \beta (\sigma+ \phi_y)}{\sigma+\phi_x + \phi_\pi \kappa} \lambda + \frac{\sigma \beta}{\sigma+\phi_x + \phi_\pi \kappa}=0\)

LaSalle (1986) showed that both roots of the equation \(x^2+bx+c=0\) are
less than 1 if and only if \(|c|<1\) and \(|b|<1+c\). This comment was
taken from Drago Bergholt notes.

We have that
\(\displaystyle \left| \frac{\sigma \beta}{\sigma+ \phi_y+\phi_\pi \kappa} \right|<1\)

As \(\sigma>0\), \(\beta>0\), \(\kappa>0\), \(\phi_\pi>0\) and
\(\phi_y>0\)

\(\displaystyle \frac{\sigma \beta}{\sigma+ \phi_y+\phi_\pi \kappa} <1 \ \ \Rightarrow \ \ \sigma \beta<\sigma+ \phi_y+\phi_\pi \kappa \ \ \Rightarrow \ \ \sigma (\beta-1)< \phi_y+\phi_\pi \kappa\)
This condition is always satisfied, as \(\beta<1\).

The second condition is
\(\displaystyle \left| \frac{\sigma+\kappa+ \beta(\sigma+\phi_y)}{\sigma+ \phi_y+\phi_\pi \kappa} \right|<1+\frac{\sigma \beta}{\sigma+ \phi_y+\phi_\pi \kappa}\)

\(\displaystyle \frac{\sigma+\kappa+ \beta(\sigma+\phi_y)}{\sigma+ \phi_y+\phi_\pi \kappa} < 1+\frac{\sigma \beta}{\sigma+ \phi_y+\phi_\pi \kappa}\)

\(\sigma+\kappa+ \beta(\sigma+\phi_y) < \sigma+ \phi_y+\phi_\pi \kappa + \sigma \beta \ \ \Rightarrow \ \ \kappa + \beta \phi_y<\phi_y+\phi_\pi \kappa \ \ \Rightarrow \ \ \kappa (\phi_\pi-1)+\phi_y(1-\beta)>0\)

\hypertarget{macroeconomic-implications}{%
\subsection{Macroeconomic
implications}\label{macroeconomic-implications}}

By equation (35)
\(\displaystyle \overline{y}_t=\Omega+\Gamma a_t -\alpha \Psi y_t^*=\frac{\nu-\mu}{\sigma_\alpha+\varphi}+\frac{1+\varphi}{\sigma_\alpha+\varphi} a_t -\alpha \frac{\Theta \sigma_\alpha}{\sigma_\alpha+\varphi}y_t^*\),
we can see that a technological shock always increase the output level,
as \(\sigma_\alpha>0\) and \(\varphi>0\).

Computing the natual level of the terms of trade, we have

\(\displaystyle \overline{s}_t=\sigma_\alpha (\overline{y}_t-y_t^*)=\sigma_\alpha \left(\frac{\nu-\mu}{\sigma_\alpha+\varphi}+\frac{1+\varphi}{\sigma_\alpha+\varphi} a_t -\alpha \frac{\Theta \sigma_\alpha}{\sigma_\alpha+\varphi}y_t^* -y_t^* \right)=\sigma_\alpha \left(\frac{\nu-\mu}{\sigma_\alpha+\varphi}+\frac{1+\varphi}{\sigma_\alpha+\varphi} a_t - \frac{\alpha \Theta \sigma_\alpha+\sigma_\alpha+\varphi}{\sigma_\alpha+\varphi}y_t^* \right)\)

\(\displaystyle \overline{s}_t=\sigma_\alpha \left(\frac{\nu-\mu}{\sigma_\alpha+\varphi}+\frac{1+\varphi}{\sigma_\alpha+\varphi} a_t - \frac{\alpha (\omega-1) \sigma_\alpha+\sigma_\alpha+\varphi}{\sigma_\alpha+\varphi}y_t^* \right)=\sigma_\alpha \left(\frac{\nu-\mu}{\sigma_\alpha+\varphi}+\frac{1+\varphi}{\sigma_\alpha+\varphi} a_t - \frac{(\alpha \omega- \alpha+1)\displaystyle \frac{\sigma}{1-\alpha+\alpha \omega} +\varphi}{\sigma_\alpha+\varphi}y_t^* \right)\)

\(\displaystyle \overline{s}_t=\sigma_\alpha \left(\frac{\nu-\mu}{\sigma_\alpha+\varphi}+\frac{1+\varphi}{\sigma_\alpha+\varphi} a_t - \frac{\sigma +\varphi}{\sigma_\alpha+\varphi}y_t^* \right)=\sigma_\alpha \Omega+ \sigma_\alpha \Gamma a_t - \sigma_\alpha \Phi y_t^*\),
where
\(\displaystyle \Phi=\frac{\sigma+\varphi}{\sigma_\alpha+\varphi}>0\)

As
\(\overline{p}_t=(1-\alpha) \overline{p}_{H,t}+\alpha\overline{p}_{F,t} = (1-\alpha)\overline{p}_{H,t}+\alpha(\overline{e}_{t}+p_t^*)\).
As domestic prices are fully stabilized,
\((1-\alpha)\overline{p}_{H,t}\) is a constant, so \(\overline{p}_t\) is
proportional to
\(\alpha(\overline{e}_{t}+p_t^*) = \alpha \overline{s}_t\)

\hypertarget{the-welfare-costs-of-deviations-from-the-optimal}{%
\subsection{The welfare costs of deviations from the
optimal}\label{the-welfare-costs-of-deviations-from-the-optimal}}

This section starts with the calculations of the appendix D. By the
Taylor's rule, the second order approximation is:

\(\displaystyle f(x_t) = f(a) + f'(a)(x_t-a) + \frac{1}{2}f''(a)(x_t-a)^2 + \frac{1}{6}f'''(\widetilde{a})(x_t-a)^3\).
We assume that the third term is small, as the deviations from the
steady-state are assumed to be small (by the intermediate value theorem,
\(\widetilde{a}\) is between \(x_t\) and \(a\).

We can take

\(\displaystyle \frac{Y_t}{Y}=e^{\ln \frac{Y_t}{Y}}=1+\ln \frac{Y_t}{Y}+\frac{1}{2} \left( \ln \frac{Y_t}{Y} \right)^2 + \frac{1}{3!}\left( \ln \frac{Y_t}{Y} \right)^3 + ... = 1+y_t+\frac{1}{2} \left( y_t \right)^2+o(||a||^n)\),
there \(a\) is the bound for the high order terms.

\(\displaystyle \frac{Y_t-Y}{Y}= 1+y_t+\frac{y_t^2}{2} +o(||a||^n)\)

Combining equations (18):
\(\displaystyle c_t=c_t^*+ \left(\frac{1-\alpha}{\sigma} \right) s_t\)
and (29):
\(\displaystyle y_t = y_t^*+\frac{1}{\sigma_\alpha}s_t = y_t^*+\frac{1}{\sigma}s_t\)
as \(\omega=1\), we have:

\(\displaystyle c_t=c_t^*+\left( \frac{1-\alpha}{\sigma} \right)\sigma(y_t-y_t^*)\).
As \(c_t^*=y_t^*\) because of global market clearing,
\(c_t=y_t^* + y_t -y_t^* -\alpha y_t + \alpha y_t^* \ \ \Rightarrow \ \ c_t = (1-\alpha)y_t+\alpha y_t^*\).

As \(x_t \equiv y_t-\overline{y}_t\), in the stabilized economy,
\(x_t=0\) and \(y_t=\overline{y}_t\). Thus,
\(\overline{c}_t = (1-\alpha)(0-\overline{y}_t)+\alpha y_t^*=\alpha y_t^*-(1-\alpha)\overline {y}_t\)

Substituting, we get:
\(c_t = (1-\alpha)(\overline{y}_t+x_t)+\alpha y_t^*=(1-\alpha)\overline{y}_t+\alpha y_t^* +(1-\alpha)x_t \ \ \Rightarrow \ \ c_t=\overline{c}_t+(1-\alpha)x_t\).

Expanding the log-deviation of the disutility of work, we have:

\(\displaystyle \left( \frac{N_t}{\overline{N} } \right)^{1+\varphi}=\exp[(1+\varphi)\widetilde{n}]=1+(1+\varphi)\widetilde{n}_t + \frac{1}{2}\widetilde{n}_t^2 + o(||a||^3) \ \ \Rightarrow \ \ N_t^{1+\varphi} = \overline{N}^{1+\varphi} \left( 1+(1+\varphi)\widetilde{n}_t + \frac{1}{2}\widetilde{n}_t^2 + o(||b||^3) \right)\)

\(\displaystyle \frac{N_t^{1+\varphi}}{1+\varphi} = \frac{\overline{N}^{1+\varphi}}{1+\varphi} +\overline{N}^{1+\varphi}\left[ \widetilde{n}_t + \frac{1}{2}(1+\varphi)\widetilde{n}_t^2 \right] + o(||a||^3)\)

Using the fact that
\(\displaystyle N_t=\left(\frac{Y_t}{A_t} \right) \int_0^1 \left(\frac{P_{H,t}(i)}{P_{H,t}} \right)^{-\varepsilon}di\),
\(\displaystyle \int_0^1 \left(\frac{P_{H,t}(i)}{P_{H,t}} \right)^{-\varepsilon}di=\frac{N_t A_t}{Y_t} \ \ \Rightarrow \ \ \log \int_0^1 \left(\frac{P_{H,t}(i)}{P_{H,t}} \right)^{-\varepsilon}di= \log \left( \frac{N_t A_t}{Y_t} \right)=n_t+a_t-y_t\)

If we define
\(\displaystyle z_t \equiv \log \int_0^1 \left(\frac{P_{H,t}(i)}{P_{H,t}} \right)^{-\varepsilon}di\),
then \(z_t=n_t+a_t-y_t=n_t+a_t-(\overline{y}_t+x_t)\).

When prices are stabilized, \(P_{H,t}(i)=P_{H,t}\) and
\(\overline{z}_t=0\). Also, there are no productivity shocks, so
\(a_t= \overline{y}_t-\overline{n}_t\). Thus,

\(z_t=\overline{n}_t+\widetilde{n}_t+\overline{y}_t-\overline{n}_t-(\overline{y}_t+x_t) \ \ \Rightarrow \ \ \widetilde{n}_t=z_t+x_t\)

Lemma 1 (appendix D): The proof is there. There's just one passage that
it took time to figure out what happened. From

From
\(\displaystyle E_i \{ \widehat{p}_{H,t}(i)\} = \frac{(\varepsilon-1)}{2}E_i \{ \widehat{p}_{H,t}(i)^2\}\)
and
\(\displaystyle \left(\frac{P_{H,t}(i)}{P_{H,t}} \right)^{-\varepsilon}=1-\varepsilon \widehat{p}_{H,t}(i)+ \frac{\varepsilon^2}{2} \widehat{p}_{H,t}(i)^2+o(||a||^3)\)

\(\displaystyle E_i\left[ \left(\frac{P_{H,t}(i)}{P_{H,t}} \right)^{-\varepsilon} \right]= E_i \left[ 1-\varepsilon \widehat{p}_{H,t}(i)+ \frac{\varepsilon^2}{2} \widehat{p}_{H,t}(i)^2+o(||a||^3) \right]\)

\(\displaystyle \int_0^1 \left(\frac{P_{H,t}(i)}{P_{H,t}} \right)^{-\varepsilon} di= 1-\varepsilon e_i[\widehat{p}_{H,t}(i)]+ \frac{\varepsilon^2}{2} E_i[\widehat{p}_{H,t}(i)^2]=1-\varepsilon \frac{(\varepsilon-1)}{2}E_i \{ \widehat{p}_{H,t}(i)^2\} + \frac{\varepsilon^2}{2} E_i[\widehat{p}_{H,t}(i)^2]=1+\frac{\varepsilon}{2}E_i \{ \widehat{p}_{H,t}(i)^2\}=1+\frac{\varepsilon}{2}var_i\{p_{H,t}(i)\}\)

\(\displaystyle z_t= \log \int_0^1 \left(\frac{P_{H,t}(i)}{P_{H,t}} \right)^{-\varepsilon} di = \log \left( 1+\frac{\varepsilon}{2}var_i\{p_{H,t}(i)\} \right)= \frac{\varepsilon}{2}var_i\{p_{H,t}(i)\} +o(||a||^3)\)

Rewriting the second-order approximation disutility of labor:

\(\displaystyle \frac{N_t^{1+\varphi}}{1+\varphi} = \frac{\overline{N}^{1+\varphi}}{1+\varphi} +\overline{N}^{1+\varphi}\left[ x_t+z_t + \frac{1}{2}(1+\varphi)(x_t+z_t)^2 \right] + o(||a||^3)\)

\(\displaystyle \frac{N_t^{1+\varphi}}{1+\varphi} = \frac{\overline{N}^{1+\varphi}}{1+\varphi} +\overline{N}^{1+\varphi}\left[ x_t+z_t + \frac{1}{2}(1+\varphi)(x_t^2+2 x_tz_t + z_t^2) \right] + o(||a||^3)\)

As \(z_t\) is a variance (second-order term), \(z_t^2\) and \(x_t z_t\)
are terms of greater order, so they can be included in the remanescent
terms \(o(||a||^3)\). Thus, we get

\(\displaystyle \frac{N_t^{1+\varphi}}{1+\varphi} = \frac{\overline{N}^{1+\varphi}}{1+\varphi} +\overline{N}^{1+\varphi}\left[ x_t+z_t + \frac{1}{2}(1+\varphi)x_t^2 \right] + o(||a||^3)\).

Under the optimal subsidy assumption, from the consumer's FOC, we have
that \(\overline{N}_t^{1+\varphi}=(1-\alpha)\) (constant employment).
Thus,

\(\displaystyle U(C_t,N_t)\equiv \frac{C_t^{1-\sigma}}{1-\sigma}-\frac{N_t^{1+\varphi}}{1+\varphi}= \frac{C_t^{1-\sigma}}{1-\sigma}- \frac{\overline{N}^{1+\varphi}}{1+\varphi} -\overline{N}^{1+\varphi}\left[ x_t+z_t + \frac{1}{2}(1+\varphi)x_t^2 \right] + o(||a||^3)\)

\(\displaystyle U(C_t,N_t) = - \frac{(1-\alpha)}{1+\varphi}-(1-\alpha)\left[ x_t+z_t + \frac{1}{2}(1+\varphi)x_t^2 \right] + \frac{C_t^{1-\sigma}}{1-\sigma}+ o(||a||^3)\)

\(\displaystyle U(C_t,N_t) = -(1-\alpha)\left[ x_t+z_t + \frac{1}{2}(1+\varphi)x_t^2 \right] + \frac{C_t^{1-\sigma}}{1-\sigma}- \frac{(1-\alpha)}{1+\varphi} o(||a||^3)= -(1-\alpha)\left[ z_t + \frac{1}{2}(1+\varphi)x_t^2 \right] + t.i.p+ o(||a||^3)\),
there t.i.p. denotes terms independent of policy and under the optimal
policy, \(x_t=0\).

Lemma 2 from Woodford(2003):
\(\displaystyle \sum_{t=0}^\infty \beta^t var_i[p_{H,t}(i)]= \frac{1}{\lambda} \sum_{t=0}^\infty \beta^t \pi_{H,t}^2\),
where
\(\displaystyle \lambda \equiv \frac{(1-\theta)(1-\beta \theta)}{\theta}\)

\(\displaystyle \mathbb{W}= \sum_{i=0}^{\infty} \beta^tU(C_t,N_t) = \sum_{i=0}^{\infty} \beta^t \left[-(1-\alpha)\left( z_t + \frac{1}{2}(1+\varphi)x_t^2 \right) + t.i.p+ o(||a||^3) \right]\)

\(\displaystyle \mathbb{W}= \sum_{i=0}^{\infty} \beta^t \left[-(1-\alpha)\left( \frac{\varepsilon}{2}var_i[p_{H,t}(i)] +o(||a||^3) + \frac{1}{2}(1+\varphi)x_t^2 \right) \right]+ t.i.p+ o(||a||^3)\).

Using Lemma 2, we have:

\(\displaystyle \mathbb{W}= -\frac{(1-\alpha)}{2}\sum_{i=0}^{\infty} \beta^t \left[ \frac{\varepsilon}{\lambda}\pi_{H,t}^2+ (1+\varphi)x_t^2 \right] + t.i.p+ o(||a||^3)\)

Now
\(\displaystyle \mathbb{E}_t \left[ -\frac{(1-\alpha)}{2}\sum_{i=0}^{\infty} \beta^t \left[ \frac{\varepsilon}{\lambda}\pi_{H,t}^2+ (1+\varphi)x_t^2 \right] + t.i.p+ o(||a||^3) \right] = -\frac{(1-\alpha)}{2}\sum_{i=0}^{\infty} \beta^t \left[ \frac{\varepsilon}{\lambda}var(\pi_{H,t})+ (1+\varphi)var(x_t) \right]\)
as \(E[x_t]=E[\pi_t]=0\)

If \(\beta \to 1\), any policy deviation in a period can be calculated
as

\(\displaystyle \mathbb{V} = -\frac{(1-\alpha)}{2}\sum_{i=0}^{\infty} \beta^t \left[ \frac{\varepsilon}{\lambda}var(\pi_{H,t})+ (1+\varphi)var(x_t) \right]\)

\end{document}
