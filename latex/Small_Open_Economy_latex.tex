% Options for packages loaded elsewhere
\PassOptionsToPackage{unicode}{hyperref}
\PassOptionsToPackage{hyphens}{url}
%
\documentclass[
]{article}
\usepackage{amsmath,amssymb}
\usepackage{lmodern}
\usepackage{iftex}
\ifPDFTeX
  \usepackage[T1]{fontenc}
  \usepackage[utf8]{inputenc}
  \usepackage{textcomp} % provide euro and other symbols
\else % if luatex or xetex
  \usepackage{unicode-math}
  \defaultfontfeatures{Scale=MatchLowercase}
  \defaultfontfeatures[\rmfamily]{Ligatures=TeX,Scale=1}
\fi
% Use upquote if available, for straight quotes in verbatim environments
\IfFileExists{upquote.sty}{\usepackage{upquote}}{}
\IfFileExists{microtype.sty}{% use microtype if available
  \usepackage[]{microtype}
  \UseMicrotypeSet[protrusion]{basicmath} % disable protrusion for tt fonts
}{}
\makeatletter
\@ifundefined{KOMAClassName}{% if non-KOMA class
  \IfFileExists{parskip.sty}{%
    \usepackage{parskip}
  }{% else
    \setlength{\parindent}{0pt}
    \setlength{\parskip}{6pt plus 2pt minus 1pt}}
}{% if KOMA class
  \KOMAoptions{parskip=half}}
\makeatother
\usepackage{xcolor}
\usepackage[left=1cm, right=1cm, top=1cm, bottom=1.5cm]{geometry}
\usepackage{graphicx}
\makeatletter
\def\maxwidth{\ifdim\Gin@nat@width>\linewidth\linewidth\else\Gin@nat@width\fi}
\def\maxheight{\ifdim\Gin@nat@height>\textheight\textheight\else\Gin@nat@height\fi}
\makeatother
% Scale images if necessary, so that they will not overflow the page
% margins by default, and it is still possible to overwrite the defaults
% using explicit options in \includegraphics[width, height, ...]{}
\setkeys{Gin}{width=\maxwidth,height=\maxheight,keepaspectratio}
% Set default figure placement to htbp
\makeatletter
\def\fps@figure{htbp}
\makeatother
\setlength{\emergencystretch}{3em} % prevent overfull lines
\providecommand{\tightlist}{%
  \setlength{\itemsep}{0pt}\setlength{\parskip}{0pt}}
\setcounter{secnumdepth}{-\maxdimen} % remove section numbering
\usepackage{bbm}
\usepackage{amsmath}
\ifLuaTeX
  \usepackage{selnolig}  % disable illegal ligatures
\fi
\IfFileExists{bookmark.sty}{\usepackage{bookmark}}{\usepackage{hyperref}}
\IfFileExists{xurl.sty}{\usepackage{xurl}}{} % add URL line breaks if available
\urlstyle{same} % disable monospaced font for URLs
\hypersetup{
  pdftitle={Small Open Economy Model},
  pdfauthor={Anna Catarina Tavella e Matheus Franciscão},
  hidelinks,
  pdfcreator={LaTeX via pandoc}}

\title{Small Open Economy Model}
\author{Anna Catarina Tavella e Matheus Franciscão}
\date{2022-xx-xx}

\begin{document}
\maketitle

\hypertarget{problem-of-the-consumer-for-reference}{%
\subsection{1 - Problem of the consumer (for
reference)}\label{problem-of-the-consumer-for-reference}}

\(\displaystyle {max} \ \mathbb{E}_0 \sum_{t=0}^\infty \beta^tU(C_t,N_t) = \displaystyle {max} \ E_0 \sum_{t=0}^\infty \beta^tU \left(\left[ (1-\alpha)^{\frac{1}{\eta}} (C_{H,t})^{\frac{\eta-1}{\eta}} + \alpha^{\frac{1}{\eta}} (C_{F,t})^{\frac{\eta-1}{\eta}} \right]^{\frac{\eta}{\eta-1}},N_t \right)\)

subject to the budget constraint (specified below), where

\(C_{H,t} \equiv \displaystyle \left( \int_0^1 C_{H,t}(j)^{\frac{\varepsilon-1}{\varepsilon}}dj \right) ^{\frac{\varepsilon}{\varepsilon-1}}\),
\(C_{F,t} \equiv \displaystyle \left( \int_0^1 (C_{i,t})^{\frac{\gamma-1}{\gamma}}di \right) ^{\frac{\gamma}{\gamma-1}}\),
\(C_{i,t} \equiv \displaystyle \left( \int_0^1 C_{i,t}(j)^{\frac{\varepsilon-1}{\varepsilon}}dj \right) ^{\frac{\varepsilon}{\varepsilon-1}}\)

and
\(C_t \equiv \left[ (1-\alpha)^{\frac{1}{\eta}} (C_{H,t})^{\frac{\eta-1}{\eta}} + \alpha^{\frac{1}{\eta}} (C_{F,t})^{\frac{\eta-1}{\eta}} \right]^{\frac{\eta}{\eta-1}}\)

Substituting, we get

\(\displaystyle {max} \ \mathbb{E}_0 \sum_{t=0}^\infty \beta^tU \left(\left[ (1-\alpha)^{\frac{1}{\eta}} \left [\displaystyle \left( \int_0^1 C_{H,t}(j)^{\frac{\varepsilon-1}{\varepsilon}}dj \right) ^{\frac{\varepsilon}{\varepsilon-1}} \right]^{\frac{\eta-1}{\eta}} + \alpha^{\frac{1}{\eta}} \left[ \displaystyle \left( \int_0^1 (C_{F,t})^{\frac{\gamma-1}{\gamma}}di \right) ^{\frac{\gamma}{\gamma-1}} \right]^{\frac{\eta-1}{\eta}} \right]^{\frac{\eta}{\eta-1}},N_t \right)\)
=

\(\displaystyle {max} \ \mathbb{E}_0 \sum_{t=0}^\infty \beta^tU \left(\left[ (1-\alpha)^{\frac{1}{\eta}} \left [\displaystyle \left( \int_0^1 C_{H,t}(j)^{\frac{\varepsilon-1}{\varepsilon}}dj \right) ^{\frac{\varepsilon}{\varepsilon-1}} \right]^{\frac{\eta-1}{\eta}} + \alpha^{\frac{1}{\eta}} \left[ \displaystyle \left( \int_0^1 \left( \displaystyle \left( \int_0^1 C_{i,t}(j)^{\frac{\varepsilon-1}{\varepsilon}}dj \right) ^{\frac{\varepsilon}{\varepsilon-1}} \right)^{\frac{\gamma-1}{\gamma}}di \right) ^{\frac{\gamma}{\gamma-1}} \right]^{\frac{\eta-1}{\eta}} \right]^{\frac{\eta}{\eta-1}},N_t \right)\)

subject to the budget constraint:

\(\displaystyle \int_0^1 P_{H,t}(j)C_{H,t}(j)dj + \int_0^1\int_0^1 P_{i,t}(j)C_{i,t}(j)dj\ di + \mathbb{E}_t\{ Q_{t,t+1}D_{t+1}\} \leq D_t + W_tN_t + Tt\)

\(\mathcal{L} = \displaystyle \mathbb{E}_t \sum_{t=0}^\infty \beta^t U \left(\left[ (1-\alpha)^{\frac{1}{\eta}} \left [\displaystyle \left( \int_0^1 C_{H,t}(j)^{\frac{\varepsilon-1}{\varepsilon}}dj \right) ^{\frac{\varepsilon}{\varepsilon-1}} \right]^{\frac{\eta-1}{\eta}} + \alpha^{\frac{1}{\eta}} \left[ \displaystyle \left( \int_0^1 \left( \displaystyle \left( \int_0^1 C_{i,t}(j)^{\frac{\varepsilon-1}{\varepsilon}}dj \right) ^{\frac{\varepsilon}{\varepsilon-1}} \right)^{\frac{\gamma-1}{\gamma}}di \right) ^{\frac{\gamma}{\gamma-1}} \right]^{\frac{\eta-1}{\eta}} \right]^{\frac{\eta}{\eta-1}},N_t \right)\)

\(+ \lambda_t \left(D_t + W_tN_t + Tt \displaystyle - \int_0^1 P_{H,t}(j)C_{H,t}(j)dj - \int_0^1\int_0^1 P_{i,t}(j)C_{i,t}(j)dj\ di - \mathbb{E}_t\{ Q_{t,t+1}D_{t+1}\}\right)\)

\hypertarget{finding-the-demand-function-for-each-specific-good}{%
\subsection{2 - Finding the demand function for each specific
good}\label{finding-the-demand-function-for-each-specific-good}}

\(\displaystyle C_{H,t}(j)= \left( \frac{P_{H,t}(j)}{P_{H,t}}\right)^{-\varepsilon}C_{H,t}\),
\(\displaystyle C_{i,t}(j)= \left( \frac{P_{i,t}(j)}{P_{i,t}}\right)^{-\varepsilon}C_{i,t}\)
and
\(\displaystyle C_{i,t}= \left( \frac{P_{i,t}}{P_{F,t}}\right)^{-\gamma}C_{F,t}\)

It's easier by calculating the MRS (marginal rate of substitution)
between \(C_{H,t}(j)\) and \(C_{H,t}\), as by the optimal allocation, it
has to be the rate of prices in every period of time (otherwise the
consumer could by a little less of the product with relative higher
price and buy another with relative lower price, increasing his
utility).

\(\displaystyle \frac{\partial U(C_t,N_t)}{\partial C_{H,t}(j)} = U_c(C_t,N_t)\frac{\eta}{1-\eta}\left( C_t^{\frac{\eta-1}{\eta}} \right)^{\frac{\eta}{\eta-1}-1} (1-\alpha)^{\frac{1}{\eta}}\frac{\eta-1}{\eta}\left( C_{H,t}\right)^{-\frac{1}{\eta}}\frac{\varepsilon}{\varepsilon-1}\left(C_{H,t}^{\frac{\varepsilon-1}{\varepsilon}} \right)^{\frac{\varepsilon}{\varepsilon-1}-1}\int_0^1 \frac{\varepsilon-1}{\varepsilon} C_{H,t}(j)^{-\frac{1}{\varepsilon}}dj\)

After simplifying, we get

\(\displaystyle \frac{\partial U(C_t,N_t)}{\partial C_{H,t}(j)} = U_c(C_t,N_t) (1-\alpha)^{\frac{1}{\eta}} C_t^{\frac{1}{\eta}}\left( C_{H,t}\right)^{-\frac{1}{\eta}}C_{H,t}^{\frac{1}{\varepsilon}} \int_0^1 C_{H,t}(j)^{-\frac{1}{\varepsilon}}dj = U_c(C_t,N_t) \left[ (1-\alpha) \frac{C_t}{C_{H,t}} \right]^{\frac{1}{\eta}} \int_0^1 \left[\frac{C_{H,t}}{C_{H,t}(j)}\right]^{\frac{1}{\varepsilon}}dj\)

Similarly,

\(\displaystyle \frac{\partial U(C_t,N_t)}{\partial C_{H,t}} = U_c(C_t,N_t)\frac{\eta}{1-\eta}\left( C_t^{\frac{\eta-1}{\eta}} \right)^{\frac{\eta}{\eta-1}-1} (1-\alpha)^{\frac{1}{\eta}}\frac{\eta-1}{\eta}\left( C_{H,t}\right)^{-\frac{1}{\eta}} = U_c(C_t,N_t) (1-\alpha)^{\frac{1}{\eta}} C_t^{\frac{1}{\eta}}\left( C_{H,t}\right)^{-\frac{1}{\eta}} = U_c(C_t,N_t) \left[ (1-\alpha) \frac{C_t}{C_{H,t}} \right]^{\frac{1}{\eta}}\)

\(\displaystyle \frac{\displaystyle \frac{\partial U(C_t,N_t)}{\displaystyle \partial C_{H,t}(j)}}{\frac{\displaystyle \partial U(C_t,N_t)}{\displaystyle \partial C_{H,t}}} = \frac{\displaystyle U_c(C_t,N_t) \left[ (1-\alpha) \frac{C_t}{C_{H,t}} \right]^{\frac{1}{\eta}} \int_0^1 \left[\frac{C_{H,t}}{C_{H,t}(j)}\right]^{\frac{1}{\varepsilon}}dj}{\displaystyle U_c(C_t,N_t) \left[ (1-\alpha) \frac{C_t}{C_{H,t}} \right]^{\frac{1}{\eta}}} = \frac{\displaystyle \int_0^1P_{H,t}(j)dj}{P_{H,t}}\)

After simplifying again, the expression is almost the demand function we
want.

\(\displaystyle \int_0^1 \left[\frac{C_{H,t}}{C_{H,t}(j)}\right]^{\frac{1}{\varepsilon}}dj = \displaystyle \int_0^1 \frac{P_{H,t}(j)}{P_{H,t}}dj\)

As the interval of both integrals are the same and the variable being
integrated is also the same, what is inside the integral in both sides
have also to be the same. So,

\(\displaystyle \left[\frac{C_{H,t}}{C_{H,t}(j)}\right]^{\frac{1}{\varepsilon}} = \frac{P_{H,t}(j)}{P_{H,t}} \Rightarrow \left[\frac{C_{H,t}(j)}{C_{H,t}}\right]^{-\frac{1}{\varepsilon}} = \frac{P_{H,t}(j)}{P_{H,t}} \Rightarrow C_{H,t}(j)= \left[ \frac{P_{H,t}(j)}{P_{H,t}} \right]^{-\varepsilon}C_{H,t}\)

Calculating now the MRS (marginal rate of substitution) between
\(C_{i,t}(j)\) and \(C_{i,t}\), which is also equal the rate of the
prices.

\(\displaystyle \frac{\partial U(C_t,N_t)}{\partial C_{i,t}(j)} =\)

\(\displaystyle U_c(C_t,N_t)\frac{\eta}{1-\eta}\left( C_t^{\frac{\eta-1}{\eta}} \right)^{\frac{\eta}{\eta-1}-1} \alpha^{\frac{1}{\eta}}\frac{\eta-1}{\eta}\left( C_{F,t}\right)^{-\frac{1}{\eta}}\frac{\gamma}{\gamma-1}\left(C_{H,t}^{\frac{\gamma-1}{\gamma}} \right)^{\frac{\gamma}{\gamma-1}-1}\int_0^1 \frac{\gamma-1}{\gamma} C_{i,t}^{-\frac{1}{\gamma}}\left[ \frac{\varepsilon}{\varepsilon-1}\left(C_{i,t}^{\frac{\varepsilon-1}{\varepsilon}} \right)^{\frac{\varepsilon}{\varepsilon-1}-1}\int_0^1 \frac{\varepsilon-1}{\varepsilon} C_{i,t}(j)^{-\frac{1}{\varepsilon}}dj \right]di\)

After simplifying, we get

\(\displaystyle \frac{\partial U(C_t,N_t)}{\partial C_{i,t}(j)} = U_c(C_t,N_t) \alpha^{\frac{1}{\eta}} C_t^{\frac{1}{\eta}}\left( C_{F,t}\right)^{-\frac{1}{\eta}}C_{F,t}^{\frac{1}{\gamma}} \int_0^1 C_{i,t}^{-\frac{1}{\gamma}} \left[ C_{i,t}^\frac{1}{\varepsilon} \int_0^1 C_{i,t}(j)^{-\frac{1}{\varepsilon}}dj \right] di\)

\(\displaystyle \frac{\partial U(C_t,N_t)}{\partial C_{i,t}(j)} = U_c(C_t,N_t) \left[ \alpha \frac{C_t}{C_{F,t}} \right]^{\frac{1}{\eta}} \int_0^1 \left[\frac{C_{F,t}}{C_{i,t}}\right]^{\frac{1}{\gamma}} \int_0^1 \left[\frac{C_{i,t}}{C_{i,t}(j)}\right]^{\frac{1}{\varepsilon}}dj \ di\)

\(\displaystyle \frac{\partial U(C_t,N_t)}{\partial C_{i,t}} = U_c(C_t,N_t)\frac{\eta}{1-\eta}\left( C_t^{\frac{\eta-1}{\eta}} \right)^{\frac{\eta}{\eta-1}-1} \alpha^{\frac{1}{\eta}}\frac{\eta-1}{\eta}\left( C_{F,t}\right)^{-\frac{1}{\eta}}\frac{\gamma}{\gamma-1}\left(C_{H,t}^{\frac{\gamma-1}{\gamma}} \right)^{\frac{\gamma}{\gamma-1}-1}\int_0^1 \frac{\gamma-1}{\gamma} C_{i,t}^{-\frac{1}{\gamma}}di\)

\(\displaystyle \frac{\partial U(C_t,N_t)}{\partial C_{i,t}} = U_c(C_t,N_t) \left[ \alpha \frac{C_t}{C_{F,t}} \right]^{\frac{1}{\eta}} \int_0^1 \left[\frac{C_{F,t}}{C_{i,t}}\right]^{\frac{1}{\gamma}} di\)

Calculating the MRS we have

\(\displaystyle \frac{\displaystyle \frac{\partial U(C_t,N_t)}{\displaystyle \partial C_{i,t}(j)}}{\frac{\displaystyle \partial U(C_t,N_t)}{\displaystyle \partial C_{i,t}}} = \frac{\displaystyle U_c(C_t,N_t) \left[ \alpha \frac{C_t}{C_{F,t}} \right]^{\frac{1}{\eta}} \int_0^1 \left[\frac{C_{F,t}}{C_{i,t}}\right]^{\frac{1}{\gamma}} \int_0^1 \left[\frac{C_{i,t}}{C_{i,t}(j)}\right]^{\frac{1}{\varepsilon}}dj \ di }{\displaystyle U_c(C_t,N_t) \left[ \alpha \frac{C_t}{C_{F,t}} \right]^{\frac{1}{\eta}} \int_0^1 \left[\frac{C_{F,t}}{C_{i,t}}\right]^{\frac{1}{\gamma}} di} = \frac{\displaystyle \int_0^1P_{i,t}(j)dj}{P_{i,t}}\)

As before, we can simplify again. Also, as there's a continuum of firms,
we can consider that the price of each product \(P_{i,t}(j)\) is only
correlated with its specific demand function and not with the demand
function of other in its country or another country, it follows that
each specific price is uncorrelated with \(C_{F,t}\) and \(C_{i,t}\).
Also, as each firm is very small, we can consider that it has negligible
influence on the aggregate index price of its country (\(P_{i,t}\)).
Whith these independence assumption, the joint distribution is equal to
the product of the marginal distributions.

\(\displaystyle \int_0^1 \int_0^1 \left[\frac{C_{F,t}}{C_{i,t}}\right]^{\frac{1}{\gamma}} \left[\frac{C_{i,t}}{C_{i,t}(j)}\right]^{\frac{1}{\varepsilon}}dj \ di = \int_0^1 \left[\frac{C_{F,t}}{C_{i,t}}\right]^{\frac{1}{\gamma}} di \int_0^1 \frac{\displaystyle P_{i,t}(j)}{P_{i,t}}dj = \int_0^1 \int_0^1 \frac{\displaystyle P_{i,t}(j)}{P_{i,t}} \left[\frac{C_{F,t}}{C_{i,t}}\right]^{\frac{1}{\gamma}} dj \ di\)

Now, as before, the integrand in both sides needs to be the same. The we
get the second demand equation.

\(\displaystyle \left[\frac{C_{F,t}}{C_{i,t}}\right]^{\frac{1}{\gamma}} \left[\frac{C_{i,t}}{C_{i,t}(j)}\right]^{\frac{1}{\varepsilon}} = \frac{\displaystyle P_{i,t}(j)}{P_{i,t}} \left[\frac{C_{F,t}}{C_{i,t}}\right]^{\frac{1}{\gamma}} \Rightarrow \left[\frac{C_{i,t}(j)}{C_{i,t}}\right]^{-\frac{1}{\varepsilon}} = \frac{\displaystyle P_{i,t}(j)}{P_{i,t}} \Rightarrow C_{i,t}(j) = \left[ \frac{\displaystyle P_{i,t}(j)}{P_{i,t}} \right]^{-\varepsilon}C_{i,t}\)

To find the aggregate demand for each country, in terms of total foreign
demand, we proceed by calculating the MRS between the aggregate
consumption for the country and the aggregate consumption of foreign
goods, which the optimal allocation resulting from the rate between the
prices, as before.

\(\displaystyle \frac{\partial U(C_t,N_t)}{\partial C_{F,t}} = U_c(C_t,N_t)\frac{\eta}{1-\eta}\left( C_t^{\frac{\eta-1}{\eta}} \right)^{\frac{\eta}{\eta-1}-1} \alpha^{\frac{1}{\eta}}\frac{\eta-1}{\eta}\left( C_{F,t}\right)^{-\frac{1}{\eta}} = U_c(C_t,N_t) \left[ \alpha \frac{C_t}{C_{F,t}} \right]^{\frac{1}{\eta}}\)

\(\displaystyle \frac{\displaystyle \frac{\partial U(C_t,N_t)}{\displaystyle \partial C_{i,t}}}{\frac{\displaystyle \partial U(C_t,N_t)}{\displaystyle \partial C_{F,t}}} = \frac{\displaystyle U_c(C_t,N_t) \left[ \alpha \frac{C_t}{C_{F,t}} \right]^{\frac{1}{\eta}} \int_0^1 \left[\frac{C_{F,t}}{C_{i,t}}\right]^{\frac{1}{\gamma}} di }{\displaystyle U_c(C_t,N_t) \left[ \alpha \frac{C_t}{C_{F,t}} \right]^{\frac{1}{\eta}} } = \frac{\displaystyle \int_0^1P_{i,t}di}{P_{F,t}}\)

As \(P_{F,t}\) doesn't depend on a specific i, we can put it inside the
integral. Then we get again two integrans which have to be the same for
the equality to hold.

\(\displaystyle \left[\frac{C_{F,t}}{C_{i,t}}\right]^{\frac{1}{\gamma}} = \frac{P_{i,t}}{P_{F,t}} \Rightarrow \left[\frac{C_{i,t}}{C_{F,t}}\right]^{-\frac{1}{\gamma}} = \frac{P_{i,t}}{P_{F,t}} \ \ \Rightarrow \ \ C_{i,t} = \left[ \frac{P_{i,t}}{P_{F,t}} \right]^{-\gamma}C_{F,t}\)

\hypertarget{aggregating-the-expenditure}{%
\subsection{3 - Aggregating the
expenditure}\label{aggregating-the-expenditure}}

Now that we have the demand functions

\(\displaystyle C_{H,t}(j)= \left( \frac{P_{H,t}(j)}{P_{H,t}}\right)^{-\varepsilon}C_{H,t}\),
\(\displaystyle C_{i,t}(j)= \left( \frac{P_{i,t}(j)}{P_{i,t}}\right)^{-\varepsilon}C_{i,t}\)
and
\(\displaystyle C_{i,t}= \left( \frac{P_{i,t}}{P_{F,t}}\right)^{-\gamma}C_{F,t}\)

let's prove that

\(\displaystyle \int_0^1 P_{H,t}(j)C_{H,t}(j)dj = P_{H,t}C_{H,t}\) ,
\(\displaystyle \int_0^1 P_{i,t}(j)C_{i,t}(j)dj = P_{i,t}C_{i,t}\) and
\(\displaystyle \int_0^1 P_{i,t}C_{i,t}dj = P_{F,t}C_{F,t}\)

using the definition of the price indexes:

\(\displaystyle P_{H,t} \equiv \left( \int_0^1 P_{H,t}(j)^{1-\varepsilon}dj \right)^{\frac{1}{1-\varepsilon}}\),
\(\displaystyle P_{i,t} \equiv \left( \int_0^1 P_{i,t}(j)^{1-\varepsilon}dj \right)^{\frac{1}{1-\varepsilon}}\)
and
\(\displaystyle P_{F,t} \equiv \left( \int_0^1 P_{i,t}^{1-\gamma}di \right)^{\frac{1}{1-\gamma}}\)

\(\displaystyle \int_0^1 P_{H,t}(j)C_{H,t}(j)dj = \int_0^1P_{H,t}(j)\left( \frac{P_{H,t}(j)}{P_{H,t}}\right)^{-\varepsilon}C_{H,t}dj = \frac{C_{H,t}}{P_{H,t}^{-\varepsilon}}\int_0^1P_{H,t}(j)^{1-\varepsilon}dj = \frac{C_{H,t}}{P_{H,t}^{-\varepsilon}}P_{H,t}^{1-\varepsilon} = P_{H,t}C_{H,t}\)

\(\displaystyle \int_0^1 P_{i,t}(j)C_{i,t}(j)dj = \int_0^1P_{i,t}(j)\left( \frac{P_{i,t}(j)}{P_{i,t}}\right)^{-\varepsilon}C_{i,t}dj = \frac{C_{i,t}}{P_{i,t}^{-\varepsilon}}\int_0^1 P_{i,t}(j)^{1-\varepsilon}dj = \frac{C_{i,t}}{P_{i,t}^{-\varepsilon}}P_{i,t}^{1-\varepsilon} = P_{i,t}C_{i,t}\)

\(\displaystyle \int_0^1 P_{i,t}C_{i,t}di = \int_0^1P_{i,t}\left( \frac{P_{i,t}}{P_{F,t}}\right)^{-\gamma}C_{F,t}di = \frac{C_{F,t}}{P_{F,t}^{-\gamma}}\int_0^1 P_{F,t}^{1-\gamma}dj = \frac{C_{F,t}}{P_{F,t}^{-\varepsilon}}P_{F,t}^{1-\varepsilon} = P_{F,t}C_{F,t}\)

With this aggregatio, the budget constraint can be simplified

\(\displaystyle \int_0^1 P_{H,t}(j)C_{H,t}(j)dj + \int_0^1\int_0^1 P_{i,t}(j)C_{i,t}(j)dj\ di + \mathbb{E}_t\{ Q_{t,t+1}D_{t+1}\} \leq D_t + W_tN_t + Tt\)

\(\displaystyle P_{H,t}C_{H,t} + \int_0^1P_{i,t}C_{i,t}di = P_{H,t}C_{H,t} + P_{F,t}C_{F,t} \leq D_t + W_tN_t + Tt - \mathbb{E}_t\{ Q_{t,t+1}D_{t+1}\}\)

As the total consumption expenditure by the representative consumer is
with the domestic produts or foreig products, the budget constraint
becomes:

\(\displaystyle P_t C_t \leq D_t + W_tN_t + Tt - \mathbb{E}_t\{ Q_{t,t+1}D_{t+1}\}\)

\hypertarget{finding-the-optimal-share-between-the-domestic-and-imported-goods}{%
\subsection{4 - Finding the optimal share between the domestic and
imported
goods}\label{finding-the-optimal-share-between-the-domestic-and-imported-goods}}

Now we will calculate the MRS between the domestic products and the
total consumption, which has to be equal to the rate of prices. After,
we will do the same for the foreign products.

\(\displaystyle \frac{\displaystyle \frac{\partial U(C_t,N_t)}{\displaystyle \partial C_{H,t}}}{\frac{\displaystyle \partial U(C_t,N_t)}{\displaystyle \partial C_t}} = \frac{\displaystyle U_c(C_t,N_t) \left[ (1-\alpha) \frac{C_t}{C_{H,t}} \right]^{\frac{1}{\eta}}}{\displaystyle U_c(C_t,N_t) } = \frac{P_{H,t}}{P_t} \ \ \Rightarrow \ \ (1-\alpha) \frac{C_t}{C_{H,t}} = \left( \frac{P_{H,t}}{P_t} \right)^\eta \ \ \Rightarrow \ \ C_{H,t} = (1-\alpha)\left( \frac{P_{H,t}}{P_t} \right)^{-\eta}C_t\)

\(\displaystyle \frac{\displaystyle \frac{\partial U(C_t,N_t)}{\displaystyle \partial C_{F,t}}}{\frac{\displaystyle \partial U(C_t,N_t)}{\displaystyle \partial C_t}} = \frac{\displaystyle U_c(C_t,N_t) \left[ \alpha \frac{C_t}{C_{F,t}} \right]^{\frac{1}{\eta}}}{\displaystyle U_c(C_t,N_t) } = \frac{P_{F,t}}{P_t} \ \ \Rightarrow \ \ \alpha \frac{C_t}{C_{F,t}} = \left( \frac{P_{F,t}}{P_t} \right)^\eta \ \ \Rightarrow \ \ C_{F,t} = \alpha \left( \frac{P_{F,t}}{P_t} \right)^{-\eta}C_t\)

\hypertarget{standard-problem-of-the-representative-consumer}{%
\subsection{5 - Standard problem of the representative
consumer}\label{standard-problem-of-the-representative-consumer}}

Now we arrive at a standard problem for the representative consumer

\(\displaystyle \underset{C_t,N_t,D_{t+1}} {max} \ \mathbb{E}_0 \sum_{t=0}^\infty \beta^t U(C_t,N_t) = \displaystyle \underset{C_t,N_t} {max} \ E_0 \sum_{t=0}^\infty \beta^t \left( \frac{C_t^{1-\sigma}}{1-\sigma}-\frac{N_t^{1+\varphi}}{1+\varphi} \right)\)

subject to
\(\displaystyle D_t + W_tN_t + Tt - \mathbb{E}_t\{ Q_{t,t+1}D_{t+1}\} - P_t C_t = 0\),
as an optimal condition.

using the separable utility function specified as
\(\displaystyle U(C_t,N_t)=\frac{C_t^{1-\sigma}}{1-\sigma}-\frac{N_t^{1+\varphi}}{1+\varphi}\)

\(\mathcal{L} = \displaystyle \mathbb{E}_t \sum_{t=0}^\infty \beta^t \left[ \left( \frac{C_t^{1-\sigma}}{1-\sigma}-\frac{N_t^{1+\varphi}}{1+\varphi} \right) + \lambda_t \left( D_t + W_tN_t + Tt - Q_{t,t+1}D_{t+1} - P_t C_t \right) \right]\)

with first order conditions (FOCs):

(\(C_t\)):
\(\displaystyle \beta^tC_t^{-\sigma} = \beta^t \lambda_t P_t \ \ \Rightarrow \ \ C_t^{-\sigma} = \lambda_t P_t\)

(\(N_t\)):
\(\displaystyle \beta^tN_t^{\varphi} = \beta^t \lambda_t W_t \ \ \Rightarrow \ \ N_t^{\varphi} = \lambda_t P_t\)

(\(D_{t+1}\)):
\(\displaystyle \beta^t \lambda_t Q_{t,t+1} = \beta^{t+1} \mathbb {E}[\lambda_{t+1}] \ \ \Rightarrow \ \ \frac{\mathbb{E}[\lambda_{t+1}]}{\lambda_t} = \frac{Q_{t,t+1}}{\beta}\)

Dividing (\(N_t\)) FOC by (\(C_t\)) FOC, we have the standard equation
of intratemporal substitution between consumption and leisure
\(\displaystyle C_t^{\sigma}N_t^{\varphi} = \frac{W_t}{P_t}\)

Advancing one period for the consumption FOC, we have
\(\mathbb{E} [C_{t+1}^{-\sigma}] = \mathbb{E}[\lambda_{t+1}P_{t+1}]\)

As the model will be log-linearized and an approximation of the first
order will be used to solve it, we can ignore the Jensen's inequality
where there is an expectation operator. Up to first order approximation,
\(\mathbb{E}[xy] \approx \mathbb{E}[x] \mathbb{E}[y]\).

Dividing the consumption FOC in t+1 by the equation in t and
substituting by the \(\mathbb{E}[\lambda_{t+1}]/\lambda_t\) in the
\(D_{t+1}\) FOC, we get the Euler equation

\(\displaystyle \mathbb{E} \left[ \left( \frac{C_{t+1}}{C_t} \right)^{-\sigma} \right] = \mathbb{E} \left[ \frac{\lambda_{t+1} P_{t+1}}{\lambda_t P_t} \right] \ \ \Rightarrow \ \  \mathbb{E} \left[ \left( \frac{C_{t+1}}{C_t} \right)^{-\sigma} \frac{P_t}{P_{t+1}}\right]= \frac{\mathbb{E}[Q_{t,t+1}]}{\beta} \ \ \Rightarrow \ \ \beta R_t \mathbb{E} \left[ \left( \frac{C_{t+1}}{C_t} \right)^{-\sigma} \frac{P_t}{P_{t+1}}\right]= 1\)

as \(Q_{t,t+1}\), is the price of a a riskless one-period discount bond
in domestic currency with gross return \(R_t\).

\hypertarget{log-linearization}{%
\subsection{6 - Log linearization}\label{log-linearization}}

Log-linearizing
\(\displaystyle C_t^{\sigma}N_t^{\varphi} = \frac{W_t}{P_t}\) is
straight forward: \(w_t - p_t = \sigma c_t + \varphi n_t\)

To log-linerize the Euler equation, we'll use the Taylor expansion:
\(f(x) \approx f(x_0) + f'(x_o)(x-x_0)\). When expanding the exponential
function around 0, we get \(e^x=e^0+e^0(x-0)=1+x\)

\(\mathbb{E} \left[ \exp \left( \ln \left[ \beta R_t \left( \frac{C_{t+1}}{C_t} \right)^{-\sigma} \frac{P_t}{P_{t+1}} \right] \right) \right] = \mathbb{E} \left[ \exp \left( \ln(\beta) + r_t -\sigma(c_{t+1}-c_t) + p_t - p_{t+1} \right) \right] = 1 \ \ \Rightarrow\)

\(\displaystyle 1 + \ln(\beta) + r_t -\sigma(\mathbb{E} \left[c_{t+1} \right]-c_t) - \mathbb{E} \left[ \pi_{t+1} \right] = 1 \ \ \Rightarrow \ \ \sigma c_t = \sigma \mathbb{E} \left[c_{t+1} \right] + \mathbb{E} \left[ \pi_{t+1} \right] - r_t - \ln(\beta)\)
\(\displaystyle \Rightarrow \ \ c_t = \mathbb{E} \left[c_{t+1} \right] -\frac{1}{\sigma} \left( r_t - \mathbb{E} \left[ \pi_{t+1} \right] -\rho\right)\)
as \(\rho \equiv \frac{1-\beta}{\beta} \approx -\ln(\beta)\)

\hypertarget{terms-of-trade}{%
\subsection{7 - Terms of trade}\label{terms-of-trade}}

Let's log-linearize the expression for the bilateral terms of trade
\(\displaystyle S_t \equiv \frac{P_{F,t}}{P_{H,t}}= \left( \int_0^1 S_{i,t}^{1-\gamma}di \right)^{\frac{1}{1-\gamma}} \ \ \Rightarrow \ \ S_t^{1-\gamma} = \int_0^1 S_{i,t}^{1-\gamma}di\)

\(\displaystyle \exp \left(\ln \left[ S_t^{1-\gamma}\right] \right) = \int_0^1 \exp \left(\ln \left[ S_{i,t}^{1-\gamma}\right] \right)di \ \ \Rightarrow \ \ \exp[(1-\gamma)s_t]=\int_0^1 \exp[(1-\gamma)s_{i,t}]di\)

Applying the exponential Taylor expansion (\(e^x = 1 + x\)) in both
sides, we get
\(\displaystyle 1+(1+\gamma)s_t = 1 + (1+\gamma) \int_0^1 s_{i,t}di \ \ \Rightarrow \ \ s_t = \int_0^1 s_{i,t}di\)

To log-linearize the CPI formula, considering that it is a symmetric
steady-state, we have \(P_{H,t}=P_{F,t}=P_t\). Now, taking logs in both
sides and using the Taylor expansion for a vector of two variables we
have
\(\displaystyle f(x,y) \approx f(x_0,y_0)+\frac{\partial f(x,y)}{\partial x} \biggr|_{x_0,y_0}(x-x_0)+\frac{\partial f(x,y)}{\partial y} \biggr|_{x_0,y_0}(y-y_0)\)
So,
\(\displaystyle (P_t)^{\eta-1} = (1-\alpha)(P_{H,t})^{1-\eta} + \alpha(P_{F,t})^{1-\eta}\)

By the CPI definition, we have
\(\displaystyle P_t \equiv \left[ (1-\alpha)(P_{H,t})^{1-\eta} + \alpha(P_{F,t})^{1-\eta} \right]^{\frac{1}{1-\eta}} \ \ \Rightarrow \ \ (P_t)^{1-\eta} = (1-\alpha)(P_{H,t})^{1-\eta} + \alpha(P_{F,t})^{1-\eta}\)

Taking logs, we have
\((1-\eta)\ln(P_t) = \ln\left[ (1-\alpha)(P_{H,t})^{1-\eta} + \alpha(P_{F,t})^{1-\eta} \right]=f(P_{H,t},P_{F,t})\)

Applying the Taylor expansion on the right side using \(x = P_{H,t}\),
\(y = P_{F,t}\) and \(x_0 = y_0 = P_t\),

\(\displaystyle (1-\eta)p_t \approx \ln \left[ (1-\alpha)(P_t)^{1-\eta} + \alpha(P_t)^{1-\eta} \right]+ \frac{(1-\alpha)(1-\eta)P_t^{-\eta}}{P_t^{1-\eta}}(P_{H,t}-P_t) + \frac{\alpha(1-\eta)P_t^{-\eta}}{P_t^{1-\eta}}(P_{F,t}-P_t)\)

\(\displaystyle (1-\eta)p_t = \ln(P_t^{1-\eta}) + (1-\alpha)(1-\eta)\frac{P_{H,t}-P_t}{P_t} + \alpha(1-\eta)\frac{P_{F,t}-P_t}{P_t}\).

\(p_t \approx p_t + (1-\alpha)[\ln(P_{H,t})-\ln(P_t)] + \alpha[\ln(P_{F,t})-\ln(P_t)] \ \ \Rightarrow \ \  p_t = (1-\alpha)p_{H,t}+\alpha \ p_{F,t}\),
as defined in the paper.

As \(s_t \equiv p_{F,t}-p_{H,t}\),
\(p_t=(1-\alpha)P_{H,t}+\alpha (s_t+p_{H,t}) = p_{H,t} + \alpha s_t\)

The domestic inflation rate is defined as
\(\displaystyle \pi_{H,t} \equiv p_{H,t}-p_{H,t-1}\), taking the
difference between the equation between t and t-1, we have
\(p_t-p_{t-1}=p_{H,t}-p_{H,t-1}+\alpha(s_t-s_{t-1}) \ \ \Rightarrow \ \ \pi_t = \pi_{H,t}+\alpha \Delta s_t\).

Assuming that the law of one price is valid in all times (the same goods
produced in different countries have the same price when converting to
the domestic currency, using the nominal interest rate), we have
\(P_{i,t}^i(j) = \mathcal{E}_{i,t}P_{i,t}^i(j)\) for all
\(i, j \in [0,1]\).

As
\(P_{i,t}^i \equiv \displaystyle \left( \int_0^1 P_{i,t}^i(j)^{{1-\varepsilon}}dj \right) ^{\frac{1}{1-\varepsilon}}\),
we have
\(\displaystyle \mathcal{E}_{i,t}P_{i,t}^i = \mathcal{E}_{i,t} \left( \int_0^1 P_{i,t}^i(j)^{{1-\varepsilon}}dj \right) ^{\frac{1}{1-\varepsilon}} = \left( (\mathcal{E}_{i,t})^{1-\varepsilon} \int_0^1 P_{i,t}^i(j)^{{1-\varepsilon}}dj \right) ^{\frac{1}{1-\varepsilon}}\)

\(\displaystyle = \left( \int_0^1 \left( \mathcal{E}_{i,t}P_{i,t}^i(j)\right) ^{{1-\varepsilon}}dj \right) ^{\frac{1}{1-\varepsilon}} = \left( \int_0^1 P_{i,t}(j)^{{1-\varepsilon}}dj \right) ^{\frac{1}{1-\varepsilon}} = P_{i,t}\)

Now, we have
\(\displaystyle P_{F,t}= \left( \int_0^1 P_{i,t}^{{1-\gamma}}di \right) ^{\frac{1}{1-\gamma}} = \left( \int_0^1 \left( \mathcal{E}_{i,t}P_{i,t}^i \right)^{{1-\gamma}}di \right)^{\frac{1}{1-\gamma}} \ \ \Rightarrow \ \ P_{F,t}^{1-\gamma} = \int_0^1 \left( \mathcal{E}_{i,t}P_{i,t}^i \right)^{{1-\gamma}}di\)

Log-linearizing the last expression, we get
\(\displaystyle \exp \left[(1-\gamma)\ln P_{F,t} \right] = \int_0^1 \exp \left[(1-\gamma) \ln \left( \mathcal{E}_{i,t}P_{i,t}^i \right) \right]di\)

\(\Rightarrow \ \  \displaystyle 1 + (1-\gamma)p_{F,t} = \int_0^1 \left[ 1+ (1-\gamma)(e_{i,t}+p_{i,t}^i)\right]di \ \ \Rightarrow \ \ p_{F,t} = \int_0^1 (e_{i,t}+p_{i,t}^i)di=e_t+p_t^*\),

where \(\displaystyle e_t \equiv \int_0^1 e_{i,t} \ di\) and
\(\displaystyle p_t^*\equiv \int_0^1 p_{i,t}^i \ di\). Also, we have
that \(s_t = p_{F,t}-p_{H,t}=e_t + p_t^*-p_{H,t}\).

Defining the bilateral real exchange rate
\(\displaystyle \mathcal{Q}_{i,t} \equiv \frac{\mathcal{E}_{i,t}P_{i,t}}{P_t}\)
and the (log) effective real exchange rate
\(\displaystyle q_t \equiv \int_0^1 q_{i,t}di\) we have
\(\displaystyle q_t = \int_0^1 \ln \left( \frac{\mathcal{E}_{i,t}P_{i,t}}{P_t} \right)di= \int_0^1 \left(e_{i,t}+p_{i,t}-p_t \right)di = e_t+p_t^*-p_t = s_t+p_{H,t}-(p_{H,t}+\alpha s_t) = (1-\alpha)s_t\)

\hypertarget{international-risk-sharing}{%
\subsection{8 - International risk
sharing}\label{international-risk-sharing}}

The problem of the representative household in any country is the same,
as the economies are all equal. There is, any country has an Euler
equation like
\(\displaystyle \beta \mathbb{E} \left[ \left( \frac{C_{t+1}}{C_t} \right)^{-\sigma} \frac{P_t}{P_{t+1}}\right]= Q_{t,t+1}\).

The condition for the clearing of the international market is that
\(Q_{t,t+1}\) is unique. So the price converted to a common current has
to be the same. So, for every foreign country, the Euler equation
becomes

\(\displaystyle \beta \mathbb{E} \left[ \left( \frac{C_{t+1}^i}{C_t^i} \right)^{-\sigma} \frac{P_t^i \mathcal{E}_t^i}{P_{t+1}^i \mathcal{E}_{t+1}^i}\right] = Q_{t,t+1}\).

Combining both equations, using the definition of the real exchange rate
\(\displaystyle \mathcal{Q}_{i,t} \equiv \frac{\mathcal{E}_{i,t}P_{i,t}}{P_t}\)
and solving for \(C_t\), we have

\(\displaystyle \beta \mathbb{E} \left[ \left( \frac{C_{t+1}}{C_t} \right)^{-\sigma} \frac{P_t}{P_{t+1}}\right] = \beta \mathbb{E} \left[ \left( \frac{C_{t+1}^i}{C_t^i} \right)^{-\sigma} \frac{P_t^i \mathcal{E}_t^i}{P_{t+1}^i \mathcal{E}_{t+1}^i}\right] \ \ \Rightarrow \ \ \mathbb{E} \left[ \left( \frac{C_{t+1}}{C_t} \right)^{-\sigma} \frac{1}{P_{t+1}}\right] = \mathbb{E} \left[ \left( \frac{C_{t+1}^i}{C_t^i} \right)^{-\sigma} \frac{\mathcal{Q}_{i,t}}{P_{t+1}^i \mathcal{E}_{t+1}^i}\right]\)

\(\displaystyle \Rightarrow \ \ (C_t)^\sigma = \mathbb{E} \left[ {C_{t+1}}^\sigma \left( C_{t+1}^i \right)^{-\sigma} \frac{P_{t+1}}{P_{t+1}^i \mathcal{E}_{t+1}^i}\right]\mathcal{Q}_{i,t}(C_t^i)^\sigma \ \ \Rightarrow \ \ C_t = \mathbb{E} \left[ \frac{C_{t+1}}{C_{t+1}^i} (\mathcal{Q}_{i,t+1})^{-\frac{1}{\sigma}}\right] C_t^i \mathcal{Q}_{i,t}^\frac{1}{\sigma} \ \ \Rightarrow \ \ C_t = \vartheta_i C_t^i \mathcal{Q}_{i,t}^\frac{1}{\sigma}\),

where
\(\displaystyle \vartheta_i = \mathbb{E} \left[ \frac{C_{t+1}}{C_{t+1}^i} (\mathcal{Q}_{i,t+1})^{-\frac{1}{\sigma}}\right]\)
is a constant and generally will depend on initial relative net asset
positions. Assuming identical conditions for all economies, the net
asset position for all of the is zero. In this case,
\(\vartheta_i = \vartheta = 1\) for all i. As the symmetric foresight
steady-state in this condition is shown in the appendix A.

The international market clearing implies that the total goods produced
in a country is consumed by domestically or it's exported. The integral
represents the sum of the demand for products of the economy analysed by
foreign countries. In a case with economies not with measure zero, we
need to exclude the economy analysed from the integral do avoid double
counting.

\(\displaystyle Y=C_H+C_i=(1-\alpha)\left( \frac{P_{H}}{P} \right)^{-\eta}C + \int_0^1 \left( \frac{P_i^i}{P_F} \right)^{-\gamma}C_F di = (1-\alpha)\left( \frac{P_{H}}{P} \right)^{-\eta}C + \alpha \int_0^1 \left( \frac{P_i^i}{P_F^i} \right)^{-\gamma} \left( \frac{P_{F}^i}{P^i} \right)^{-\eta}C^idi\)

\(\displaystyle Y= (1-\alpha)\left( \frac{P_{H}}{P} \right)^{-\eta}C + \alpha \int_0^1 \left( \frac{P_F^i}{P_i^i} \right)^{\gamma} \left( \frac{P_{F}^i}{P^i} \right)^{-\eta}C^idi= (1-\alpha)\left( \frac{P_{H}}{P} \right)^{-\eta}C + \alpha \int_0^1 \left( \frac{\mathcal{E}_i P_F^i }{P_H} \right)^{\gamma} \left( \frac{P_{F}^i}{P^i} \right)^{-\eta}C^idi\),

where \(P_i^i\) is the price in the domestic economy converted to the
currency of country i, or
\(\displaystyle P_i^i = \frac{P_i}{\mathcal{E}_i} = \frac{P_H}{\mathcal{E}_i}\),
as the goods have the same price in the international market, after
converting to the same currency. After simplifying, we have

\(\displaystyle Y = \left( \frac{P_{H}}{P} \right)^{-\eta} \left[ (1-\alpha)C + \alpha \int_0^1 \left( \frac{ \mathcal{E}_i P_F^i }{P_H} \right)^{\gamma-\eta} \left( \frac{\mathcal{E}_i P_i}{P} \right)^\eta C^idi \right] = \left( \frac{P_{H}}{P} \right)^{-\eta} \left[ (1-\alpha)C + \alpha \int_0^1 \left( \frac{ \mathcal{E}_i P_F^i }{P_H} \right)^{\gamma-\eta} Q_i^\eta C^idi \right]\)

Considering that
\(\displaystyle P = \left[ (1-\alpha) \left( P_H \right)^{1-\eta} + \alpha \left( P_F\right)^{1-\eta} \right]^{\frac{1}{1-\eta}}\)
in the steady-state,
\(\displaystyle P^{1-\eta} = (1-\alpha) \left( P_H \right)^{1-\eta} + \alpha \left( P_F \right)^{1-\eta}\),
as \(\displaystyle \mathcal{S}_i \equiv \frac{P_i}{P_H}\). So,

\(\displaystyle \left( \frac{P}{P_H} \right)^{1-\eta} = (1-\alpha) + \alpha \left( \frac{P_F}{P_H}\right)^{1-\eta} = (1-\alpha) + \alpha \mathcal{S}_i^{1-\eta} \ \ \Rightarrow \ \ \frac{P}{P_H}=\left[ (1-\alpha)+\alpha \mathcal{S}^{1-\eta}\right]^{\frac{1}{1-\eta}}= \left[ (1-\alpha) + \alpha \int_0^1(\mathcal{S}_i)^{1-\eta}di \right]^{\frac{1}{1-\eta}} \equiv h(\mathcal{S})\)

Defining
\(\displaystyle \mathcal{S}^i = \frac{\mathcal{E}_i P_F^i}{P_i}\) and
using the fact that
\(\displaystyle C^i=C \mathcal{Q}^{-\frac{1}{\sigma}}\) as
\(\vartheta_i = 1\) in a symmetric steady-state, we have

\(\displaystyle Y = h(\mathcal{S})^{\eta}C \left[ (1-\alpha) + \alpha \int_0^1 \left( \frac{ \mathcal{E}_i P_F^i }{P_i} \frac{P_i}{P_H} \right)^{\gamma-\eta} Q_i^{\eta-\frac{1}{\sigma}} di \right]= h(\mathcal{S})^{\eta}C \left[ (1-\alpha) + \alpha \int_0^1 \left( \mathcal{S}^i \frac{P_F}{P_H} \right)^{\gamma-\eta} Q_i^{\eta-\frac{1}{\sigma}} di \right]\)

As we will work with a first order approximation, the equality below is
valid.

\(\displaystyle Y = h(\mathcal{S})^{\eta}C \left[ (1-\alpha) + \alpha \int_0^1 \left( \mathcal{S}^i \mathcal{S}_i \right)^{\gamma-\eta} Q_i^{\eta-\frac{1}{\sigma}} di \right] = h(\mathcal{S})^{\eta}C \left[ (1-\alpha) + \alpha \int_0^1 ( \mathcal{S}^i)^{\gamma} di \int_0^1 ( \mathcal{S}_i)^{-\eta} di \int_0^1 Q_i^{\eta-\frac{1}{\sigma}} di \right]\)

\(\displaystyle Y = h(\mathcal{S})^{\eta}C \left[ (1-\alpha) + \alpha \mathcal{S}^{-\eta} \int_0^1 \left( \frac{P_F^i}{P_H} \right)^{\gamma} di \int_0^1 \left( \frac{\mathcal{E}_i P_F^i}{P} \right)^{\eta-\frac{1}{\sigma}} di \right]\),

as if
\(\displaystyle \mathcal{S}^{1-\gamma}=\int_0^1 \mathcal{S^{1-\gamma}}di\),
we can substitute variables \(-\eta=1-\gamma\) and we get the result.

\(\displaystyle Y = h(\mathcal{S})^{\eta}C \left[ (1-\alpha) + \alpha \mathcal{S}^{-\eta} \left( \frac{1}{P_H} \right)^{\gamma} \int_0^1 \left( P_F^i \right)^{\gamma} di \int_0^1 \left( \frac{\mathcal{E}_i P_F^i}{P} \right)^{\eta-\frac{1}{\sigma}} di \right]\)

\(Y= h(\mathcal{S})^{\eta}C \left[ (1-\alpha) + \alpha \mathcal{S}^{-\eta} \left( \frac{1}{P_H} \right)^{\gamma} (P^*)^{\gamma} \int_0^1 \left( \frac{\mathcal{E}_i P_F^i}{P_H} \frac{P_H}{P} \right)^{\eta-\frac{1}{\sigma}} di \right]\),

using the fact that
\(\left( P_F^i \right)^{1-\gamma} = \displaystyle \int_0^1 \left( P_i^i \right)^{1-\gamma}di\)
and using P* for the international price index of imported goods.

\(\displaystyle Y = h(\mathcal{S})^{\eta}C \left[ (1-\alpha) + \alpha \mathcal{S}^{-\eta} \left( \frac{P^*}{P_H} \right)^{\gamma} \int_0^1 \left( \frac{\mathcal{S}^i}{h(\mathcal{S})} \right)^{\eta-\frac{1}{\sigma}} di \right] =h(\mathcal{S})^{\eta}C \left[ (1-\alpha) + \alpha \mathcal{S}^{-\eta} \mathcal{S}^\gamma \left( \frac{1}{h(\mathcal{S})} \right)^{\eta-\frac{1}{\sigma}} \int_0^1 ( \mathcal{S}^i )^{\eta-\frac{1}{\sigma}} di \right]\)

\(\displaystyle Y = h(\mathcal{S})^{\eta}C \left[ (1-\alpha) + \alpha \mathcal{S}^{\gamma-\eta} \left( \frac{1}{h(\mathcal{S})} \right)^{\eta-\frac{1}{\sigma}} \mathcal{S} ^{\eta-\frac{1}{\sigma}} \right] = h(\mathcal{S})^{\eta}C \left[ (1-\alpha) + \alpha \mathcal{S}^{\gamma-\eta} \left( \frac{\mathcal{S}}{h(\mathcal{S})} \right)^{\eta-\frac{1}{\sigma}} \right]\),

which yields the result.
\(\displaystyle Y= h(\mathcal{S})^{\eta}C \left[ (1-\alpha) + \alpha \mathcal{S}^{\gamma-\eta} q(\mathcal{S})^{\eta-\frac{1}{\sigma}} \right]\),
where
\(\displaystyle \mathcal{Q}=\frac{\mathcal{S}}{h(\mathcal{S})} \equiv q(\mathcal{S})\)

\end{document}
