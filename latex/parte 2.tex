\documentclass{article}
\usepackage[utf8]{inputenc}
\usepackage{amsthm}
\usepackage{amsfonts}
\usepackage{bigints}

\newcommand{\Et}{\mathbb{E}_t}

\begin{document}


\section{Technology}

A representative firm has a technology with constant returns:

$$Y_t(j) = A_t N_t(j)$$

where $a_t = \ln(A_t)$ follows the AR(1) process $a_t = \rho_a a_{t-1} + \varepsilon_t$. Aggregating across firms and log-linearizing we get that:

$$\int^1_0 Y_t(j) dj = \int^1_0 A_t N_t(j) dj \Rightarrow Y_t = A_t N_t \Rightarrow y_t = a_t + n_t$$

This technology leads to a real marginal cost that does not deppend on the firm or the produced quantity:

$$MC_t = \frac{\partial}{\partial Y_t(j)} \frac{Y_t(j)}{A_t} * \frac{W_t(1-\tau)}{P_{H,t}} =\frac{W_t(1-\tau)}{P_{H,t} A_t}$$

Where $\tau$ is an employment subsidy. Log-linearizing this expression, and setting $\nu \equiv -\ln(1-\tau)$ we get:

$$mc_t(j) = -\nu + w_t - p_{H,t} - a_t$$

If the firm can adjust price at time $t$, we will set its price at $\bar P_{H,t} (j)$ which maximizes the present value of its future profit:

$$\max_{\bar P_{H,t}(j)} \sum^\infty_{k=0} \theta^k \Et \big[ Q_{t, t+k}[Y_{t+k}(j) (\bar P_{H,t}(j) - MC_{t+k} P_{H,t+k})] \big]$$


As we get in (xxx) that $C_t(j) = \Big( \frac{P_{H,t}(j)}{P_{H,t}} \Big)^{-\varepsilon} C_{H, t}$ if the price remains unchanged at $\bar P_{H,t}$ until $t+k$ period then: $C_{t+k}(j) = \Big( \frac{\bar P_{H,t}(j)}{P_{H,t+k}}\Big)^{-\varepsilon} C_{H, t+k}$. Similarly the foreign consumption of this domestic good is $C^i_{t+k}(j) = \int^1_0 \Big( \frac{\bar P_{H,t}(j)}{P_{H,t+k}} \Big)^{-\varepsilon} C^i_{H, t+k} di$. Market clearing imposes that:

$$Y_{t+k}(j) = C_{H,t+k}(j) + \int_0^1 C^i_{H, t+k}(j) di = \bigg( \frac{\bar P_{H,t}(j)}{P_{H,t+k}} \bigg)^{-\varepsilon} (C_{H,t} +  \int_0^1 C^i_{H, t+k} di)$$

Calculating the derivated with respect to $\bar P_{H,t}$:

$$\frac{\partial Y_{t+k}(j)}{\bar P_{H,t}} = -\varepsilon \bigg(C_{H,t} +  \int_0^1 C^i_{H, t+k} di \bigg)  \bigg( \frac{1}{P_{H,t+k}}  \bigg)^{-\varepsilon} \bar P_{H,t}(j)^{-\varepsilon -1} = - \varepsilon \frac{Y_{t+k}(j)}{\bar P_{H,t}}$$

First order condition of the maximization implies that:

\begin{equation*}
    \begin{split}
        0 &= \sum^\infty_{k=0} \theta^k \Et \bigg[ Q_{t, t+k} \Big[ \frac{\partial Y_{t+k}(j)}{\bar P_{H,t}} (\bar P_{H,t}(j) - MC_{t+k} P_{H,t+k}) + Y_{t+k}(j)  \Big] \bigg]\\
        0 &= \sum^\infty_{k=0} \theta^k \Et \bigg[ Q_{t, t+k} \Big[ -\varepsilon \frac{Y_{t+k}(j)}{\bar P_{H,t}} \Big(\bar P_{H,t}(j) - MC_{t+k} P_{H,t+k} \Big) + Y_{t+k}(j)  \Big] \bigg]\\
        0 &= \sum^\infty_{k=0} \theta^k \Et \bigg[ Q_{t, t+k} \Big[ Y_{t+k}(j) \Big(-\varepsilon  + -\varepsilon\frac{MC_{t+k} P_{H,t+k}}{\bar P_{H,t}} +1 \Big) \Big] \bigg] \\
        0 &= \sum^\infty_{k=0} \theta^k \Et \bigg[ Q_{t, t+k} Y_{t+k}(j) \Big(\bar P_{H,t} - \frac{\varepsilon}{\varepsilon-1} MC_{t+k} P_{H,t+k} \Big) \bigg]
    \end{split}
\end{equation*}


By consumer Euler equation, we know that $Q_{t, t+k} = \beta^k \Big( \frac{C_{t+k}}{C_t} \Big)^{-\sigma} \Big( \frac{P_{t}}{P_{t+k}} \Big)$, then:

\begin{equation*}
    \begin{split}
        0 &= \sum^\infty_{k=0} \theta^k \Et \bigg[ \beta^k \Big( \frac{C_{t+k}}{C_t} \Big)^{-\sigma} \Big( \frac{P_t}{P_{t+k}} \Big) Y_{t+k}(j) \Big(\bar P_{H,t} - \frac{\varepsilon}{\varepsilon-1} MC_{t+k} P_{H,t+k} \Big) \bigg] \\
        0 &= \sum^\infty_{k=0} (\beta\theta)^k \Et \bigg[ C_{t+k}^{-\sigma} P_{t+k}^{-1}  Y_{t+k}(j) \Big(\bar P_{H,t} - \frac{\varepsilon}{\varepsilon-1} MC_{t+k} P_{H,t+k} \Big) \bigg] \\
        0 &= \sum^\infty_{k=0} (\beta\theta)^k \Et \bigg[ C_{t+k}^{-\sigma} \frac{P_{H, t-1}}{P_{t+k}}  Y_{t+k}(j) \Big(\frac{\bar P_{H,t}}{P_{H,t-1}} - \frac{\varepsilon}{\varepsilon-1} MC_{t+k} \frac{P_{H,t+k}}{P_{H,t-1}} \Big) \bigg]
    \end{split}
\end{equation*}


Log-linearizing this expression we get that:

$$\bar p_{H,t} = p_{H,t-1} + \sum^\infty_{k=0} (\beta \theta)^k \Et[\bar p_{H,t+1} - p_{H, t}] + \pi_{H, t} + (1- \beta \theta)(mc_t + \mu)$$

Where $\mu = \ln\big(\frac{\varepsilon}{\varepsilon-1}\big)$ In steady state without price stickness $p_{H,t} = p_{H,t-1}$, $\bar p_{H,t+1} = p_{H, t}$, $\pi_{H, t} = 0$ and $\theta =0$, so we get that $mc_t = -\mu$. Rearranging:

$$\bar p_{H,t} = \mu + (1- \beta \theta) \sum^\infty_{k=0} \Et[mc_{t+k} + p_{H, t+k}]$$

Moreover, after find $\bar P_{H,t}$ we can define the domestic prices dynamics of the economy:

$$P_{H,t} [\theta P_{H, t-1}^{1-\epsilon} + (1-\theta) \bar P_{H, t-1}^{1-\epsilon}]^\frac{1}{1- \epsilon}$$

Which log-linearized is:

$$\pi_{H, t} = (1-\theta)(\bar p_{H,t} - p_{H,t-1})$$

As we already found an expression for $\bar p_{H,t} - p_{H,t-1}$ above, we get (vou explicar melhor):

$$\pi_{H, t} = \beta \Et[\pi_{H, t+1}] + \lambda (mc_t + \mu)$$

Where $\lambda \equiv \frac{(1-\theta)(1 - \beta \theta)}{\theta}$.

\section{Equilibrium}

Market clearing in goods market imposes:

$$Y_{t}(j) = \bigg( \frac{P_{H,t}(j)}{P_{H,t}} \bigg)^{-\varepsilon} (C_{H,t} +  \int_0^1 C^i_{H, t} di)$$

Substituing $C_{H,t} = (1-\alpha) \Big( \frac{P_{H,t}}{P_t} \Big)^{-\eta} C_t$ and $C^i_{H,t} = \alpha \Big( \frac{P_{H,t}}{\mathcal{E}_{i,t} P^i_{F,t}} \Big)^{-\gamma} \Big( \frac{P^i_{F,t}}{P^i_{t}} \Big)^{-\eta} C^i_t$

$$Y_{t}(j) = \bigg( \frac{P_{H,t}(j)}{P_{H,t}} \bigg)^{-\varepsilon} \bigg((1-\alpha) \Big( \frac{P_{H,t}}{P_t} \Big)^{-\eta} C_t +  \alpha \int_0^1 \Big( \frac{P_{H,t}}{\mathcal{E}_{i,t} P^i_{F,t}} \Big)^{-\gamma} \Big( \frac{P^i_{F,t}}{P^i_{t}} \Big)^{-\eta} C^i_t di \bigg)$$

As $Y(t) = [\int^1_0 Y_t(j)^{\frac{\varepsilon-1}{\varepsilon}}]^\frac{\varepsilon}{\varepsilon-1}$

\begin{equation*}
    \begin{split}
    Y_{t} &= \Bigg\{ \int^1_0 \Bigg[ \bigg( \frac{P_{H,t}(j)}{P_{H,t}} \bigg)^{-\varepsilon} \bigg((1-\alpha) \Big( \frac{P_{H,t}}{P_t} \Big)^{-\eta} C_t +  \alpha \int_0^1 \Big( \frac{P_{H,t}}{\mathcal{E}_{i,t} P^i_{F,t}} \Big)^{-\gamma} \Big( \frac{P^i_{F,t}}{P^i_{t}} \Big)^{-\eta} C^i_t di \bigg)\Bigg]^{\frac{\varepsilon-1}{\varepsilon}}dj \Bigg\}^\frac{\varepsilon}{\varepsilon-1}\\
    Y_{t} &= \Bigg\{ \int^1_0 \Big( P_{H,t}(j)^{-\varepsilon} \Big)^{\frac{\varepsilon-1}{\varepsilon}}dj \Bigg\}^\frac{\varepsilon}{\varepsilon-1} \Big(\frac{1}{P_{H,t}}\Big)^{-\epsilon} \bigg((1-\alpha) \Big( \frac{P_{H,t}}{P_t} \Big)^{-\eta} C_t +  \alpha \int_0^1 \Big( \frac{P_{H,t}}{\mathcal{E}_{i,t} P^i_{F,t}} \Big)^{-\gamma} \Big( \frac{P^i_{F,t}}{P^i_{t}} \Big)^{-\eta} C^i_t di \bigg)
    \end{split}
\end{equation*}


Considering that $\Big[ \int^1_0 \Big( P_{H,t}(j)^{-\varepsilon} \Big)^{\frac{\varepsilon-1}{\varepsilon}}dj \Big]^\frac{\varepsilon}{\varepsilon-1} = \Big[ \int^1_0 P_{H,t}^{1-\varepsilon}(j)  dj \Big]^\frac{\varepsilon}{\varepsilon-1} = P_{H,t}^{-\epsilon}$ we get:

\begin{equation*}
    \begin{split}
        Y_{t} &= (1-\alpha) \Big( \frac{P_{H,t}}{P_t} \Big)^{-\eta} C_t +  \alpha \int_0^1 \Big( \frac{P_{H,t}}{\mathcal{E}_{i,t} P^i_{F,t}} \Big)^{-\gamma} \Big( \frac{P^i_{F,t}}{P^i_{t}} \Big)^{-\eta} C^i_t di \\
        Y_{t} &= \Big( \frac{P_{H,t}}{P_t} \Big)^{-\eta} \bigg[(1-\alpha)  C_t +  \alpha \int_0^1 \Big( \frac{P_{H,t}}{\mathcal{E}_{i,t} P^i_{F,t}} \Big)^{-\gamma}  \Big( \frac{\mathcal{E}_{i,t} P^i_{F,t}}{P_{H,t}} \Big)^{-\eta} \Big(\frac{P_t}{\mathcal E_{i,t} P_t^i} \Big)^{-\eta} C^i_t di \bigg] \\
        Y_{t} &= C_t \Big( \frac{P_{H,t}}{P_t} \Big)^{-\eta} \bigg[(1-\alpha)  +  \alpha \int_0^1 \Big(\mathcal S^i_t \mathcal S_{i,t} \Big)^{\gamma - \eta} \mathcal Q^{\eta - \frac{1}{\sigma}}_{i,t} di \bigg] 
    \end{split}
\end{equation*}


Log-linearizing and defining $\omega \equiv \sigma \gamma + (1- \alpha)(\sigma \eta - 1)$ we get that:

$$y_t = c_t + \alpha \gamma s_t + \alpha \bigg( \eta - \frac{1}{\sigma} \bigg) q_t = c_t + \frac{\alpha\omega}{\sigma} s_t$$

The previous expressions holds for any foreign country $y_t^i = c_t^i + \frac{\alpha\omega}{\sigma} s^i_t$, but as $\int^1_0 s^i_t di = 0$ we conclude that $y^*_t = \int_0^1 y^i_t di = \int_0^1 c^i_t di + \frac{\alpha\omega}{\sigma} \int_0^1 s^i_t di =\int_0^1 c^i_t di = c^*_t $. As we already know that $c_t = c^*_t + \frac{1}{\sigma} q_t$ and $q_t = (1-\alpha) s_t$ then, defining $\sigma_\alpha \equiv \frac{\sigma}{(1 - \alpha) + \alpha \omega}$ :

$$y_t = c_t + \frac{\alpha\omega}{\sigma} s_t = c^*_t + \frac{1-\alpha}{\sigma} s_t + \frac{\alpha \omega}{\sigma} = y_t^* + \frac{1}{\sigma_\alpha}s_t \Rightarrow (y_{t+1} - y_{t+1}^*) - (y_{t} - y_{t}^*) = \frac{1}{\sigma_\alpha} \Delta s_{t+1}$$

Using log-linearized Euler equation and that $\pi_t = \pi_{H,t} + \alpha \Delta s_t$, and defining $\Theta = \omega-1$:

\begin{equation*}
    \begin{split}
        c_t & = \Et[c_{t+1}] - \frac{1}{\sigma} \Big(r_t - \Et[\pi_{t+1}] - \rho \Big) \Rightarrow\\
        y_t - \frac{\alpha\omega}{\sigma} s_t &= \Et\Big[y_{t+1} - \frac{\alpha\omega}{\sigma} s_{t+1}\Big] - \frac{1}{\sigma} \Big(r_t - \Et[\pi_{t+1}] - \rho \Big)\\
        y_t &= \Et[y_{t+1} ] - \frac{1}{\sigma} \Big(r_t - \Et [\pi_{t+1} ] - \rho \Big) - \frac{\alpha\omega}{\sigma} \Et[\Delta s_{t+1}]\\
        y_t &= \Et[y_{t+1} ] - \frac{1}{\sigma} \Big(r_t - \Et [\pi_{H, t+1} + \alpha \Delta s_{t+1}] - \rho \Big) - \frac{\alpha\omega}{\sigma} \Et[\Delta s_{t+1}]\\
        y_t &= \Et[y_{t+1}] - \frac{1}{\sigma} \Big(r_t - \Et [\pi_{H, t+1}] - \rho \Big) - \frac{\alpha(\omega-1)}{\sigma} \Et[\Delta s_{t+1}]\\
        y_t &= \Et[y_{t+1} ] - \frac{1}{\sigma} \Big(r_t - \Et [\pi_{H, t+1}] - \rho \Big) - \frac{\alpha(\omega-1)}{\sigma} \sigma_\alpha \Et[(y_{t+1} - y_{t+1}^*) - (y_{t} - y_{t}^*)]\\
        y_t &= \Et[y_{t+1}] - \frac{1}{\sigma_\alpha} \Big(r_t - \Et [\pi_{H, t+1}] - \rho \Big) - \alpha \Theta \Et[\Delta y^*_{t+1}]\\
    \end{split}
\end{equation*}


Combining the previous results of marginal cost: $mc_t(j) = -\nu + w_t - p_{H,t} - a_t$, labor supply: $w_t - p_t = \sigma c_t + \varphi n_t$, firm technology aggregation: $y_t = a_t + n_t$, $p_t - p_{H,t} = \alpha s_t$ and $y_t = y_t^* + \frac{1}{\sigma_\alpha} s_t$:

\begin{equation*}
    \begin{split}
    mc_t & = -\nu + (w_t - p_{H, t}) - a_t\\
    & = -\nu + (w_t - pt) + (p_t - p_{H, t}) - a_t\\
    & = -\nu + \sigma c_t + \varphi n_t + \alpha s_t - a_t\\
    & = -\nu + \sigma y_t^* + \varphi + s_t - (1+ \varphi) a_t\\
    & = -\nu + (\sigma_\alpha + \varphi) y_t + (\sigma - \sigma_\alpha) y_t^* - (1+ \varphi) a_t
    \end{split}
\end{equation*}

Defining $\bar y_t$ as the natural level of domestic output (obtained when there is not price stickness, so $mc_t = -\mu$) we get:

$$-\mu = -\nu + (\sigma_\alpha + \varphi) y_t + (\sigma - \sigma_\alpha) y_t^* - (1+ \varphi) a_t \Rightarrow \bar y_t = \Omega + \Gamma a_t + \alpha \Psi y_t^*$$

Where $\Omega \equiv \frac{\nu - \mu}{\sigma_\alpha + \varphi}$, $\Gamma = \frac{1 + \varphi}{\sigma_\alpha + \varphi}$ and $\Psi = -\frac{\Theta \sigma_\alpha}{\sigma_\alpha + \varphi}$. Now, defining $x_t \equiv y_t - \bar y_t$ where $x_t$ as the output gap as the (log) deviation of domestic output from its natural level we can derive the dynamic IS:

\begin{equation*}
    \begin{split}
        y_t &= \Et[y_{t+1}] - \frac{1}{\sigma_\alpha} \Big(r_t - \Et [\pi_{H, t+1}] - \rho \Big) - \alpha(\omega-1) \Et[\Delta y^*_{t+1}]\\
        y_t - \bar y_t &= \Et[y_{t+1} -\bar y_{t+1}] \frac{1}{\sigma_\alpha} \Big(r_t - \Et [\pi_{H, t+1}]  - \rho \Big) - \alpha(\omega-1) \Et[\Delta y^*_{t+1}] + \Gamma(a_{t+1}- a_t)  + \alpha \Psi (y_{t+1}^* - y_t^*)\\
        x_t & = \Et [x_{t+1}] - \frac{1}{\sigma_\alpha} \Big(r_t - \Et[\pi_{H,t+1} - \rho + \sigma_\alpha \Gamma (1-\rho_a)a_t - \alpha \sigma_\alpha (\Theta + \Psi) \Et[\Delta y_{t+1}^*] \Big)\\
        x_t & = \Et [x_{t+1}] - \frac{1}{\sigma_\alpha} \Big(r_t - \Et[\pi_{H,t+1} - \bar{rr}_t \Big)
    \end{split}
\end{equation*}

Where $\bar{rr}_t = \rho - \sigma_\alpha \Gamma (1-\rho_a)a_t + \alpha \sigma_\alpha (\Theta + \Psi) \Et[\Delta y_{t+1}^*]$ is the small open economy's natural rate of interest.\\

In order to define the New Keynesian Philips Curve we use that $\pi_{H, t} = \beta \Et[\pi_{H, t+1}] + \lambda (mc_t + \mu)$ and use the previous results on $mc_t$:

\begin{equation*}
    \begin{split}
        mc_t & = -\nu + (\sigma_\alpha + \varphi) y_t + (\sigma - \sigma_\alpha) y_t^* - (1+ \varphi) a_t \\
        \mu & = + \nu -(\sigma_\alpha + \varphi) \bar y_t - (\sigma - \sigma_\alpha) y_t^* + (1+ \varphi) a_t\\
        mc_t + \mu &= (\sigma_\alpha + \varphi)(y_t - \bar y_t) = (\sigma_\alpha + \varphi)x_t\\
        \pi_{H, t} &= \beta \Et[\pi_{H, t+1}] + \lambda (\sigma_\alpha + \varphi)x_t\\
        \pi_{H, t} &= \beta \Et[\pi_{H, t+1}] + \kappa_\alpha x_t
    \end{split}
\end{equation*}

Where $\kappa_\alpha \equiv \lambda (\sigma_\alpha + \varphi)$.

\section{Monetary Policy}

This framework, as in the usual closed economy new Keynesian model, has two sources of inefficiency which prevents its equilibrium to be Pareto optimal: the first one is market power due to imperfect competition in goods market, with firms causing an intentional decrease in quantities to maximize profit, the second one is the price rigidity, as the firms cannot keep the optimal price at each period. Moreover, with open economy there are still an incentive to monetary authority to deviate from optimal policy as it can use the rate definition to impact terms of trade in a way beneficial to domestic consumers, reducing . However, these problems can be addressed.\\

The first and the third issues can be solved with an employment subsidy for the firms (which can be compensated with a non-distortionaty lump-sum tax in firms or consumers), causing an increase in labor hiring and then in production. With the correct subsidy its possible to replicate exactly the Pareto optimal result of perfect competition, as showed in (Benigno and Benigno (2003)), in the particular case of unitary coefficients of relative risk aversion and substituted between domestic and foreign goods, setting $\tau$ such that $(-\tau)(1-\alpha) = 1 - \frac{1}{\varepsilon}$ or, equivalently, $\nu = \mu + \ln(1-\alpha)$ its sufficient to solve this problem.\\

Finally, the price stickiness distortion can be solved with a monetary policy that at each period set the rate in a way that optimal price decision of the firms is exactly the current price, so price rigidity will not have any impact and the result is efficient, as showed in (Gali, 2003). By NKPC the full stabilization of price level implies on full stabilization also in output gap, using this results to dynamics IS we see that in this optimal policy $r_t = \bar{rr}_t$. As this rule is equivalent to $r_t =  \phi_\pi \pi_{H,t} + \bar{rr}_t$ where $\phi_\pi = 0 < 1$ it would violate the Blanchard and Khan conditions and result in multiple equilibrium. However, by setting an arbitrary $\phi_\pi >1$ (we will use 1.5 ahead), as this policy keeps inflation constant at 0 it does not cause distortions to optimal policy and this coefficient serves only as a threat of that central bank will respond strongly enough to inflation and merely the existence of this threat makes this application unnecessary.


In addition to this optimal rule $r_t = \phi_\pi \pi_{H,t} + \bar{rr}_t$ there are 3 suggested monetary policy rules:
\begin{itemize}
    \item DITR: $r_t = \rho + \phi_\pi \pi_{H,t}$
    \item CITR: $r_t = \rho + \phi_\pi \pi_{t}$
    \item PEG: $e_t = 0$
\end{itemize}

Where DITR is domestic inflation-based Taylor Rule, CIPR is CPI inflation-based Taylor Rule, and PEG is the exchange rate peg. 

\end{document}
