\documentclass{article}
\usepackage[a4paper, total={6in, 8in}, hmargin=0.8in, vmargin=0.8in]{geometry}
\usepackage[utf8]{inputenc}
\usepackage{amsmath}
\usepackage{amsthm}
\usepackage{amsfonts}
\usepackage{bigints}
\usepackage{float}
\usepackage{svg}
\usepackage{multirow}
\usepackage[round]{natbib}
\usepackage{url}
\def\UrlBreaks{\do\/\do-}
\usepackage{breakurl}
\usepackage[hidelinks]{hyperref}
%\usepackage{layout}

\setlength{\topskip}{0mm}
\setlength{\parindent}{0mm}


\newcommand{\Et}{\mathbb{E}_t}
\newcommand{\E}{\mathcal{E}}

\title{\large{\textbf{Macroeconomics III - Project}} \\
\LARGE{Replication of New Keynesian Model for a Small Open Economy and Inclusion of Staggered Wages }
\author{Anna Catarina Batista Tavella\\Matheus Roberto de Bona Franciscão}}
\date{}

\begin{document}
\maketitle

\section*{Abstract}
Replicates and extends \cite{gali_monacelli} model to a small open economy. Terminar

\section{Introduction}
Terminar

\section{Description of the model}

\subsection{Consumers Problem}
The model considers a standard representative consumer with separable preferences that maximizes his expected discounted payoff in a infinite time horizon:

\begin{equation}
    \max_{C_t, N_t} E_0 \sum^\infty_{t=0} \beta^t \left[\frac{C_t^{1-\sigma}}{1-\sigma} - \frac{N_t^{\varphi+1}}{\varphi+1} \right] \hspace{1em}  \textrm{ st. } \hspace{1em} \forall t, \hspace{0.8em} P_t C_t + \Et[Q_{t,t+1} D_{t+1}] \leq D_t +  W_t N_t + T_t
\end{equation}

Where $D_t$ is the nominal payoff in period $t$ of the portfolio held at the end of period $t-1$. The consumer chooses between domestic goods ($C_{H,t}$) and foreign goods ($C_{F,t}$) with $\alpha \in [0,1]$ as the opening index of the economy and elasticity of substitution $\eta > 0$ between domestic and imported goods. The imported goods consumption is divided in goods imported from a continuum of different countries $i$ with elasticity of substitution $\gamma > 0$. Finally, he chooses between goods $j$ made in the same country (domestic or foreign) with elasticity of substitution $\varepsilon > 0$.

\begin{equation}
    C_t \equiv \left[ (1-\alpha)^{\frac{1}{\eta}} (C_{H,t})^{\frac{\eta-1}{\eta}} + \alpha^{\frac{1}{\eta}} (C_{F,t})^{\frac{\eta-1}{\eta}} \right]^{\frac{\eta}{\eta-1}}
\end{equation}

\begin{equation*}
    C_{F,t} \equiv \left( \int^1_0 C_{i,t}^{\frac{\gamma-1}{\gamma}} di \right)^{\frac{\gamma}{\gamma-1}} \quad C_{H,t} \equiv \left( \int^1_0 C_{H,t}(j)^{\frac{\varepsilon-1}{\varepsilon}} dj \right)^{\frac{\varepsilon}{\varepsilon-1}} \quad C_{i,t} \equiv \left( \int^1_0 C_{i,t}(j)^{\frac{\varepsilon-1}{\varepsilon}} dj \right)^{\frac{\varepsilon}{\varepsilon-1}}
\end{equation*}


%Moreover, the price indexes of domestic goods ($P_{H,t}$), imported goods ($P_{F,t}$) and goods imported from a specific country $i$ ($P_{i,t}$) are given by:

The detailed solution of the consumer problem is detailed in the appendix A and results are, as usual, the consumer Euler equation and the labor supply:

\begin{equation}
    \label{euler}
    \beta R_t \Et\left[\left(\frac{C_{t+1}}{C_t} \right)^{-\sigma} \left(\frac{P_t}{P_{t+1}} \right) \right] = 1
\end{equation}

\begin{equation}
    \label{labour_supply}
    C_t^\sigma N_t^\varphi = \frac{W_t}{P_t} \equiv W_t^R
\end{equation}

Where $W_t^R$ is the real wage.

\subsection{Terms of trade}
We define the bilateral terms of trade $S_{i,t}$ between domestic economy and the country $i$ as $S_{i,t} \equiv \frac{P_{i,t}}{P_{H,t}}$. The effective terms of trade given by:

\begin{equation}
    S_t \equiv \left(\int^1_0 S_{i,t}^{1-\gamma} di \right)^{\frac{1}{1-\gamma}} = \left(\int^1_0 \left(\frac{P_{i,t}}{P_{H,t}}  \right)^{1-\gamma}  di \right)^{\frac{1}{1-\gamma}} = \frac{1}{P_{H,t}} \left(\int^1_0 \left(P_{i,t}^{1-\gamma} di \right)^{\frac{1}{1-\gamma}}\right) = \frac{P_{F,t}}{P_{H,t}}
\end{equation}

Using the that price level are given by $P_t^{1-\eta} = (1-\alpha) P_{H,t}^{1-\eta} + \alpha P_{F,t}^{1-\eta}$ we get:

\begin{equation}
    \label{tot_level}
    P_t^{1-\eta} = (1-\alpha) P_{H,t}^{1-\eta} + \alpha \left( P_{H,t}^{1-\eta} S_{t}^{1-\eta} \right) = P_{H,t}^{1-\eta} \left[(1-\alpha) + \alpha S_t^{1-\eta} \right] \Rightarrow P_t = P_{H,t} \left[(1-\alpha) + \alpha S_t^{1-\eta} \right]^{\frac{1}{1-\eta}} 
\end{equation}

Dividing $P_t$ by $P_{t-1}$:

\begin{equation}
    \label{tot}
    \frac{P_t}{P_{t-1}} = \frac{P_{H,t} \left[(1-\alpha) + \alpha S_t^{1-\eta} \right]^{\frac{1}{1-\eta}}} {P_{H,t-1} \left[(1-\alpha) + \alpha S_{t-1}^{1-\eta} \right]^{\frac{1}{1-\eta}}} \Rightarrow \Pi_t = \Pi_{H,t} \left(\frac{(1-\alpha) + \alpha S_t^{1-\eta}} {(1-\alpha) + \alpha S_{t-1}^{1-\eta}} \right)^{\frac{1}{1-\eta}}
\end{equation}

\subsection{Exchange Rate}
We assume that law of one price holds for all goods and all times, defining $\E_{i,t}$ as the bilateral exchange rate between domestic economy and country $i$ and $P^i_{i,t}(j)$ as the price expressed in the producer's currency, then:

\begin{equation}
    P_{i,t}(j) = \E_{i,t} P^i_{i,t}(j) \Rightarrow \left( \int^1_0 \left(P_{i,t}(j) \right)^{1-\epsilon} dj \right)^{\frac{1}{1-\epsilon}} = \left( \int^1_0 \left(\E_{i,t} P^i_{i,t}(j) \right)^{1-\epsilon} dj \right)^{\frac{1}{1-\epsilon}} \Rightarrow P_{i,t} = \E_{i,t} P^i_{i,t}
\end{equation}


Aggregating for all countries $i$

\begin{equation}
    \left( \int^1_0 \left(P_{i,t} \right)^{1-\gamma} di \right)^{\frac{1}{1-\gamma}} = \left( \int^1_0 \left(\E_{i,t} P^i_{i,t} \right)^{1-\gamma} di \right)^{\frac{1}{1-\gamma}} \Rightarrow S_t P_{H,t} = \left( \int^1_0 \left(\E_{i,t} P^i_{i,t} \right)^{1-\gamma} di \right)^{\frac{1}{1-\gamma}}
\end{equation}

Dividing $S_t P_{H,t}$ by $S_{t-1} P_{H,t-1}$ we get:

\begin{equation}
    \label{fx}
    \frac{S_t P_{H,t}}{S_{t-1} P_{H,t-1}} = \frac{\left( \int^1_0 \left(\E_{i,t} P^i_{i,t} \right)^{1-\gamma} di \right)^{\frac{1}{1-\gamma}}}{\left( \int^1_0 \left(\E_{i,t-1} P^i_{i,t-1} \right)^{1-\gamma} di \right)^{\frac{1}{1-\gamma}}} \Rightarrow \frac{S_t}{S_{t-1}}\Pi_{H,t} = \left(\frac{\int^1_0 \left(\E_{i,t} P^i_{i,t} \right)^{1-\gamma} di }{\int^1_0 \left(\E_{i,t-1} P^i_{i,t-1} \right)^{1-\gamma} di} \right)^{\frac{1}{1-\gamma}}
\end{equation}

Which defines implicitly $\E_{t} \equiv \left( \int^1_0 \E_{i,t}^{\frac{\gamma-1}{\gamma}} di \right)^{\frac{\gamma}{\gamma-1}}$ as the nominal effective exchange rate \footnote{Log-linearizing we get $\pi_{H,t+1} + s_{t+1} - s_t = \pi^*_{t+1} + \Delta e_{t+1}$, where $e_t \equiv \int^1_0 e_{i,t}$, $p_t^* \equiv \int^1_0 p^i_{i,t}$ and $\pi^*_t \equiv p_t^* - p_{t-1}^*$. The baseline model of the paper assumes $p_t^* = p^* = 0$, implying that $\pi_{H,t+1} + s_{t+1} - s_t = \Delta e_{t+1}$}.

The bilateral real exchange rate with country $i$ is defined as $\mathcal{Q}_{i,t} \equiv \frac{\E_{i,t} P^i_{i,t}}{P_t}$ and using that $P_{i,t} = \E_{i,t} P^i_{i,t}$ then:

\begin{equation}
\label{real_rate}
\begin{split}
\mathcal{Q}_{i,t} & = \frac{\E_{i,t} P^i_{i,t}}{P_t} = \frac{P_{i,t}}{P_t} \Rightarrow \left( \int^1_0 \mathcal Q_{i,t}^{1-\gamma} di \right)^{\frac{1}{1-\gamma}} = \left( \int^1_0 \left(\frac{P_{i,t}}{P_t}\right)^{1-\gamma} di \right)^{\frac{1}{1-\gamma}} = \frac{1}{P_t}\left( \int^1_0 P_{i,t}^{1-\gamma} di \right)^{\frac{1}{1-\gamma}} \\
\mathcal{Q}_{t} & = \frac{P_{F,t}}{P_t} = \frac{P_{H,t} S_t}{P_t} = \frac{P_{H,t} S_t}{P_{H,t} \left[(1-\alpha) + \alpha S_t^{1-\eta} \right]^{\frac{1}{1-\eta}}} = \left[(1-\alpha)S_t^{\eta-1} + \alpha \right]^{\frac{1}{\eta-1}}
\end{split}
\end{equation}


\subsection{Firms Problem}
A representative firm has a technology with constant returns:

\begin{equation}
    \label{technology}
    Y_t(j) = A_t N_t(j) \Rightarrow \int^1_0 Y_t(j) dj = \int^1_0 A_t N_t(j) dj \Rightarrow \int^1_0 Y_t(j) dj = A_t N_t 
\end{equation}

\begin{equation}
    \label{tfp}
    A_{t+1} = {A_t}^{\rho_a}\exp( \varepsilon^a_t)
\end{equation}

Where $A_t$ is such that $a_t \equiv \ln(A_t)$ follows an AR(1) process and in steady state we normalize $A = 1$.\\

The model assume sticky prices following Calvo (1983): at each period a random selected measure $\theta$ of firms have to keep their prices unchanged from the previous period, while the remaining $1-\theta$ firms can reset them. If a firm could adjust price at time $t$, we will set its price at $\bar P_{H,t}$ which maximizes the present value of its future profit:

\begin{equation}
    \bar P_{H,t} = \max_{\bar P_{H,t}} \sum^\infty_{k=0} \theta^k \Et \left[ Q_{t, t+k}[Y_{t+k}(j) (\bar P_{H,t} - MC_{t+k} P_{H,t+k})] \right]
\end{equation}

Where $MC_t$ is the real marginal cost. Combining the first order of this problem with the price dynamics (detailed solution is in appendix B) and defining $\widehat{MC}_t = \frac{MC_t}{1/ \mathcal M}$ as the marginal cost deviation from steady state we get:

\begin{equation}
    \label{nkpc}
    \Pi_{H,t} = \left[ \theta + \frac{(1-\theta)}{P_{H,t-1}^{1-\varepsilon}} \left(\frac{\Et\left[ \sum^\infty_{k=0} (\beta\theta)^k C_{t+k}^{-\sigma} \frac{1}{P_{t+k}}\tilde C_{H,t+k} P_{H,t+k} \widehat{MC}_{t+k}\right] }{\Et\left[ \sum^\infty_{k=0} (\beta\theta)^k C_{t+k}^{-\sigma} \frac{1}{P_{t+k}} \tilde C_{H,t+k}  \right]} \right)^{1-\varepsilon}\right]^\frac{1}{1-\varepsilon}
\end{equation}

Finally, due to constant returns to scale, the real marginal cost does not depend on quantities. Using the consumers labor supply (\ref{labour_supply}), the technology (\ref{technology}) and (\ref{tot_level}) we can derive the marginal cost deviation from its steady state:

\begin{equation}
    \label{marginal_cost}
    \widehat{MC}_t = \mathcal M MC_t = \frac{W_t (1 - \tau) \mathcal M}{P_{H,t}A_t} = \frac{W_t P_t (1 - \tau)\mathcal M}{P_t P_{H,t}A_t}
    = \frac{W_t^R \left[(1-\alpha) + \alpha S_t^{1-\eta} \right]^{\frac{1}{1-\eta}} (1 - \tau) \mathcal M}{A_t^{1+\varphi}}
\end{equation}

Where $\tau$ is an employment subsidy \footnote{See appendix C for discussion on optimal subsidy} and $\frac{W_t}{P_{H,t}}$ is the real wage in the firm's perspective\footnote{On consumer problem's the real wage $W_t^R \equiv \frac{W_t}{P_t}$ was defined considering the total price index, however when calculating the real marginal cost of the domestic firms the paper uses the real wage considering only the domestic price index ($\frac{W_t}{P_{H,t}}$).}.\\

\subsection{Rest of the world}
Given the small domestic economy, we assume the rest of the world's output $Y_t^*$ can be taken as exogenous and such that $y_t^* = \ln(Y^*_t)$ follows and AR(1) process:

\begin{equation}
    \label{y_row}
    Y^*_{t+1} = {Y^*_t}^{\rho_y}\exp( \varepsilon^*_t)
\end{equation}

Consumer's Euler equations also holds for a representative consumer in any other country:

$$\beta \left( \frac{C^i_{t+1}}{C_t^i} \right) \left( \frac{P_t^i}{P^i_{t+1}} \right) \left( \frac{\E^i_t}{\E^i_{t+1}} \right) = Q^i_{t, t+1}$$

The model assumes complete markets, then there are perfect domestic and international risk sharing, implying that $Q^i_{t, t+1} = Q_{t,t+1}$:

\begin{equation}
\begin{array}{cc}
     &\beta \left( \frac{C^i_{t+1}}{C_t^i} \right)^{-\sigma} \left( \frac{P_t^i \E^i_t}{P^i_{t+1} \E^i_{t+1}} \right) = \beta \left( \frac{C_{t+1}}{C_t^i} \right)^{-\sigma} \left( \frac{P_t}{P_{t+1}} \right) \Rightarrow \left( \frac{C_{t+1}}{C_t^i} \right) = \left( \frac{C^i_{t+1}}{C_t^i} \right) \left(\frac{\frac{\E_{i,t+1} P^i_{t+1}}{P_{t+1}}}{\frac{\E_{i,t} P^i_{t}}{P_{t}}} \right)^{\frac{1}{\sigma}} = \\
     &\left( \frac{C_{t+1}}{C_t} \right) = \left( \frac{C^i_{t+1}}{C_t^i} \right) \left(\frac{\mathcal{Q}_{i,t+1}}{\mathcal{Q}_{i,t}} \right)^{\frac{1}{\sigma}} \Rightarrow \left[\frac{C_{t+1}}{C^i_{t+1} \mathcal Q_{i,t+1}^{\frac{1}{\sigma}}}\right] =  \left[\frac{C_t}{C^i_t \mathcal Q_{i,t}^{\frac{1}{\sigma}}} \right] \\
\end{array}
\end{equation}

Defining $t = 0$ and iterating we get that, for all period $t$

\begin{equation}
    \left[\frac{C_{t}}{C^i_{t} \mathcal Q_{i,t}^{\frac{1}{\sigma}}}\right] =  \left[\frac{C_0}{C^i_0 \mathcal Q_{i,0}^{\frac{1}{\sigma}}} \right] \equiv v_{i, 0} \Rightarrow C_t = v_{i,0} C^i_t \mathcal Q_{i,t}^{\frac{1}{\sigma}}
\end{equation}

Where $v_{i,0}$ depends on initial conditions, which the model assumes be symmetric, then $v_{i,0} = 1$ for all $i$. Aggregating over all countries:

\begin{equation}
    \label{consumption}
    \left(\int^1_0 C_t^{1-\gamma} di \right)^{\frac{1}{1-\gamma}} \equiv C_t = \left(\int^1_0  \left(C_t^i \mathcal Q_{i,t}^{\frac{1}{\sigma}}\right)^{1-\gamma} di \right)^{\frac{1}{1-\gamma}}
\end{equation}


\subsection{Market clearing}
For each domestic good, the total production is equal to domestic + external demands. Substituting the expressions obtained in the consumer problem and aggregating for all domestic goods (detailed steps are in appendix D) we obtain:

\begin{equation}
    \label{output_st}
    Y_{t} = C_t \left[(1-\alpha) + \alpha S_t^{1-\eta} \right]^{\frac{\eta}{1-\eta}} \left[(1-\alpha)  +  \alpha \int_0^1 \left(\mathcal S^i_t \mathcal S_{i,t} \right)^{\gamma - \eta} \mathcal Q^{\eta - \frac{1}{\sigma}}_{i,t} di \right] 
\end{equation}



As we are considering the country as a small open economy, for the rest of the world the domestic consumption and production are insignificant and can be ignored, so market clearing implies:

\begin{equation}
    \label{world_equilibrium}
    C_t^* = Y_t^*
\end{equation}

\section{Equilibrium conditions}
The equilibrium conditions (without a monetary policy equation) are given by equations (\ref{euler}), (\ref{labour_supply}), (\ref{tot}), (\ref{fx}), (\ref{real_rate}), (\ref{technology}), (\ref{tfp}), (\ref{marginal_cost}), (\ref{nkpc}), (\ref{y_row}), (\ref{consumption}), (\ref{output_st}) and (\ref{world_equilibrium}) obtained above:

\begin{subequations}
    \label{se}
    \begin{align}
        &\beta R_t \Et\left[\left(\frac{C_{t+1}}{C_t} \right)^{-\sigma} \left(\frac{P_t}{P_{t+1}} \right) \right] = 1\\
        & C_t^\sigma N_t^\varphi = W_t^R\\
        &\Pi_t = \Pi_{H,t} \left(\frac{(1-\alpha) + \alpha S_t^{1-\eta}} {(1-\alpha) + \alpha S_{t-1}^{1-\eta}} \right)^{\frac{1}{1-\eta}}\\
        &\frac{S_t}{S_{t-1}}\Pi_{H,t} = \left(\frac{\int^1_0 \left(\E_{i,t} P^i_{i,t} \right)^{1-\gamma} di }{\int^1_0 \left(\E_{i,t-1} P^i_{i,t-1} \right)^{1-\gamma} di} \right)^{\frac{1}{1-\gamma}}\\
        &\mathcal{Q}_{t}  = \left[(1-\alpha)S_t^{\eta-1} + \alpha \right]^{\frac{1}{\eta-1}}\\
        &\int^1_0 Y_t(j) dj = A_t N_t\\
        &A_{t+1} = {A_t}^{\rho_a}\exp( \varepsilon^a_{t+1})\\
        &\widehat{MC}_t  = \frac{W_t^R \left[(1-\alpha) + \alpha S_t^{1-\eta} \right]^{\frac{1}{1-\eta}} (1 - \tau) \mathcal M}{A_t^{1+\varphi}}\\
        &\Pi_{H,t} = \left[ \theta + \frac{(1-\theta)}{P_{H,t-1}^{1-\varepsilon}} \left(\frac{\Et\left[ \sum^\infty_{k=0} (\beta\theta)^k C_{t+k}^{-\sigma} \frac{1}{P_{t+k}}\tilde C_{H,t+k} P_{H,t+k} \widehat{MC}_{t+k}\right] }{\Et\left[ \sum^\infty_{k=0} (\beta\theta)^k C_{t+k}^{-\sigma} \frac{1}{P_{t+k}} \tilde C_{H,t+k}  \right]} \right)^{1-\varepsilon}\right]^\frac{1}{1-\varepsilon}\\
        &Y^*_{t+1} = {Y^*_t}^{\rho_y}\exp( \varepsilon^*_{t+1})\\
        &C_t = \left(\int^1_0  \left(C_t^i \mathcal Q_{i,t}^{\frac{1}{\sigma}}\right)^{1-\gamma} di \right)^{\frac{1}{1-\gamma}}\\
        &Y_{t} = C_t \left[(1-\alpha) + \alpha S_t^{1-\eta} \right]^{\frac{\eta}{1-\eta}} \left[(1-\alpha)  +  \alpha \int_0^1 \left(\mathcal S^i_t \mathcal S_{i,t} \right)^{\gamma - \eta} \mathcal Q^{\eta - \frac{1}{\sigma}}_{i,t} di \right] \\
        & C_t^* = Y_t^*
    \end{align}
\end{subequations}

The log-linear approximation of these equations around the steady state is \footnote{Lower case letters represent the neperian logarithm of the original variables. We define $\rho \equiv \beta^{-1} -1$,  $\nu \equiv -ln(1-\tau)$, $\mu \equiv \ln(\mathcal M)$, $\omega \equiv \sigma \gamma + (1-\alpha)(\sigma \eta - 1)$, $\sigma_\alpha \equiv \frac{\sigma}{(1-\alpha) + \alpha \omega}$, $\Omega \equiv \frac{\nu - \mu}{\sigma_\alpha + \varphi}$, $\Gamma \equiv \frac{1 + \varphi}{\sigma_\alpha + \varphi}$, $\Psi \equiv \frac{(1-\omega) \alpha}{\sigma_\alpha + \varphi}$}:

\begin{subequations}
\label{sistema_loglin}
    \begin{align}
        &c_t = \Et[c_{t+1}] - \frac{1}{\sigma} (r_t - \Et[\pi_{t+1}] - \rho)\\
        & w_t^R = \sigma c_t + \varphi n_t \\
        &\pi_t = \pi_{H,t} + \alpha s_t - \alpha s_{t-1}\\
        &s_{t} - s_{t-1} + \pi_{H,t} = \Delta e_t \\
        &q_t = (1-\alpha) s_t\\
        &y_t = a_t +  n_t\\
        &a_{t+1} = \rho_a a_t + \varepsilon^a_{t+1}\\
        &\hat{mc}_{t} = -\nu + w_t^R + \alpha s_t - (1+ \varphi )a_t + \mu\\
        & \pi_{H,t} = \beta \Et[\pi_{H,t+1}] + \lambda \hat{mc}_t\\
        &y^*_{t+1} = \rho_y y^*_t + \varepsilon^*_{t+1}\\
        &c_t = c_t^* + \frac{1}{\sigma} q_t\\
        &y_t = c_t + \alpha \gamma s_t + \alpha \left( \eta - \frac{1}{\sigma} \right) q_t\\
        &c_t^* = y_t^*
    \end{align}
\end{subequations}

To these equations we also add a definition of potential output $\bar y_t$ obtained from (\ref{sistema_loglin}h) when $\hat{mc_t} = 0$ (and manipulating to be a function only of exogenous variables) and a definition of (log) output gap given by output deviation from its potential:

\begin{subequations}
    \begin{align}
        &\bar y_t = \Omega + \Gamma a_t - \alpha \Psi y_t^* \tag{\ref{sistema_loglin}n}\\
        &x_t = y_t - \bar y_t \tag{\ref{sistema_loglin}o}
    \end{align}
\end{subequations}

To complete the model its necessary to include a last equation, relative to monetary policy. The paper suggests 3 different monetary rules and a optimal monetary rule benchmark\footnote{See appendix E for details} and estimates the model in each case:

\begin{itemize}
    \item Optimal Monetary Policy: $r_t = \bar{rr}_t + \phi_\pi \pi_{H, t}$
    \item Domestic Inflation Taylor Rule: $r_t = \rho + \phi_\pi \pi_{H, t}$
    \item CPI Inflation Taylor Rule: $r_t = \rho + \phi_\pi \pi_{t}$
    \item Exchange Rate Peg\footnote{The paper defines Peg as $e_t = 0$ but as the nominal variables are not determined in equilibrium, depending on the initial conditions (only their variations like inflation rates and $\Delta e_t$ are determined) we considered peg as $\Delta e_t = 0$, imposing an initial condition $e_0 = 0$ which implies in $e_t = 0$}: $\Delta e_t = 0$.
\end{itemize}

Where $\bar{rr}_t = \rho + \sigma_\alpha \Gamma (1 - \rho_a) a_t + \alpha \sigma_\alpha (\Theta + \Psi)(\Et[y_{t+1}^*] - y_t^*)$ is the natural interest rate.


\section{Calibration}
The paper assumes the following calibration for the structural parameters:

\begin{table}[H]
    \centering
    \begin{tabular}{clc}
        \hline
        \textbf{Parameter} & \textbf{Description} & \textbf{Value}\\
        \hline
        \multicolumn{3}{l}{\textit{Common to Real Business Cycles Model}}\\
        $\beta$ & Intertemporal discount factor & $0.99$\\
        $\sigma$ & Inverse elasticity of intertemporal subsititution & $1$\\
        $\varphi$ & Inverse Frisch elasticity of labour supply & $3$\\
        $\rho_a$ & Productivity shock smoothing & $0.66$\\
        $\sigma_a$ & Standard deviation of the productivity shocks & $0.0071$ \\
        \multicolumn{3}{l}{\textit{Common to New Keynesian Model}}\\
        $\varepsilon$ & Substitutability between varieties (from the same country) & $6$\\
        $\theta$ & Calvo price stickiness & $0.75$\\
        $\phi_\pi$ & Taylor rule response to inflation & $1.5$ \\
        \multicolumn{3}{l}{\textit{Specific to Galí and Monacelli Model}}\\
        $\alpha$ & Opening index of the economy  & $0.4$\\
        $\eta$ & Substitutability between domestic and imported goods & $1$\\
        $\gamma$ & Substitutability between goods from different foreign countries & $1$\\
        $\rho_{y^*}$ & World GDP shock smoothing & $0.86$\\
        $\sigma_{y^*}$ & Standard deviation of the world GDP shocks & $0.0078$ \\
        $\rho_{ay^*}$ & Correlation between prod. and world GDP shocks & $0.3$ \\
        \hline
    \end{tabular}
\end{table}

In the paper's text and tables the parameter $\rho_a$ is defined as $0.66$, however the IRFs charts are compatible with value $\rho_a = 0.90$ (apparently authors reused the charts from an working paper version \citep{gali_monacelli2002} without update the figure or inform the different parameter in the text). We use this same value in the charts to be comparable with original ones in the paper. 

\section{Steady State Properties}
In the steady state: (i) Purchasing Power Parity holds symmetrically for all other countries (then $\mathcal Q_i = \mathcal Q$, $S_i = S$, $S^i=1$ and $C^i = C^*$), (ii) all the stationary variables are constant and (iii) there is not uncertainty ($\varepsilon^* = \varepsilon^a = 0$).  Using the equilibrium conditions showed in (\ref{se}) immediately by (g), (j) and (m) we obtain that $A = Y^* = C^* = 1$. The equation (a) defines $R = \beta^{-1}$. From (d) and (c) we have $\Pi = \Pi_H = 1$. Using (i) we find $\widehat{MC} = 1$. The remaining system (composed by equations (b), (e), (f), (l) and (l)) is:

\begin{subequations}
    \label{ser}
    \begin{align}
        & C^\sigma N^\varphi = W^R \\
        &\mathcal{Q} = \left[(1-\alpha)S^{\eta-1} + \alpha \right]^{\frac{1}{\eta-1}}\\
        &Y =  N\\
        &1 = W^R \left[(1-\alpha) + \alpha S^{1-\eta} \right]^{\frac{1}{1-\eta}} (1 - \tau) \mathcal M\\
        &C = \mathcal Q^{\frac{1}{\sigma}}\\
        &Y = C \left[(1-\alpha) + \alpha S^{1-\eta} \right]^{\frac{\eta}{1-\eta}} \left[(1-\alpha)  +  \alpha \mathcal S^{\gamma - \eta} \mathcal Q^{\eta - \frac{1}{\sigma}} \right]
    \end{align}
\end{subequations}

This system cannot be solved analytically in the general case, however, using the values showed in calibration for $\alpha, \sigma, \eta, \gamma, \varepsilon$ and $\varphi$ and defining the optimal subsidy ($ 1-\tau = \frac{1}{(1-\alpha)\mathcal M}  $), as showed in appendix C, we obtain the following steady state values:

\begin{table}[H]
    \centering
    \begin{tabular}{clc}
        \hline
        \textbf{Variable} & \textbf{Description} & \textbf{Value}\\
        \hline
        $Y$ & Output & 1.13622 \\
        $C$ & Consumption & 1.07964 \\
        $W^R$ & Real wage & 1.58367 \\
        $C/Y$ & Consumption-to-GDP Ratio & 0.95020 \\
        $S$ & Terms of trade & 1.13622 \\
        $NX/Y$ & New exports in terms of domestic output & $0.00000$\\
        $(R^4 -1)$ & Real annual interest rate &  0.04102\\
        \hline
    \end{tabular}
\end{table}

Note that consumption-to-GDP ratio is lower than 1 while net exports is zero, showing an apparent contradiction. However it is justified as domestic output and domestic consumption uses different prices indexes ($P_H$ and $P$, respectively), then net exports in terms of domestic output is given by $\frac{NX}{Y} = \frac{(Y - C \frac{P}{P_H})}{Y} = 1 - \frac{C}{Y} \frac{P}{P_H} = 1 - \frac{C}{Y} S^{0.4} = 0 $.

\section{Dynamic Properties}

Simulating 1000 samples with 201 periods each we obtained the following dynamic properties:

\begin{table}[H]
    \centering
    \begin{tabular}{lcccc}
        \hline
        & Optimal & DI Taylor & CPI Taylor & Peg\\
        & sd\% & sd\% & sd\% & sd\% \\
        \hline
        Output & 0.93 & 0.66 & 0.70 & 0.84 \\
        Domestic inflation & 0.00 & 0.27 & 0.26 & 0.35 \\
        CPI inflation & 0.38 & 0.40 & 0.26 & 0.21 \\
        Nominal int. rate & 0.32 & 0.40 & 0.40 & 0.21 \\
        Terms of trade & 1.52 & 1.43 & 1.33 & 1.08 \\
        Nominal depr. rate & 0.95 & 0.85 & 0.52 & 0.00 \\
        \hline
        \multicolumn{5}{l}{\textit{Note: } Sd denotes standard deviation in \%}
    \end{tabular}
\end{table}

The contribution to welfare losses are:

\begin{table}[H]
    \centering
    \begin{tabular}{lcccc}
        \hline
        &  DI Taylor & CPI Taylor & Peg\\
        \hline
        \multicolumn{4}{c}{Benchmark $\mu = 1.2$, $\varphi = 3$}\\
        Var(Domestic infl.) & 0.0150 & 0.0143 & 0.0259 \\
        Var(Output gap) & 0.0009 & 0.0019 & 0.0052 \\
        Total & 0.0159 & 0.0162 & 0.0311 \\
        \multicolumn{4}{c}{Low steady state markup $\mu = 1.1$, $\varphi = 10$}\\
        Var(Domestic infl.) & 0.0277 & 0.0262 & 0.0472 \\
        Var(Output gap) & 0.0009 & 0.0019 & 0.0052 \\
        Total & 0.0287 & 0.0281 & 0.0524 \\
        \multicolumn{4}{c}{Low elasticity of labour supply $\mu = 1.2$, $\varphi = 3$}\\
        Var(Domestic infl.) & 0.0226 & 0.0230 & 0.0556 \\
        Var(Output gap) & 0.0005 & 0.0020 & 0.0064 \\
        Total & 0.0231 & 0.0250 & 0.0620 \\
        \multicolumn{4}{c}{Low markup and elasticity of labour supply $\mu = 1.1$, $\varphi = 10$}\\
        Var(Domestic infl.) & 0.0415 & 0.0421 & 0.1011 \\
        Var(Output gap) & 0.0005 & 0.0020 & 0.0063 \\
        Total & 0.0420 & 0.0440 & 0.1074 \\
        \hline
        \multicolumn{4}{l}{\textit{Note: } Values are \% units of steady state consumption}
    \end{tabular}
\end{table}

\section{Impulse Response Functions}
We show the IRFs relative to both shocks present in the model (TFP and world output shocks) in the main variables in the economy.

\begin{figure}[H]
\centering
\includesvg[inkscapelatex=false, width=0.9\textwidth]{img/figure1.svg}
\includesvg[inkscapelatex=false, width=0.9\textwidth]{img/figure2.svg}
\end{figure}

Explicar!

\section{Modification}
We include wage stickiness following \cite{erceg} and \citet[Chap.~6]{gali2015} in the small open economy model. This framework add imperfect competition in the labor market and consider unions that can define labor supply and set nominal wages aiming to maximize workers utility and that decision are subject to nominal rigidities following Calvo, with the same mechanism of the price setting by the firms. This modification changes the consumer problem (as they do not decides individually their work hours) and the firms (by way of technology), while the external sector relations and the monetary policy rules are unaffected.

\subsection{Firms}
Each represent firm $j$ now uses a continuum of different labor types ($x \in [0,1]$) as inputs:

\begin{equation}
    Y_t(j) = A_t N_t(j) \quad N_t(j) \equiv \left( \int^1_0 N_t(j, x)^{\frac{\zeta-1}{\zeta}} dx \right)^{\frac{\zeta}{\zeta-1}}
\end{equation}

Where $\zeta$ represents the elasticity of substitution among labor varieties and $W_t(x)$ is the nominal wage per unit of $x$-type labor. Analogously to the consumer problem that solve for the optimal demand for each type of good given the individual aggregate consumption, for the firm cost minimization problem given $N_t(j)$ we have that the optimal demand for $n$-type labor is:

\begin{equation}
    N_t(j,x) = \left(\frac{W_t(x)}{W_t} \right)^\zeta N_t(j) \quad \textrm{ where } \ W_t \equiv \left(\int^1_0 W_t(x)^{1-\zeta} dx \right)^{\frac{1}{1-\zeta}}
\end{equation}

The $W_t$ and $N_t(j)$ above are such that in aggregate terms $\int^1_0 W_t(x) N_t(j,x) dx = W_t N_t(j)$. Follows that the firms price setting problem remains unchanged as marginal cost is constant due to constant returns to scale and $W_t$ is calculated by aggregator above.

\subsection{Households}
Moreover we assume that all consumers homogeneously supply all the labor types implying that his income is given by $\int^1_0 W_t(x) N_t(x) dx$, leading to same problem discussed in original model:

\begin{equation}
    \max_{C_t} E_0 \sum^\infty_{t=0} \beta^t \left[\frac{C_t^{1-\sigma}}{1-\sigma} - \frac{N_t^{\varphi+1}}{\varphi+1} \right] \hspace{1em}  \textrm{ st. } \hspace{1em} \forall t, \hspace{0.8em} P_t C_t + \Et[Q_{t,t+1} D_{t+1}] \leq D_t +  \int^1_0 W_t(x) N_t(x) dx + T_t
\end{equation}

The major difference is that now the consumers do not chose anymore the hours worked, thus labor income is now exogenous and the maximization problem solution consists only in the Euler equation showed in (\ref{euler}).\\

\subsection{Wage setting}

The model assumes that there exists a continuum of representative unions that can determine the nominal wages for each labor type. Analouslgy to price setting problem we assume that each union is not completely free to adjust their wage at any period, but only a random selected fraction of them with measure $\varsigma$ can reset the wages in a given period, while the remaining fraction must keep the nominal wage unchanged. The union wage setting problem consider the utility maximization of the workers, taking as exogenous the other union decisions and the labor demand by the firms. Formally:

\begin{equation}
    \label{wage_setting}
    \max_{W_t^*} \Et \sum^\infty_{k=0} (\beta \varsigma)^k \left( \frac{C_{t+k}^{-\sigma}}{C_t^{-\sigma}} \frac{P_t}{P_{t+k}} W_t^* N_{t+k|t} - \frac{N_{t+k|t}^{1+\varphi}}{1+\varphi} \right)Z_{t+k} \quad \textrm{such that} \quad N_{t+k|k} = \left(\frac{W_t^*}{W_{t+k}} \right)^{-\zeta} \left(\int_0^1 N_t(j) dj \right)
\end{equation}

The first order condition is given by:

\begin{equation}
    \label{foc_wage}
    \sum^\infty_{k=0} (\beta \varsigma)^k \Et \left[N_{t+k|t} Z_{t+k} \left(\frac{C_{t+k}^{-\sigma} W_t^*}{P_{t+k}} - \Xi N_{t+k|t}^\varphi \right) \right] = 0
\end{equation}

Where $\Xi = \frac{\zeta}{\zeta-1}$. Log-linearizing and rearranging the expression above and defining $\xi = \ln(\Xi)$ we get:

\begin{equation}
    w_t^* = (1 - \beta \varsigma) \sum^\infty_{k=0} (\beta \rho)^k \Et[\xi +  \sigma c_{t+k} + \varphi n_{t+k|t} + p_{t+k}]
\end{equation}


Log-linearizing the constraint in (\ref{wage_setting}) we obtain that $n_{t+k|t} = -\zeta w_t^* + \zeta w_{t+k} + n_{t+k}$, thus:

\begin{equation}
    w_t^* = (1 - \beta \varsigma) \sum^\infty_{k=0} (\beta \rho)^k \Et[\xi +  \sigma c_{t+k} + \varphi( \zeta w_{t+k} -\zeta w_t^* + n_{t+k}) + p_{t+k}]
\end{equation}

\begin{equation}
    w_t^* = \frac{1-\beta \varsigma}{1 + \zeta \varphi} \sum^{\infty}_{k=0} (\beta \varsigma)^k \Et[\xi + \sigma c_{t+k} + \varphi n_{t+k} + \zeta \varphi w_{t+k} + p_{t+k}]
\end{equation}

Writing in a recursive form:

\begin{equation}
    \label{opt_wage}
    w_t^* = \beta \varsigma \Et[w^*_{t+1}] + (1 - \beta \varsigma) \left[w_t - (1+\zeta \varphi)^{-1} (w_t^R  - \sigma c_t - \varphi n_t - \xi) \right]
\end{equation}

\subsection{Wage inflation dynamics}
The wage inflation is defined as $\Pi_{w,t} = \frac{W_t}{W_{t-1}}$ or, log-linearizing, $\pi_{w,t} = w_t - w_{t-1}$. As we defined real wage $W_t^R = \frac{W_t}{P_t}$ we can derive a relation between goods and wage inflation:

\begin{equation}
    \Pi_{w,t} = \frac{W_t}{W_{t-1}} = \frac{W_t^R P_t}{W_{t-1}^R P_{t-1}} = \frac{W_t^R \Pi_t}{W_{t-1}^R}
\end{equation}

Log-linearizing:

\begin{equation}
    \label{wage_inflation}
    \pi_{w,t} = w_t^R - w_{t-1}^R + \pi_t
\end{equation}

The aggregate wage dynamics is given by:

\begin{equation}
    W_t = \left(\varsigma W_{t-1}^{1-\zeta} + (1-\varsigma) (W_t^*)^{1-\zeta} \right)^{\frac{1}{1-\zeta}}
\end{equation}

Log-linearizing and calculating $\pi_{w,t}$:

\begin{equation}
    \Rightarrow w_t = \varsigma w_{t-1} + (1-\varsigma) w_t^* \Rightarrow \pi_{w,t} = (1-\varsigma)(w_t^* - w_{t-1})
\end{equation}

Substituting the expression for $w_t^*$ found in (\ref{opt_wage}) and rearranging:

\begin{equation}
    \label{wage_pc}
    \pi_{w,t} = \beta \Et[\pi_{w,t+1}] - \Lambda (w_t^R  - \sigma c_t - \varphi n_t - \xi)
\end{equation}

Where $\Lambda \equiv \frac{(1-\varsigma)(1- \beta \varsigma)}{\varsigma(1+ \zeta \varphi)}$.

\subsection{Equilibrium}
The new log-linearized equilibrium is the same showed in (\ref{sistema_loglin}) but removing the labor supply equation $(\ref{sistema_loglin}b)$ and adding (\ref{wage_inflation}) and (\ref{wage_pc}). \\

Note that this new model generalizes the original one in \cite{gali_monacelli}: assuming that workers have no market power to set wages due to perfect substitutability among labor types $\zeta \to \infty$ (thus $\Xi \to 1$ and $\xi \to 0$), and there is not wage stickiness $\varsigma \to 0$ (thus $\Lambda \to \infty$) we are back to baseline model.

\section{The new implications for the static and dynamics properties of the model}

\subsection{Steady state}
Using the first order condition in wage setting problem in (\ref{foc_wage}) evaluated in steady state it becomes $W^R = \Xi N^\varphi C^\sigma$. Thus, the exact same steady state found in (\ref{ser}) must hold, with only change the substitution of labor supply equation (\ref{ser})a) to the expression above:

\begin{subequations}
    \label{ser_wage}
    \begin{align}
        & \Xi C^\sigma N^\varphi = W^R \\
        &\mathcal{Q} = \left[(1-\alpha)S^{\eta-1} + \alpha \right]^{\frac{1}{\eta-1}}\\
        &Y =  N\\
        &1 = W^R \left[(1-\alpha) + \alpha S^{1-\eta} \right]^{\frac{1}{1-\eta}} (1 - \tau) \mathcal M\\
        &C = \mathcal Q^{\frac{1}{\sigma}}\\
        &Y = C \left[(1-\alpha) + \alpha S^{1-\eta} \right]^{\frac{\eta}{1-\eta}} \left[(1-\alpha)  +  \alpha \mathcal S^{\gamma - \eta} \mathcal Q^{\eta - \frac{1}{\sigma}} \right]
    \end{align}
\end{subequations}

It is immediate that if $\zeta \to \infty$ then $\Xi \to 1$ and the system above results in the same of the original model. Assuming the same calibration showed in section 4 and, for the new parameters, adopting $\varsigma = 0.75$ (consistent 1 year as average time to change wages) and $\zeta = 4$ (following \cite{erceg}, that considered a wage markup of $1/3$) we obtain that steady state values are identical to original model except the real wage:

\begin{table}[H]
    \centering
    \begin{tabular}{clcc}
        \hline
        \multirow{2}*{\textbf{Variable}} & \multirow{2}*{\textbf{Description}} & \textbf{Original} & \textbf{Model with}\\
        & & \textbf{Model} & \textbf{Price Stickiness} \\
        \hline
        $Y$ & Output & 1.13622 & 1.13622 \\
        $C$ & Consumption & 1.07964 & 1.07964 \\
        $W^R$ & Real wage & 1.58367 & 2.11156 \\
        $C/Y$ & Consumption-to-GDP Ratio & 0.95020 & 0.95020 \\
        $S$ & Terms of trade & 1.13622 & 1.13622 \\
        $NX/Y$ & New exports in terms of domestic output & $0.00000$ & $0.00000$\\
        $(R^4 -1)$ & Real annual interest rate &  0.04102 &  0.04102\\
        \hline
    \end{tabular}
\end{table}

\subsection{Dynamic Properties}
Again, simulating 1000 samples with 201 periods, we obtained the following dynamic properties\footnote{The previously obtained optimal policy rule 
$r_t = \bar{rr}_t + \phi_\pi \pi_{H,t}$ is no longer optimal in that new framework due to change in the welfare cost. We proceed our comparison with only DITR, CITR and Peg.}

\begin{table}[H]
    \centering
    \begin{tabular}{lcccccc}
        \hline
        & \multicolumn{2}{c}{DI Taylor} & \multicolumn{2}{c}{CPI Taylor} & \multicolumn{2}{c}{Peg}\\
        & Model & Modif.  & Model & Modif.  & Model & Modif. \\
        &  sd\% & sd\% & sd\% & sd\% & sd\% & sd\% \\
        \hline
        Output &  0.66 & 1.81 & 0.70 & 1.75 & 0.84 & 1.43 \\
        Domestic inflation & 0.27 & 0.37 & 0.26 & 0.33 & 0.35 & 0.25 \\
        CPI inflation & 0.40 & 0.41 & 0.26 & 0.33 & 0.21 & 0.15 \\
        Nominal int. rate & 0.40 & 0.55 & 0.40 & 0.50 & 0.21 & 0.20 \\
        Terms of trade & 1.43 & 1.29 & 1.33 & 1.12 & 1.08 & 1.03 \\
        Nominal depr. rate & 0.85 & 0.61 & 0.52 & 0.43 & 0.00 & 0.00 \\
        \hline
        \multicolumn{5}{l}{\textit{Note: } Sd denotes standard deviation in \%}
    \end{tabular}
\end{table}

Talvez colocar tambem correlação das variáveis com y para ver se é pro-cíclico?

\subsection{Impulse Response Functions}
The new calculated impulse response functions to both shocks are:

\begin{figure}[H]
\centering
\includesvg[inkscapelatex=false, width=0.9\textwidth]{img/figure3.svg}
\end{figure}

\begin{figure}[H]
\centering
\includesvg[inkscapelatex=false, width=0.9\textwidth]{img/figure4.svg}
\end{figure}

Explicar! Ressaltar que as variáveis reais / quantidades reagem mais fortemente dada a maior rigidez nominal\\

\begin{figure}[H]
\centering
\includesvg[inkscapelatex=false, width=0.9\textwidth]{img/figure5.svg}
\end{figure}

\subsubsection{Welfare Losses}
Following the derivation for welfare losses in a closed economy with wage stickiness in \citet[Appendix.~6.1]{gali2015} and in a open economy with openness index $\alpha$ in \cite{rhee} the the expected welfare losses as fraction of steady state consumption are given by:

\begin{equation}
    \mathbb V = -\frac{(1-\alpha)}{2} \left[ (1 + \varphi) var(x_t) + \frac{\varepsilon}{\lambda} var(\pi_t) + \frac{\zeta}{\Lambda} var(\pi_{w,t}) \right] 
\end{equation}

Now calculating the contributions to welfare losses in the new model, considering different levels of wage rigidities:

\begin{table}[H]
    \centering
    \begin{tabular}{lcccc}
        \hline
        &  DI Taylor & CPI Taylor & Peg\\
        \hline
        \multicolumn{4}{c}{No wage rigidity: $\varsigma = 0$}\\
        Var(Domestic infl.) & 0.01506 & 0.01570 & 0.04082\\
        Var(Output gap) & 0.00088 & 0.00224 & 0.00662\\
        Var(Wage infl.) & 0.00000 & 0.00000 & 0.00000\\
        Total & 0.01594 & 0.01794 & 0.04744\\
        \multicolumn{4}{c}{Low wage rigidity: $\varsigma = 0.25$}\\
        Var(Domestic infl.) & 0.01587 & 0.01334 & 0.02549\\
        Var(Output gap) & 0.00106 & 0.00420 & 0.01076\\
        Var(Wage infl.) & 0.01473 & 0.02009 & 0.02418\\
        Total & 0.03167 & 0.03763 & 0.06043\\
        \multicolumn{4}{c}{Intermediate wage rigidity: $\varsigma = 0.50$}\\
        Var(Domestic infl.) & 0.02171 & 0.01706 & 0.01889\\
        Var(Output gap) & 0.00353 & 0.00764 & 0.01460\\
        Var(Wage infl.) & 0.05409 & 0.04948 & 0.01989\\
        Total & 0.07933 & 0.07417 & 0.05338\\
        \multicolumn{4}{c}{Standard wage rigidity: $\varsigma = 0.75$}\\
        Var(Domestic infl.) & 0.03002 & 0.02320 & 0.01485\\
        Var(Output gap) & 0.02904 & 0.02963 & 0.02146\\
        Var(Wage infl.) & 0.29458 & 0.21234 & 0.01422\\
        Total & 0.35364 & 0.26516 & 0.05053\\
        \hline
        \multicolumn{4}{l}{\textit{Note: } Values are \% units of steady state consumption}
    \end{tabular}
\end{table}

Comentar! Quanto mais rígido comparativamente a PEG fica superior

\section{Conclusion}
a


\appendix
\section{Consumer problem}
Consumer problem solution.

\section{Labor subsidy definition}
As the model depends on imperfect competition assumption the firms have market power implying that competitive equilibrium is not Pareto Optimal due to lower production and hiring. In this case, the domestic benevolent social planner maximizes the representative household discounted utility subject to technology (\ref{technology}), domestic/foreign consumption relation due to international risk sharing (\ref{consumption}) and market clearing (\ref{output_st}). For a analytically tractable solution we need to impose $\gamma = \sigma = \eta = 1$ (as adopted in the calibration). The problem becomes:

$$\max_{C_t, N_t} E_0 \sum^\infty_{t=0} \beta^t \left[\ln(C_t) - \frac{N_t^{\varphi+1}}{\varphi+1} \right] \ \textrm{ s.t. } \begin{matrix}
    Y_t = A_t N_t\\
    C_t = C_t^* \mathcal Q_{t}\\
    Y_t = C_t S_t^\alpha
\end{matrix}$$

By (\ref{real_rate}) with $\gamma = \sigma = \eta = 1$ we have that $\mathcal{Q}_t = S_t^{1-\alpha}$, and using the global market clearing condition in (\ref{world_equilibrium}): $Y_t^* = C_t^*$ we rewrite the social planner problem:

$$\max_{C_t, N_t} E_0 \sum^\infty_{t=0} \beta^t \left[\ln(C_t) - \frac{N_t^{\varphi+1}}{\varphi+1} \right] \ \textrm{ s.t. } \begin{matrix}
    C_t = (A_t N_t)^{1-\alpha}(Y_t^*)^\alpha
\end{matrix}$$

As this problem is a static one we can solve separately for $C_t$ and $N_t$ at any period. The Lagrangean and the first order conditions are:

\begin{equation*}
    \begin{split}
        \mathcal L = & \ln(C_t) - \frac{N_t^{\varphi+1}}{\varphi+1} - \lambda_t(C_t - (A_t N_t)^{1-\alpha}(Y_t^*)^\alpha) \\
        (C_t) \ & \frac{1}{C_t} - \lambda_t = 0\\
        (N_t) \ & -N_t^\varphi + \lambda_t (Y_t^*)^\alpha (1 - \alpha) A_t^{(1 - \alpha)}N_t^{-\alpha} = 0\\
    \end{split}
\end{equation*}

Manipulating we get that $N_t^{1+\varphi} = N^{1+\varphi} = (1-\alpha)$. In the competitive steady state in (\ref{ser}) again assuming $\gamma = \sigma = \eta = 1$ we obtain that $\frac{1}{\mathcal M} = (1-\tau) N^{1+\varphi}$. So, if the subsidy is such that $(1-\tau) = \frac{1}{(1-\alpha) \mathcal M}$ the steady state employment level coincides to Pareto optimal and, then, the efficiency is restored. Note that the solution above holds only for this specific parameter selection.

\section{Optimal price setting}
The firms optimal price setting problem following the Calvo's model is given by:

\begin{equation}
    \bar P_{H,t} = \max_{\bar P_{H,t}} \sum^\infty_{k=0} \theta^k \Et \left[ Q_{t, t+k}[Y_{t+k}(j) (\bar P_{H,t} - MC_{t+k} P_{H,t+k})] \right]
\end{equation}

The domestic demand for a specific variety is $C_{H,t}(j) = \left( \frac{P_{H,t}(j)}{P_{H,t}} \right)^{-\varepsilon} C_{H, t}$ if the price remains unchanged at $\bar P_{H,t}$ until $t+k$ period then: $C_{t+k}(j) = \left( \frac{\bar P_{H,t}(j)}{P_{H,t+k}}\right)^{-\varepsilon} C_{H, t+k}$. Similarly the foreign consumption of this domestic good is $C^i_{t+k}(j) = \int^1_0 \left( \frac{\bar P_{H,t}(j)}{P_{H,t+k}} \right)^{-\varepsilon} C^i_{H, t+k} di$. Market clearing imposes that:

\begin{equation}
    Y_{t+k}(j) = C_{H,t+k}(j) + \int_0^1 C^i_{H, t+k}(j) di = \left( \frac{\bar P_{H,t}}{P_{H,t+k}} \right)^{-\varepsilon} \left(C_{H,t+k} +  \int_0^1 C^i_{H, t+k} di \right) \equiv \left( \frac{\bar P_{H,t}}{P_{H,t+k}} \right)^{-\varepsilon} \tilde C_{H,t+k}
\end{equation}

Substituting $Q_{t, t+k}$ for the expression obtained in the consumer problem and $Y_{t+k}(j)$ for the expression above in the firms problem:

\begin{equation}
    \bar P_{H,t} = \max_{\bar P_{H,t}} \sum^\infty_{k=0} \theta^k \Et \left[ \beta^k \left( \frac{C_{t+k}}{C_t} \right)^{-\sigma} \left( \frac{P_{t}}{P_{t+k}} \right) \left( \frac{\bar P_{H,t}}{P_{H,t+k}} \right)^{-\varepsilon} \tilde C_{H,t+k} (\bar P_{H,t} - MC_{t+k} P_{H,t+k})  \right]
\end{equation}

Calculating the first order condition with respect to $\bar P_{H,t}$ and rearranging we get:

\begin{equation}
    \bar P_{H,t} =  \frac{\Et\left[ \sum^\infty_{k=0} (\beta\theta)^k C_{t+k}^{-\sigma} \frac{1}{P_{t+k}}\tilde C_{H,t+k} P_{H,t+k} MC_{t+k} \mathcal M \right] }{\Et\left[ \sum^\infty_{k=0} (\beta\theta)^k C_{t+k}^{-\sigma} \frac{1}{P_{t+k}} \tilde C_{H,t+k}  \right]}
\end{equation}

In the zero inflation steady state $\bar P_{H,t} = P_{H,t} = P_{t} = P_H$, implying that, by the previous formula, $MC_{t}  = \frac{1}{\mathcal M} \equiv \frac{\varepsilon-1}{\varepsilon}$. Thus we define $\widehat{MC}_t = \frac{MC_t}{1/ \mathcal M}$ as the marginal cost deviation from steady state. Now using the price dynamics:

\begin{equation}
    P_{H,t} = [ \theta P_{H,t-1}^{1-\varepsilon} + (1-\theta) \bar P_{H,t}^{1-\varepsilon}]^\frac{1}{1-\varepsilon} \Rightarrow \Pi_{H,t} = \frac{P_{H,t}}{P_{H,t-1}} =  \frac{\left[ \theta P_{H,t}^{1-\varepsilon} + (1-\theta) \bar P_{H,t}^{1-\varepsilon}\right]^\frac{1}{1-\varepsilon}}{P_{H,t-1}} 
\end{equation}

\begin{equation}
    P_{H,t} = [ \theta P_{H,t-1}^{1-\varepsilon} + (1-\theta) \bar P_{H,t}^{1-\varepsilon}]^\frac{1}{1-\varepsilon} \Rightarrow \Pi_{H,t} = \frac{P_{H,t}}{P_{H,t-1}} =  \left[ \theta + (1-\theta) \frac{\bar P_{H,t}^{1-\varepsilon}}{P_{H,t-1}^{1-\varepsilon}} \right]^\frac{1}{1-\varepsilon} 
\end{equation}

Substituing $\bar P_{H,t}$ we reach:

\begin{equation}
    \Pi_{H,t} = \left[ \theta + \frac{(1-\theta)}{P_{H,t-1}^{1-\varepsilon}} \left(\frac{\Et\left[ \sum^\infty_{k=0} (\beta\theta)^k C_{t+k}^{-\sigma} \frac{1}{P_{t+k}}\tilde C_{H,t+k} P_{H,t+k} \widehat{MC}_{t+k}\right] }{\Et\left[ \sum^\infty_{k=0} (\beta\theta)^k C_{t+k}^{-\sigma} \frac{1}{P_{t+k}} \tilde C_{H,t+k}  \right]} \right)^{1-\varepsilon}\right]^\frac{1}{1-\varepsilon}
\end{equation}


Log-linearizing we get:

\begin{equation}
    \frac{\pi_{H,t} + p_{H,t-1}}{(1-\theta)(1-\beta \theta) } = \sum^\infty_{k=0} (\beta \theta)^k \Et[\hat{mc}_{t+k} + p_{t+k}] 
\end{equation}

To obtain a recursive form, subtract for the same expression in $t+1$ multiplied by $\beta$, apply the Law of Iterated Expectations and rearrange:

\begin{equation}
    \frac{\pi_{H,t} - \beta \Et[\pi_{H,t+1}]}{(1-\theta)(1-\beta \theta) } = \sum^\infty_{k=0} (\beta \theta)^k \Et[\hat{mc}_t + p_{t+k}]  - \beta \Et \sum^\infty_{k=0} (\beta \theta)^k  \mathbb E[\hat{mc}_{t+1+k} + p_{t+k+1}] 
\end{equation}


\begin{equation}
    \pi_{H,t} = \beta \Et[\pi_{H,t+1}] + \lambda \hat{mc}_t
\end{equation}

Where $\lambda \equiv \frac{(1-\theta) (1 - \beta \theta)}{\theta}$.

\section{Goods market clearing}
Market clearing in goods market imposes that, for each domestic good, the total production is equal to domestic + external demands:

\begin{equation}
    Y_{t}(j) = \left( \frac{P_{H,t}(j)}{P_{H,t}} \right)^{-\varepsilon} C_{H,t} + \left( \frac{P_{H,t}(j)}{P_{H,t}} \right)^{-\varepsilon} \int_0^1 C^i_{H, t} di
\end{equation}

Substituing $C_{H,t} = (1-\alpha) \left( \frac{P_{H,t}}{P_t} \right)^{-\eta} C_t$ and $C^i_{H,t} = \alpha \left( \frac{P_{H,t}}{\mathcal{E}_{i,t} P^i_{F,t}} \right)^{-\gamma} \left( \frac{P^i_{F,t}}{P^i_{t}} \right)^{-\eta} C^i_t$

\begin{equation}
    Y_{t}(j) = \left( \frac{P_{H,t}(j)}{P_{H,t}} \right)^{-\varepsilon} \left((1-\alpha) \left( \frac{P_{H,t}}{P_t} \right)^{-\eta} C_t +  \alpha \int_0^1 \left( \frac{P_{H,t}}{\mathcal{E}_{i,t} P^i_{F,t}} \right)^{-\gamma} \left( \frac{P^i_{F,t}}{P^i_{t}} \right)^{-\eta} C^i_t di \right)
\end{equation}

As $Y_t = \left[\int^1_0 Y_t(j)^{\frac{\varepsilon-1}{\varepsilon}} \right]^\frac{\varepsilon}{\varepsilon-1}$

\begin{equation}
    \label{}
    \begin{split}
    Y_{t} &= \left\{ \int^1_0 \left[ \left( \frac{P_{H,t}(j)}{P_{H,t}} \right)^{-\varepsilon} \left((1-\alpha) \left( \frac{P_{H,t}}{P_t} \right)^{-\eta} C_t +  \alpha \int_0^1 \left( \frac{P_{H,t}}{\mathcal{E}_{i,t} P^i_{F,t}} \right)^{-\gamma} \left( \frac{P^i_{F,t}}{P^i_{t}} \right)^{-\eta} C^i_t di \right)\right]^{\frac{\varepsilon-1}{\varepsilon}}dj \right\}^\frac{\varepsilon}{\varepsilon-1}\\
    Y_{t} &= \left\{ \int^1_0 \left( P_{H,t}(j)^{-\varepsilon} \right)^{\frac{\varepsilon-1}{\varepsilon}}dj \right\}^\frac{\varepsilon}{\varepsilon-1} \left(\frac{1}{P_{H,t}}\right)^{-\epsilon} \left((1-\alpha) \left( \frac{P_{H,t}}{P_t} \right)^{-\eta} C_t +  \alpha \int_0^1 \left( \frac{P_{H,t}}{\mathcal{E}_{i,t} P^i_{F,t}} \right)^{-\gamma} \left( \frac{P^i_{F,t}}{P^i_{t}} \right)^{-\eta} C^i_t di \right)
    \end{split}
\end{equation}


Considering that $\left[ \int^1_0 \left( P_{H,t}(j)^{-\varepsilon} \right)^{\frac{\varepsilon-1}{\varepsilon}}dj \right]^\frac{\varepsilon}{\varepsilon-1} = \left[ \int^1_0 P_{H,t}^{1-\varepsilon}(j)  dj \right]^\frac{\varepsilon}{\varepsilon-1} = P_{H,t}^{-\epsilon}$ we get:

\begin{equation}
    \begin{split}
        Y_{t} &= (1-\alpha) \left( \frac{P_{H,t}}{P_t} \right)^{-\eta} C_t +  \alpha \int_0^1 \left( \frac{P_{H,t}}{\mathcal{E}_{i,t} P^i_{F,t}} \right)^{-\gamma} \left( \frac{P^i_{F,t}}{P^i_{t}} \right)^{-\eta} C^i_t di \\
        Y_{t} &= \left( \frac{P_{H,t}}{P_t} \right)^{-\eta} \left[(1-\alpha)  C_t +  \alpha \int_0^1 \left( \frac{P_{H,t}}{\mathcal{E}_{i,t} P^i_{F,t}} \right)^{-\gamma}  \left( \frac{\mathcal{E}_{i,t} P^i_{F,t}}{P_{H,t}} \right)^{-\eta} \left(\frac{P_t}{\mathcal E_{i,t} P_t^i} \right)^{-\eta} C^i_t di \right]
    \end{split}
\end{equation}

\begin{equation}
    Y_{t} = C_t \left( \frac{P_{H,t}}{P_t} \right)^{-\eta} \left[(1-\alpha)  +  \alpha \int_0^1 \left(\mathcal S^i_t \mathcal S_{i,t} \right)^{\gamma - \eta} \mathcal Q^{\eta - \frac{1}{\sigma}}_{i,t} di \right] 
\end{equation}

And finally, using (\ref{tot_level}) we obtain:

\begin{equation}
    Y_{t} = C_t \left[(1-\alpha) + \alpha S_t^{1-\eta} \right]^{\frac{\eta}{1-\eta}} \left[(1-\alpha)  +  \alpha \int_0^1 \left(\mathcal S^i_t \mathcal S_{i,t} \right)^{\gamma - \eta} \mathcal Q^{\eta - \frac{1}{\sigma}}_{i,t} di \right] 
\end{equation}

\section{Optimal Monetary Policy and Welfare Losses}
a

\nocite{*}
\bibliographystyle{plainnat}
\bibliography{bibliografia}


\end{document}
